\section{Conclusion and Future Work}\label{conclusion}
This article describes a bytecode weakest precondition calculus applied to a bytecode specification language (BCSL).
BCSL is defined as suitable extensions of the Java class file format.
Implementations for a proof obligation generator and a JML compiler to BCSL have been developed and are part of the Jack 1.8 release\footnote{http://www-sop.inria.fr/everest/soft/Jack/jack.html}.
At this step, we have built a framework for Java program verification. This validation can be done at source or at bytecode level in a common environment: for instance, to prove lemmas ensuring bytecode correctness all the current and future provers plugged in Jack can be used.

 We envisage several directions for future work:
\begin{itemize}
  \item perform case studies
  \item build the missing part of the PCC architecture described in Section \ref{architecture} where the proofs are done interactively over the source code.
 As we discussed in Section \ref{results} in the performed tests the 
proof obligations generated over a source program and over its compilation with non optimizing compiler are
 syntactically the same modulo names and basic types. We aim to establish formally this property.  
  \item an extension of the framework applying previous research results in automated annotation generation for
 Java bytecode (see~\cite{PBBHL}). The client thus will have the possibility to verify a security policy by
 propagating properties in the loaded code and then by verifying that the code verify the propagated properties.

\end{itemize}

%\todo{je ne sais pas ou mettre la reference de Jver}


