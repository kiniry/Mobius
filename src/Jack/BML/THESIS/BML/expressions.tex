
\newtheorem{Expression}{Definition}[section]



\subsection{Expressions}\label{lang}

We are now going to look at the expressions that our bytecode language manipulates. 
Similarly to the Java Bytecode language, the values of our language are of primitive data 
type\footnote{ we support only the integer primitive data type, which is sufficient for the purposes of this document} or references
where  references can be either references to class instance objects or to array objects. 
The special reference \Mynull \ does not point to any object.  
$$ \begin{array}{ll}
             \Values & := i, i \in \  \Myint \mbox{\rm literal } \mid \RefValues \\
	     \RefValues & := Ref : \AllRefs \mid \Mynull 
  \end{array} $$


 We introduce several notations which will be used in the following.
 If we want to say explicitely that a reference points to a class instance object of type \clazz \ we denote this with $\reference{\clazz}$ and if
 a reference points to a location of an array object whose elements are of type \anyType \ we denote this
 with $\RefArr{\anyType \lbrack \ \rbrack}$.
 

% The field with unique name  $i$  in a class \clazz is denoted by the function $\field{i}{C}$ and its domain is the set of
% references of class \clazz and has the following signature:
% $$ \begin{array}{l}
%           \field{i}{\clazz}: \reff \rightarrow \Values \\
%	   \dom{\field{i}{\clazz}} = \{ \Ref{\clazz} \}
%   \end{array} $$

 Our language supports field access expressions $\fieldd(\expression)$, where \fieldd \ is of type \FieldSet (see the previous subsection \ref{clazz}).
 When assigning value $v$ to the object field \fieldd \ of the object pointed by the reference $r$,
 the  corresponding field is updated resulting in a new field function $\fieldd[ \oplus r \longrightarrow v]: \AllRefs \longrightarrow \Values $.
 Next, our language supports arrays
 and the access to the $\rm{i^{th}}$ element in the array \textrm{arr} is denoted with $ \arrayAccess{\rm{arr}}{\rm{i}}$.
 Similarly, we use update functions for array access expressions  $ \update{\arrayAccess{\rm{arr}}{\rm{i}}}{ (ref, ind)  } { val}$.

The list of registers is denoted with \locVarOnly. The $\rm{i^{th}}$ element of \textrm{reg} is denoted by 
$\locVar{i}$. Note that $\locVar{0}$ will stand for the current object, 
which corresponds to \texttt{this} in the Java language. 
 The language deals also with arithmetic expressions that evaluate to an integer value and have the form $\expression_1 \ \op \ \expression_2 $,
 where  $ \expression_1$ and $\expression_2$ are also arithmetic expressions and $op$ is an arithmetic operation symbol
 ranging in $+, - , div, rem , xor, xand $. 

As the Java virtual machine is stack based we will also 
need expressions that model respectively the operand stack and the stack counter.
Thus, the expressions $\counter$ and  \textrm{stack} stand respectively for the stack counter and the stack. 
With \stack{i}, we denote the expression at the \textrm{i}-th position 
on the stack is. 


The following definition gives the formal grammar for the expressions of the bytecode language.
\begin{Expression}[Formal Grammar of Expressions] \label{expr}
    
\end{Expression}

  $$ 
  \begin{array}{ll}  
    \op             & ::= + \mid - \mid div \mid rem \mid xor \mid xand  \\
		    &                                                \\
		    &                                                \\
    \expression     & ::= \\
                    &  \Values \\
	            &  \mid  \fieldd(\expression), \fieldd : \FieldSet \\
		    &  \mid  \update{\fieldd}{ \expression}{\expression}(\expression) \\
		    &  \mid  \arrayAccess{\expression} {\expression} \\	   
		    &  \mid \update{ \arrayAccessOnly}{ (\expression , \expression)}{ \expression} (\expression,\expression) \\	
		    &  \mid  \locVar{i} \\ 
		    &  \mid  \expression \ \op \ \expression   \\
		    &  \mid  \counter \\
		    &  \mid  \stack{ \expression} \\
		  %  &  \mid  \result 		 
 \end{array} 
 $$






\subsubsection{Substitution} \label{subst}
 Expression substitution is defined inductively in a standard way over the expression structure. Still, we allow also
 substitution over objects that are not from our language, i.e. we apply substitution over field objects which results 
 in an update version of the field. 
 This is done by establishing the substitution rule for field access as follows:
 $$  \substitution{\fieldd (o)}{R_1}{R_2} =\substitution{\fieldd}{R_1}{R_2} (  \substitution{o}{R_1}{R_2}) $$


 In the next, we define a substitution over field function objects:
$$  
\begin{array}{l}
\substitution{\fieldd}{R_1}{R_2} = \\
\left\{\begin{array}{ll}
\fieldd  & if \ R_1 \neq \fieldd\\ 
&\\ 

R_2  &  if \  R_1 = \fieldd \wedge \\
               &  R_2 = \fieldd[ \oplus \mbox{ \rm \texttt{r} } \longrightarrow \mbox{  \rm \texttt{v} } ]
\end{array}
\right. \\
\\
\\
\substitution {\update{\fieldd}{ \mbox{ \rm \texttt{r} }}{ \mbox{  \rm \texttt{v} } }} {R_1}{R_2} = \\
\left\{\begin{array}{ll}
\update{\fieldd}{ \substitution{\mbox{ \rm \texttt{r'} }}{ R_1 }{ R_2}}{ \substitution{\mbox{ \rm \texttt{v'} }}{ R_1 }{ R_2}  } 

 & if \ R_1 \neq \fieldd\\
& \\ 
\fieldd \begin{array}{l}
             \lbrack \oplus \substitution{ \mbox{ \rm \texttt{r'}}}{R_1}{R_2} \longrightarrow  \substitution{ \mbox{ \rm \texttt{v'}} }{R_1}{R_2} \rbrack \\
	     \lbrack \oplus \mbox{ \rm \texttt{r} } \longrightarrow \mbox{  \rm \texttt{v} } \rbrack
	     \end{array}
&  

                if \   R_1 = \fieldd \wedge \\
               &  R_2 = \fieldd[ \oplus \mbox{ \rm \texttt{r} } \longrightarrow \mbox{  \rm \texttt{v} } ]

\end{array}
\right. 
\end{array}
$$ 


For example, consider the following substitution  
$$ \substitution{\fieldd(o)}{\fieldd}{\update{\fieldd}{e }{v }} $$
This results in the new expression : 
$$\update{\fieldd}{ e}{ v} (\substitution{o}{ \fieldd}{ \update{ \fieldd}{ e}{ v }}) $$


The same kind of substitution is allowed for array access expressions, where the array object \arrayAccessOnly \  can be updated. 
$$  
\begin{array}{l}
\substitution{\arrayAccessOnly}{R_1}{R_2} = \\
\left\{\begin{array}{ll}
\arrayAccessOnly  & if \ R_1 \neq \arrayAccessOnly \\ 
&\\ 

R_2  &  if \  R_1 = \arrayAccessOnly \wedge \\
               &  R_2 = \arrayAccessOnly  [ \oplus \mbox{ \rm \texttt{r} , ind } \longrightarrow \mbox{  \rm \texttt{v} } ]
\end{array}
\right. \\
\\
\\
\substitution {\update{\arrayAccessOnly}{ \mbox{ \rm \texttt{r} }}{ \mbox{  \rm \texttt{v} } }} {R_1}{R_2} = \\
\left\{\begin{array}{ll}
\update{\arrayAccessOnly}{ \substitution{\mbox{ \rm \texttt{r'} }}{ R_1 }{ R_2}}{ \substitution{\mbox{ \rm \texttt{v'} }}{ R_1 }{ R_2}  } 

 & if \ R_1 \neq \arrayAccessOnly \\
& \\ 
\arrayAccessOnly  \begin{array}{l}
                        \lbrack \oplus \substitution{ \mbox{ \rm \texttt{r'}}}{R_1}{R_2} \longrightarrow  \substitution{ \mbox{ \rm \texttt{v'}} }{R_1}{R_2} \rbrack \\
	                \lbrack \oplus \mbox{ \rm \texttt{r} } \longrightarrow \mbox{  \rm \texttt{v} } \rbrack
	     \end{array}
&  

                if \   R_1 = \arrayAccessOnly \wedge \\
               &  R_2 = \arrayAccessOnly[ \oplus \mbox{ \rm \texttt{r} } \longrightarrow \mbox{  \rm \texttt{v} } ]

\end{array}
\right. 
\end{array}
$$ 
