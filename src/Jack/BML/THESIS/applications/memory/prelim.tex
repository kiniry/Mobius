\subsection{Java class files} \label{classFileFormat}
The standard format for Java bytecode programs is the so-called class
file format which is specified in the Java Virtual Machine
Specification~\cite{JVMspec}. For the purpose of this paper, it is
sufficient to know that class files contain the definition of a single
class or interface, and are structured into a hierarchy of different attributes that
contain information such as the class name, the name of its superclass
or the interfaces it implements, a table of the methods declared in the class. 
Moreover an attribute may contain other attributes. For example the attribute that 
describes a single method contains an \verb!Local_Variable_Table! attribute that describes 
the method parameters and its local variables; further in this section we will denote the 
table of local variables by $l$ and the $i^{th}$ variable by $l[i]$. In addition to these attributes which
provide all the information required by a standard implementation of
the Java Virtual Machine, class files can accommodate user-defined
attributes, which are not used by standard implementations of the Java
Virtual Machine but can be used for other purposes. We take advantage
of this possibility and introduce additional attributes that contain
annotations such as method preconditions and postconditions, variants
and invariants. Annotations are given in the Bytecode Modeling
Language, which we describe below.

\subsection{The  Bytecode Modeling Language}
The bytecode modeling language \annotation\ is a variant of the Java
Modeling Language (JML) \cite{JMLRefMan} tailored to Java bytecode;
the BML specification language is described in~\cite{LM05:acc}. For
our purposes, we only need to consider a restricted fragment of BML,
which is given in Fig.~\ref{fig:bml}; we let $\expression$ and
$\predicate$ denote respectively the set of BML expressions and BML
predicates. As for JML, BML specifications contain different forms of
statements, in the form of BML predicates tagged with appropriate
keywords. BML predicates are built from BML expressions using standard
predicate logic; furthermore BML expressions are bytecode programs
that correspond to effect-free Java expressions, or BML specific
expressions.  The latter include expressions of the form
\verb!\old(exp)! which refers to the value of the expression
\verb!exp! at the beginning of the method, or
$\mbox{\tt exp}^{\mbox{{\tt pc}}}$ which refers to the
value of the expression \verb!expr! at program point
\verb!pc!. Note that the latter is not a standard expression in JML
but can be emulated introducing a ghost variable 
$\mbox{\tt exp}^{\mbox{{\tt pc}}}$ and performing the ghost 
assignment \verb!set exp!$\mbox{}^{\mbox{{\tt pc}}}$\verb!= exp! at
program point \verb!pc!.

Statements can be used for the following purposes:
\begin{itemize}
\item Specifying method preconditions, which following the design by
contract principles, must be satisfied upon method invocation. Such
preconditions are formulated using statements of the form $\requires \
\predicate$;


\item Specifying method postconditions, which following the design by
contract principles, must be guaranteed upon returning normally from the
method. Such postconditions are formulated using statements of the form
$\ensures\ \predicate$;

\item Specifying method exceptional postconditions, which must be
guaranteed upon returning exceptionally from the method. Such
postconditions are formulated using statements of the form \\
$\exsures{Exception} $ $ \predicate$, that record the reason for
exceptional termination;

\item Stating loop invariants, which are predicates that must hold
every time the program enters the loop;

\item Guaranteeing termination of loops and recursive methods, using
statements of the form $\variant\ \expression$ which provide a measure (in
the case of BML a positive number) that strictly decreases at each
iteration of the loop/recursive call;


\item Local assertions, using $\assert \ \predicate$, which asserts
that $\predicate$ holds at the program point immediately after the
assertion;

\item Declaring and updating ghost variables, using statements of the
form $\declare \ \ghost \ Type \ name$ and $ \set \ \expression = \expression$;

\item Keeping track of variables that are modified by a method or in a
loop, using declarations of the form $\modifies \ var$. During the
generation of verification conditions, one checks that variables that
are not declared as modifiable by the clause above will not be
modified during the execution of the method/loop, and one also uses
the information about modified variables to generate the verification
conditions.
\end{itemize}

\begin{figure}
%\begin{frameit}
$$
\begin{array}{lll} 
\mbox{\annotation}-{\sf stmt} & = &
                                       \requires \ \predicate \\
                              & \mid & \ensures  \ \predicate  \\
                           & \mid  & \exsures{Exception} \ \predicate  \\
                               & \mid  &  \assert  \ \predicate  \\
                                & \mid & \invariant \  \predicate  \\
                                & \mid & \variant \  \expression  \\
                                & \mid &  \declare \ \ghost \ Type \ name \\
                                 & \mid & \modifies  \ var  \\
                                 & \mid & \set \ \expression = \expression

\end{array}
$$
\caption{{\sc Specification language}}\label{fig:bml}
%\end{frameit}
\end{figure}

Note that, as alluded in the previous paragraph, annotations are not
inserted directly into bytecode; instead they are gathered into
appropriate user defined attributes of an extended class file. Such
extended class files can be obtained either through direct
manipulation of standard class files, or using an extended compiler
that outputs extended class files from JML annotated programs,
see~\cite{LM05:acc}.

\subsection{Verification of annotated bytecode}
In order to validate annotated Java bytecode programs, we resort to a
verification environment for Java bytecode (described in~\cite{LM05:acc}), which is an adaptation of JACK~\cite{BRL03:fme}. It consists of two main components:

%The environment, which is an adaptation of JACK~\cite{BRL03:fme}, and was developed by L.~Burdy and the second author, see~\cite{LM05:acc}, consists of two main components:

\begin{itemize}
\item A verification condition generator, which takes as input an annotated
applet and generates a set of verification conditions which are sufficient
to guarantee that the applet meets its specification;

\item A proof engine that attempts to discharge the verification
conditions automatically using automatic tools such as B and Simplify,
and then sends the remaining verification conditions to proof assistants
where they can be discharged interactively by the user.  We are
currently generating verification conditions for the proof assistants
Coq~\cite{coq} and PVS~\cite{pvs}.
\end{itemize}


\subsubsection{Generating the Verification Conditions}\label{subsec:verification}
The verification condition generator, or VCGen for short, takes as
input an extended class file and returns as outputs a set of proof
obligations, whose validity guarantees that the program satisfies its
annotations. The VCGen proceeds in a modular fashion in the sense that
it addresses each method separately, and is based on computing weakest
preconditions. More precisely, for every method
$\method$, postcondition $\psi$ that must hold after normal
termination of $\method$, and exceptional postcondition $\psi'$ that
must hold after exceptional termination of $\method$ (for simplicity
we consider only one exception in our informal discussion), the VCGen computes a predicate $\phi$ whose validity at the onset of method
execution guarantees that $\psi$ will hold upon normal termination,
and $\psi'$ will hold upon exceptional termination. The VCGen will
then return several proof obligations that correspond, among other
things, to the fact that the precondition of $\method$ given by the
specification entails the predicate $\phi$ that has been computed, and
to the fact that variants and invariants are correct.


The procedure for computing weakest preconditions is described in
detail in~\cite{LM05:acc}. In a nutshell, one first defines for each
bytecode a predicate transformer that takes as input the
postconditions of the bytecode, i.e. the predicates to be satisfied
upon execution of the bytecode (different predicates can be provided
in case the bytecode is a branching instruction), and returns a
predicate whose validity prior to the execution of bytecode guarantees
the postconditions of the bytecode. The definition of such functions is
completely generic and independent of any program, so the next step is
to use these functions to compute weakest preconditions for programs.
This is done by building the control flow graph of the program, and then
by computing the weakest preconditions of the program using its control
flow graph.

Note that the verification condition generator operates on BML
statements which are built from extended BML expressions. Indeed,
predicate transformers for instructions need to refer to the operand
stack and must therefore consider expressions of the form
\verb!st(top -+ i)! which represent the \verb!st(top -+ i)!-th 
element from the stack top:
$$wp( \store \ l(i) , \psi , \psi') = \psi[\verb!top! \leftarrow \verb!top-1!][l[i] \leftarrow \verb!st(top)!].$$


\subsubsection{Discharging verification conditions}
Verification conditions are expressed in an intermediate language
for which translations to automatic theorem provers and proof
assistants exist.


\subsection{Correctness of the method}\label{subsec:sound}
The verification method is correct in the sense that one can prove
that for all methods $\method$ of the program the postcondition (resp.
exceptional postcondition) of the method holds upon termination (resp.
exceptional termination) of the method provided the method is called
in a state satisfying the method precondition and provided all
verification conditions can be shown to be valid.


The correctness of the verification method is established relative to
an operational semantics that describes the transitions to be taken by
the virtual machine depending upon the state in which the machine is
executed. There are many formalizations of the operational semantics
of the Java Virtual Machine, see
%e.g.~\cite{BSS:jbook,FM03:jar,KN02:tcs,Siveroni:03:JavaCardSemantics}. Such
e.g.~\cite{BSS:jbook,FM03:jar,KN02:tcs,SH01pjo}. Such
semantics manipulate states of the form
$\configMem{h,\fram{m,\pc,l,s},\stf}$,
 where $h$ is the heap of objects,
$\fram{m,\pc,l,s}$ is the current \emph{frame} and $\stf$ is the
current call stack (a list of frames). A frame $\fram{m,\pc,l,s}$
contains a method name $m$ and a program point $\pc$ within $m$, a set
of local variables $l$, and a local operand stack~$s$.
%The rule for the generic instruction \instr\ is formalized as a
The operational semantics for each instruction is formalized as rules specifying
 transition between states, or between a state and some tag that
indicates abnormal termination. For example, the semantics of
 the instruction $\store$ is given by the transition
rule below, where  $\instrAt (m,\pc)$ is the function that extracts
the $\pc$-th instruction from the body of method $\method$.

$$\frac{\begin{array}[c]{c}
\ \instrAt (m,\pc)=\store \ i
\end{array}}%
{\begin{array}[t]{c}
\configMem{h,\fram{m,\pc,l,v::s},\stf} \to_{\store\ i}
\configMem{h,\fram{m,\pc+1,l[i \mapsto v],s},\stf}.
\end{array}}$$
In order to establish the correctness of our method, one first needs
to establish the correctness of the predicate transformer for each
bytecode. For example for the instruction $\store$ we show that:

$$\begin{array}[t]{c} 
wp(\store \ i , \psi )( \configMem{h,\fram{m,\pc,l,v::s},\stf} ) \\ 
\Rightarrow  
\\
\psi( \configMem{h,\fram{m,\pc+1,l[i \mapsto v],s},\stf})
\end{array}$$
In the above $\psi(\configMem{h,\fram{m,\pc,l,v::s},\stf} )$ is to be
understood as the instance of the formula $\psi$ in which all local
variables $l$ and field references are substituted with their
corresponding values in state $\configMem{h,\fram{m,\pc,l,v::s},\stf} $.


The proof proceeds by a case analysis on the instruction to be
executed, and makes an intensive use of auxiliary substitution
lemmas that relate e.g. the stack of the pre-state with the stack
of the post-state of executing an instruction. Then one proves the
correctness of the method by induction on the length of the
execution sequence. We have proved the correctness of our method
for a fragment of the JVM that includes the following constructs:
\begin{itemize}
\item Stack manipulation: \push, \pop, \dup, \dup 2, \swap etc;
\item Arithmetic instructions: \arithOp;
\item Local variables manipulation: type\_\load, type\_\store, etc;
\item Jump instructions: \ifCond, \goto, etc;
\item Object creation and  object manipulation: \new, \putfield, \getfield, \newarray, etc;
\item Array instructions: \arrstore, \arrload, etc;
\item Method calls and return: \invoke, \return, etc;
\item subroutines: \jsr\ and \ret.
\end{itemize}


Note however that our method imposes some mild restrictions on the
structure of programs: for example, we require that $\jsr$ and $\throw$ instructions
are not entry for loops in the control flow graph in order to prevent pathological recursion.
Lifting such restrictions is left for future work.




%\subsubsection{Building the Control Flow Graph}\label{cntrlFlow}
%\section{Defintitions}
We introduce definitions related to the control flow graph of a
method.

We assume that a bytecode is an array of instructions from our
language. The $k-th$ instruction from a given bytecode is denoted
with $\instrAt{k}$.

We denote by $\Gamma  = ( \Omega, \ll)$ the execution graph of a
bytecode $\Pi$ where the set of nodes $\Omega$ is the set of
blocks. The set of edges $\ll$ defines the execution relation
between blocks.Using standard terminology \cite{ARU1986com}, a
basic block is a code segment that has no unconditional jump or
conditional branch statements except for possibly the last
statement, and none of its statements, except possibly the first,
is a target of any jump or branch statement.

We denote a block that starts at instruction $\instrAt{j}$
respectively by \blockm{j}. Let's have the bytecode $\Pi$ and the
set of its blocks  be $\bbb$. The edges ( $\ll \ \subseteq \bbb \ \times \ \bbb $ ) define the order of block execution.


\begin{defn}[Entry block for a method]\label{entryBlock}
Let's have the control flow graph $\Gamma  = ( \Omega, \ll)$ for a bytecode $\Pi$. We assume that every execution path of $\Gamma$ starts at the block starting at
 $\instrAt{0}$. We denote this block with \blockm{entry}. We say
 also that $\instrAt{0}$ is the entry instruction of the program.
\end{defn}

\begin{defn}[Execution relation between blocks]\label{execRel}
\begin{tabbing}
\\Let \=  have \= the blocks \blockm{j} and   \blockm{k}. They
are\\
in execution relation  and we note it by \blockm{j} $\ll$ \blockm{k}\\
(this means that \blockm{k} can immediately follow\\
\> \blockm{j} in some execution sequence  ) if : \\
\>  if  \blockm{j} ends with a conditional or unconditional jump to the first instruction of \blockm{k}\\
\>  if  \blockm{j}  immediately follows \blockm{k} in the order of the program and does not end in an\\
\> unconditional jump \\
\end{tabbing}
We say that \blockm{j}  is a \textit{predecessor}of \blockm{k} and
that \blockm{k} is a \textit{successor} of \blockm{j}.
\end{defn}

\begin{defn}[Execution relation between instructions]\label{execRelInstr}
We also  define execution relation between instructions. If
\blockm{j} $\ll$ \blockm{k} (see def.~\ref{execRel}) and if
\blockm{j} terminates with instruction $\instrAt{m}$ then we also
say that $\instrAt{m}$ is a \textit{predecessor} of $\instrAt{k}$.
If two instructions $\instrAt{k}$ and $\instrAt{k+1}$ are in the
same basic block then $\instrAt{k}$ is a \textit{predecessor} of
$\instrAt{k+1}$. The predecessors of instructions $\instrAt{k}$
are denoted by $preds(\instrAt{k}) $. In the same way is defined
\textit{successor} for instructions.
\end{defn}

%We say that there exists a path between \blockm{i} and \blockm{j}
%and we note it with  \pathm{i}{j}, if there exists blocks
%\blockm{s_{1}}... \blockm{s_{n}} such that \blockm{i}$\ll$
%\blockm{s_{1}}$\ll$ \blockm{s_{2}} $\ll$ ...$\ll$
%\blockm{s_{n}}$\ll$ \blockm{j}

\begin{defn}[Loop Definition]\label{defLoop}
Let's have a well formed bytecode  $\Pi$ representing the body of
method \method. We say that  \loopSet{l} is the set of blocks of a
loop $l$ , \blockm{s} $ \in \loopSet{l}$ is the entry block of
loop $l$ in $\Pi$ and \loopEndsSet{l} $ \subseteq \loopSet{l}$ is
the set of end blocks of $l$ if:
\begin{itemize}
\item if $\forall  $ \blockm{e} $ \in  \loopSet{l}$ every path
from the entry block \blockm{entry} of the method \method to
\blockm{e} goes through \blockm{s}

%\item $\forall  $ \blockm{e} $ \in  \loopEndsSet{l}$ there exists
%a path in the execution graph starting at the entry block
%\blockm{entry} of the method \method (i.e. the one that starts
%with the entry instruction) of $\Pi$ and that passes first through
%\blockm{s} and later through \blockm{e}, i.e.\blockm{entry}
%$\ll^{*}$ \blockm{s} $\ll^{*}$ \blockm{e}

\item $\forall  $ \blockm{e} $ \in  \loopEndsSet{l}$  \blockm{e}
$\ll$ \blockm{s}
\end{itemize}
The instruction $\instrAt{s}$ at which the entry loop block
\blockm{s} starts is denoted with \loopEntry{l}

\end{defn}




%Note that every loop in the control graph has at least one backedge: an edge in the control flow graph that goes from some $\instrAt{j}$ to $\instrAt{k}$ such that $k \le j$ (% that makes possible that there is a path from the start instruction of the loop to itself. We denote a backedge as $\backedge{\instrAt{j}}{\instrAt{k}}$). For reasons of being simpler we assume here that there is only one backedge per loop (actually it is quite normal to have more than one backedge : for example if in the loop there is a \texttt{continue} in a loop but this will make the explanations more complicated).
% Note also that the instructions that are between the end of a loop and the start of a loop have their number in the bytecode array smaller than the start of the corresponding backedge and bigger than the end of the backedge.

\begin{defn}[Nested Loops]
\label{nestedLoops}
We say that a loop \progLoop{l2} is nested into  \progLoop{l1} that in the body of a method \method \   iff :
\begin{itemize}
\item every path starting at the entry block \blockm{entry} of the method \method \  and that reaches the entry block of \progLoop{l2} passes through the entry block of \progLoop{l1}
\item no path that starts at the entry block of \progLoop{l2} and that reaches an end instruction of \progLoop{l2} passes through the end of \progLoop{l1}
\end{itemize}
\end{defn}

%\begin{defn}[Loop Ordering]
%\label{loopOrder}
%For every loop \progLoop{l} in the control flow graph a number is
%associated to its entry block $\numLoop{\progLoop{l}} $. Thus
 %               \begin{enumerate}
  %                      \item if a loop \progLoop{l2} is nested in \progLoop{l1}, \numLoop{\progLoop{l1}} $ <$ \numLoop{\progLoop{l2}}
   %                     \item if a loop \progLoop{l2}  is not nested in \progLoop{l1} but their entry blocks are  \blockm{i} and \blockm{j} respectively such that
    %                            $ i < j $ then  \numLoop{\progLoop{l1}} $ <$ \numLoop{\progLoop{l2}}
     %           \end{enumerate}
%\end{defn}


%\subsubsection{Generating the Verification Conditions}
%%Verifying programs being correct
\label{verification}
Once an appropriate annotation is attached to the code, a
static verification procedure is applied to the bytecode and
the annotation. This static analysis is based on a predicate transformer
function. The predicate transformer is defined over almost the whole
set of sequential Java instructions except for 64 bit data and floating
point arithmetic, including instructions treating references,
exceptions and subroutines. The calculus supports a superset of the
specification language BML that is considered in the present work. The predicate
transformer takes a predicate $\psi$ that must hold in the poststate of
the program and translates the bytecode and its specification into
predicate$\phi$. The relation between the two predicates $\psi$ and
$\phi$ must be the following: \textit{if $\phi$ holds in the prestate
of the program then the predicate $\psi$ must hold in the program
poststate. }

For example the rule for the instructions \srcCode{putfield} is given
at fig.~\ref{wpInstr}. The rule considers also the cases for
exceptional termination of the instruction execution (e.g. the rule
for the \srcCode{putField} instruction has two conjuncts : the one
assumes that the reference is not null thus this is the normal
execution case and the other assumes that the reference whose field is
accessed is null). Aliasing is treated by an overriding function.

\begin{figure}[!hbp] 
$
\begin{array}{ll}
\wpi{putfield \ \it{ind} }{\normalPost}{\excPost} & =   \left\{ \begin{array}{l}
                               \stack{\topStack - 1} != \javaNull \Rightarrow \\
                                \phantom{aaaaa} \normalPost \substitution{\topStack}{\topStack - 2} \substitution{cp(ind)( \stack{\topStack - 1} )}{\stack{\topStack}} \\
                \wedge \\
        \stack{\topStack - 1} == \javaNull \Rightarrow \\
             \phantom{aaaaa}\excPost(NullPointerExc) \begin{array}{l} \substitution{\topStack}{0}\\
                                     \substitution{\stack{\topStack}}{\Ref{NullPointerExc}  }
                                    \end{array}
                    \end{array} \right.\\

\end{array}
$
$\stack$ stands for the method stack \\
$\topStack$ is the stack counter \\
\caption{wp rule for the instruction \srcCode{putfield} }
\label{wpInstr}
\end{figure}

As bytecode programs lack structure we define wp rules also for basic
blocks. Thus the rule for treating the conditional jump is given at
fig.~\ref{wpBlock}.

\begin{figure}[!hbp]
 $$ \begin{array}{ll}
                 \wpExe{\If \ k} & =  \left\{ \begin{array}{l}
                                                cond( \stack{\topStack}, \stack{\topStack - 1}) \Rightarrow  \psi_{1}\substitution{\topStack }{\topStack - 2}   \\                                                              \wedge \\
                                        not (cond( \stack{\topStack}, \stack{\topStack - 1})) \Rightarrow \psi_{2}\substitution{\topStack}{\topStack - 2 }  
                                        \end{array} 
                                   \right. 
         \end{array}
     $$


$\stack$ stands for the method stack \\
$\topStack$ is the stack counter \\
$\psi_{1}$ is the weakest precondition for the block \blockm{k} \\
$\psi_{2}$ is the weakest precondition for the block \blockm{\nextIns{\srcCode{\If \  k}}}
\caption{Weakest precondition rule for conditional jump.}
 \label{wpBlock}
\end{figure}




%\subsubsection{Correctness of the method}
%
In the verification mechanism there are two points that are not trusted
- the first one is the weakest precondition calculus -what guarantees
that the truthfulness of the formulas generated implies the program
correctness w.r.t. to its specification and the second one is the
specification - what guarantees that it expresses the property that we
have in mind.


We discuss now the first point. We have a proof of correctness of the
wp calculus which is in a draft version. The proof establishes that the
predicate transformer has the property stated in ~\ref{verification}. 
The proof is done over the operational semantics of our language. The operational semantics in terms of program states; a program state is a vector of the values of
the program variables and the program counter.  
Assume that we have a  bytecode \srcCode{bc}  and its operational semantics is
\srcCode{bc}:  $\tau_0 \rightarrow \tau_1 $ and we have a predicate $\psi$ that must hold in the state after its execution.  
What the proof of correctness establishes is that if \textit{wp(\srcCode{bc}, $\psi$) } holds in state$\tau_0$ then $\psi$
holds in state $\tau_1$. The proof is done inductively by cases for all the instructions for which
we have a rule, and the case for \srcCode{return} instruction gives us
that when the execution terminates the postcondition holds.


The other question is our methodology ( the specification that
we prescribe in section modeling constraint memory use policy )
for specifying the policy in question. Actually as we see in the chapter
for modeling the policy in question we use a special variable called
\Mem. We can instrument the operational semantics with the variable
\Mem / $\{ \tau_0, Heap, \Mem\}$ and what we can prove is the
following property : $\{ \tau_0, Heap, \Mem \} \rightarrow \{ \tau_1,
Heap' ,\Mem' \} , \ then \ Heap - Heap' \le \Mem - \Mem' $.This is due
to the fact that whenever there is an allocation in the heap - in our
language this is the instruction \srcCode{\new},i.e. the heap grows
the variable \Mem also grows.

