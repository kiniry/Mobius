
In the end of this thesis, we would like to summerize 
its contents and to focus on the future objectives. 

% done. summary
So far, we have seen how to perform verification over bytecode
programs without compromising the compiler. In particular, the verification 
scheme proposed allows to check a bytecode program against 
non trivial functional and security properties. For this, we have 
introduced a specification language BML tailored to Java bytecode, a compiler
from the Java source specification language JML to BML and a verification 
condition generator for Java bytecode programs. Moreover, we have
shown that the verification procedure over Java like source program 
and Java like bytecode are syntactically equivalent (modulo names and types). 
This scheme is actually part of the PCC architecture of 
European project Mobius\footnote{the site name} which aims to resolve the problems
of mobile and ubicuous computing via PCC.

% towards.PCC
For building a PCC framework from the components cited above 
there is still missing the proof certificate, the decision procedure
that will be used by the producer for the certificate generation and the type checker used by the code
client for checking the certificate. Important problems in this direction are
\begin{itemize}
  \item light weight verification condition generators. In particular, we refer 
        to verification condition generation techniques which are simple and do not need
	much computational resources. Because a verification condition generator always
	form part of the trusted computing base on the client side, building such verification 
	condition generators is important for on - device checking which rely on limitted computational 
	resources  
  
  \item generation of compact certificates. This is important for several reasons.
        The certificate may certainly  arrive via the network and should not corrupt the performance 
 

 \item efficient type checker on the client site. This is in particular important 
        if the device is with limitted resources where a complex certificate checking procedure
        may corrupt the performance of the device
       
    
\end{itemize}


 %To do this,  it is still missing the proof
%certificate, the decision procedure used by the code producer 
%for building the certificate  as well as the type checker used by the code
%client for checking the certificate. 

% to do. type systems
Another perspective in this direction is how   to encode
 type systems into the bytecode logic. 
Type systems provide a high level of automation. 
Their encoding in the verification condition generator can be useful
especially on where the certificate checking is performed on a 
 device which is potentially with limited resources.

 So far, we have shown that the verification procedures over bytecode produced by
a non optimizing compiler  and its respective source produce syntactically 
equivalent verification conditions. The next step is to explore 
the relation between  the verification conditions on bytecode produced by an optimizing compiler and its source.  

