%intro
This chapter aims to acquaint the reader with  the basic principles in the formal verification  
of Java programs. As we already stated in the introduction, a formal program verification
relies on three elements: a specification language in which to express the requirements 
that a program must satisfy, a verification procedure for generating verification conditions
whose validity implies that the program respects its specification and finally,
 a decision procedure which for deciding the validity of the verification conditions.
In this chapter, we shall focus on the first two components. 
In particular, we shall present JML,
the de facto Java specification language  tailored to Java. This will be done in Section \ref{BCSLprelim}.
In Section \ref{source}, we shall present a Java like programming language supporting the most important features of Java
 and in Section \ref{pog:wpSrc}, we shall define over it a  verification condition calculus.

