
\subsection{Exceptional Functions}
We now look at how the exceptional postconditions for expressions(statements) are managed. 
Remind that an expression(statement) may terminate normally or by throwing an exception. 
In both cases the expression(statement) has to satisfy some condition : 
the normal postcondition in case of normal termination and the exceptional postcondition
for exception $\texttt{Exc}$ if it terminates on exception $\texttt{Exc}$


For every method \method~ there is a function $ \excPostSpecSrc_{\method}$ defined as follows :

$$ \excPostSpecSrc_{\tt{m}} : \texttt{ Exception  } \longrightarrow Predicate $$ 

$\excPostSpecSrc_{\tt{m}}$ \ returns the predicate $\excPostSpecSrc_{\tt{m}} (\texttt{Exc}) $ that must hold if the method \method \ 
ends by throwing an exception $\texttt{Exc}$. 
The function $\excPost$ that determines the exceptional postcondition of a statement or expression $exp$
 upon throwing an exception \texttt{Exc} is defined as follows:

$$
\excPost(\texttt{Exc}) = 
       \left\{\begin{array}{ll} 
         \wpiSrc{handler}{\normalPostSrc}{\excPost} &  the \ first \ handler \ that  \ protects \\
                                                    &  exp \ from \ \texttt{Exc} \\
                                                    & \\
         \excPostSpecSrc_{\tt{m}}(\texttt{Exc} )    & else 
     \end{array}\right.$$

We also use updates of the function $\excPost$ which are defined in the usual way


$$
\excPost \oplus[\texttt{Exc'} \longrightarrow P](\texttt{Exc})  = 
       \left\{\begin{array}{ll} 
         P & if \texttt{Exc} <: \texttt{Exc'} \\
         \excPost(\texttt{Exc} ) & else 
     \end{array}\right.$$
