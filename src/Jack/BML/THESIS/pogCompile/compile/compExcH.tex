
\subsection{Exception handler table}\label{pogEq:compile:excHandlers}

Our source language contains exception handler constructions, and thus, when compiling a method body
the compiler should keep track of the exception handlers by adding information 
in the exception handler table array \excHandlerTable \ (presented in Section \ref{clazz})   every time it sees one.
 To do this, the compiler is provided with a procedure with the following name and signature:

$$ \addExceptionTableOnly : nat \rightarrow nat \rightarrow nat \rightarrow  \ClassSet_{exc}  \rightarrow    \MethodSet  $$
	
The definition of the procedure is the following:

$$ \begin{array}{l}
  \addExceptionTable{ start }{ end  }{ h}{  \mbox{\rm\texttt{Exc}} } ) = \\
   \begin{array}{l}
         \methodd.\excHandlerTable := \{ (start, end, h,  \mbox{\rm\texttt{Exc}}   ) , \methodd.\excHandlerTable \}

   \end{array}
\end{array}$$
The function adds a  new element in the exception handler table
of a method \methodd. The meaning of the new element is that 
every exception of type \mbox{\rm\texttt{Exc}}  thrown in between the instructions $start \ldots end$ can be handled by 
the code starting at index $h$.

We can remark that when the function  \addExceptionTableOnly \  adds a new element in the exception handler table array
of a method \methodd, the new element is added at the beginning of the  exception handler table   \methodd.\excHandlerTable.

