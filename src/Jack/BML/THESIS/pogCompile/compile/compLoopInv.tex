\subsection{Compiling loop invariants}\label{pogEq:compile:loopInv}
When compiling a method \methodd,
the compiler will also take care of the loop specification in the source loops
by adding it in the loop specification table \methodd.\loopSpecTable (defined in Section \ref{methExtend}) of the method \methodd.

This is done by the procedure  \addLoopTableOnly \ which has the following signature
$$ \addLoopTableOnly : \MethodSet  *  nat *  \formulaBc  *  list \ \expression $$
The procedure takes a method \methodd, an index in the array of bytecode instructions $i$ of \methodd, 
a predicate $\invariant$, a list of modified expressions \modLoop \ 
and adds a new element in the table of loop invariants 
\methodd.\loopSpecTable \ of method \methodd.
 Or more formally :
 


$$ \begin{array}{l}
  \addLoopTable{\methodd}{ i }{ \invariant  }{ \modLoop} =  \\
   
   \begin{array}{l}
         
         \methodd.\loopSpecTable := \{ (i,\invariant , \modLoop  ) , \methodd.\loopSpecTable \}

   \end{array}
\end{array}$$

