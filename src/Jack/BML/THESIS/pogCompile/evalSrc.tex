
\subsection{Evaluating expressions }

An expression in our source language can be evaluated either to an object reference, to an integer value  or to a boolean value. We denote the set of object references with \Refs. We denote the set of values with \Values
$$ 
\Values := \Refs \mid \Myint \mid \Mybool
$$

Thus, the evaluation function \textrm{eval} has the following signature:

$$\textrm{eval}: \expressionSrc \rightarrow Values $$

 
For the purposes of this study we do not need the definition of the function eval.

 % \[ \eval{\expressionSrc} =
 % \begin{array}{l}
 %  \constant  	& if \ \expressionSrc in \{  \constantInt , \Mytrue , \Myfalse
 % , \constantRef \} \\
 %    
 % \eval{\expressionSrc_1}         &  \expressionSrc = ( \expressionSrc_1 = \expressionSrc_2) \\
 % 				&  m(\expressionSrc_1 \ldots \expressionSrc_k ) \\
 % 				&  \expressionSrc.\expressionSrc \\
 % 				&  id \\
 %   				
 %                                 &  \expressionSrc \ op \ \expressionSrc \\  
 % \eval{	\expressionSrc}		&  (Class) \ \expressionSrc  \\
 %                                 &  \expressionSrc \ instanceof \ Class \\
 %            		        & \mid  \expressionSrc = \Mynull  \\
 % 				& \mid \this \\
 %               	                & \mid  \newSrc \ Class  ( \expressionSrc_1 \ldots \expressionSrc_k   ) \\
 %  %                                 & \mid \expressionSrc \ \rel \ \expressionSrc  
 % 
 % \end{array} 
 % \]



