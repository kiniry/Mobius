\subsubsection{Expressions}\label{pog:wpSrc:wpExpr}
In the following, we  give a standard  definition of a weakest
predicate transformer function for source expressions. %The weakest precondition function for expressions has the following signature:
%$$ \wpNameSrcExpr : \expressionSrc \rightarrow \formulaSrc \rightarrow ( \mbox{ \rm \texttt{Exc}} \rightarrow  \formulaSrc ) \rightarrow  
%\MethodSet \rightarrow   (\expressionSpecSrc \cup \formulaSrc)  \rightarrow  \formulaSrc $$
For calculating the  $\wpNameSrcExpr$ \ predicate  of  an expression $\expressionSrc$ declared in method \methodd,
 the function $\wpNameSrcExpr$ \ takes as arguments  $\expressionSrc$, a postcondition $\normalPostSrc$, an exceptional postcondition 
function $\excPostSrc$  and 
  returns the    formula \\
$\wpSrcExpr{\expressionSrc}{\psi}{\excPostSrc}{ v }$ which is the \wpName \ precondition of expression $\expressionSrc$ 
if the its evaluation is represented by  specification expression $v$.  

In Fig. \ref{pog:wpSrc:wpExpr:wpSrcExpr} we can see the $\wpNameSrcExpr$ rules for  most of the expressions of the source language (except for method invokation 
and instance creation). As we may notice, for some of the expressions   
the definition of the weakest precondition function is trivial is the identity function
 as they have  no side effects. For instance, the rule for constant expressions does not change the state and thus,
 if a predicate $ \normalPostSrc$ holds
 after its execution this means that it held in the prestate of the expression. 
 However, this is not the case for expressions that might throw an exception or which may change values of program variables.
 The predicate returned by the weakest precondition predicate transformer for expressions which may throw an exception 
will basically acumulate the hypothesis under which the evaluation 
of the expression terminates normally and the  conditions under which it  terminates exceptionally.
\footnote{Note that here we do not consider arithmetic expressions that may throw
an arithmetic exception, i.e. we discard the division operations.
 However,   we do not lose any particular feature of the language by discarding this case 
while gaining clearer representation.} The rule for a cast expression $( \class ) \ \expressionSrc$ shows that the evaluation of the expression
does not change the program state in case the $ \expressionSrc$ is of subtype of of class  $\class$. In case the latter is not true 
the rule reflects the case that the a  \ClassCastExc{} exception is thrown. The rule for the field access expression takes into account
the two possible  outcomings of its evaluation. If the evaluation $v$ of the expression $\expressionSrc$ 
is different from $\Mynull$ then the evaluation terminates normally, otherwise the exceptional postcondition  
for \NullPointerExc{}  must hold.

 

\begin{figure}[ht!]
\begin{frameit}
$${\scriptsize 
        \begin{array}{l} 
           \wpSrcExpr{const}{\normalPostSrc }{ \excPostSrc }{v}  =  \normalPostSrc\subst{v}{const}  \\ 
	   const \in \{  \constantInt , \Mytrue , \Myfalse, \constantRef, \Mynull, \this,\var\} \\
	   \\\\
	   %\wpSrcExpr{\Mynull  }{\normalPostSrc }{ \excPostSrc }{\Mynull} =  \normalPostSrc \\ 
	    % \\
	   \wpSrcExpr{ \expressionSrc_1 \ op \ \expressionSrc_2 }{\normalPostSrc }{ \excPostSrc }{v}  =   \\
	    \begin{array}{l}
                \wpSrcExpr{ \expressionSrc_1  }{\wpSrcExpr{\expressionSrc_2 }{ \normalPostSrc \subst{v}{v_1 op v_2}  }{ \excPostSrc }{v_2}  }{ \excPostSrc }{v_1} 
	    \end{array}\\
	  where \  op = \{ +, - , *, \}  \\
	   \\\\
	   %\wpSrcExpr{\this  }{\normalPostSrc }{ \excPostSrc }{\this} = \normalPostSrc	      	    \\
	   %\\
           \wpSrcExpr{ \expressionSrc \ \instanceofSrc \  \class }{\normalPostSrc  }{ \excPostSrc }{ v}= \\
	                \wpSrcExpr{ \expressionSrc  }
                                  { \normalPostSrc\subst{v}{ v_1 \neq \Mynull \wedge \typeof{ v_1 } <:  \class }  }
				  { \excPostSrc }{v_1 } \\
           \\\\

           \wpSrcExpr{ \expressionSrc_1 \ \rel \ \expressionSrc_2 }{\normalPostSrc }{ \excPostSrc }{v} = \\
	       \begin{array}{l}
                       \wpSrcExpr{ \expressionSrc_1  }{  \wpSrcExpr{ \expressionSrc_2  }{ \normalPostSrc \subst{v}{v_1 \rel v_2 } }{ \excPostSrc }{v_2}  }{ \excPostSrc }{v_1}
               \end{array}\\ \\\\

		     
             \wpSrcExpr{\expressionSrc.\fieldd  }{\normalPostSrc }{ \excPostSrc }{v}  = \\
	                        \begin{array}{l} \wpSrcExpr{\expressionSrc }{\\
			                   \phantom{wpiSr} \begin{array}{l} 
						        v_1 \neq \Mynull \Rightarrow \normalPostSrc \subst{v}{v_1.\fieldd}\\
			                                \wedge \\
						        v_1 = \Mynull \Rightarrow\\
							 \Myspace 
							  \begin{array}{l} 
							      \forall \freshVar, 
							      \neg \instances(\freshVar) \wedge \\
							       \freshVar \neq \Mynull \Rightarrow\\ 
							       \Myspace\excPostSrc(\NullPointerExc )\subst{\EXC}{\freshVar}
							  \end{array}
							        
		                                   \end{array}}{\\ 
                                           \phantom{wpiSrc} \excPostSrc }{v_1}  
				\end{array}  \\
		 \\\\	
 

\wpSrcExpr{ ( \class ) \ \expressionSrc  }{\normalPostSrc }{ \excPostSrc }{v} = \\
	         \begin{array}{l}   \wpSrcExpr{ \expressionSrc } { \\ 
                      
                    \phantom{wpSr} \begin{array}{l}
                    \typeof{v_1 } <:\class  \vee v = \Mynull \Rightarrow   % \\
		      %\phantom{wpSr} \phantom{wpSr}  
		       \normalPostSrc \subst{v}{v_1}  \\
		     \wedge \\
		    \neg \   \typeof{v_1 } <:  \class \Rightarrow  \\ 
		    \Myspace 
							  \begin{array}{l} 
							      \forall \freshVar, 
							      \neg \instances(\freshVar) \wedge \\
							       \freshVar \neq \Mynull \Rightarrow\\ 
							       \Myspace \excPostSrc( \mbox{ \rm \ClassCastExc }  ) \subst{\EXC}{\freshVar}
							  \end{array}
                     
                    \end{array}  } { \\ \phantom{wpSrc}  \excPostSrc }{v_1} \end{array}
				
        

    \end{array} 
} $$

\caption{\sc WP for source expressions }
\label{pog:wpSrc:wpExpr:wpSrcExpr}
\end{frameit}
\end{figure}


In Fig.\ref{pog:wpSrc:wpExpr:wpSrcInvoke}, we give the rule of instance creation. It is given separately
as it is more complicated than the rest of the cases. 
 The rule for instance creation states that  for any reference which is
 fresh for the heap  ($\neg \ \instances(\freshVar ) $) ,  which is not \Mynull{} and whose type is a subtype of \class{}  
the precondition of the constructor  $\Constructor{\class}{}$  of class  \class{} must hold and that the 
normal postcondition $\Constructor{\class}.\normalPostSrc{}$ of the constructor 
must imply the postcondition  $\normalPostSrc$ with the respective substitutions. A similar condition we
 get for the cases when the constructor may terminate on an exception.
% Let us now look at the rule for method invokation expression $ \expressionSrc.\methodd()$ in Fig. \ref{pog:wpSrc:wpExpr:wpSrcInvoke}.
%The resulting precondition  takes into account the case when the  object $v$ to which $\expressionSrc $ evaluates and
% on which  the method is called is \Mynull{} or not. If it is \Mynull{} then the weakest precondition of the handler against \NullPointerExc{}
%must hold. In the opposite case, we want several predicates to hold. 
%First, the precondition $ \methodd.\preSrc $  of the invoked method \methodd{} must hold. 
%Then, the  normal postcondition $\methodd.\normalPostSrc $ 
%of \methodd{} must imply the postcondition  $\normalPostSrc{}$ whatever is the value of the returned object and whatever are the values of the 
%locations in the modifies list $\methodd. \mod{}$  of the method \methodd. Finally, if the method \methodd{} terminates on an exception \mbox{\rm\texttt{E}},
% then the respective postcondition    $\methodd.\excPostSpecSrc(\mbox{\rm \texttt{E}})$ must imply the predicate $\excPostSrc(\mbox{\rm\texttt{E}} )$  returned by the exceptional function 
% $\excPostSrc$ whatever are the values of the locations in the modifies list  $\methodd. \mod{} $  of the method~\methodd.


\begin{figure}[ht!]
\begin{frameit}
$${\scriptsize 
        \begin{array}{l} 
	     

\wpSrcExpr{ \newSrc \ \class  (   ) } {\normalPostSrc }{ \excPostSrc  }{v} = \\
 	  
 	             \begin{array}{l}  \forall \freshVar, \\
 				               \neg \ \instances(\freshVar ) \wedge \\
 					       \freshVar  \neq \Mynull \\
					       \typeof{\freshVar} <: \class  \Rightarrow \\ 
		     
 		                 \begin{array}{l}
                                   
 				    \Myspace  \Constructor{\class}.\preSrc
 				          \begin{array}{l} 
                                                \subst{ \this}{ \freshVar } \\
                                		\subst{ \fieldd} { \update{\fieldd} { \freshVar}{\defaultValue{ \fieldd.  \fieldType } } }_{{ \subtype{\fieldd.\declaredIn}{  \class}} } \\
  					  \end{array} \\
  				 \Myspace	       \wedge \\
				 \Myspace \left( \begin{array}{l} \forall \ m \in \Constructor{\class}. \mod , \\
				 \Myspace    \Constructor{\class}.\normalPostSrc       
  					       \begin{array}{l}   
  						     \subst{ \this}{ \freshVar  } \\
						     \subst{ \fieldd} { \update{\fieldd} { \freshVar}{\defaultValue{ \fieldd.  \fieldType } } }_{{ \subtype{\fieldd.\declaredIn}{  \class}} } 
                                                \end{array}  \\
                                \Myspace   \Myspace     \Rightarrow  
						   \normalPostSrc \begin{array}{l}
						    \subst{ v }{ \freshVar  } \\ 
						    \subst{ \fieldd} { \update{\fieldd} { \freshVar}{\defaultValue{ \fieldd.  \fieldType } } }_{{\subtype{\fieldd.\declaredIn}{  \class}} } \end{array} \end{array}\right)\\
  					           \Myspace  \wedge \\
  						   \Myspace  \forall \mbox{\rm \texttt{Exc}} \in  \Constructor{\class}.\exceptionSrc, \\
  						   \Myspace  \forall \ m \in \Constructor{\class}. \mod, \\
  						   \Myspace  \forall \freshVar_{exc},\\
						 \Myspace \left(\begin{array}{l}
						 \freshVar \neq \Mynull \wedge \\
						 \typeof{\freshVar_{exc}} <: \mbox{\rm\texttt{Exc}} \Rightarrow \\
						  \Constructor{\class}.\excPostSpecSrc ( \mbox{\rm\texttt{Exc}}) 
                                                           \Rightarrow\\ 
                                         \Myspace  \Myspace   \Myspace \Myspace \excPostSrc( \mbox{\rm\texttt{Exc}})
					 \end{array}\right) 
					 \begin{array}{l}
					   \subst{\EXC}{\freshVar_{exc}}\\ 
					   \subst{ v }{ \freshVar  } \\  
					   \subst{ \fieldd}{\update{\fieldd} { \freshVar}{\defaultValue{ \fieldd.  \fieldType } } }_{{ \subtype{\fieldd.\declaredIn}{  \class}} }
					   \end{array}
  		              \end{array}
			      \end{array}
  		     

        \end{array} } $$
\caption{\sc Weakest precondition for instance creation }
\label{pog:wpSrc:wpExpr:wpSrcInvoke}
\end{frameit}
\end{figure}





