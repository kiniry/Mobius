
\subsubsection{Statements}\label{pog:wpSrc:wpStmt}

In the following, we discuss  the weakest precondition predicate transformer for  control statements.
 We will not give an exhaustive overview of the \wpName{} rules for statements as they are standard 
but we will rather  concentrate on few which we consider illustrative. 

Fig. \ref{pog:wpSrc:wpStmt:withoutExc} gives  the rule for  field assignment and conditional control transfer
statements which do not deal with exceptions. They are defined in a standard way. 
The rule for field assignment shows that the statement may terminate normally or on an exception. 
In particular, if the  dereferenced object reference $\expressionSrc_1$ does not evaluate to \Mynull{} 
 then  the postcondition $\normalPostSrc{}$ must hold where the value of the field \fieldd{} for the 
evaluation $v_1$ of $\expressionSrc_1$ is changed to the value  $v_2$ of  $\expressionSrc_2$.
If  the  dereferenced object reference $\expressionSrc_1$  evaluates to \Mynull{} then the postcondition 
$ \excPostSrc(\NullPointerExc) $ must hold.
The rule for the conditional statement is also standard.
 
\begin{figure}[ht!]
\begin{frameit}
$${\scriptsize 
        \begin{array}{l} 
	  
    \wpSrcStmt{ \expressionSrc_1.\fieldd = \expressionSrc_2}{\normalPostSrc }{ \excPostSrc } = ^{\mbox{\rm\textsf{fieldAssign}}}\\
               \begin{array}{l} 
	       \wpSrcStmt{\expressionSrc_1}{  \\
	        \phantom{wpiSr} \wpSrcExpr{\expressionSrc_2 }{ \\
		\phantom{wpiSr}\phantom{wpiSr}
		\begin{array}{l}
		    v_1 \neq \Mynull  \Rightarrow \\
	         \phantom{wpiSr}  \normalPostSrc \subst{\fieldd}{\update{\fieldd}{ v_1}{  v_2}}   \\
		 \wedge \\
		 v_1  = \Mynull  \Rightarrow  \\
		   \phantom{wpiSr}   \excPostSrc(\NullPointerExc) 
		   \end{array}}{ \\ \phantom{wpiSr}\phantom{wpiSr}  \excPostSrc}{v_2} 
		   }{ \\ \phantom{wpiSr}  \excPostSrc}{v_1} 
		   \end{array} \\ \\ \\

    \wpSrcStmt{ \begin{array}{l} \Myif \  (\expressionSrcRel)\\  \Mythen  \{ \stmt_1 \} \\  \Myelse \ \{ \stmt_2 \} \end{array}}{\normalPostSrc}{ \excPostSrc } 
    =^{\mbox{\rm\textsf{if}}}\\

         \begin{array}{l} 
	 \wpSrcExpr{ \expressionSrcRel }{ \\
	              \phantom{wpiSr}  \begin{array}{l}  
		      v = \Mytrue  \Rightarrow \wpSrcStmt{\stmt_1 }{\normalPostSrc }{ \excPostSrc } \\
		      \wedge \\
		      v = \Myfalse    \Rightarrow \wpSrcStmt{\stmt_2 }{\normalPostSrc }{ \excPostSrc } \\
	 \end{array}
	 } {\\  \phantom{wpiSr}  \excPostSrc }{v} 
     \end{array} \\ \\ \\


\end{array} } $$

\caption{\sc WP for source control statements without exceptions }
\label{pog:wpSrc:wpStmt:withoutExc}
\end{frameit}
\end{figure}


The control statements related to the exception handling and throwing as well as the finally statements
 have a more particular definition. They are given in Fig. \ref{pog:wpSrc:wpStmt:withExc}. 
Let us look at the rule \textsf{try catch}. %Actually, it is similar to the  rule  \textsf{seq} from  Fig. \ref{pog:wpSrc:wpStmt:withoutExc}, but dual in the way the postcondition modifications.
 The weakest predicate of a try catch statement $ \try \ \{ \stmt_1 \} \ $ 
$\catch (  \mbox{\rm\texttt{Exc}} \ c )\ \{ \stmt_2 \} $  w.r.t. a normal postcondition   $\normalPostSrc$  and exceptional postcondition function $ \excPostSrc$
is the weakest predicate of the try statement $\stmt_1$ w.r.t. the normal postcondition  $\normalPostSrc$  and the updated exceptional function
$\update{\excPostSrc}{ \mbox{\rm\texttt{Exc}} }{\wpSrcStmt{\stmt_2}{ \normalPostSrc}{\excPostSrc} }$.


\begin{figure}[ht!]
\begin{frameit}
$${\scriptsize 
        \begin{array}{l} 
 \wpSrcStmt{ \throw \ \expressionSrc }{ \normalPostSrc}{\excPostSrc} =^{\mbox{\rm\textsf{throw}}}\\
	       \begin{array}{l} 
	      \wpSrcExpr{\expressionSrc}{\\ \phantom{wpiSr} 
                  \begin{array}{l}
		       
	                v = \Mynull \Rightarrow \excPostSrc( \NullPointerExc  )  \\
			\wedge \\
			 v \neq \Mynull \Rightarrow\\
			 \Myspace
			  \begin{array}{l}  
				 \forall \mbox{\rm\texttt{Exc} } , \\
				 \typeof{ v } <: \mbox{\rm\texttt{Exc} }   \Rightarrow \\
				 \Myspace   \methodd.\excPostSrc( \mbox{\rm\texttt{Exc} }    ) \subst{\EXC}{v}
			  \end{array} 

			  
	          \end{array}}   
		     { \\ \phantom{wpiSr}  \excPostSrc}{v} 
	   \end{array}     \\ \\ \\

 \wpSrcStmt{ \try \ \{ \stmt_1 \} \ \catch (  \mbox{\rm\texttt{Exc}} \ c )\ \{ \stmt_2 \} } { \normalPostSrc}{\excPostSrc} =^{\mbox{\rm\textsf{try catch}}}\\
              \begin{array}{l}
	      \wpSrcStmt{ \stmt_1}{ \\ 
	                 \phantom{wpiSr} \normalPostSrc}{ \\ 
			 \phantom{wpiSr} \update{\excPostSrc}{ \mbox{\rm\texttt{Exc}} }{\wpSrcStmt{\stmt_2}{ \normalPostSrc}{\excPostSrc} }} 
	      \end{array} \\ \\ \\

\wpSrcStmt{ \try \ \{ \stmt_1 \} \ \finally \ \{ \stmt_2 \} } { \normalPostSrc}{\excPostSrc} =^{\mbox{\rm\textsf{try finally}}}\\
       \begin{array}{l} 
	 \wpSrcStmt{ \stmt_1}{\\ \phantom{wpiSr}
                     \wpSrcStmt{\stmt_2}{ \normalPostSrc} { \excPostSrc}}{ \\ \phantom{wpiSr}
		     \update{\excPostSrc}{\mbox{\rm\texttt{Exception}}}{ \wpSrcStmt{\stmt_2}{ \excPostSrc( \mbox{\rm\texttt{Exception}}  )}{\excPostSrc}  } }
       \end{array} 
       
\end{array} } $$

\caption{\sc WP for the control statements with exceptions }
\label{pog:wpSrc:wpStmt:withExc}
\end{frameit}
\end{figure}





