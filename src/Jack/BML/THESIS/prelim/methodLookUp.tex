\index{\lookupOnly}
\section{Method lookup}

Java allows that a class   declares a method \methodd{} with the same signature as in one of its super types.
In this case, we say that method \methodd{}  overrides the corresponding method in the super type or interface.
When defining subtypes, this feature of Java allows to change the behavior of the subtype
 w.r.t. the superclass behavior.

This however complicates slightly the invocation mechanism. Let us see why.
 The problem arises because the types of variables  on execution time differ from the type 
with which they are declared. In particular, in Java we talk about  dynamic type and static type of expressions.
Dynamic types can not be determined statically, they can only be determined on execution time.
For instance, if we have a Java expression 
$a.\methodd()$ where  the variable $a$ is declared with type $A$ and class $A$ has a subclass $B$,
 in the general case it is not possible  to determine statically if $a$ contains 
a reference value of dynamic type $B$ or dynamic type $A$.
 If a method  \methodd{} is invoked on object of dynamic type $B$
accordingly to the semantics of Java, we intend to call the method declared in the class $B$ if a method with such signature is declared in $B$.


Because this cannot be resolved statically,  it is the virtual machine which on execution time determines which method to execute. 
 In order, to simulate this behavior we define the function \lookupOnly, which takes a method signature - its name, its argument types and its return value
 and a class \class{} and determines either the method is declared
in \class{} via the function \findMethodOnly{} which searches in the list of the method of class \class. If this is the case, then the lookup procedure returns the method in \class, otherwise the procedure continues recursively
to look for the method in the super type \class.\superClass{} of \class. Finally, we give the function for the method look up:

$$\begin{array}{l}
\lookup{name}{ argTypes}{retType }{\class} = \\
       \left\{\begin{array}{ll}
             \methodd & \findMethod{name}{ argTypes} {retType}{\class} = \methodd \\
	              & \\
	     \methodd &    \begin{array}{l}
                                  \findMethod{name}{ argTypes} {retType}{\class} = \bottom \wedge \\
				  \lookup{name}{ argTypes} {retType}{\class.\superClass} = \methodd
			\end{array}	  
       \end{array}\right.
\end{array}$$
