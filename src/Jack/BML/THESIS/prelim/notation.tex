
 \section{Notation}\label{notation}
 Here we introduce several notations used in the rest of this chapter.
 If we have a function $f$ with  domain type  $A$ and range type $B$
 we note it with $f : A \rightarrow B$. If the function receives $n$ arguments of type $A_1 \ \ldots \ A_n$ respectively
 and maps them to elements of type $B$
 we note the function signature with $f : A_1 * ...*A_n \rightarrow B$.
 The set of elements which  represent the domain of the  function $f$
 is given by the function $\Dom(f)$ and the elements in its range are given
 by $\Range(f)$. 

 Function updates of function $f$ with n arguments is denoted with $ \update{f}{x_1 \ldots x_n}{y} $ and the definition of such function is :
 $$  \update{f}{x_1 \ldots x_n }{y}(z_1 \ldots z_n) = 
    \left\{\begin{array}{ll}
                  y & if \ x_1 = z_1 \wedge ...\wedge x_n = z_n \\
		  f(z_1 \ldots z_n ) & else
           \end{array}\right.$$
 

 The type $list$ \ is used to represent a sequence of elements. The empty list is denoted with $\emptyList$.
 If it is true that the element $e$ is in the  list $l$, we use the notation $e \in l$.
 The function \assocList \ receives two arguments an element $e$ and a list $l$ and returns a new list 
 $e \assocList l$ whose head and tail are respectively $e$ and $l$. The number of elements in a list $l$ is denoted with  $l.\listLen$.
 The $i$-th element in a list $l$  is denoted with $l[i]$. Note that the indexing in a list $l$ starts at $0$, thus the last
 index in  $l$ being $l.\listLen - 1$.




 
