\section{Related Work}\label{relWork}
 A considerable effort has been done on the formalization of 
 the JVM semantics as a reply to the holes and ambiguities encountered in
 the specification of the Java bytecode verifier. 
 Thus,  most of the existing formalizations are
 used for reasoning over the Java bytecode well-typedness.
 
 %cover a representative subset of the language.
 We can start with the work of Stata and Abadi \cite{stata98type} in which they propose a semantics  and 
  typying rules for checking the correct behavior of  subroutines in the presence of polymorphism.
 In \cite{FM99FFJ}  N.Freund and J.Mitchell extend the aforementioned work to 
 a language which supports all the Java features e.g.
 object manipulation and instance initialization, exception handling, arrays.
 In \cite{qian99formal}  Qian  gives a formalization in terms of a small step operational semantics of a
 large subset of the Java bytecode language including method calls, object creation and manipulation,
 exception throwing and handling as well subroutines, which is used for the formal specification of the language and the bytecode verifier.
 Based on the work of Qian, in \cite{pusch98proving} C.Pusch gives a formalization of the JVM and the Java Bytecode Verifier
 in Isabelle/HOL and proves in it the soundness of the specification of the verifier.
 In \cite{KleinN04}, Klein and Nipkow give a formal small step and big step operational
 semantics of a Java-like language called Jinja, an operational semantics of a Jinja VM and its type system and a bytecode verifier as well as 
 a compiler from Jinja to the language of the JVM. They prove the equivalence between the small and big step
 semantics of Jinja, the  type safety for the Jinja VM, the correctness of the bytecode verifier w.r.t. the type system
 and finally that the compiler preseves semantics and well-typedness.
  

% Java Card
  The small size and complexity of the JavaCard platform (the JVM version tailored to smart cards) 
 simplifies the formalization of the system and thus,
 has attracted  particularly the scientific interest. CertiCartes \cite{barthe01formal,barthe02formal}
 is an in-depth formalization of JavaCard. It has a formal executable
 specification written in Coq of a defensive and an offensive JCVM and an abstract JCVM together with the specification
 of the Java Bytecode Verifier.  Siveroni proposes a formalization of the JCVM in \cite{siveroni04operational} in terms of
 a small step operational semantics. 


