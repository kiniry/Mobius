\subsection{Related Work}\label{relWork}
 A considerable effort has been done on the formalization of 
 the semantics of the JVM. Most of the existing formalizations cover a representative subset 
 of the language. Among them is the work \cite{FM99FFJ} by N.Freund and J.Mitchell and \cite{qian99formal} by Qian,
 which give a formalization in terms of a small step operational semantics of a
 large subset of the Java bytecode language including method calls, object creation and manipulation,
 exception throwing and handling as well subroutines, which for the formal specification of the language and the bytecode verifier.

 Based on the work of Qian, in \cite{pusch98proving} C.Pusch gives a formalization of the JVM and the Java Bytecode Verifier
 in Isabelle/HOL and proves in it the soundness of the specification of the verifier.
 In \cite{KleinN04}, Klein and Nipkow give a formal small step and big step operational
 semantics of a Java-like language called Jinja, an operational semantics of a Jinja VM and its type system and a bytecode verifier as well a compiler from Jinja
 to the language of the JVM. They prove the equivalence between the small and big step semantics of Jinja, the  type safety for the Jinja VM, the correctness
 of the bytecode verifier w.r.t. the type system and finally that the compiler preseves semantics and well-typedness.
  
 The small size and complexity of the JavaCard platform simplifies the formalization of the system and thus,
 has attracted  particularly the scientific interest. CertiCartes \cite{barthe01formal,barthe02formal}
 is an in-depth formalization of JavaCard. It has a formal executable
 specification written in Coq of a defensive and an offensive JCVM and an abstract JCVM together with the specification
 of the Java Bytecode Verifier.  
