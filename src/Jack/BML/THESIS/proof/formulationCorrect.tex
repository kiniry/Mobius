\newtheorem{defCorrect}{Definition}[section]
\newtheorem{vcGenCorrect}{Theorem}[section]

\section{Formulation of the correctness statement}\label{proof:defineCorrect}

%The correctness of our verification  condition generator is established w.r.t.to the operational semantics described in Section \ref{opSem}.
In order to define what it means for the verification calculus to be correct we will need to 
 precise what does it mean that a  method respects its specification. As we have stated 
 before the intuition behind this notion is that if the method starts execution in a state 
 where its precondition  holds then if it terminates execution then its postcondition holds.
 Another condition for the correctness of a method is that if it overrides a method from a super class
 of the class where it is declared  it must respect the specification of the method it overrides. As we explained 
 earlier in Chapter \ref{assertLang}, Section \ref{assertLang:lang} the specification upon which the verification 
 of an overriding method is performed guarantees that the method respects automatically the specification of the overriden method.
 Thus, we shall not discuss in the following the behavioral subtyping as it holds trivially because of the way we construct specifications 
 for overriding methods. Thus, we formulate method correctness as follows:

\begin{defCorrect}[A method respects its specification] \label{defCorrect}
For every method \methodd \ with precondition \methodd.\pre, normal postcondition \methodd.\normalPost
and exceptional postcondition function \methodd.\excPostSpec, we say that \methodd \ respects its specification if 
for every two states $s_0$ and $s_1$ such that :
\begin{itemize}
      
      \item   $\methodd : s_0  \stateTransTerm s_1   $
      \item   $ \interp{\methodd.\pre }{ s_0 }$
\end{itemize}
Then if \methodd \ terminates normally then the normal postcondition holds in the final state $s_1 $:  $\interp{\methodd.\normalPost}{s_1}$. 
Otherwise, if  \methodd \ terminates on an exception \mbox{ \rm \texttt{Exc}} the exceptional postcondition holds in the poststate $s_1 $
$ \interp{\methodd.\excPostSpec(  \mbox{ \rm \texttt{Exc}} )}{s_1} $
\end{defCorrect}


Let us now see informally the correctness condition for the verification calculus. 
We would like to establish that if the verification conditions
generated for a method 
are valid 
and the precondition of the method holds in the initial state of the method 
then the postcondition of  the method holds in its final state if the method terminates.
Formally, we want to establish the following statement:

\begin{vcGenCorrect}\label{vcGenCorrect}
For any  method \methodd \  if the verification condition is valid:
$$ \numConclusion{1} \ \validFormula{\methodd.\pre \Rightarrow \wpi{\method.\body[0]}{\methodd}{}} $$
 then \methodd \ respects its specificcation in the sense of the Def. \ref{defCorrect}. 
\end{vcGenCorrect}

The main purpose of the current chapter is to establish this theorem. Before entering in technical part of the proof
we shall give an outline of the steps to be taken for establishing it.
