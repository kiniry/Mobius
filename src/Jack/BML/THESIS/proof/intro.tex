%\section{Introduction}
 In the previous chapter \ref{wpGeneral}, we defined a verification
 condition generator for a Java bytecode like language. We used a weakest precondition
 to build the verification conditions. In this section, we will argue formally that the
 proposed verification condition generator is correct, or in other words that it is sufficient
 to prove the verification conditions generated over a bytecode program and its specification 
 for establishing that the program respects the specification. 

 In particular, we will prove the correctness of our methodology w.r.t. the operational semantics of our bytecode language
 given in chapter \ref{opSem}. The way in which the proof is done is standard. Note that the formalization
 of the operational semantics in terms of relation on states serves us to give a model for our assertion
 language. 

 Another point which deserves our attention, is that we will prove the partial correctness of our methodology, i.e.
 we suppose that our programs always terminate.

 In the following, we will proceed with a section which establishes how we define substitution. 
 In section \ref{interpret}, we give a meaning of formulas from our assertion language in a state.
 The last section \ref{proof} skatches the proof of correctness of the verification condition generator.  


