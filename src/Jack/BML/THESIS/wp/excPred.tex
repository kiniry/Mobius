

\subsection{Weakest precondition in the presence of runtime exceptions}\label{wp:interExc} 

For every method \methodd \  we define also the function $ \methodd.\excPost $ \ with signature:
$$\methodd.\excPost : int   \longrightarrow \excType \longrightarrow  \predWp  $$
 
$ \methodd.\excPost(\mbox{ \rm \textit{i}},  \mbox{ \rm \texttt{Exc}})$ 
returns the predicate that must hold in the poststate of the instruction at index \textit{i} if the
 instruction throws a runtime exception of type \texttt{Exc}.
Note that the function $\methodd.\excPost$  does not deal with programmatic exceptions thrown by the instruction \athrow,
 neither exception caused by a method invokation
(execution of instruction \invoke). Those two cases are handled in a different way as we
 shall see later in the definition of the \fwpi \ function in Section
\ref{wpRules}.  

The formal definition follows.
\begin{defExc}[Postcondition in case of throwing an exception]\label{defExc}
The function application $\methodd.\excPost( \mbox{ \rm \textit{i}} ,  \mbox{ \rm \texttt{Exc}} )   $ is defined as follows:
 \begin{itemize}
    \item if $  \ins{i} \neq \athrow \wedge \ins{i} \neq \invoke \wedge   \findExcHandler{\mbox{ \rm \texttt{Exc}} }{i}{\methodd.\excHandlerTable} = \pcHandler \wedge \pcHandler \neq \bottom    $ then 
      $$ \begin{array}{l}
           \methodd.\excPost( \mbox{ \rm \textit{i}} ,  \mbox{ \rm \texttt{Exc}}) = \\
                     \forall \referenceOnly,\\
                      \Myspace    not \ \instances(\referenceOnly) \wedge \\
		      \Myspace \referenceOnly \neq \Mynull \Rightarrow	\\

       \Myspace    \Myspace    \inter{i}{\pcHandler}\\
                      \begin{array}{l}
                        \subst{\counter}{0} \\
			\subst{\stack{0}}{\referenceOnly} \\
                         \subst{ \fieldd} { \update{\fieldd} { \referenceOnly }{\defaultValueOnly( \fieldd.  \fieldType ) } }_{{\small \forall \fieldd : \FieldSet, \ 
                         \subtype{\fieldd.\declaredIn}{ \mbox{ \rm \texttt{Exc}}} } } \\
			  \subst{ \typeof{\referenceOnly}}{  \mbox{ \rm \texttt{Exc}} } \\
                       \end{array} 
        \end{array}$$   


   


    \item else if $\ins{i} \neq \athrow \wedge \ins{i} \neq \invoke \wedge   \findExcHandler{ \mbox{ \rm \texttt{Exc}}   }{i}{\methodd.\excHH} = \bottom $ then 
            $$\begin{array}{l}
                    \methodd.\excPost( \mbox{ \rm \textit{i}} ,  \mbox{ \rm \texttt{Exc}}) = \\

                     \forall \referenceOnly,\\
                      \Myspace    not \ \instances(\referenceOnly) \wedge \\
		      \Myspace \referenceOnly \neq \Mynull \Rightarrow	\\

                 \Myspace    \Myspace      \methodd.\excPostSpec( \mbox{ \rm \texttt{Exc}}  ) \\
                     \begin{array}{l}
		           \subst{ \EXC }{\referenceOnly  }\\
                           \subst{ \fieldd}{ \update{\fieldd } { \referenceOnly }{\defaultValueOnly( \fieldd.\fieldType ) } }_{ {\small \forall \fieldd: \FieldSet, 
			   \subtype{\fieldd.\declaredIn}{ \mbox{ \rm \texttt{Exc}}} } }    \\
			   \subst{ \typeof{\referenceOnly}}{  \mbox{ \rm \texttt{Exc}} } \\
                    \end{array}  
             \end{array} $$
	
    




\end{itemize}
\end{defExc}

The above definition considers two cases depending on if  the thrown exception   is handled or not.

The function $\methodd.\excPost$ will return the weakest precondition of the exception handler that protects the instruction from the thrown exception \texttt{Exc},
if such an exception handler exists ( $  \findExcHandler{\mbox{ \rm \texttt{Exc}} }{i}{\methodd.\excHH} $ $ \neq \bottom $ ). 
Otherwise, the function will return the exceptional postcondition if method \methodd \ terminates on
an exception  \texttt{Exc}.
  
 If the thrown exception is not caught  then the exceptional postcondition predicate must hold, where the specification variable \EXC, which stands for the thrown object in exceptional 
postocnditions, is substituted with the reference to the thrown object.


%Note that in what follows we do not treat the Java virtual machine errors.
%x Thus, we deal only with user defined exceptions and \texttt{JavaRuntimeException} subclasses.
