\newtheorem{defExc}{Definition}[subsection]
\newtheorem{defExcRuntime}[defExc]{Definition}% auxiliary function
\subsection{Weakest precondition in the presence of exceptions } \label{wp:Exc} % this section describes a function which decides what is the exceptional postcondition if an instruction throws an exception
Our weakest precondition calculus deals with exceptional termination and thus, we need
a way for calculating the postcondition of an instruction in  case it terminates on an exception.
In particular, the postcondition should depend on if there is an exception handler or not. 
In the first case, the execution continues at the exception handler entry point and thus
 the postcondition of the exceptionally terminating instruction will be the precondition 
of the instruction from which the exception handler starts. In the case where there is no exception handler,
 this means that the current method also terminates on exception and thus, the specified exceptional exceptional
postcondition of the method for this exception should hold.

We define the function \getExcPost \  with signature :
 $$\getExcPost : int   \longrightarrow \excType \longrightarrow  \predWp  $$
The function \methodd.\getExcPost \ 
takes as arguments an index $i$ in the array of instructions of method \methodd \ and an exception type
 \mbox{ \rm \texttt{Exc}}  and returns the predicate \methodd.\getExcPost($i$, \mbox{ \rm \texttt{Exc}}   )
  that must hold after the instruction at index $i$ throws an exception of type \mbox{ \rm \texttt{Exc}}. We give a formal definition hereafter.


\begin{defExc}[Postcondition in case of a thrown exception]\label{defExc}
$$  \begin{array}{l}
           \methodd.\getExcPost(i, \mbox{ \rm \texttt{Exc}}   ) = \\
   \left\{\begin{array}{ll}
        \wpi{}{\methodd}{\pcHandler}   & if \  \findExcHandler{\mbox{ \rm \texttt{Exc}} }{i}{\methodd.\excHandlerTable} = \pcHandler \\
	\methodd.\excPostSpec( \mbox{ \rm \texttt{Exc}}  ) & \findExcHandler{\mbox{ \rm \texttt{Exc}} }{i}{\methodd.\excHandlerTable} = \bottom 
  \end{array}\right.
\end{array}
$$
\end{defExc}

%\subsection{Weakest precondition in the presence of runtime exceptions}\label{wp:interExc} 
Next, we introduce an auxiliary function which will be used in the definition of the \fwpi \ function for instructions
that may throw runtime exceptions.
Thus, for every method \methodd \  we define  the auxiliary function $ \methodd.\excPost $ \ with signature:
$$\methodd.\excPost : int   \longrightarrow \excType \longrightarrow  \predWp  $$
 
$ \methodd.\excPost(\mbox{ \rm \textit{i}},  \mbox{ \rm \texttt{Exc}})$ 
returns the predicate that must hold in the preststate of the instruction at index \textit{i}  which may throw a runtime exception of type \texttt{Exc}.
Note that the function $\methodd.\excPost$  does not deal with programmatic exceptions thrown by the instruction \athrow,
 neither exception caused by a method invokation
(execution of instruction \invoke) as the exceptions thrown by those instructions are handled in a different way as we
 shall see later in the definition of the \fwpi \ function in the next subsection.  




The function application $\methodd.\excPost( \mbox{ \rm \textit{i}} ,  \mbox{ \rm \texttt{Exc}} )   $ is defined as follows:
\begin{defExcRuntime}[Auxuliary function for instructions throwing runtime exceptions]   \label{wp:exc:defExcRuntime}
      $$ \begin{array}{l}
            \ins{i} \neq \athrow \wedge \ins{i} \neq \invoke \Rightarrow  \\
           \methodd.\excPost( \mbox{ \rm \textit{i}} ,  \mbox{ \rm \texttt{Exc}}) = \\
                     \forall \referenceOnly,\\
                      \Myspace    \neg \ \instances(\referenceOnly) \wedge \\
		      \Myspace \referenceOnly \neq \Mynull	\\
		      \Myspace \typeof{\referenceOnly} <: \mbox{ \rm \texttt{Exc}}  \Rightarrow\\
          \Myspace  \left(   \methodd.\getExcPost(\mbox{ \rm \textit{i}},  \mbox{ \rm \texttt{Exc}})
                    %\begin{array}{l}
                        \subst{\counter}{0} %\\
			\subst{\stack{0}}{\referenceOnly} %\\
                         \subst{ \fieldd} { \update{\fieldd} { \referenceOnly }{\defaultValueOnly( \fieldd.  \fieldType ) } }_{{\small \forall \fieldd : \FieldSet, \ 
                         \subtype{\fieldd.\declaredIn}{ \mbox{ \rm \texttt{Exc}}} } } %\\
			  %\subst{ \typeof{\referenceOnly}}{  \mbox{ \rm \texttt{Exc}} } \\
                    %   \end{array} 
\right)
        \end{array}$$   

\end{defExcRuntime} 
   


 




The function $\methodd.\excPost$ will return a predicate which states that for every newly created exception reference  
the predicate returned by the function \getExcPost \ for the exception type \mbox{ \rm \texttt{Exc}}   and program point $i$ must hold.


 
%Note that in what follows we do not treat the Java virtual machine errors.
%x Thus, we deal only with user defined exceptions and \texttt{JavaRuntimeException} subclasses.
