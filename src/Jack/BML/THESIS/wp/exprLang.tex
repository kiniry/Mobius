

\newtheorem{deepExpr}{Definition}[section]

\section{The expression language}

In the following, we will introduce a deep encoding of the expressions and
predicates over which the \fwpi \ calculus will be defined. 
Most of the expressions are directly taken from the specification language BML introduced
in Chapter \ref{bcsl}. However, there are several construct which does not belong to the
BML grammar. We keep the same  set of predicates as in BML. 
The next definition gives the set of expressions and formulas.

\begin{deepExpr}[Language of expressions]\label{deepExpr}
$$\begin{array}{ll}
\ConstantsWp           & ::=  \Values \mid \ClassSet \\
% & \\
%\refWp               & ::= \refWp\_\intLiteral \\
%& \\
%\boundVar            & ::= \bound\_\intLiteral \\ 
& \\
  \exprWp            & ::= \ConstantsWp \\
                     &  \mid  \locVar{nat} \\ 
       	             &  \mid  \fieldAccess{\exprWp}{\fieldd}, \fieldd :\FieldSet\\
		%     &  \mid \ident \\
		     &  \mid  \update{\fieldd}{ \exprWp}{\exprWp}(\exprWp), \fieldd :\FieldSet\\
		     &  \mid  \arrayAccess{\exprWp} {\exprWp} \\	   
		     &  \mid \update{ \arrayAccessOnly}{ (\exprWp , \exprWp)}{ \exprWp} (\exprWp,\exprWp) \\	
		     &  \mid  \exprWp \ \op \ \exprWp   \\
		     &  \mid  \counter \\
		     &  \mid  \stack{ \exprWp} \\
                     &  \mid \EXC    \\
		     &  \mid  \result \\
	%	     &  \mid  \var \\
    & \\

    \typeExpWp       & :: =  \typeof{ \exprWp} \\
                     &  \mid \type{\exprWp} \\
                     &  \mid \elemtype{\exprWp  }\\
		     &  \mid \TYPE\\   
   & \\ 
   \expressionsWp    & :: = \exprWp \mid  \typeExpWp  \\
   & \\

 \predWp  & ::=   \expressionsWp \ \predicates \  \expressionsWp    \\
	  & \mid  \true \\
	  & \mid  \false  \\	
          & \mid not \ \predWp  \\
	  & \mid \predWp  \wedge  \predWp \\
	  & \mid \predWp \vee  \predWp \\
	  & \mid \predWp  \Rightarrow \predWp \\
          & \mid \predWp \iff  \predWp \\
	  & \mid \forall \ x , \predWp\\
	  & \mid \exists \ x  , \predWp	
\end{array}$$
\end{deepExpr}

From the above definition, that the constants in the language are the values as defined in Chapter \ref{opSem},  section \ref{types} as well 
the set of classes \ClassSet.
Note that the expression language also supports update expressions for field and array access.
 These expressions appear in the intermediate states of the \fwpi \ calculus.
 
 In the following, we will proceed with subsection \ref{subst} which establishes some particularities of the substitution. 
 In section \ref{interpret}, we give a meaning of formulas from our assertion language in a state.
