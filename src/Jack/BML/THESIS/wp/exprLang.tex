

\newtheorem{deepExpr}{Definition}[section]

\section{The expression language}

In the following, we will introduce a deep encoding of the expressions and
predicates over which the \fwpi \ calculus will be defined. 
Most of the expressions are directly taken from the specification language BML introduced
in Chapter \ref{bcsl}. However, there are several construct which does not belong to the
BML grammar. We keep the same  set of predicates as in BML. 
The next definition gives the set of expressions and formulas.

\begin{deepExpr}[Language of expressions]\label{deepExpr}
$$\begin{array}{ll}
\Constants   & ::= \intLiteral  \mid \signedInt  \mid \Mynull  \mid \refWp \\
& \\
\refWp & ::= \refWp\_\intLiteral \\
& \\
  \exprWp      & ::= \ConstantsWp \\
                     &  \mid  \locVar{ \digits } \\ 
       	             &  \mid  \fieldAccess{\exprWp}{\fieldd}, \fieldd :\FieldSet\\
		     &  \mid \ident \\
		     &  \mid  \update{\fieldd}{ \exprWp}{\exprWp}(\exprWp) \\
		     &  \mid  \arrayAccess{\exprWp} {\exprWp} \\	   
		     &  \mid \update{ \arrayAccessOnly}{ (\exprWp , \exprWp)}{ \exprWp} (\exprWp,\exprWp) \\	
		     &  \mid  \exprWp \ \op \ \exprWp   \\
		     &  \mid  \counter \\
		     &  \mid  \stack{ \exprWp} \\
                     &  \mid \EXC    \\
		     &  \mid  \result \\
		     &  \mid  \boundVar \\
    & \\

    \typeExpWp       & :: =  \typeof{ \exprWp} \\
                     &  \mid \type{\ident} \\
                     &  \mid \elemtype{\exprWp  }\\
		     &  \mid \TYPE\\   
   & \\ 
   \expressionsWp    & :: = \exprWp \mid  \typeExpWp  \\
   & \\

 \predWp  & ::=   \expressionsWp \ \predicates \  \expressionsWp    \\
	  & \mid  \true \\
	  & \mid  \false  \\	
          & \mid not \ \predWp  \\
	  & \mid \predWp  \wedge  \predWp \\
	  & \mid \predWp \vee  \predWp \\
	  & \mid \predWp  \Rightarrow \predWp \\
          & \mid \predWp \iff  \predWp \\
	  & \mid \forall \ \boundVar , \predWp\\
	  & \mid \exists \ \boundVar  , \predWp	
\end{array}$$
\end{deepExpr}

From the above definition, we can notice that field expressions are defined this time by using an identifier.
The deep expression language also supports
 update expressions for field and array access.  These expressions appear in the intermediate states of \fwpi \ calculus.
We also get a special syntactic construct for references \refWp \ which is not part of BML. \refWp \ expressions stand 
for the reference values and are used basically in cases when a new reference is created.
 In the following, we will proceed with a subsection which establishes some particularities of the substitution. 
 In section \ref{interpret}, we give a meaning of formulas from our assertion language in a state.
