\subsection{Substitution} \label{subst}
 Expression substitution is defined inductively in a standard way over the expression structure. Still, we allow also
 substitution over objects that are not from our language, i.e. we apply substitution over field objects which results 
 in an update version of the field. 
 This is done by establishing the substitution rule for field access as follows:
 $$  \substitution{\fieldd (o )}{ \expressionsWp^1}{ \expressionsWp^2} =\substitution{\fieldd}{ \expressionsWp^1}{ \expressionsWp^2} (  \substitution{o}{ \expressionsWp^1}{ \expressionsWp^2}) $$


 In the next, we define a substitution over field function objects:
$$  
\begin{array}{l}
\substitution{\fieldd^1}{\fieldd^2}{ \fieldd^2[ \oplus  \expressionsWp^1 \longrightarrow  \expressionsWp^2 ] } = \\\\
\left\{\begin{array}{l}
 if \ \fieldd^1 \neq \fieldd^2 \ then \\ 
\fieldd^1  \\

\\ 
 else \ if \  \fieldd^1 = \fieldd^2 \ then \\ 
\fieldd^2[ \oplus  \expressionsWp^1 \longrightarrow  \expressionsWp^2 ]   \\

           
\end{array}
\right. \\
\\
\\
\substitution {\update{\fieldd^1}{  \expressionsWp^1 }{  \expressionsWp^2}} {\fieldd^2}{  \update{\fieldd^2}{  \expressionsWp^3 }{  \expressionsWp^4}       } = \\\\
\left\{\begin{array}{l}  
if \ \fieldd^1 \neq \fieldd^2 \ then \\
\update{\fieldd^1}{ \substitution { \expressionsWp^1}{\fieldd^2}{  \update{\fieldd^2}{  \expressionsWp^3 }{  \expressionsWp^4}}  }{ \substitution { \expressionsWp^2}{\fieldd^2}{  \update{\fieldd^2}{  \expressionsWp^3 }{  \expressionsWp^4}} } \\

%\fieldd[ \oplus \mbox{ \rm \texttt{r} }\substitution{ \expressionsWp^1}{ \expressionsWp^2} \longrightarrow  \mbox{ \rm \texttt{v} }\substitution{ \expressionsWp^1}{ \expressionsWp^2} ] 

 \\ 
 else \   \fieldd^1 = \fieldd^2  \ then \\ 
\fieldd^1 \begin{array}{l}
             \lbrack \oplus  \substitution { \expressionsWp^1}{\fieldd^2}{  \update{\fieldd^2}{  \expressionsWp^3 }{  \expressionsWp^4}}  \longrightarrow  
                 \substitution { \expressionsWp^2}{\fieldd^2}{  \update{\fieldd^2}{  \expressionsWp^3 }{  \expressionsWp^4}}       \rbrack \\
	     \lbrack \oplus  \expressionsWp^2 \longrightarrow  \expressionsWp^3 \rbrack
	     \end{array} \\

               

\end{array}
\right. 
\end{array}$$ 


For example, consider the following  substitution expression:  
$$ \substitution{\fieldd(  \expressionsWp^1)}{\fieldd}{\update{\fieldd}{ \expressionsWp^2 }{  \expressionsWp^3 }} $$
This results in the new expression : 
$$\update{\fieldd}{ \expressionsWp^2  }{ \expressionsWp^3 } (\substitution{  \expressionsWp^1}{ \fieldd}{ \update{ \fieldd}{   \expressionsWp^2  }{  \expressionsWp^3 }}) $$


The same kind of substitution is allowed for array access expressions, where the array object \arrayAccessOnly  can be updated. 
