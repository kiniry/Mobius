\subsection{Substitution} \label{subst}
 Expression substitution is defined inductively in a standard way over the expression structure. Still, we allow also
 substitution over objects that are not from our language, i.e. we apply substitution over field objects which results 
 in an update version of the field. Such a substitution has the form :

$$ \substitution{\expressionsWp}{ \fieldd }{ \update{\fieldd}{\expressionsWp}{\expressionsWp} } $$
This substitution does not affect any of the ground expressions,, i.e. it does not affect 
local variables ($\locVar{i}$), the constants of our language (\ConstantsWp),  the stack counter (\counter), the result expression
(\result), the thrown exception instance variable (\EXC). For instance, the following substitution does not change $\locVar{1}$: 
$$
 \substitution{\locVar{1}}{ \fieldd }{ \update{\fieldd}{\expressionsWp}{\expressionsWp} } = \locVar{1}
$$   

Field substitution affects only field objects as we see in the following: 



 
 %This is done by establishing the substitution rule for field access as follows:
 %$$  \substitution{\fieldd ( \expressionsWp^1)}{ \expressionsWp^2}{ \expressionsWp^3} =
 %\substitution{\fieldd}{ \expressionsWp^2}{ \expressionsWp^3} (  \substitution{ \expressionsWp^1  }{ \expressionsWp^2}{ \expressionsWp^3}) $$


% Let us see how we define  substitution over field objects:
$$  
\begin{array}{l}
\substitution{\fieldd^1}{\fieldd^2}{ \fieldd^2[ \oplus  \expressionsWp^1 \longrightarrow  \expressionsWp^2 ] } = \\\\
\left\{\begin{array}{l}
 if \ \fieldd^1 \neq \fieldd^2 \ then \  
\fieldd^1  \\

\\ 
 else \ if \  \fieldd^1 = \fieldd^2 \ then \
\fieldd^2[ \oplus  \expressionsWp^1 \longrightarrow  \expressionsWp^2 ]   \\

           
\end{array}
\right. \\
\\
\\
\substitution {\update{\fieldd^1}{  \expressionsWp^1 }{  \expressionsWp^2}} {\fieldd^2}{  \update{\fieldd^2}{  \expressionsWp^3 }{  \expressionsWp^4}       } = \\\\
\left\{\begin{array}{l}  
if \ \fieldd^1 \neq \fieldd^2 \ then \\
\update{\fieldd^1}{ \substitution { \expressionsWp^1}{\fieldd^2}{  \update{\fieldd^2}{  \expressionsWp^3 }{  \expressionsWp^4}}  }{ \substitution { \expressionsWp^2}{\fieldd^2}{  \update{\fieldd^2}{  \expressionsWp^3 }{  \expressionsWp^4}} } \\

%\fieldd[ \oplus \mbox{ \rm \texttt{r} }\substitution{ \expressionsWp^1}{ \expressionsWp^2} \longrightarrow  \mbox{ \rm \texttt{v} }\substitution{ \expressionsWp^1}{ \expressionsWp^2} ] 

 \\ 
 else \   \fieldd^1 = \fieldd^2  \ then \\ 
\fieldd^1 \begin{array}{l}
             \lbrack \oplus  \substitution { \expressionsWp^1}{\fieldd^2}{  \update{\fieldd^2}{  \expressionsWp^3 }{  \expressionsWp^4}}  \longrightarrow  
                 \substitution { \expressionsWp^2}{\fieldd^2}{  \update{\fieldd^2}{  \expressionsWp^3 }{  \expressionsWp^4}}       \rbrack \\
	     \lbrack \oplus  \expressionsWp^3 \longrightarrow  \expressionsWp^4 \rbrack
	     \end{array} \\

               

\end{array}
\right. 
\end{array}$$ 


For example, consider the following  substitution expression:  
$$ \substitution{\fieldd(  \locVar{1})}{\fieldd}{\update{\fieldd}{ \locVar{2} }{  3 }} $$
This results in the new expression : 
$$\update{\fieldd}{ \locVar{2}  }{ 3} ( \locVar{1})$$ 


The same kind of substitution is allowed for array access expressions, where the array object \arrayAccessOnly  \ can be updated. 
