
\section{Weakest Precondition Calculus For Java Bytecode}\label{wpbc}
In this section, we define a bytecode logic in terms of a weakest precondition calculus.
We assume that the bytecode program has passed the bytecode verification procedure (we discuss the issue in section \ref{relWork}),
 thus the calculus is concerned only with program functional properties. We also assume that code is generated by a non optimizing compiler. 

The proposed weakest precondition has those features:
\begin{itemize}
%\item modular, design by contract verification, in particular every method is verified separately method calls being translated to their specification 
\item it supports all Java sequential instructions except for floating point arithmetic instructions and 64 bit data (\java{long} and \java{double} types), including 
exceptions, object creation, references and subroutines. The calculus is defined over the method control flow graph

\item it supports BCSL (section \ref{bcSpecLg}), i.e. method's specification written in BCSL like pre- and postconditions, assertions at particular program point among 
which loop invariants (if there is nothing special specified the specification by default: preconditions, postconditions and invariants are taken to be true) is taken into account. %The verification procedure assumes that the bytecode is specified enough, i.e. we do not try to infer specification, as we assume that they are compiled from the source program
\end{itemize}

The calculus is defined over the control flow graph of the program and has two levels of definitions --- the first one is the set of rules for single Java bytecode instructions (discussed in subsection~\ref{wpInstr} ) and the second one takes into account how control
 flows in the bytecode(subsection ~\ref{wpGraph}). A related problem is how the loops in the control flow are treated. 
As we mentioned earlier we assume that every method is specified in sufficient details, i.e. if there are loops, the corresponding 
loop invariant is present. This allows us to ``cut'' the loops in the graph at the program point where the invariant must hold. 
These ``cuts'' generate an abstract control flow graph which is acyclic and over which the verification conditions are generated. Subsection ~\ref{abstrCntrFlow} discusses 
how the abstract control flow graph is generated.

%In the rest of the section we describe the bytecode logic  and how the verification conditions are generated:
% the weakest precondition rules for single Java bytecode instructions in subsection \ref{wpInstr}, 
% the method abstract control flow graph is explained in subsection ~\ref{abstrCntrFlow}, 
% the definition of the weakest precondition over the abstract control flow graph is discussed in ~\ref{wpGraph} where how exception handling and subroutines
%are described.



\input wpSingleInstr.tex
\input abstractCntrFlow.tex
\input wpOverGraph.tex

%\begin{center} \texttt{wp} : \texttt{STMT} $\longrightarrow$ \texttt{Predicate} $\longrightarrow$ \texttt{Predicate}\end{center}





