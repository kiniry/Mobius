\documentclass[a4paper,10pt]{article}
\usepackage{form}

\title{Formalization}
\author{Alejandro Tamalet}

\begin{document}
\maketitle

\section{Notation}\label{sec:Notation}
The abstract syntax of Java programs is described by production rules of the form
\begin{haskell}
T & \geqdef\,\,\, & C_1\, a_{11}: T_{11}; a_{12}: T_{12} \dots\\
 & \gbar & \ldots\\
 & \gbar & C_n\,\, a_{n1}: T_{n1}; a_{n2}: T_{n2} \dots\\
\end{haskell}
An element $a_{ij}$ is called a component and it is described by the rule $T_{ij}$. The number of components
in each branch may be different but finite. 
Components of the same type can be joined as in $a_1, a_2 \colon T$ and mutual recursive definitions of the
rules are allowed.
To obtain an specimen of type $T$ we use one of the labels $C_i$ as a constructor, e.g. $C_i(e_1, e_2, \ldots,
e_k)$ where each $e_j$ with $j=1..k$ is an specimen described by the rule $T_{ij}$; this implies that the
order in which components are named matters.
When there is just one way to construct an specimen of $T$ we omit the label and we use $T$ as constructor.
By convention constructors begin with capital letters.

To access a component of a specimen we use the standard dot notation: $x.p$ refers to the component $p$ of the
specimen $x$.
The notation is lifted over lists and sets.
When $xs$ is a set, $xs.p$ is the set of the $p$ component of the elements of $xs$, that is,
\begin{haskell}
  xs.p \eqdef \{x.p | x \in xs\}.
\end{haskell}
Similarly, when $xs$ is a list we define
\begin{haskell}
  xs.p \eqdef [x.p | x \leftarrow xs].
\end{haskell}

The update notation $c(a := new\_a)$ is used to construct a specimen that is like $c$ but where its $a$
component is updated to $new\_val$. There can be several updates separated by commas.

When a construct can be repeated many times but the order is not important and each element is unique we use
sets, in those cases we write $xs \colon \powerset(X)$ instead of $xs \colon (X)^*$.
An optional component is enclosed in square bracket, e.g. $[a \colon T]$; to represent an absent component we
use $\bot$.

Production rules will be used only to define program constructs; other objects like automata, events or
transitions will be described as tuples, e.g.
\begin{haskell}
Event \eqdef (type \colon EventType; mname \colon Name).
\end{haskell}
To describe an event we don't use $Event$ as constructor, instead we just give a tuple like $(Entry, m)$,
nevertheless, we still use the dot notation to access its components.
Note that we use $\geqdef$ for production rules and $\eqdef$ for other definitions.

We use parenthesis for function application, e.g. $f(x)$, instead of the style used by some functional
languages that represent function application with a space or a dot. 
Functions defined  will be uncurryfied hence we will define $f \colon X \times Y \to Z$ instead of $f \colon X
\to Y \to Z$.
When the argument is a tuple we omit the outer pairs of parentheses, e.g. we write $f(x,y)$ instead of
$f((x,y))$ when $f \colon X \times Y \to Z$. 

A function can be applied to a set of elements of one of its arguments yielding a set of values, for instance
for a function $f \colon X \to Y$ and subset $S$ of $X$ we have that:
\begin{haskell}
  f(\cdot S \cdot) \eqdef \{y | x \in S \wedge f(x) = y\}.  
\end{haskell}

Finally we remark that the description of the annotation process is a specification and not an implementation
thus there are not issues about performance. Furthermore, an implementation may annotate programs in a
different but equivalent way, though this equivalence must be proved.


\section{Assumptions and Simplifications}\label{sec:Assumptions}
This section describes the general assumptions of the approach.
\subsubsection*{MVAs} 
Imposed by the definition of MVAs:
\begin{itemize}
  \item \<halted\> is not in the set of control points.
  \item All sets are finite.
  \item If many MVAs annotate a program, they have disjoints sets of control points and variables.
  \item Ghost variables added in the process are valid identifiers and that they do not clash with the
identifiers of the program.
  \item Guards are side-effect free.
  \item The evaluation of guards and actions terminates normally.
  \item Expressions appearing in actions do not involve any of the methods being monitored. These expressions
will be translated into the expressions of a set statement and JML requires these expressions not to have
side-effects. We may allow such method calls if we assume that in those cases the set annotations will not be
executed.
\end{itemize}

\subsubsection*{Programs}
\begin{itemize}
  \item The code is available.
  \item Methods involved in the specification are not abstract nor native. (Probably we allow native methods
in inherited classes, although I'm not sure if it is possible to override native methods).
  \item Methods involved in events that are inherited but not overridden must not be final and must be have a
protected or public access modifier.
  \item There is no static overloading of methods. This can be achieved by \emph{mangling} names, that is,
adding information of the types of the arguments to the name of the methods that are overloaded to
disambiguate them.
  \item Variables that can be used by the automata to evaluate guards are not hidden by parameters or local
varibles. In an implementation this assumption can be drop by prefixing \texttt{this.} to the conflictive
variables.
  \item One of the classes has a function called \texttt{main} (with the required signature).
\end{itemize}

\subsubsection*{Annotations}
In this section we do some assumptions and simplifications that will shorten the description of the
annotation process.
We will benefit from this simplification when proving properties of annotations.

%TODO
\begin{quote}
\small
Explain why it is possible to do these assumptions and simplifications and to manage the general case.
\end{quote}

The first simplification is to think that there is only one automaton that annotates a class, thus there is
no need to disambiguate between the variables that keep track of the current control point of each automaton;
we will only need one variable that we call \texttt{cp}.

We also assume that there is a JML conditional instruction that can evaluate ghost variables:
\begin{haskell}
Statement &\geqdef \ldots \gbar IfJML \gbar \ldots\\
IfJML &\geqdef (test: BoolExpr, body: Statement)^*
\end{haskell}

It takes a list of pairs of boolean conditions and statements to model embedded ifs. The statements must
be a sequence of \<Set\> or other \<IfJML\> instructions.
It is possible to do without it using conditional expressions and auxiliary ghost variables.


\section{Simplified Java programs}\label{sec:JavaProgs}
This section describes the abstract syntax for an interesting subset of Java programs.

\subsection{Abstract Syntax}
For sake of abbreviation we define
\begin{haskell}
\Methods \eqdef \powerset(Method)
\end{haskell}

The main simplifications are:
\begin{itemize}
  \item No modeled: interfaces, types, arrays, modifiers, object creation, static data, packages, type casts,
\texttt{instanceof}.
  \item No \texttt{for} loops (just \texttt{while}). No \texttt{break} nor \texttt{continue} instructions.
  \item There is only one kind of exceptions thus it is not specified in the catch clause and in the throw.
instruction.
  \item Methods have just one argument and one return statement at their end.
  \item All methods are public thus the invariant is check for every method call.
\end{itemize}

\renewcommand{\arraystretch}{1.3}
\setlength{\arraycolsep}{.5\arraycolsep}
\begin{displaymath}
\begin{array}[l]{lcl}
Program         & \geqdef & classes \gcolon \powerset(Class)\\
Class           & \geqdef & name \gcolon Name \gsemi superClass \gcolon Class \gsemi inv \gcolon
\powerset(BoolExpr);\\
                &         & gvds \gcolon \powerset(GhostVarDecl) \gcolon fds \gcolon \powerset(FieldDecl)
\gsemi methods \gcolon \Methods\\
GhostVarDecl    & \geqdef & type \colon Type\gsemi name \colon Name\gsemi ival \colon Expr\\
FieldDecl       & \geqdef & type \gcolon Type\gsemi name \gcolon Name\gsemi ival \gcolon Expr\\
Type            & \geqdef & IntType \gbar BoolType \gbar Void \gbar RefType\\
Method          & \geqdef & name \gcolon Name\gsemi argDecls \gcolon (ArgDecl)^*\gsemi pre, post \gcolon
Expr \gsemi\\
                &         & body \gcolon Stmt \gsemi pset, qset, xset \gcolon Stmt \gsemi res \gcolon Expr
\gsemi resType \gcolon Type\\
ArgDecl         & \geqdef & type \gcolon Type\gsemi name \gcolon Name\\
\end{array}
\end{displaymath}

\begin{displaymath}
\begin{array}[l]{lcl}
Stmt   & \geqdef & If\,\, test \gcolon BoolExpr\gsemi then \gcolon Stmt\gsemi else \gcolon
Stmt\\
            & \gbar   & Sequence\,\, s1, s2 \gcolon Stmt\\
            & \gbar   & Set\,\, target \gcolon Name\gsemi source \gcolon Expr\\
            & \gbar   & Skip\\
            & \gbar   & StmtExpr\,\, expr \gcolon Expr\\
            & \gbar   & Throw\\
            & \gbar   & TryCatch\,\, try \gcolon Stmt\gsemi [finally, catch \gcolon Stmt]\\
            & \gbar   & While\,\, test \gcolon BoolExpr\gsemi body \gcolon Stmt\\
\end{array}
\end{displaymath}

\begin{displaymath}
\begin{array}[l]{lcl}
Expr  & \geqdef & Assign\,\, target \gcolon Name\gsemi source \gcolon
Expr\\
            & \gbar   & BExpr\,\, b \gcolon BoolExpr\\
            & \gbar   & CondExpr\,\, test \gcolon BoolExpr\gsemi then,else \gcolon Expr\\
            & \gbar   & MethodCall\,\, tgt \gcolon Expr \gsemi name \gcolon Name\gsemi args\gcolon
(Expr)^*\\
            & \gbar   & NExpr\,\, n \gcolon NumExpr \\ 
            & \gbar   & QAssign\,\, tgt \gcolon Expr \gsemi target \gcolon Name\gsemi source \gcolon
Expr\\
            & \gbar   & VarEval\,\, var \gcolon Name\\              
\end{array}
\end{displaymath}

\begin{displaymath}
\begin{array}[l]{lcl}
BoolExpr        & \geqdef & True \gbar False \gbar Not\,\, b \gcolon BoolExpr\\
                & \gbar   & And\,\, b1, b2 \gcolon BoolExpr \gbar Or\,\, b1, b2 \gcolon BoolExpr\\
                & \gbar   & Lt\,\, b1, b2 \gcolon NumExpr \gbar Le\,\, b1, b2 \gcolon NumExpr\\
                & \gbar   & Eq\,\, b1, b2 \gcolon NumExpr \gbar Ne\,\, b1, b2 \gcolon NumExpr\\
                & \gbar   & Ge\,\, b1, b2 \gcolon NumExpr \gbar Gt\,\, b1, b2 \gcolon NumExpr\\
\end{array}
\end{displaymath}

\begin{displaymath}
\begin{array}[l]{lcl}
NumExpr         & \geqdef & Int\,\, i \gcolon \Integer\\
                & \gbar   & Plus\,\, b1, b2 \gcolon NumExpr \gbar Minus\,\, b1, b2 \gcolon NumExpr\\
                & \gbar   & Times\,\, b1, b2 \gcolon NumExpr \gbar Div\,\, b1, b2 \gcolon NumExpr\\
\end{array}
\end{displaymath}


\subsection{Operational Semantics}
This is an adaptation of von Oheimb's work \cite{Oheimb-Nipkow-Java-LNCS}.

The evaluation rules are given as mutually inductive set of tuples. These sets define functions which will
be writen as shown next.
\begin{list}{--}{}
  \item $\stp{c,\sigma}{\sigma'}$  means that the execution of the $Statement$ $c$ transforms the state
$s$ into $s'$ in the context of the $Program$ P.
  \item $\etp{c,\sigma}{v,\sigma'}$  means that the $Expression$ $e$ evaluates to $v$, transforming
$s$ into $s'$ in the context of the $Program$ P.
\end{list}

\begin{haskell}
\hskwd{datatype} Maybe(a) \eqdef Some(a) | None
\end{haskell}

\begin{haskell}
Excpt &\eqdef&\, Throwable | NullPointer
\end{haskell}

An state of a program is composed by an optional exception and a \texttt{small state} that keeps track of the
values of the fields, ghost variables and local variables. For simplification we assume that ghost variables
are global, local ghost variables could be added easily.
\begin{haskell}
PState \eqdef ex \gcolon Maybe Excpt \times st \gcolon PSt\\
PSt \eqdef (fvs \gcolon Var, gvs \gcolon Var, lvs \gcolon Var)\\
\end{haskell}

Exceptions propagate when a series of statements or a series of expressions is evaluated:
$$\stp{c,(Some(x), \sigma)}{(Some(x), \sigma)}$$
$$\etp{e,(Some(x), \sigma)}{\bot, (Some(x), \sigma)}$$

All the other rules can assume that in their concerning initial state no exception has been thrown. For such
states, we define the abbreviation $Norm(\sigma)$, which stands for $(None, \sigma)$.

\subsection*{Statements}
The rules for statements not explicitly involving exceptions are the standards ones in big-step semantics:

%THINK: It may be better to make the exception explicit
\begin{displaymath}
\begin{array}{lc}
\br{Skip_P}     & \B\stp{ Skip, Norm(\sigma)}{Norm(\sigma)} \\

\br{Seq_P}      & 
\begin{array}{c}
\stp{c_1, Norm(\sigma_0)}{\sigma_1} \quad \stp{c_2, \sigma_1}{\sigma_2}\\
\hline
\B\stp{Seq(c_1, c_2), Norm(\sigma_0)}{\sigma_2}
\end{array}\\

\br{StmtExpr_P} &
\begin{array}{c}
\etp{e, Norm(\sigma_0)}{v,\sigma_1}\\
\hline
\B\stp{StmtExpr(e), Norm(\sigma_0)}{\sigma_1}
\end{array}\\

\br{If_P} &
\begin{array}{c}
\etp{e, Norm(\sigma_0)}{v,\sigma_1}\\
\stp{\pif{the\_Bool(v)}{c_1}{c_2}, \sigma_1}{\sigma_2}\\
\hline
\B \stp{If(e,c_1,c_2), Norm(\sigma_0)}{\sigma_2}
\end{array}\\

\br{While_P} &
\begin{array}{c}
\stp{If(e, While(e,c), Skip), \sigma_1}{\sigma_1}\\
\hline
\stp{While(e,c), Norm(\sigma_0)}{\sigma_2}
\end{array}\\

\end{array}
\end{displaymath}

Since it is not our intention to give a realistic model of Java programs we have simplified the treatment of
exceptions. We assume that \texttt{throw} takes no argument and always raises an exception of type
\texttt{Throwable}.

\begin{displaymath}
\begin{array}{lc}
\br{Throw_P}     &
\stp{Throw, Norm(\sigma_0)}{(Throwable, \sigma_0)}
\end{array}
\end{displaymath}

In a \texttt{try} $c_1$ \texttt{catch} $c_2$ \texttt{finally} $c_3$ statement we do not distinguish between
types of exceptions. For a more complete treatment of exceptions see~\cite{Oheimb-Nipkow-Java-LNCS}. If
the execution of $c_1$ throws an exception $c_2$ is evaluated, otherwise it is skipped; then $c_3$ is
evaluated even if $c_2$ raised an exception. If both $c_2$ and $c_3$ threw an exceptions then $c_2$ will be
given precedence.

% \renewcommand{\arraystretch}{1.3}
\begin{displaymath}
\begin{array}{lc}
\br{TryCatch_P}     &  
\begin{array}{c}
\stp{c_1, Norm(\sigma_0)}{(x_1,\sigma_1)}\\
% (x_2, \sigma_2) = \begin{cases}
% Norm(\sigma_1)& \text{if } x_1 = None\\
% \Delta_P(c_2, Norm(\sigma_1))& \text{otherwise}
% \end{cases}\\
\stp{\textbf{if}\, x_1=None\, \textbf{then}\, Skip\, \textbf{else}\, c_2, Norm(\sigma_0)}{(x_2, \sigma_2)}\\
\stp{c_3, Norm(\sigma_2)}{(x_3, \sigma_3)}\\
x_4 = \begin{cases}
x_2& \text{if } x_2 \neq None\\
x_3& \text{otherwise}
\end{cases}\\
\hline
\stp{TryCatch(c_1, c_2, c_3), Norm(\sigma_0)}{(x_4, \sigma_3)}
\end{array}
\end{array}
\end{displaymath}

\begin{displaymath}
\begin{array}{lc}
\br{LU_P}     &
\begin{array}{c}
is\_ghost(vn, Norm(\sigma_0))\\
\etp{e, Norm(\sigma_0)}{v, (x_1, \sigma_1)}\\
\sigma_2 = \pif{x = None}{\sigma_1.gvs[vn \mapsto v]}{\sigma_1}\\
\hline
\etp{Set(vn, e), Norm(\sigma_0)}{v, (x_1, \sigma_2)}
\end{array}
\end{array}
\end{displaymath}

%TODO: Revise the syntax and define or change the function $the\_Bool$.

\subsection*{Expressions}
We assume that the sets of fields, ghost variables and local variable names are disjoint. The following
functions, of type $Identifier \times Program \to Bool$ are use to distinguish between the kind of variables.

\begin{haskell}
is\_local(vn, \sigma) &\eqdef\,& vn \in \dom(\sigma.lvs)\\
is\_field(vn, \sigma) &\eqdef& vn \in \dom(\sigma.fvs)\\
is\_ghost(vn, \sigma) &\eqdef& vn \in \dom(\sigma.gvs)
\end{haskell}

%NOTE: To be more precise, this variables should take as argument the class but I don't do that because
% I am flattering the namespace. This simplification will also bring problems in the semantics of qualified
% constructs where the target expression is evaluated just to produce its side-effect.

An access to a local variable (or a parameter or the \texttt{this} pointer) reads from the local state
component:
\begin{displaymath}
\begin{array}{lc}
\br{LA_P}     &
\begin{array}{c}
is\_local(vn, Norm(\sigma))\\
\hline
\etp{VarEval(vn), Norm(\sigma)}{\sigma.lvs(vn), Norm(\sigma)}
\end{array}
\end{array}
\end{displaymath}

An assignment to a local variables updates the state, unless the evaluation of the subexpression raises an
exception:
\begin{displaymath}
\begin{array}{lc}
\br{LU_P}     &
\begin{array}{c}
is\_local(vn, Norm(\sigma_0))\\
\etp{e, Norm(\sigma_0)}{v, (x_1, \sigma_1)}\\
\sigma_2 = \pif{x = None}{\sigma_1.lvs[vn \mapsto v]}{\sigma_1}\\
\hline
\etp{LAssign(vn, e), Norm(\sigma_0)}{v, (x_1, \sigma_2)}
\end{array}
\end{array}
\end{displaymath}
%TODO: Assign takes three args. Diff between local assign (LAssign) and qualified assignments (QAssign)


\begin{displaymath}
\begin{array}{lc}
\br{QA_P}     &
\begin{array}{c}
is\_field(vn, Norm(\sigma_0))\\
\etp{e, Norm(\sigma_0)}{v, (x_1, \sigma_1)}\\
x_2 = \pif{x_1 = None \wedge v = Null}{NullPointer}{x_1}\\
\hline
\etp{QAccess(e, vn), Norm(\sigma_0)}{\sigma_1.fvs(vn), (x_2, \sigma_1)}
\end{array}
\end{array}
\end{displaymath}
%TODO: This is a too simplistic view of the heap!

\begin{displaymath}
\begin{array}{lc}
\br{QU_P}     &
\begin{array}{c}
is\_field(vn, Norm(\sigma_0))\\
\etp{e_1, Norm(\sigma_0)}{v_1, (x_1, \sigma_1)}\\
x_2 = \pif{x_1 = None \wedge v_1 = Null}{NullPointer}{x_1}\\
\etp{e_2,\sigma_1}{v_2, (x_3, \sigma_2)}\\
\sigma_3 = \pif{x_2 = None}{\sigma_2.fvs[vn \mapsto v_2]}{\sigma_2}\\
\hline
\etp{QAssign(e_1, vn, e_2), Norm(\sigma_0)}{v_2, (x_3, \sigma_3)}
\end{array}
\end{array}
\end{displaymath}


%TODO: Define init_vars and this

\begin{displaymath}
\begin{array}{lc}
\br{MC_P}     &
\begin{array}{c}
\etp{e,Norm(\sigma_0)}{a,(x_1,\sigma_1)}\\
\etp{p, (x_1, \sigma_1)}{pv, (x_2, \sigma_2)}\\
x_3 = \pif{x_2 \neq None \wedge a = Null}{NullPointer}{x_2}\\
md = lookup\_mthd(P, a, mn, p)\quad inv = lookup\_inv(P, a)\\
\sigma_3 = \sigma_2(lvs:=init\_vars(md.lvars[this \mapsto a][md.pn \mapsto pv]))\\
P, \sigma_3 \vDash inv\quad P, \sigma_3 \vDash md.pre\quad \stp{md.pset, (x_3, \sigma_3)}{(x_4, \sigma_4)}\\
\stp{md.body, (x_4, \sigma_4)}{(x_5, \sigma_5)}\\
\etp{md.res, (x_5, \sigma_5)}{v,(x_6, \sigma_6))}\\
\stp{\pif{x_6 = None}{md.qset}{md.xset}, (x_6, \sigma_6)}{(x_7, \sigma_7)}\\
P, \sigma_7 \vDash inv\quad P, \sigma_7 \vDash md.post\\
\hline
\etp{MethodCall(e,mn,p), Norm(\sigma_0)}{v, (x_7, \sigma_7(lvs:=\sigma_2.lvs)}\\
\end{array}
\end{array}
\end{displaymath}

Recall that we assumed that there is no clash between local variables and class variables, otherwise we
should evaluate the invariant in a state with no local variables like $\sigma_7(lvs:=\emptyset)$ so that the
value of the variables are taken from $fvs$ or $gvs$.

\medskip
The following theorems enunciate that the evaluation of expressions and the execution of statements are
deterministic. They can be proved by the \emph{special rule principle of induction} (see~\cite{Winskel}).
\begin{theorem}
If $\etp{e,\sigma_0}{v,\sigma_1}$ and $\etp{e,\sigma_0}{v,\sigma_2}$ then $\sigma_1 = \sigma_2$.
\end{theorem}

\begin{theorem}
If  $\stp{c,\sigma_0}{\sigma_1}$ and $\stp{c,\sigma_0}{\sigma_2}$ then $\sigma_1 = \sigma_2$.
\end{theorem}

\medskip
The execution of a program starts with a call to the entry method \texttt{main} with some argument, and is
evaluated in a state where fields and ghost variables are mapped to their initial values.

\begin{definition}\label{def:PE}
The execution of a program $P$ with input $e$ is defined as
$$P(e) \Rightarrow_{PE} (x,\sigma) \eqdef \stp{MethodCall(this, main, e), Norm(\sigma_0)}{(x, \sigma)}$$
where $\sigma_0$ is such that
\begin{haskell}
\sigma_0.fvs &=\,& \{fd.name \mapsto fd.ival \,|\, fd \in c.fds \wedge c \in P.classes\}\\
\sigma_0.gvs &=& \{gvd.name \mapsto gvd.ival \,|\, gvd \in c.gvds \wedge c \in P.classes\}\\
\sigma_0.lvs &=& \{this \mapsto v\}
\end{haskell}
\end{definition}

%NOTE: To be more precise I should evaluate fs.ival and gvd.ival (they are expressions). The problem is to say
% in which state are evaluated these expressions. A way out would be to require them (in the syntax) to be
% literals

The value $v$ represents the initial object; in a more realistic semantics such object would be created.
%NOTE: If I model the heap and object creation, as David does, I will have problems when defining the
% equivalence relation since I will have to define a heap equivalence (which should be independent of the
% actual addresses) and then I will have to make proofs with that definition.


\section{MVA}

\subsection{Definition}

An MVA is defined as the following tuple
\begin{haskell}\label{def:MVA}
MVA &\eqdef\,& (name,className \dcolon Name; cps \dcolon \powerset(CP); icp \dcolon CP; evs \dcolon
\powerset(Event);\\
  && vdsA \dcolon \powerset(VarDeclA); vdsP \dcolon \powerset(VarDeclP); ts \dcolon \powerset(Transition))
\end{haskell}
where
\begin{itemize}
  \item \<name\> is the name of the automaton. It is needed when working with many automatas that annotate a
class but in this work we will restrict ourself to just one automaton thus its role will be downplayed.
  \item \<className\> is the name of the class the automaton will monitor.
  \item \<cps\> is a the set of control points. We assume that \<halted\> $\notin$ \<cps\>.
  \item \<icp\> is the initial control point. It must be a member of \<cps\>.
  \item \<evs\> is a set of events.
  \item \<vdsA\> is a set of variable declarations for the automaton and \<vdsP\> is a set of variable
declarations for variables of the class that the automaton may use in guards.
  \item \<ts\> is a set of transitions.
\end{itemize}

A variable of the automaton is described by an identifier, a type and its initial value.
\begin{haskell}
VarDeclA \eqdef (name \dcolon Name, type \dcolon Type, ival \dcolon Expression)
\end{haskell}

For a variable of the program we only require its name and its type.
\begin{haskell}
VarDeclP \eqdef (name \dcolon Name, type \dcolon Type)
\end{haskell}

Let $Val$ be the set of possible variable values and $Var$ a mapping from identifiers to values, that is,
\begin{haskell}
Var \eqdef Name \to Val
\end{haskell}

A guard is a boolean expression that includes variables from the automaton and from the program.
\begin{haskell}
Guard \eqdef Var \times Var \to Bool
\end{haskell}

An action is a function of type $Var \to Var$. Concretly, an action is a list of assignments to variables
in $vdsA.name$.
\begin{haskell}
Action \eqdef (source \colon Name; target \colon Expression)^*\\
\end{haskell}

\<skip\>, the empty action, is just an empty list. Do not confuse \<skip\> (an \<Action\>) with \<Skip\> (a
\<Statement\>).
An event is composed by an event type and a method name.
\begin{haskell}
Event \eqdef (type \colon EventType; mname \colon Name)
\end{haskell}

There are three flavours of events: entry, exit normal and exit exceptional.
\begin{haskell}
EventType \eqdef Entry | ExitNormal | ExitExcept
\end{haskell}

The set of transitions is described as a table with entries for the source control point, event, guard, action
and target control point.
\begin{haskell}
Transition \eqdef (scp \dcolon CP; event \dcolon Event; guard \dcolon Guard; action \dcolon Action; tcp
\dcolon CP)
\end{haskell}

To make the specification more consice we define
\begin{haskell}
  \Transitions \eqdef \powerset(Transition)
\end{haskell}

We say that an automaton is \emph{deterministic} if for every source control point and event there is just one
guard that holds, in any state of the variables. Formally,
\begin{haskell}\label{def:MVADeterm}
determ(a) \eqdef \forall q \in a.cps; e \in a.evs; va \in VarA; vp \in VarP \cdot\\
 \quad\quad (\exists p \in Guard; t \in a.ts \cdot t.scp = q \wedge t.event = e \wedge t.guard = p \wedge\\
 \quad\quad p(va,vp)) \Longrightarrow (\forall t' \in ts \cdot t.scp = q \wedge t'.event = e \wedge\\
 \quad\quad t.guard \neq p \Longrightarrow \neg t'.guard(va, vp))\\
\end{haskell}

Note that the guard may not exist, but if it does then it is the only one that holds for the given values of
the variables in the transitions that leave from $q$ when $e$ occurs.

We say that an MVA is \emph{total} if there is always one and only one guard that holds for each set
transitions with a common source control point and event.
\begin{haskell}\label{def:MVATotal}
total(a) \eqdef determ(a) \wedge \forall q \in a.cps; e \in a.evs; va \in VarA; vp \in aVarP \cdot\\
 \quad\quad \exists p \in Guard; t \in a.ts \cdot t.scp = q \wedge t.event = e \wedge t.guard = p \wedge\\
 \quad\quad p(va, vp).
\end{haskell}

\subsection{Operational Semantics} 
In this subsection we give the operational semantics for MVAs. We restrict our atention to total automata.
The state of an MVA is composed by its current control point and values for its variables.
\begin{haskell}
AState \eqdef (cp \dcolon CP; vars \dcolon VarA)
\end{haskell}

Notice that the value of the variables of the program are not part the automaton state. They will be provided
as paramenters to transitions, together with an event in a $PEvent$.

\begin{haskell}
PEvent \eqdef Var_P \times Event
\end{haskell}

A single judgment for an MVA $a$ is given by the function $\Delta_a$ which for a given state of $a$ and a
\<PEvent\> it updates the current control point and the variables of the automaton according the transition
that applies (recall that we assume that $a$ is total).
\begin{haskell}
\Delta_a &\ofType& AState \times PEvent \to AState\\
\Delta_a((q,va), (vp,e)) &\eqdef& (t.tcp, t.action(va))
  \hswhere{t \in a.ts \wedge t.scp = q \wedge t.event = e \wedge t.guard(va, vp)}
\end{haskell}

We write $s \stackrel{pe}{\Rightarrow} s'$  when $\Delta(s, pe) = s'$.
% We generalize the single-step judgment to (finite) multi-step judgment using standard rules of reflexivity
% and transitivity. The multi-step partial function $\Delta^*_a$ is recursively defined as
% \begin{haskell}
% \Delta^*_a &\ofType& AState \times [PEvent] \to AState\\
% \Delta^*_a(s, \emptylist) &\eqdef& s\\
% \Delta^*_a(s, w:a) &\eqdef& \Delta_a(\Delta^*_a(s, w), a)
% \end{haskell}
% and we write $s \stackrel{pe}{\Rightarrow_a^*} s'$  when $\Delta_a^*(s, pe) = s'$.


\section{Monitored Programs}
\subsection{Definition}
A monitored program is a program together with an MVA that monitors its behaviour.
\begin{haskell}
MonitoredProgram \eqdef Program \times MVA
\end{haskell}

The following definition states when a MVA $A$ can monitor a program $P$.
\begin{haskell}
\sqsubseteq &\ofType& MVA \times Program \to Bool\\
A \modeled P &\eqdef&
 \hsalign{
   \exists c \in P.classes \cdot\, c.name = A.className \wedge (\forall v \in A.vdsA,\\
    \exists pv \in c.gvs \cdot\, v.name = pv.name \wedge v.type = pv.type) \wedge\\
   (\forall v \in A.vdsP, \exists pv \in c.fields \cdot\, v.name = pv.name \wedge\\
   v.type = pv.type)
  }
\end{haskell}

In this article we will work with a MVA $A$ and a program $P$ such that $A \modeled P$. 

\subsection{Operational Semantics}
\subsubsection*{Statements}
% The rules for the evaluation of statements are similar to that of annotated programs since their premises
% don't have into account the state of the automaton. For example,

\begin{displaymath}
\begin{array}{lc}
\br{If_{MP}} &
\begin{array}{c}
\etmp{e, \alpha_0 \monitor Norm(\sigma_0)}{v,\alpha_1 \monitor \sigma_1}\\
\stmp{\pif{the\_Bool(v)}{c_1}{c_2}, \alpha_1 \monitor \sigma_1}{\alpha_2 \monitor \sigma_2}\\
\hline
\B \stmp{If(e,c_1,c_2), \alpha_0 \monitor Norm(\sigma_0)}{\alpha_2 \monitor \sigma_2}
\end{array}\\
\end{array}
\end{displaymath}


\subsubsection*{Expressions}
\begin{displaymath}
% \begin{array}{lc}
% \br{MC_{MP}} &
\begin{array}{c}
\etmp{e, \alpha_0 \monitor Norm(\sigma_0)}{a, \alpha_1 \monitor (x_1, \sigma_1)}\\

\etmp{p, \alpha_1 \monitor (x_1, \sigma_1)}{pv, \alpha_2 \monitor (x_2, \sigma_2)}\\

x_3 = \pif{x_2 \neq None \wedge a = Null}{NullPointer}{x_2}\\

md = lookup\_mthd(P, a, mn, p)\quad inv = lookup\_inv(P, a)\\

\sigma_3 = \sigma_2(lvs:=init\_vars(md.lvars[this \mapsto a][md.pn \mapsto pv]))\\

P, \sigma_3 \vDash inv \quad \alpha_0.cp \neq halted \quad P, \sigma_3 \vDash md.pre\\

\alpha_3 = \begin{cases}
\Delta_A(\alpha_2, \sigma_3.fvs, (mn, Entry))& \text{if } (m, Entry) \in A.evs\\
\alpha_2& \text{otherwise}
\end{cases}\\

\stmp{md.body, \alpha_3 \monitor (x_3, \sigma_3)}{\alpha_4 \monitor (x_4, \sigma_4)}\\

\etmp{md.res, \alpha_4 \monitor (x_4, \sigma_4)}{v, \alpha_5 \monitor (x_5, \sigma_5))}\\

\alpha_6 = \begin{cases}
\Delta_A(\alpha_5, \sigma_5.fvs, (mn, ExitNormal))\!\!\!& \text{if } (mn, ExitNormal) \in A.evs \wedge x_5 =
None\\
\Delta_A(\alpha_5, \sigma_5.fvs, (mn, ExitExcept))& \text{if } (mn, ExitExcept) \in A.evs \wedge x_5 \neq
None\\
\alpha_5& \text{otherwise}
\end{cases}\\

P, \sigma_5 \vDash md.post\quad P, \sigma_5 \vDash inv\quad \alpha_6.cp \neq halted\\

\hline
\etmp{\T MethodCall(e,mn,p), \alpha_0 \monitor Norm(\sigma_0)}{v, \alpha_6 \monitor (x_5,
(\sigma_5(lvs:=\sigma_2.lvs)))}\\
\end{array}\\
% \end{array}
\end{displaymath}

\bigskip
The execution of a program $P$ monitored by an automaton $A$ executes $P$ begining in the state $\alpha_0
\monitor \sigma_0$ where $\alpha_0$ and $\sigma_0$ are the initial states of $A$ and $P$, respectively.

\begin{definition}\label{def:MPE}
The execution of a program $P$ with input $e$, monitored by a MVA $A$ is defined as
$$A \monitor P(e) \Rightarrow_{MPE} \alpha \monitor (x,\sigma) \eqdef \stp{MethodCall(this, main, e),
\alpha_0 \monitor Norm(\sigma_0)}{\alpha \monitor (x, \sigma)}$$
where $\sigma_0$ is such that
\begin{haskell}
\alpha_0.cp &=\,\,& A.icp\\
\alpha_0.vars &=\,& \{v.name \mapsto v.ival \,|\, v \in A.vdsA\}\\
\sigma_0.fvs &=& \{field.name \mapsto field.ival \,|\, field \in c.fields \wedge c \in P.classes\}\\
\sigma_0.gvs &=& \{gv.name \mapsto gv.ival \,|\, gv \in c.gvs \wedge c \in P.classes\}\\
\sigma_0.lvs &=& \{this \mapsto v\}
\end{haskell}
\end{definition}



\section{Annotation Generation}\label{sec:AnnGen}
This section describes the annotation generation process. 
The description is top-down; subsections describe different phases of the procedure.


\subsection{Completing an MVA}\label{sec:CompMVA}
The first step in the annotation process is to make transition function of the automata a total function (see
section \ref{def:MVADeterm}).

Given an MVA \<a\>, \<completeMVA(a)\> is a new MVA that is like \<a\> but its set of control points is
augmented with \<halted\> and its transition function is made total. Hence in this automaton there is no stuck
configuration ---the automaton not being able to take any step further---, but it may reach a trap control
point.

\begin{haskell}\label{def:completeMVA}
completeMVA &\ofType& MVA \to MVA\\
completeMVA(a) &\eqdef&
    \hsalign{
      a(&cps := a.cps \cup \{halted\},\\
        &ts := a.ts \cup completeTrans(cps, a.evs, a.ts))
    }
\end{haskell}

To complete a set of transitions so that it becomes a total function with respect to a set of control points
$cps$ and a set of events $es$ we choose a control point $cp \in cps$ and an event $e \in es$, and we add the
transition \<(cp, p, e, skip, halted)\> where $p$ covers any case not covered by the guards of the transitions
having $cp$ as source control point and $e$ as event.
We also add an unconditional transition for any event in $es$ from the halted state to itself.

\begin{haskell}\label{def:completeTrans}
completeTrans &\ofType& \powerset(CP) \sstimes \powerset(Event) \sstimes \Transitions \ssto \Transitions\\
completeTrans(cps, es, ts) &\eqdef&\relax
  \hsbody{
    \{(cp, e, p, skip, halted) | cp \in cps \wedge e \in es \wedge \\
    \quad\quad p = Not(disjunct(\{t.guard | t \in ts \wedge t.scp = cp \wedge t.event = e\}))\} \cup\\
    \{(halted, e, True, skip, halted) | e \in es\}
  }
\end{haskell}

The function \<disjunct\> returns the disjunction of the guards given as input. It may be tempting to define
it simply as $\bigvee_{i=1}^n p_i$, but recall that here a guard is a program construct, not a boolean
function.

\begin{haskell}\label{def:disjunct}
disjunct(\emptyset) &\eqdef\,& True\\
disjunct(\{p\} \sscup ps) &\eqdef& Or(p, disjunct(ps))
\end{haskell}

The definition is by cases on the structure of the set. For the empty set it returns $True$ and for a
non-empty set the definition is recursive. The well-foundness of the function is justified by the assumption
that the given set is finite. The order in which elements are taken from the set is irrelevant since $Or$ is
conmutative when both arguments are defined, which is also an assumption.

From now on we assume that automaton we work with has been completed with \<completeMVA\>.

\subsection{Annotating programs}\label{sec:AnnPrograms}
The function \<annProgram\> is in charge of annotating programs; it is the top level function. It receives a
monitored program and returns a program where the automaton has been encoded as JML annotations.

\begin{haskell}\label{def:annProgram}
annProgram &\ofType& MonitoredProgram \to Program\\
annProgram(cs, a) &\eqdef& \{Program(annClass(c, a)) |\, c \in cs\}
\end{haskell}

Given a (possibly already annotated) class \<c\> and an automaton \<a\>, \<annClass\> returns \<c\> with
additional annotations and some simple code transformations that encode \<a\>.

\begin{haskell}\label{def:annClass}
annClass &\ofType& Class \times MVA \to Class\\
annClass(c, a) &\eqdef&\relax
  \hsbody{
    \hsif{c.name = a.className}{
      \hsalign{
        c(&inv := c.inv \cup \{Ne(VarEval(\texttt{"cp"}), VarEval(halted))\}\\
          &gvs := c.gvs \cup newGVDs(a),\\
          &methods := annMethods(methods', a))
      }
    }{
      c}\\
  \hskwd{where}\\
    \quad\quad methods' = completeMethods(c.methods, c.superClass, a.evs.mname)
  }
\end{haskell}

\<annClass\> returns a class that is like \<c\> but with some additions to its invariants, ghost variables and
methods. 
An invariant stating that the current control point is not \<halted\> is added to the resulting class.
Declarations of ghost variables are enlarged with \<newGVDs(a)\> (see page~\pageref{def:newGVDs}) and methods
are first completed with \<completeMethods\> (see next definition) and then annotated with the transitions of
\<a\> (see page~\pageref{def:annMethods}).


\subsection{Completing methods}\label{sec:CompMethods}
An automaton may have within its events a call to a method $m$ that is not defined in the class $c$ that it
is monitoring. In that case $m$ will be redefined with just a call to its implementation in a superclass of
$c$. This redefinition is needed to have a body where the ghost variables could be modified.

\<completeMethods\> receives a list of methods $ms$, their superclass \<c\>, and a list of the names of the
methods that have to be annotated, $ns$, and returns $ms$ plus redefinitions of methods whose names are in
$ns$ but not in $ms$; \<c\> is needed to look for inherited methods.

\begin{haskell}\label{def:completeMethods}
completeMethods &\ofType& \Methods \sstimes Class \sstimes \powerset(Name) \ssto \Methods\\
completeMethods(ms, c, ns) &\eqdef&\relax
  \hsbody{
%     ms \cup \{\,overrideMethod(n, c) |\, n \in ns \wedge n \notin ms.name\,\}
    ms \cup \{\,overrideMethod(n, c) |\, n \in ns \setminus ms.name\,\}
  }
\end{haskell}

The function \<overrideMethod\> takes the name of a method $n$ and its superclass $c$ and returns a
redefinition of that method with just a call to the implementation in its ancestor.
It is assumed that $n$ is defined along the hierarchy of $c$.

\begin{haskell}\label{def:overrideMethod}
overrideMethod &\ofType& Name \times Class \to Method\\
overrideMethod(n, c) &\eqdef& m(body := body')
  \hswhere{
    m & \eqdef lookupMethod(n,c)\\
    body' &\eqdef \hsifone{m.resultType = Void}{StmtExpr(mc)}{Return(mc)}\\
    mc &\eqdef MethodCall(Super, m.name, VarEval(\cdot m.argDecls.name \cdot))
  }
\end{haskell}

Since the returned method has the same signature as $m$, $m$ will be overridden.
If the return type of $m$ is $Void$, the call is treated as an statement, otherwise it is used as the return
value of the method.
For the method call each formal argument has been turned into an expression.

The function \<lookupMethod\> takes a name of method and a class and returns the first method with that name
in the hierarchy of $c$ (going upwards from $c$) or $\bot$ if there is no such method. It is not further
specified.


\subsection{Generating ghost variables}\label{sec:GenGVs}
This subsection describes \<newGVDs\> which that generates the declarations of the ghost variables that encode
an automaton. Its definition is the following.

\begin{haskell}\label{def:newGVDs}
newGVDs &\ofType& MVA \to \powerset(GhostVarDecl)\\
newGVDs(a) &\eqdef&
  \hsbody{
    \{\,GhostVarDecl(IntType, \texttt{"cp"}, VarEval(a.icp)\,\} \,\cup\\
    \{\,GhostVarDecl(IntType, q, NExpr(Int(unique(q)))) \,|\, q \in a.cps\,\} \,\cup\\
    \{\,GhostVarDecl(v.type, v.name, v.ival) \,|\, v \in a.vdsA\,\}
  }
\end{haskell}

The result is the union of declarations that encode the current control point, possible values for it and 
variables of the automanton. 
The function \<unique\> is an injective function from $CP$ to $\Integer$.

It is assumed that the name of the control points and variables of the automaton do not clash with other
identifiers of the program.


\subsection{Annotating methods}\label{sec:AnnMethods}
Here we describe the core of the annotation process, that is, the annotation of the individual methods.

The function \<annMethods\> annotates the given methods with the given automata.
\begin{haskell}\label{def:annMethods}
annMethods &\ofType& \Methods \times MVA \to \Methods\\
annMethods(ms, a) &\eqdef& \{annMethod(m, a) | m \in ms\}\\
\end{haskell}

\<annMethod\> adds the annotations of the methods at the right places.
\begin{haskell}\label{def:annMethod}
annMethod &\ofType& Method \times MVA \to Method\\
annMethod(m, a) &\eqdef& 
  \hsalign{
    m(&pset:=annEvent(filterEv(Entry)),\\
      &qset:=annEvent(filterEv(ExitNormal)),\\
      &xset:=annEvent(filterEv(ExitNormal)))
  }
  \hswhere{
    filterEv(type) &\eqdef& \{\,t | t \in a.ts \wedge t.event = (type, m.name)\,\}
  }
\end{haskell}


\subsection{Annotating Events}\label{sec:AnnEvents}
Given a set of control points $\{q_i\}_{i=1}^n$, an event $e$ and a set of transitions $\{(q_1, p_{11},e,$
$a_{11}, q_{11}),$ $(q_1, p_{12}, e, a_{12}, q_{12}),$ $\ldots,$ $(q_n, p_{n1}, e, a_{n1},
q_{n1}),$ $\ldots,$ $(q_n, p_{nk}, e, a_{nk}, q_{nk})\}$ (note that all transitions have $e$ as event), the
idea is to encode the triggering of $e$ with annotations similar to the following.

%   @     \textit{perform action \(a\sb{11}\)};
\begin{alltt}
/*@
  @ if (cp == \(q\sb{1}\)) \{
  @   if (\(p\sb{11}\)) \{
  @     set cp = \(q\sb{11}\);
  @     \(a\sb{11}\);
  @   \} else if (\(p\sb{12}\)) \{
  @     set cp = \(q\sb{12}\);
  @     \(a\sb{12}\);
  @   \}
  @   \ldots
  @ \ldots
  @ \} else if (cp == \(q\sb{n}\)) \{
  @   if (\(p\sb{n1}\)) \{
  @     set cp = \(q\sb{n1}\);
  @     \(a\sb{n1}\);
  @   \ldots
  @   \} else if (\(p\sb{nk}\)) \{
  @     set cp = \(q\sb{nk}\);
  @     \(a\sb{nk}\);
  @   \}
  @ \}
  @*/
\end{alltt}

\<annEvent\> generates the annotations to update the control point and the variables for an event,
independently of its type. It receives the transitions that involve the event that is being annotated.

\begin{haskell}\label{def:annEventSimp}
annEvent &\ofType& \Delta \to Statement\\
annEvent(ts) &\eqdef& IfJML(branches)
   \hswhere{
     branches &\eqdef& [annTransCP(cp, \{t | t \in ts \wedge t.scp = cp\}) | cp \in ts.scp]
   }
% annEvent(ts) &\eqdef& IfJML([annTransCP(cp, filterCP(cp, ts)) | cp \in ts.scp])
\end{haskell}

Each element of \<branches\> will produce annotations for the transitions that have the same event and the
same source control point.
Note that the set of control points is taken from the the transitions; in general it may be a proper subset of
all the control points of the automaton but having completed the transitions we are sure that they are the
same. It is obvious that \texttt{cp} can only have values within the set of control points thus at least one
of the branches will be selected.
The order in which the elements of the list \<branches\> are generated from the set \<ts.scp\> is
irrelevant because the conditions are mutually exclusive.

\bigskip
\<annTransCP\> takes a control point \<cp\> and a set of transitions \<ts\> and returns a pair containing a
boolean expression that checks if the current control point is \<cp\> and a statement that will choose one of
the guards from \<ts\> and take the corresponding action.
\begin{haskell}\label{def:annTransCP}
annTransCP &\ofType& CP \to \Delta \to (BoolExpr, Statement)\\
annTransCP(q, ts) &\eqdef& (cond, stmt)
  \hswhere{
    cond &\eqdef\,& Eq(VarEval(\texttt{"cp"}), VarEval(q))\\
    stmt &\eqdef& IfJML(annGuardsActions(ts))
  }
\end{haskell}

\bigskip
\<annGuardsActions\> receives a set of transitions that have the same source control point and event and
returns a list of boolean expressions obtained from their guards and statements representing their execution.
\begin{haskell}\label{def:annGuardsActions}
annGuardsActions &\ofType& \Delta \to [(BoolExpr, Statement)]\\
annGuardsActions(ts) &\eqdef& [(t.guard, updVars(t)) | t \in ts]
\end{haskell}
Again, the order in which the list is generated from the set of transitions is not important because having
completed the set transitions we can assure that there is always one and only one guard that holds for each
pair of control point and event, in any state of the variables.
%TODO: put a link to the definition of functions for transitions.

\bigskip
\<updVars\> updates the control point and the variables of the automaton according to the transition given as
argument.
\begin{haskell}\label{def:updVars}
updVars &\ofType& Transition \to Statement\\
updVars(t) &\eqdef& Sequence(Set(\texttt{"cp"}), VarEval(t.tcp), annAction(t.action))
\end{haskell}

\bigskip
Recall that an action is a list of pairs (\<source: Name, target: Expression\>). The function \<annAction\>
simply translates an action into a sequence of set instructions.
\begin{haskell}\label{def:annAction}
annAction &\ofType& Action \to Statement\\
annAction(\emptylist) &\eqdef& Skip\\
annAction(a:as) &\eqdef& Sequence(Set(a.source, a.target), annAction(as))
\end{haskell}

Notice that the action \<skip\> is translated to the program construct \<Skip\>.

\subsection{Refining the specification}

\begin{haskell}
newGVDs &\ofType& MVA \to \powerset(GhostVarDecl)\\
newGVDs(a) &\eqdef&
  \hsbody{
    \{\,GhostVarDecl(IntType, \texttt{"cp"}, VarEval(a.icp)\,\} \,\cup\\
    \{\,GhostVarDecl(BoolType, \texttt{"ex"}, BExpr(False)\,\} \,\cup\\
    \{\,GhostVarDecl(IntType, q, NExpr(Int(unique(q)))) \,|\, q \in a.cps\,\} \,\cup\\
    \{\,GhostVarDecl(v.type, v.name, v.ival) \,|\, v \in a.vdsA\,\}
  }
\end{haskell}

The boolean variable \texttt{ex} will be used as a flag to mark an exceptional termination.

\<annMethod\> adds the annotations of the methods at the right places.
\begin{haskell}
annMethod &\ofType& Method \times MVA \to Method\\
annMethod(m, a) &\eqdef& m(body := TryCatch(tryBlock, catchBlock, finallyBlock))
  \hswhere{
    tryBlock &\eqdef\,& annEvent(filterEv(Entry)) ; body\\
    catchBlock &\eqdef& \hsiftwo{filterEv(ExitExecpt) = \emptyset}{Throw}{
      \hsalign{
        &annEvent(filterEv(ExitExecpt)) ;\\
        &Set(\texttt{"ex"}, True) ; Throw}}\\
%TODO: This is not accurate because VarEval is an Expression not a BoolExpr
    finallyBlock &\eqdef& IfJML [(Not(VarEval(\texttt{"ex"})), annEvent(\\
                 && filterEv(ExitNormal))), (True, Skip)] ; Set(\texttt{"ex"}, False)\\
    filterEv(type) &\eqdef& \{\,t | t \in a.ts \wedge t.event = (type, m.name)\,\}
  }
\end{haskell}
To be able to differentiate between normal and exceptional exits, a try-catch-finally block is put around the
body of each annotated method.
The annotations for entry events are placed at the entrance of the method, exit exceptional events are
annotated in the catch close and exit normal events are annotated in the finally clause. The catch block also
flags the exceptional termination setting the variable \texttt{ex} and re-throws the exception to avoid
disturbing the semantics of the method.
The finally block executes annotations for exit normal events only if the method did not throw an exception.
Then it resets the variable \texttt{ex}.

The set of transitions passed to \<annEvent\> is filter according to the type of event what it is being
annotated in each place.
To simplify the notation we have written $a; b$ instead of \<Sequence(a, b)\>.


\section{Properties of Annotations}
This sections discusses properties of the annotation process. Many of them are trivial but we will rely on
them when proving the correctness of the approach.

\subsection{General}
\begin{property}
  The function \<annProgram\> finishes for all entries (that meet the assumptions).
\begin{proof}
  This follows from the fact that the sets are all finite and the recursive definitions are primitive
recursive.
\end{proof}
\end{property}

\begin{property}\label{prop:cClass}
Given a program $P$ and a MVA $A$ such that $A \modeled P$, there is a class $c \in P.classes$ such that
\begin{haskell}
annProgram(P, A) = (P.Classes \setminus \{c\}) \cup annClass(c, A)
\end{haskell}
\begin{proof}
Since $A \modeled P$, there is a class $c \in P.classes$ such that $c.name = A.className$. Since class names
are all different, this is the only class that gets annotated with \<annClass\> and the other classes remain
unchanged.
\end{proof}
\end{property}

\begin{property}
  The ghost variables that encode the different control points are constant throughout the program.
\begin{proof}
  This follows from the assumption that the variables of the automata and \texttt{cp} are only modified by the
annotation that produced by the annotation process and the fact that there is no assignment to those
variables, thus their values do not change once they are initialized.
\end{proof}
\end{property}

\begin{property}
  In the program $annProgram(P, A)$, the value of $cp$ is always one of the possible control points of $A$.
\begin{proof}
The variable $cp$ is initialized to $A.icp$ which is requiered to be a member of $A.cps$.
The only places in the code where $cp$ is assigned are those generated by $updVars$ which assigns to it
the value of $t.tcp$, for some transition $t$ of $A.ts$. This value is also requiered to be in $A.cps$.
\end{proof}
\end{property}


\subsection{Automaton completition}
\begin{property}
  If for an input string $s$ an MVA $a$ gets stuck then \<completeMVA(a)\> will be in the state \<halted\>
after reading $s$, that is
  $$\Delta^*_a(q_0, s) = \bot \textrm{ if and only if } \Delta^*_{a'}(q_0, s) = halted,$$
where \<a' = completeMVA(a)\> and $q_0$ is the initial state of $a$.
\end{property}

\begin{property}
  Let $a$ be an MVA that has been completed with \<completeMVA\>, then for any event $e$ in $a.evs$,
$(halted, True, e, skip, halted) \in a.ts$
\end{property}

\begin{property}
  Let $a$ be an MVA that has been completed with \<completeMVA\>, then $a.ts$ is a total function.
\end{property}


\subsection{Code Transformations}
\begin{property}
  The code transformations made by \<completeMethods\> and \<annMethod\> do not change the semantic of the
program.
\end{property}

\begin{property}
  There is an equivalent way of encoding the transitions that does not need the assumption of the \<IfJML\>
instruction.
\end{property}


\subsection{Inital states}
  Let $\sigma_0^P$ and $\sigma_0^{\gamma(A,P)}$ be the initial (small) states of $P$ and $annProgram(P, A)$ as
defined
in Definition \ref{def:PE} and \ref{def:MPE}, respectively.

\begin{property}\label{prop:InitalMPModelsA}
%   Let $\sigma_0^P$ and $\sigma_0^{\gamma(A,P)}$ be the initial states of $P$ and $annProgram(A, P)$ as
% defined in Definition \ref{def:PE} and \ref{def:MPE}, respectively, then
\ 
\begin{haskell}
\sigma_0^{\gamma(A,P)}.gvs =\,\,& \sigma_0^P.gvs \sscup \{ cp \mapsto A.icp \} \sscup \{ q.name \mapsto
unique(q) \,|\, q \in A.cps \}\\
& \cup \{ v.name \mapsto v.ival\,|\,v \in A.vdsA \}
\end{haskell}
\begin{proof}
Let $c$ be the class of $P.classes$ that gets annotated (see Property \ref{prop:cClass}). Then
\begin{haskell}
\siAP.gvs &=\,\,& \setc{gvd.name \mapsto gvd.ival}{gvd \in c.gvds \wedge c \in
annProgram(P,A).classes}\\
&=& \setc{gvd.name \mapsto gvd.ival}{gvd \in c.gvds \wedge c \in P.classes \setminus \{c\}} \cup\\
&&  \setc{gvd.name \mapsto gvd.ival}{gvd \in annClass(c,A).gvds}\\
&=& \setc{gvd.name \mapsto gvd.ival}{gvd \in c.gvds \wedge c \in P.classes \setminus \{c\}} \cup\\
&&  \setc{gvd.name \mapsto gvd.ival}{gvd \in c.gvds} \cup\\
&&  \setc{gvd.name \mapsto gvd.ival}{gvd \in newGVDs(c, A)}\\
&=& \setc{gvd.name \mapsto gvd.ival}{gvd \in c.gvds \wedge c \in P.classes} \cup\\
&&  \{cp \mapsto A.icp\} \cup \setc{q.name \mapsto unique(q)}{q \in A.cps} \cup\\
&&  \{v.name \mapsto v.ival \,|\, v \in A.vdsA\}\\
&=& \sigma_0^P.gvs \cup \{cp \mapsto A.icp\} \cup \setc{q.name \mapsto unique(q)}{q \in A.cps} \cup\\
&&  \setc{v.name \mapsto v.ival}{v \in A.vdsA}
\end{haskell}
\end{proof}
\end{property}

\begin{property}\label{prop:InitalMPModelsP}
\ 
\begin{haskell}
\siAP.fvs = \siP.fvs \wedge \siAP.gvs \subseteq \siP \wedge \siAP.lvs = \siP.lvs
\end{haskell}
\begin{proof}
Let $c$ be the class in $P.classes$ that gets annotated (see Property~\ref{prop:cClass}). For this class the
ghost variable declarations are enlarged with $newGVDs(c, A)$ and hence the function $\siAP.gvs$ is a subset
of $\siP.gvs$. Since fields and local variables declarations are not changed, $\siAP.fvs = \siP.fvs$ and
$\siAP.lvs = \siP.lvs$.
\end{proof}
\end{property}


\section{Correctness}\label{sec:Correctness}
\begin{definition}
Let $A$ be an MVA and $P$ a program such that $A \modeled P$, then we define
\begin{haskell}\label{def:AModeled}
\sqsubseteq &\ofType& AState \times PSt \to Bool\\
\sA \modeled \sP &\eqdef& \sA.cp = \sP.gvs(cp) \wedge \sA.vars \subseteq \sP.gvs
\end{haskell}

\begin{haskell}\label{def:PModeled}
\sqsubseteq &\ofType& PSt \times PSt \to Bool\\
\sP \modeled \sAP &\eqdef& \sP.fvs = \sAP.fvs \wedge \sP.gvs \subseteq \sAP.gvs \wedge\\
&&\sP.lvs = \sAP.lvs
\end{haskell}

\begin{haskell}\label{def:StateEquiv}
\approx &\ofType& MPState \times PState \to Bool\\
% \alpha \monitor (x, \sigma) \approx (x', \sigma') &\eqdef& (x = x' \vee (x' = JMLExc \wedge \alpha.cp =
% halted)) \wedge\\
%   && \alpha \modeled \sigma' \wedge \sigma \modeled \sigma'\\
%TODO: Say that this relation is applies to "compatible" states
%   && \dom(\alpha.vars) \cup \dom (\sigma.gvs) \cup \{cp\} = \dom(\sigma')
\sA \monitor (x, \sP) \approx (x', \sAP) &\eqdef& x = x' \wedge \sA \modeled \sAP \wedge \sP \modeled \sAP\\
\end{haskell}
\end{definition}

\begin{lemma}\label{prop:InitialEquiv}
If $\siA, Norm(\siP)$ and $Norm(\siAP)$ are the initial states of $A, P$ and $annProgram(P, A)$,
respectively, then $\siA \monitor Norm(\siP) \approx Norm(\siAP)$.
\begin{proof}
Since both states are non-exceptional, to prove the equivalence we have to prove that $\siA \modeled \siAP$,
which follows from Property~\ref{prop:InitalMPModelsA}, and that $\siP \modeled \siAP$ which is proved in
Property~\ref{prop:InitalMPModelsP}.
\end{proof}
\end{lemma}

\subsection*{Rule Induction for Commands}
To avoid simultaneosly combining structural induction on commands with that on expressions we use the
\emph{special principle rule induction} described in~\cite[p44]{Winskel}.

% \renewcommand{\arraystretch}{1.3}
\begin{displaymath}
\begin{array}{l}
\forall c \in Stmt; \siA \monitor (x_0, \siP), \soA \monitor (x_1, \soP) \in MPState; (x'_0, \siAP),
(x'_1, \soAP) \in PState\, \cdot\\
\quad \stmp{c, \siA \monitor (x_0, \siP)}{\soA \monitor (x_0, \soP)} \wedge\\
\quad \stap{c, (x'_0, \siAP)}{(x'_1, \soAP)} \longrightarrow\\
\quad Q(c, \siA \monitor (x_0, \siP), \soA \monitor (x_1, \soP), (x'_0, \siAP), (x'_1, \soAP)) \quad
\text{iff}\\
\lbrack\\ %Exceptions
\forall c \in Stmt; \siA \monitor (Some(x_0), \siP) \in MPState; (Some(x'_0), \siAP) \in PState\, \cdot\\
\quad Q(c, \siA \monitor (Some(x_0), \siP), \siA \monitor (Some(x_0), \siP), (Some(x'_0), \siAP), (Some(x'_0),
\siAP))\\
\wedge\\ %Skip
\forall \siA \monitor Norm(\siP) \in MPState; Norm(\siAP) \in PState\, \cdot\\
\quad Q(Skip, \siA \monitor Norm(\siP), \siA \monitor Norm(\siP), Norm(\siAP), Norm(\siAP))\\
\wedge\\
\ldots\\
\wedge\\
\forall \sinMP, \soA \monitor \soP, \stA \monitor \stP \in MPState; Norm(\siAP), \soAP, \stAP \in PState\,
\cdot\\
\quad \etmp{e, \sinMP}{v, \soMP} \,\wedge\\
\quad \stmp{\pif{the\_Bool(v)}{c_1}{c_2}, \soMP}{\stMP} \,\wedge\\
\quad \etap{e, \sinAP}{(v', \soAP)} \,\wedge\\
\quad \stap{\pif{the\_Bool(v)}{c_1}{c_2}, \soAP}{\stAP} \,\wedge\\
\quad Q(\pif{the\_Bool(v)}{c_1}{c_2}, \soMP, \stMP, \soAP, \stAP) \longrightarrow\\
\quad Q(If(e, c_1, c_2), \sinMP, \stMP, \sinAP, \stAP)\\
\wedge\\
\ldots\\
\rbrack
\end{array}
\end{displaymath}

\subsection*{Rule Induction for Expressions}
% \newpage
\begin{displaymath}
\begin{array}{l}
\forall e \in Expr; \siA \monitor (x_0, \siP), \soA \monitor (x_1, \soP) \in MPState; (x'_0, \siAP),
(x'_1, \soAP) \in PState;\\
v,v' \in Val\, \cdot\\
\quad \etmp{e, \siA \monitor (x_0, \siP)}{v, \soA \monitor (x_0, \soP)} \,\wedge\\
\quad \etap{e, (x'_0, \siAP)}{v', (x'_1, \soAP)} \longrightarrow\\
\quad Q(e, \siA \monitor (x_0, \siP), v, \soA \monitor (x_1, \soP), (x'_0, \siAP), v', (x'_1, \soAP))
\quad \text{iff}\\
\lbrack \forall e \in Expr; \siA \monitor (Some(x_0), \siP) \in MPState; (Some(x'_0), \siAP) \in PState\,
\cdot\\
\quad Q(e, \siA \monitor (Some(x_0), \siP), \bot, \siA \monitor (Some(x_0), \siP), (Some(x'_0), \siAP),\\
\quad\quad \bot, (Some(x'_0), \siAP))\\
\wedge \ldots\\
\end{array}
\end{displaymath}

\begin{displaymath}
\begin{array}{l}
% \wedge \qquad \texttt{// Method call rule}\\ %Method calls
\wedge
%Monitored Programs
\forall e,p \in Expr; a,a',pv,pv',v,v' \in Val; \alpha^A_0,\ldots\alpha^A_6 \in AState;\\
\sinP,(x_1,\soP), \ldots, (x_5, \sigma^P_5), \sinAP, (x'_1,\soAP),\ldots,(x'_7,\sigma^{\AP}_7) \in PState
\,\cdot\\
\quad \etmp{e,\sinMP}{a, \alpha^A_1 \monitor (x_1, \soP)} \,\wedge\\
\quad \etmp{p, \alpha^A_1 \monitor (x_1, \soP)}{pv, \alpha^A_2 \monitor (x_2, \sigma^P_2)} \,\wedge\\
\quad x_3 = \pif{x_2 \neq None \wedge a = Null}{NullPointer}{x_2} \,\wedge\\
\quad md = lookup\_mthd(P, a, mn, p) \wedge inv = lookup\_inv(P, a) \,\wedge\\
\quad \sigma^P_3 = \sigma^P_2(lvs:=init\_vars(md.lvars[this \mapsto a][md.pn \mapsto pv])) \,\wedge\\
\quad P, \sigma^P_3 \vDash inv \wedge \alpha_0.cp \neq halted \wedge P, \sigma^P_3 \vDash md.pre \,\wedge\\
\quad \alpha^A_3 = \begin{cases}
\Delta_A(\alpha^A_2, \sigma^P_3.fvs, (mn, Entry))& \text{if } (m, Entry) \in A.evs\\
\alpha^A_2& \text{otherwise}
\end{cases} \,\wedge\\
\quad \stmp{md.body, \alpha^A_3 \monitor (x_3, \sigma^P_3)}{\alpha^A_4 \monitor (x_4, \sigma^P_4)}
\,\wedge\\
\quad \etmp{md.res, \alpha^A_4 \monitor (x_4, \sigma^P_4)}{v, \alpha^A_5 \monitor (x_5, \sigma^P_5))}
\,\wedge\\
\alpha^A_6 = \begin{cases}
\Delta_A(\alpha^A_5, \sigma^P_5.fvs, (mn, ExitNormal))\!\!\!& \text{if } (mn, ExitNormal) \in A.evs \wedge x_5
= None\\
\Delta_A(\alpha^A_5, \sigma^P_5.fvs, (mn, ExitExcept))& \text{if } (mn, ExitExcept) \in A.evs \wedge x_5 \neq
None\\
\alpha^A_5& \text{otherwise}
\end{cases}\\
\quad \wedge\, P, \sigma^P_5 \vDash md.post \wedge P, \sigma^P_5 \vDash inv \wedge \alpha_6.cp \neq halted
\,\wedge\\
%Annotated Programs
\quad \etap{e,\sinAP}{a',(x'_1,\soAP)} \,\wedge\\
\quad \etap{p, (x'_1, \soAP)}{pv', (x'_2, \stAP)} \,\wedge\\
\quad x'_3 = \pif{x'_2 \neq None \wedge a' = Null}{NullPointer}{x'_2} \,\wedge\\
\quad md' = lookup\_mthd(\gamma(P,A), a', mn, p') \wedge inv' = lookup\_inv(\gamma(P,A), a') \,\wedge\\
\quad \sigma^{\AP}_3 = \sigma^{\AP}_2(lvs:=init\_vars(md.lvars[this \mapsto a'][md.pn \mapsto pv']))
\,\wedge\\
\quad \AP, \sigma^{\AP}_3 \vDash inv' \wedge \gamma(P,A), \sigma^{\AP}_3 \vDash md'.pre \,\wedge\\
\quad \stap{md'.pset, (x'_3, \sigma^{\AP}_3)}{(x'_4, \sigma^{\AP}_4)} \,\wedge\\
\quad \stap{md'.body, (x'_4, \sigma^{\AP}_4)}{(x'_5, \sigma^{\AP}_5)} \,\wedge\\
\quad \etap{md'.res, (x'_5, \sigma^{\AP}_5)}{v,(x'_6, \sigma^{\AP}_6))} \,\wedge\\
\quad \stap{\pif{x'_6 = None}{md'.qset}{md'.xset}, (x'_6, \sigma^{\AP}_6)}{(x'_7, \sigma^{\AP}_7)} \,\wedge\\
\quad \AP, \sigma^{\AP}_7 \vDash inv' \wedge \AP, \sigma^{\AP}_7 \vDash md'.post \,\wedge\\
\quad Q(e, \sinMP, a, \soxMP, \sinAP, a', (x'_1,\soAP))\\
\quad Q(p, \alpha^A_1 \monitor (x_1, \soP), pv, \alpha^A_2 \monitor (x_2, \sigma^P_2), (x'_1, \soAP), pv',
(x'_2, \stAP))\\
%THINK: I'm saying that Q must hold for md.res, what about md'.res?
\quad Q(md.res, \alpha^A_4 \monitor (x_4, \sigma^P_4), v, \alpha^A_5 \monitor (x_5, \sigma^P_5), (x'_5,
\sigma^{\AP}_5), v, (x'_6, \sigma^{\AP}_6)) \longrightarrow\\
\quad Q(MethodCall(e,mn,p), \sinMP, v', \alpha^A_6 \monitor (x_5, \sigma^P_5(lvs:=\sigma_2.lvs)),\\
\quad\quad \sinAP, v', (x'_7, \sigma^{\AP}_7(lvs:=\sigma^{\AP}_2.lvs)))\\
\wedge \ldots \rbrack
\end{array}
\end{displaymath}

\mycomm{\quad I'm saying that Q must hold for md.res, what about md'.res?}

\medskip
The next pair of theorems express that the eqivalence relation is maintained by the execution of the
same command or the evaluation an expression in both systems.

%TODO: Agregar Norm
\begin{theorem}\label{prop:InvComm}
Let $c$ be a $Stmt$; $\alpha^A$, $(x, \sP)$ and $(x', \sAP)$ be states of $A$, $P$ and $annProgram(P, A)$,
respectively, then if
\begin{enumerate}
  \item $\sixMP \approx (x'_0, \sAP)$,
  \item $\stmp{c, \sixMP}{\soxMP}$, and
  \item $\stap{c, (x'_0, \sAP)}{(x'_1, \soAP)}$
\end{enumerate}
then $\soxMP \approx (x'_1, \soAP)$.
\end{theorem}

\begin{theorem}\label{prop:InvExpr}
Let $e$ be an $Expr$; $\alpha^A$, $(x, \sP)$ and $(x', \sAP)$ be states of $A$,
$P$ and $annProgram(P, A)$, respectively, then if
\begin{enumerate}
  \item $\sixMP \approx (x'_0, \sAP)$,
  \item $\etmp{e, \sixMP}{v, \soxMP}$, and
  \item $\etap{e, (x'_0, \sAP)}{v', (x'_1, \soAP)}$
\end{enumerate}
then $v = v'$ $\wedge$ $\soxMP \approx (x'_1, \soAP)$.
\end{theorem}

We prove both theorems together in a mutual recursion using the rules of induction for commands and
expressions. In the case of commands, $Q$ is $\sixMP \approx (x'_0, \sAP)$ $\longrightarrow$ $\soxMP \approx
(x'_1, \soAP)$ and for expressions we add $v = v'$ to the consequent.
For instance, to prove Theorem~\ref{prop:InvComm} in the $If$ case, in the antecedent we have

\[
\begin{array}{l}
\etmp{e, \sinMP }{v, \soMP} \,\wedge\\
\etap{e, Norm(\siAP)}{v', \soAP}\\
\end{array}
\]
assuming that $\sixMP \approx (x'_0, \sAP)$ we can apply Theorem~\ref{prop:InvExpr} to obtain that $\soxMP
\approx (x'_1, \soAP)$. In the antecedent we also have
\[
\begin{array}{l}
\stmp{\pif{the\_Bool(v)}{c_1}{c_2}, \soMP}{\stMP} \,\wedge\\
\stap{\pif{the\_Bool(v)}{c_1}{c_2}, \soAP}{\stAP} \,.\\
\end{array}
\]
Since $v$ is evaluated in equivalent states, it will have the same value and thus the same command will be
chosen. Applying the inductive hypotesis to this command we get that $\stMP \approx \stAP$ and hence
$Q(\pif{the\_Bool(v)}{c_1}{c_2}, \soMP, \stMP, \soAP, \stAP)$ holds. It follows that $Q(If(e, c_1, c_2),
\sinMP, \stMP, \sinAP, \stAP)$ holds.

\begin{theorem}
Let $\siA$ be a state of $A$ and $Norm(\siP)$ a state of $AnnProgram(P, A)$.
Let $c$ be the statement $annEvent(\setc{t}{t \in A.ts \wedge t.event = (k,mn)})$ for some method name $mn$
and event type $k$ and $pe$ the $PEvent$ $(\siP.fvs, (mn, k))$. If
\begin{enumerate}
  \item $\atr{\siA}{pe}{\soA}$,
  \item $\siA \modeled \siAP$, and
  \item $\stap{c, Norm(\siAP)}{(x, \soAP)}$
\end{enumerate}
then $\soA \modeled \soAP$.
\end{theorem}

\medskip
\begin{theorem}\label{prop:Correctness}
Given a program $P$ and a MVA $A$ such that $A \modeled P$,
\[A \monitor P \Rightarrow_{PE} \alpha \monitor \sigma \wedge annProgram(P, A)(e) \Rightarrow_{MPE} \sigma' 
\longrightarrow \alpha \monitor \sigma \approx \sigma'\]
\begin{proof}
It follows from Lemma~\ref{prop:InitialEquiv}, Theorem~\ref{prop:InvComm} and the fact that $annProgram(P,
A)$ does not add or modify any command in $P$, it just adds some ghost variables declarations and modifies
some method declarations.
\end{proof}
\end{theorem}

\bibliographystyle{plain}
\bibliography{overview}
\end{document}
