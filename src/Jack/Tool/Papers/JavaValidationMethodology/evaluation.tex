\chapter{Evaluations}
\section{Industrial evaluation: Oberthur}
\section{Industrial evaluation: Axalto}
The new version of Jack (i.e. 1.8, then 1.8.1) has been evaluated on the 
Payflex case study with the automatic mode using Simplify combined with 
the interactive mode using Coq.\\

Relevant changes on the Coq mode include :
\begin{itemize}
\item the transitivity relation for subtypes is now explicit :
      as a result, proofs about the payflex file system are better handled
      when one must decide whether a file of type T is also of type T' 
or not.
  \item  the display of proof obligation hypothesis is clearer and more 
readable than on previous version
  \item an prototype editor for coq inside Eclipse has been developed. It 
can be used as an   
      alternative to ProofGeneral or CoqIDE.
  \item the use of pure (i.e. side-effect free) method calls in JML 
annotations  is better handled:
      the method call is no longer automatically replaced by its 
postcondition, instead it is    
      displayed as a Java method call and the user can unfold it when 
desired (as is done in the
      Krakatoa tool for instance)
  \item JML model variables (i.e. used for specification purpose only) are 
now well supported.
\end{itemize}

\noindent Other add-in :
one can use some methods declared with the keyword 'native' (it is 
not standard JML) in annotations.These methods are native with respect to a particular prover, that is they must be defined directly in the prover.\\
      That is a way to develop some 'specification libraries' to reuse 
when needed.\\ 
      This is still very experimental and only works for Coq for the 
moment. No way to exploit this feature in our case
     study has been found yet.\\

It seems no improvement has been done on the Simplify mode as the rate 
of proof obligations fulfilled
 automatically in our case study did not change.

\paragraph{JML}
Another aspect is the work we continue on the JML specification of the
JC applet in general (as the tool uses as input the JML
specification). Some theoretical results on the verification of
specific properties have been obtained, and a paper has been issued on
the subject\cite{Rousset}. We start the development of a tool with
graphical interface to assist the writing of annotations:\\ From an
UML class diagram or the code itself of an application, it proposes
some specification patterns to the user (non-null by default for
references, ranges for variable of integer type and arrays length),
then translation to adequate class invariants and method preconditions
is performed. In practice, annotations generated in that manner reveal
sufficient to prove that each method 'does not go wrong' (i.e. is
runtime error free) and all the verification conditions are filled in
automatic mode with Simplify using tools like Krakatoa or Jack.


\section{Bytecode Verifier}
The tools have also been tested on a bytecode verifier java implementation. A termination proof has been provided.
A specific implementation has been coded with on one hand the main loop which remain unchanged whatever the specifications of the virtual machine Java chosen, and other instructions and memory states which depends on selected model.
\subsection{Implementation and Modelisation}
The main loop is in a package which contains abstract classes: 
the instructions and the states are implemented in a more generic way.
The package containing the implementation is composed from the instructions for the standard Java types and of the states of memory typing. 
\subsubsection {Memory states}
The memory states are represented by the State class, which is an abstract class.  It does not contain any precise definition of the memory: 
one has no information on the stack or on the local variables table. 
The implementation is relatively simple: 
it is a class which contains a type stack and a table of the types of the local variables. 
Functions allowing to read simply these structures and to generate verification error in the cases of misuse are defined.   
\subsubsection{Instructions}
The instructions are also represented by an abstract class: 
the class Instruction.  
Since in the Kildall algorithm each instruction is associated to a memory state,  the Instruction class has a field of the State type. 
An instruction can also have one or more successors. 
This relation is represented by a field which is the list of the successors of the instruction.  
One of the other aspects is the fact that on associate to each instruction a boolean field to determine if it has been modified or not.

Several properties of the bytecode verifier are formalised in this class.
First of all one verifies that the successors of the instruction are well included in the others instructions of the program. 
If these successors pointed towards external instructions, an verification error would be returned. 

The others important properties concern the pure function {\tt buildNewState}.
This function builds the typing state of the execution of an instruction  on the current state.
This construction can fail if the instruction tries for example to pop an element when the stack is empty.
If it succeeds, a new non null state is build.

Around ten instructions have been implemented:  {\tt load} and {\tt blind} for the access to local variables, {\tt push} and {\tt pop}   to obtain or put element on the stack, {\tt op1} and {\tt op2} which is two operators who consume both the two top element of the stack and which replaces them by a result of a certain type, {\tt ifle} and {\tt jump} instructions of jump towards another instruction successor, {\tt nop}  the instruction which does not do anything   and finally {\tt stop} which is an instruction which does not have a successor.
These instructions have an associated type in the OperandType class,
who can be None, Type1 or Type2.  Those are the minimal instructions to have a Java-like program.  
\subsubsection {The main loop}
The main loop is implemented in the Verifier class.
It is not an abstract class because it uses the properties of the State and Instruction abstract classes  to verify an instruction set on particular states.
This class provides two functions, the function  {\tt verify} in which the loop is written and the function {\tt check} which verify an instruction.

%The m \ 'ethode check ensures that all \ 'states of the successors of an instruction donn \ 'ee,    are larger or \ 'equal that the \ 'states before the ex \ 'ecution of the m \ 'ethode. This   propri \ 'and \ 'E seems simple \ `has to express but it implies several Pr \ 'erequis.  First of all it should be guaranteed that the successors of the instructions point all worms of   valid instructions. Then that all the instructions are diff \ 'erentes of no one and that theirs  \ 'states are too diff \ 'erents of no one them.    The Pr \ calculation weaker 'econdition of Jack forces us \ `has to add these propri \ 'and \ 'be    Li \ 'ees \ `with the S \ 'emantic of the language Java.  .   %Pour to facilitate the evidence I have \ 'and \ 'E oblig \ 'E  %de to add a certain number of assertions.    

The {\ tt verify} method is the main loop of the bytecode verifier.
It contains two nested loops. 
The internal one is a {\tt for} loop which iterates on the instructions  and verify all the quoted instructions (as described in the Kildall algorithm). 
The termination of the internal loop is easy to prove.
The {\tt for} loop executes as many time as there are numbers in the table.
The external {\tt while} loop stops the algorithm when no more instruction typing state is modified.
This termination is not obvious to prove, especially with JML, since it only allow to prove loop termination by giving an integer variant.

Since the states have to be used to show the algorithm termination,  
one has to make correspond each state  with an integer. 
Thus at each loop iteration, the integer associated with the state either increase or preserve the same value; and it exists a maximum value.


\begin{figure}[ht]  
\begin{center}    
\begin{tabular}{p {0.4 \textwidth} c c c c}  
{\bf Classes:} & State & Instruction & Verifier \\  
{\bf Lines of code:} & 14 & 47 & 66 \\  
{\bf Lines of annotations:} & 20 & 54 & 81 \\  \raggedright 
{\bf Proof obligations:} & 26 & 129 & 627 \\  \raggedright 
{\bf Automatically proved proof obligations:} & 17 & 93 & 112 \\  
{\bf Average length of a non-automatic proof:} & 3 & 6 & 12 \\    
\end{tabular}  
\end{center}  
\caption{Some statistics on proof}  
\label{stats}  
\end{figure}    
\subsection{Proofs}
The first proofs are relatively easy. 
The State class is proved almost automatically; 
except for the constructor where it is necessary to break a disjunction ({\tt instance S $\vee$ S = null}) to prove the invariants.

The Instruction class has also been relatively easy to prove. 
A significant number of proof was done automatically (approximately 90 \%); 
then majority of proof could be trivially resolved, except some lemma concerning  a loop termination.    

Finally the Verifier class was harder to prove.
The lemmas were containing too many hypotheses to be automatically proved.
Around 500 proof obligations have to be resolved manually.
Some of them was obvious and were resolved with quite the same script, but the script cannot be automated.
Some of them was complex: the proof script became little large (an average of 30 steps).
The lemmas concerning the loop invariant of the verify method and its initialization were the most difficult.
\section{Low-Footprint Java-to-Native Compilation}


% Keywords: embedded devices, Java, Java card, exceptions, ahead of time compilation






\subsection{Generation of the Verification Conditions}
\label{vcGen}
For generating the verification conditions, we use a bytecode verification condition generator (vcGen) based on a bytecode weakest precondition calculus~\cite{JBL05MP}. The weakest precondition function $wp$ returns for every instruction \texttt{ins}, normal postcondition $\psi$, and exceptional function $\excPost$ the weakest predicate \\ $\wpi( \texttt{ins} ,\psi ,\excPost)$ such that if it holds in the pre-state of the instruction \texttt{ins} and if the instruction terminates normally, then the normal postcondition $\psi$ holds in the poststate and if \texttt{ins} terminates on an exception \texttt{Exc}, then the predicate $\excPost(\texttt{Exc})$ holds. From the annotated bytecode the vcGen calculates a set of verification conditions for every method of the application. The verification conditions for a method are generated by tracing all the execution paths in it starting at every \texttt{return}, \texttt{athrow} and loop end instruction up to reaching the method entry point. During the process of generation of the verification conditions, for every instruction that may throw a \verb!Runtime! exception a new verification condition is generated.

In figure \ref{fig:wpRule}, we show the weakest precondition rule for the \texttt{getfield} instruction. As the virtual machine is stack-based, the rule mentions the stack \texttt{stack} and the stack counter \texttt{cntr}, thus the stack top element is referred as \stack{\counter}. If the top stack element \stack{\counter} is not null, \texttt{getfield} pops \stack{\counter} which is an object reference and pushes the value of the referenced field onto the operand stack in \stack{\counter}. If the stack top element is null, the Java Virtual Machine specification says that the \texttt{getfield} instruction throws a \texttt{NullPointerException}.

When the verification condition generator works over a method, it labels the formula related to the exceptional termination of every instruction with the index of the instruction in the bytecode array of the method. For example, if a \texttt{getField} instruction is met in the bytecode of a method, a conjunction is generated and the conjunct related to the exception is labeled as shown by figure \ref{fig:wpRule}. Finally, indexing the verification conditions allows to identify later in the proof phase which instructions can be optimized.



\begin{figure}
$$
\begin{array}{l}
\wpi(ind : \texttt{getfield} \ \texttt{Cl.f}, \ \psi, \ \excPost) = \\
\biggl( 
\begin{array}{l}
	%\begin{array}{l}
   		\stack{\counter} \not= \Mynull\Rightarrow   \\
	\Myspace \psi\begin{array}{l} \subst{\stack{\counter}}{\texttt{Cl.f} (\stack{ \counter}) } \\[0 mm]

		\end{array}\\
	%\end{array}
   \wedge \\
    ind : \stack{\counter} = \Mynull 	\Rightarrow\\
   \Myspace	 \excPost(\texttt{NullPointerException})
        \begin{array}{l}
          \subst{ \counter }{ 0} \\
          \subst{\stack{0}}{ \texttt{ref}_{NullPointer} }
	\end{array}
    \end{array} \biggr)
\end{array}
 $$
\caption{The weakest precondition rule for the \texttt{putfield} instruction}
\label{fig:wpRule}
\end{figure}



%\nocite{Adl-Tabatabai1998, Cierniak2000}


\section{Memory consumption}
%\subsection{Introduction}

%\input BML/cmdBML.tex

\newcommand{\code}{\textit{code}}
\newcommand{\indexComp}{\textit{index}}





%\section{Introduction} \label{bcsl}
This chapter presents the bytecode level specification language, called for short BML and a compiler from a
 subset of the high level Java specification language JML to BML which from now we shall call \JMLtoBML. 
The chapter is organized as follows.
 In section \ref{BCSLprelim}, we give an overview of the main features of JML. A detailed overview of BML is given in section \ref{BCSLgrammar}.  
  As we stated before, we support also a compiler from the high level specification language JML into BML. The 
 compilation process from JML to BML is discussed in section  \ref{BCSLcompile}.
 The full specification of the new user defined Java attributes in which the JML specification is compiled is given in the appendix.





%\subsection{Preliminaries}\label{sec:prelim}
%\subsubsection{Java class files} \label{classFileFormat}
The standard format for Java bytecode programs is the so-called class
file format which is specified in the Java Virtual Machine
Specification~\cite{VMSpec}. For the purpose of this paper, it is
sufficient to know that class files contain the definition of a single
class or interface, and are structured into a hierarchy of different
attributes that contain information such as the class name, the name
of its superclass or the interfaces it implements, a table of the
methods declared in the class. Moreover an attribute may contain other
attributes. For example the attribute that describes a single method
contains a \verb!Local_Variable_Table! attribute that describes the
method parameters and its local variables.
%; further in this section we will denote the table of local variables
%by $l$ and the $i^{th}$ variable by $l[i]$.

In addition to these attributes which provide all the information
required by a standard implementation of the JVM, class files can
accommodate user-defined attributes.  We take advantage of this
possibility and introduce additional attributes given in the Bytecode
Specification Language, described below.


\subsubsection{The Bytecode Specification Language}
The {\it Bytecode Specification Language} (BCSL) \cite{LM05:acc} is a
variant of the Java Modelling Language (JML) \cite{JMLRefMan} tailored
to Java bytecode. For our purposes, we only need to consider a
restricted fragment of BCSL, which is given in Fig.~\ref{fig:bml}; we
let $\expression$ and $\predicate$ denote respectively the set of BCSL
expressions and predicates. As for JML, BCSL specifications contain
different forms of statements, in the form of predicates tagged with
appropriate keywords. BCSL predicates are built from expressions using
standard predicate logic; furthermore BCSL expressions are bytecode
programs that correspond to effect-free Java expressions, or BCSL
specific expressions.  The latter include expressions of the form
\verb!\oldp(exp)! which refers to the value of the expression
\verb!exp! at the beginning of the method, or $\mbox{\tt
exp}^{\mbox{{\tt pc}}}$ which refers to the value of the expression
\verb!expr! at program point \verb!pc!. Note that the latter is not
standard in JML but can be emulated introducing a ghost variable
$\mbox{\tt exp}^{\mbox{{\tt pc}}}$ and performing the ghost assignment
\verb!set exp!$\mbox{}^{\mbox{{\tt pc}}}$\verb!= exp! at program point
\verb!pc!.


Statements can be used for the following purposes:
\begin{itemize}
\item Specifying method preconditions, which following the design by
contract principles, must be satisfied upon method invocation. They
are formulated using statements of the form $\requires \ \predicate$;


\item Specifying method postconditions, which must be guaranteed upon
returning normally from the method. Such postconditions are formulated
using statements of the form $\ensures\ \predicate$;

\item Specifying method exceptional postconditions, which must be
guaranteed upon returning exceptionally from the method. Such
postconditions are formulated using statements of the form \\
$\exsures{Exception} \predicate$, that record the reason for
exceptional termination;

\item Stating loop invariants, which are predicates that must hold
every time the program enters the loop: $\invariant\ \predicate$;

\item Guaranteeing termination of loops and recursive methods, using
statements of the form $\variant\ \expression$ which provide a measure (in
the case of BCSL, a positive number) that strictly decreases at each
iteration of the loop/recursive call;


\item Local assertions, using $\assert \ \predicate$, which asserts
that $\predicate$ holds at the program point immediately after the
assertion;

\item Declaring and updating ghost variables, using statements of the
form $\declare \ \ghost \ Type \ name$ and $ \ghostSet \ \expression =
\expression$;


\item Keeping track of variables that are modified by a method or in a
loop, using declarations of the form $\modifies \ var$. During the
generation of verification conditions, one checks that variables that
are not declared as modifiable by the clause above will not be
modified during the execution of the method/loop. This information is
also used to generate the verification conditions.
\end{itemize}

\begin{figure}
%\begin{frameit}
$$
\begin{array}{lll} 
\mbox{\annotation}-{\sf stmt} & = &
                                       \requires \ \predicate \\
                              & \mid & \ensures  \ \predicate  \\
                           & \mid  & \exsures{Exception} \ \predicate  \\
                               & \mid  &  \assert  \ \predicate  \\
                                & \mid & \invariant \  \predicate  \\
                                & \mid & \variant \  \expression  \\
                                & \mid &  \declare \ \ghost \ Type \ name \\
                                 & \mid & \modifies  \ var  \\
                                 & \mid & \ghostSet \ \expression = \expression

\end{array}
$$
\caption{{\sc Specification language}}\label{fig:bml}
%\end{frameit}
\end{figure}

Note that, as alluded above, annotations are not
inserted directly into bytecode; instead they are gathered into
appropriate user defined attributes of an extended class file. Such
extended class files can be obtained either through direct
manipulation of standard class files, or using an extended compiler
that outputs extended class files from JML annotated programs,
see~\cite{LM05:acc}.

\subsubsection{Verification of annotated bytecode}
In order to validate annotated Java bytecode programs, we resort to a
verification environment for Java bytecode, which is an adaptation by
L.~Burdy and the second author~\cite{LM05:acc} of
JACK~\cite{BRL-JACK}. The environment consists of two main
components:
\begin{itemize}
\item A verification condition generator, which takes as input an annotated
applet and generates a set of verification conditions which are sufficient
to guarantee that the applet meets its specification;

\item A proof engine that attempts to discharge the verification
conditions automatically, and then sends the remaining verification
conditions to proof assistants where they can be discharged
interactively by the user.

\end{itemize}


\paragraph{Generating the Verification Conditions}\label{subsec:verification}
The verification condition generator (VCGen) takes as input an
extended class file and returns as output a set of proof obligations,
whose validity guarantees that the program satisfies its
annotations. The VCGen proceeds in a modular fashion in the sense that
it addresses each method separately, and is based on computing weakest
preconditions. More precisely, for every method $\method$,
postcondition $\psi$ that must hold after normal termination of
$\method$, and exceptional postcondition $\psi'$ that must hold after
exceptional termination of $\method$ (for simplicity we consider only
one exception in our informal discussion), the VCGen computes a
predicate $\phi$ whose validity at the onset of method execution
guarantees that $\psi$ will hold upon normal termination, and $\psi'$
will hold upon exceptional termination. The VCGen will then return
several proof obligations that correspond, among other things, to the
fact that the precondition of $\method$ given by the specification
entails the predicate $\phi$ that has been computed, and to the fact
that variants and invariants are correct.


The procedure for computing weakest preconditions is described in
detail in~\cite{LM05:acc}. In a nutshell, one first defines for each
bytecode a predicate transformer that takes as input the
postconditions of the bytecode, i.e. the predicates to be satisfied
upon execution of the bytecode (different predicates can be provided
in case the bytecode is a branching instruction), and returns a
predicate whose validity prior to the execution of bytecode guarantees
the postconditions of the bytecode. The definition of such functions
is based on a single instruction, so the next step is to use these
functions to compute weakest preconditions for programs.  This is done
by building the control flow graph of the program, and then by
computing the weakest preconditions of the program, using the graph.

Note that the verification condition generator operates on BCSL
statements which are built from extended BCSL expressions. Indeed,
predicate transformers for instructions need to refer to the operand
stack and must therefore consider expressions of the form
\verb!st(i)! which represent the \verb!i!-element of the stack \verb!st!.

%$$wp( \store \ l(i) , \psi , \psi') = \psi[\verb!top! \leftarrow \verb!top-1!][l[i] \leftarrow \verb!st(top)!].$$


\paragraph{Discharging verification conditions}
Verification conditions are expressed in an intermediate language and
then translated to automatic theorem provers and proof assistants.  In
our examples, we have used Simplify~\cite{simplify} as automatic
prover and Coq~\cite{coq} as proof assistant. The Coq plug-in for Jack
was developed by J.~Charles, and adapted to Java bytecode by L.~Burdy.


\subsubsection{Correctness of the method}\label{subsec:sound}
The verification method is correct in the sense that one can prove
that for all methods $\method$ of the program the (exceptional)
postcondition of the method holds upon (exceptional) termination of
the method provided the method is called in a state satisfying the
method precondition and provided all verification conditions can be
shown to be valid.


The correctness of the verification method is established relative to
an operational semantics that describes the transitions to be taken by
the virtual machine depending upon the state in which the machine is
executed. There are many formalisations of the operational semantics
of the JVM, see
e.g.~\cite{FM03:jar,KN02:tcs,siv04:jlap,BSS:jbook}. 

%Such semantics manipulate states of the form
%$\config{h,\fram{m,\pc,l,s},\stf}$, where $h$ is the heap of objects,
%$\fram{m,\pc,l,s}$ is the current \emph{frame} and $\stf$ is the
%current call stack (a list of frames). A frame $\fram{m,\pc,l,s}$
%contains a method name $m$ and a program point $\pc$ within $m$, a set
%of local variables $l$, and a local operand stack~$s$.
%%The rule for the generic instruction \instr\ is formalized as a
%The operational semantics for each instruction is formalised as rules specifying transition between states, or between a state and some tag that
%indicates abnormal termination. For example, the semantics of
% the instruction $\store$ is given by the transition
%rule below, where  $\InstAt(m,\pc)$ is the function that extracts
%the $\pc$-th instruction from the body of method $\method$:

%$$\frac{
%%\begin{array}[c]{c}
%\InstAt(m,\pc)=\store \ i
%%\end{array}}%
%}
%{\begin{array}[t]{c}
%\config{h,\fram{m,\pc,l,v::s},\stf} \to_{\store\ i} \\
%\ \ \ \ \ \ \ \ \ \ \ \ \ \ \ \ \ \ \ \ 
%\config{h,\fram{m,\pc+1,l[i \mapsto v],s},\stf}
%\end{array}}$$
%In order to establish the correctness of our method, one first needs
%to establish the correctness of the predicate transformer for each
%bytecode. For example for the instruction $\store$ we show that:
%$$\begin{array}[t]{c} 
%wp(\store \ i , \psi )( \config{h,\fram{m,\pc,l,v::s},\stf} ) \ \ \Rightarrow  
%\\
%\psi( \config{h,\fram{m,\pc+1,l[i \mapsto v],s},\stf})
%\end{array}$$
%In the above $\psi(\config{h,\fram{m,\pc,l,v::s},\stf} )$ is to be
%understood as the instance of the formula $\psi$ in which all local
%variables $l$ and field references are substituted with their
%corresponding values in state $\config{h,\fram{m,\pc,l,v::s},\stf} $.


%The proof proceeds by a case analysis on the instruction to be
%executed, and makes an intensive use of auxiliary substitution
%lemmas that relate e.g. the stack of the pre-state with the stack
%of the post-state of executing an instruction. Then one proves the
%correctness of the method by induction on the length of the
%execution sequence.

We have proved the correctness of our method for a fragment of the JVM
that includes the following constructs: Stack manipulation: \push,
\pop, \dup, \dup 2, \swap, \numop, etc; Arithmetic instructions:
type\_\add, type\_\sub, etc; Local variables manipulation:
type\_\load, type\_\store, etc; Jump instructions: \If, \goto; Object
creation and object manipulation: \new, \putfd, \getfd, \newarray,
etc; Array instructions: \arrst, \arrld, etc; Method calls and return:
\invvir, \return; Subroutines: \jsr\ and \ret.

Note however that our method imposes some mild restrictions on the
structure of programs: for example, we require that $\jsr$ and
$\throw$ instructions are not entry for loops in the control flow
graph in order to prevent pathological recursion.  Lifting such
restrictions is left for future work.


\subsection{Modeling Memory Consumption}\label{sec:verif}
The objective of this section is to demonstrate how the user can
annotate and verify programs in order to obtain an upper bound on
memory consumption. We begin by describing the principles of our
approach, then turn to establish its soundness, and finally show
how it can be applied to non-trivial examples involving recursive
methods and exceptions.


\subsection{Principles}
Let us begin with a very simple memory consumption policy which aims
at enforcing that  programs do not consume more than
some fixed amount of memory \Max . To enforce this policy, we first
introduce a ghost variable \Mem\ that represents at any given point of
the program the memory used so far. Then, we annotate the program both
with the policy and with additional statements that will be used to
check that the application respects the policy.



\paragraph{The precondition} of the method $\method$ should ensure that
there must be enough free memory for the method execution. Suppose
that we know an upper bound of the allocations done by method $\method$
in any execution. We will denote this upper bound by
\allocMethod{\method}. Thus there must be at least
\allocMethod{\method}\ free memory units from the allowed \Max\ when
method $\method$ starts execution. Thus the precondition for the method
$\method$ is:
$$
\requires \ \Mem + \allocMethod{\method}  \leq \Max.
$$
%\todo{ to leave this paragraph or not. It is about the initialization of the variable \Mem} 
The precondition of the
program entry point (i.e., the method from which an application
may start its execution) should state that the program has not
allocated any memory, i.e. require that variable \Mem \ is  0:
$$
\requires \ \Mem == 0.
$$
\paragraph{The normal postcondition} of the method $\method$ must
guarantee that the memory allocated during a normal execution of
$\method$ is not more than some fixed number \allocMethod{\method}\
of memory units. Thus for the method $\method$ the postcondition is:
$$
\ensures \  \Mem \leq \old(\Mem) + \allocMethod{\method}.
$$

\paragraph{The exceptional postcondition} of the method $\method$ must
say that the memory allocated during an execution of $\method$ that
terminates by throwing an exception \texttt{Exception} is not more
than \allocMethod{\method}\ units. Thus for the method $\method$ the
exceptional postcondition is:

$$
\exsures{Exception} \  \Mem \leq \old(\Mem) + \allocMethod{\method}.
$$


\paragraph{Loops} must also be annotated with appropriate invariants. 
%Assuming that we know that loop $\progLoop{l}$ iterates no more than $\maxIter{l}$ as well as an upper bound  $\allocLoop{l}$ of the allocations done per iteration in $l$. 
Let us assume that loop $\progLoop{l}$ iterates no more than $\maxIter{l}$ and let $\allocLoop{l}$ be an upper bound of the memory allocated per iteration in $l$.
Below we give a general form of loop specification w.r.t. the property for constraint memory consumption. The loop invariant of a loop $\progLoop{l}$ states that at every iteration the loop body is not going to allocate more than $\allocLoop{l}$ memory units and that the iterations are no more than $\maxIter{l}$. We also declare an expression which guarantees loop termination, i.e. a variant (here an integer expression whose values decrease at every iteration  and is always bigger or equal to 0).
$$\begin{array}{ll}
\modifies &  \ i, \Mem \\
\invariant: & \ \Mem \le \atState{\Mem}{Before_{l}} + i * \allocLoop{l} \\
                & \wedge \\
                & i \le \maxIter{l}\\
\variant: & \maxIter{l} - i \\
\end{array}$$
 A special variable appears in the invariant, $\atState{\Mem}{Before_{l}}$. It denotes the value of the consumed memory just before entering for the first time the loop \progLoop{l}. At every iteration the consumed memory must not go beyond the upper bound given for the body of loop.

\paragraph{For every instruction that allocates memory} the ghost
variable \Mem\ must also be updated accordingly. For the purpose of
this paper, we only consider dynamic object creation with the bytecode
\new; arrays are left for future work and briefly discussed in the
conclusion. 

The function $allocInstance: Class \rightarrow int$ gives an estimation of the memory used by an instance of a class.
%In order to perform the update for \new\ bytecodes, we must assume given a function $allocInstance: Class \rightarrow int$ which maps classes to an estimation of the memory that any instance of the class may occupy. 
At every program point where a bytecode \srcCode{\new \ A} is found, the ghost variable \Mem\ must be incremented by $\allocInstance{A}$. This
is achieved by inserting a ghost assignment immediately after any \new\ instruction, as shown below:
$$
\begin{array}{l}
\srcCode{\new \ A} \\
 // \set \ \Mem = \Mem + $\allocInstance{A}$.
\end{array}
$$

\subsection{Correctness}
%An important question is if the annotations that we prescribe here guarantees that the memory used in a program is not more than a fixed upper bound \Max. 
We want to guarantee that the memory allocated by a given program is bounded by a constant \Max.
We can prove that our annotation is correct w.r.t. to the policy for constraint memory use, by instrumenting the operational semantics of the bytecode language given in
 Section \ref{subsec:sound}. The instrumented operational semantics
manipulates states as before, but it is extended with the special variable \Mem. Thus, states in the new semantics have the form:

$$\configMem{h,\fram{m,\pc,l,s},\stf,\Mem}$$

%The variable \Mem \ changes its value only for instructions that allocate space in the heap, i.e. \new\ instructions:

%$$\small{\frac{
%\begin{array}[c]{c}
%\ \InstAt(m,\pc)=\new \ A ,
%\end{array}}
%{\begin{array}[t]{c} \config{h,\fram{m,\pc,l,v::s},\stf, \Mem} \to_{\new\ A} \\ \config{h + \allocInstance{A},\fram{m,\pc+1,l,s},\stf ,\Mem + \allocInstance{A}}
%\end{array}}}$$



The other instructions do not affect \Mem, so the corresponding rules of the operational semantics are as before. As we saw in the previous section to every
instruction of the form $\new\ A$ we attach the annotation $\set\ \Mem = \Mem + \allocInstance(A)$. The proof obligation generator converts this annotation into new value for the variable \Mem:

$$
\begin{array}{l}
wp(\set \ \Mem = \Mem + \allocInstance{A}, \psi) = \\
\ \ \ \ \ \ \ \ \ \ \ \ \psi[ \Mem \leftarrow \Mem + \allocInstance{A} ]
\end{array}
$$

We can prove that whenever the allocated space in the heap increments, 
the ghost variable \Mem\ also increments, which is a sufficient condition to guarantee the correctness of the annotations. 
So far we do not deal with garbage collection (see discussion in Section \ref{sec:conc}).

\subsection{Examples}
We illustrate hereafter our approach by several examples. 
%\alarm{talk about number of proof obligations, which are discharged automatically in Coq, etc}

\subsubsection{Inheritance and overridden methods} Overriding methods are treated as follows: whenever a call is performed to a method \method,
we require that there is enough free memory space for the maximal
consumption by all the  methods that override or are overridden by
\method. In Fig. \ref{classExt} we show a class \verb!A! and its
extending class \verb!B!, where \verb!B! overrides the method \method\ from class \verb!A!. Method \method\ is invoked by $n$. Given that the dynamic type of the parameter passed to $n$ is not known, we cannot know which of the two
methods will be invoked. This is the reason for requiring enough memory space for the execution of any of these methods.
%After the method execution we consider the extreme case where there is executed the method \method\ that consumes the most.

\begin{figure}[!htp]
Specification of method $m$ in class A:
$$
\begin{array}{ll}
\requires & \Mem + k  \leq \Max \\
\modifies & \Mem \\
\ensures & \Mem  \leq \old(\Mem) + k
\end{array}
$$

Specification for method $m$ in class B:
$$
\begin{array}{ll}
\requires & \Mem + l  \leq \Max \\
\modifies & \Mem \\
\ensures & \Mem  \leq \old(\Mem) + l
\end{array}
$$

\begin{verbatim}
method n(A a)
...
//{ prove Mem <= Mem +max(l,k) }
invokevirtual m <A>
//{ assume Mem <= \old(Mem) + max(l,k)}
...
\end{verbatim}
\caption{\sc Example of overridden methods}
\label{classExt}
\end{figure}


\subsubsection{Recursive Methods} In Fig. \ref{recMeth} the bytecode of the recursive method $m$ and its specification is shown. For sake of space we show only a simplified version of the bytecode; we assume that the constructors for the class \srcCode{A} and \srcCode{C} do not allocate memory. Besides the precondition and the postcondition, the specification also includes information about the termination of the method: \variant\ $\local{1}$, meaning that the local variable $\local{1}$ decreases on every recursive call down to and no more than $0$, guaranteeing that the execution of the method will terminate.
 
%Now we explain why such a precondition is required for method \textbf{m} in order to specify the property for constraint memory consumption. 

We explain first the precondition. If the condition of line \srcCode{1} is not true, the execution continues at line \srcCode{2}.

\begin{figure}[!hbp]
\begin{alltt}
public class D \{
  public void m( int i) \{
    if (i > 0) \{
      new A();
      m(i - 1);
      new A();
    \} else \{
      new C();
      new A();
   \}
  \}
\}
\end{alltt}

$$
\begin{array}{ll}
 \requires & ( \Mem + \local{1}*2*\allocInstance{A} + \\
           &  \allocInstance{A} + \allocInstance{C}) \le \Max \\
 \variant  & \local{1} \\
 \ensures  & \local{1} \ge 0 \\
           & \wedge \\
           & \Mem <= \old(\Mem) +  \old(\local{1})*2*\allocInstance{A} + \allocInstance{A}\\
           &  +  \allocInstance{C})
\end{array}$$

\begin{alltt}
\srcCode{\textbf{public void m()}}
//\small{\textit{local variable loaded on} }
//\small{\textit{the operand stack of method \textbf{m}}}
\srcCode{0 \load\_1}
//\small{ \textit{ if \local{1} <= 0 jump}}
\srcCode{1 ifle 12}
\srcCode{2 new <A>} //\small{ \textit{ here \local{1} > 0  } }
//set \Mem = \Mem +  \allocInstance{A}
\srcCode{3 invokespecial <A.<init>>}
\srcCode{4 aload\_0}
\srcCode{5 iload\_1}
\srcCode{6 iconst\_1}
//\small{\textit{\local{1} decremented with 1}}
\srcCode{7 isub}
//\small{ \textit{ recursive call with the new value of \local{1}}}
\srcCode{8 invokevirtual <D.m>}//
\srcCode{9 new <A>}
//set \Mem = \Mem +  \allocInstance{A}
\srcCode{10 invokespecial <A.<init>>}
\srcCode{11 goto 16}
//\small{\textit{target of the jump at \srcCode{1}}}
\srcCode{12 new <A>}
//set \Mem = \Mem +  \allocInstance{A}
\srcCode{13 invokespecial <A.<init>>}
\srcCode{14 new  <C>}
//set \Mem = \Mem +  \allocInstance{C}
\srcCode{15 invokespecial <C.<init>>}
\srcCode{16 return}
\end{alltt}

\caption{\sc Example of a recursive method}
 \label{recMeth}
\end{figure}

In the sequential execution up to line \srcCode{7}, the program allocates at most $\allocInstance{A}$ memory units and decrements by $1$ the value of $\local{1}$. The instruction at line \srcCode{8} is a recursive call to $m$, which either will take the same branch if $\local{1} > 0 $ or will jump to line \srcCode{12} otherwise, where it allocates at most $\allocInstance{A} +  \allocInstance{C}$ memory units. On returning from the recursive call one more allocation will be performed at line \srcCode{9}.
 Thus $m$ will execute, $\local{1}$ times, the instructions from lines \srcCode{4} to \srcCode{35},  and it finally will execute all the instructions from lines  \srcCode{12} to \srcCode{16}.
The postcondition states that the method will perform no more
than $\old(\local{1})$ recursive calls (i.e., the value of the register variable in the pre-state of the method) and that on every recursive call it allocates no more than two instances of class \texttt{A} and that it will finally allocate one instance of class \texttt{A} and another of class \texttt{C}.


\subsubsection{More precise specification} We can be more precise in specifying the precondition of a method by considering what are the field values of an instance, for example. Suppose that we have the method \method\ as shown in Fig. \ref{excMeth}. We assume that in the constructor of the class \texttt{A} no allocations are done. The first line of the method \method\ initializes one of the fields of field \texttt{b}. Since nothing guarantees that field \texttt{b} is not \Mynull, the execution may terminate with
\texttt{NullPointerException}. Depending on the values of the parameters passed to \method, the memory allocated will be different. The precondition establishes what is the expected space of free resources depending on if the field
\texttt{b} is \Mynull  or not. In particular we do not require anything for
the free memory space in the case when \texttt{b} is \Mynull. In the
normal postcondition we state that the method has allocated an
object of class \texttt{A}. The exceptional postcondition states
that no allocation is performed if \texttt{NullpointerException} causes the execution termination.

\begin{figure}[!hbp]
$$
\begin{array}{ll}
 \requires &  \local{1} != \Mynull \Rightarrow  \\
           & \phantom{\local{1}} \Mem +  \allocInstance{A} \le \Max \\
       %& \wedge \\
       %&  \local{1} == \Mynull \Rightarrow  \\
           %& \phantom{\local{1}} \Mem +  \allocInstance{B} + \allocInstance{A}   \le \Max \\
  \modifies & \Mem \\
  \ensures  & \Mem \le \old(\Mem) +  \allocInstance{A} \\
  \exsures{NullPointerException}  & \Mem == \old(\Mem)   \\
\end{array}$$

\begin{tabular}{lr}
\begin{minipage}[t]{170pt}
\begin{alltt}
\srcCode{0 aload\_0}
\srcCode{1 getfield<C.b>}
\srcCode{2 iload\_2}
\srcCode{3 putfield <B.i>}
\srcCode{4 new <A>}
//set \Mem = \Mem +
      \allocInstance{A}
\srcCode{5 dup}
\srcCode{6 invokespecial <A.<init>>}
\srcCode{7 astore\_1}
\srcCode{8 return}
\end{alltt}
\end{minipage}
 &
\begin{minipage}[t]{170pt}
\begin{alltt}
public class C \{
  B b;
  public void m(A a, int i) \{
    b.i = i ;
    a = new A();
  \}
\}
\end{alltt}
\end{minipage}
\end{tabular}
\caption{\sc Example of a method with possible exceptional termination}
\label{excMeth}
\end{figure}

\subsection{Inferring Memory Allocation}\label{sec:infer}
In the previous section, we have described how the memory consumption
of a program can be modeled in BML and verified using an appropriate
verification environment. While our examples illustrate the benefits
of our approach, especially regarding the precision of the analysis,
the applicability of our method is hampered by the cost of providing
the annotations manually. In order to reduce the burden of manually
annotating the program, one can rely on annotation assistants that
infer automatically some of the program annotations (indeed such
assistants already exist for loop invariants, loop variants, or
class invariants). In this section, we describe an implementation of
an annotation assistant dedicated to the analysis of memory consumption,
and illustrate its working on an example.

\subsection{Annotation assistant}
The inference algorithm proposed here works on programs
 without recursive methods, exception throwing and handling.  


The user must provide information about the memory required to create objects of the given classes,
 i.e. he must give the definition  of the function $\allocInstance{\cdot}$.
 The variant and the maximum number of iterations $\maxIter{l}$ for every loop $l$  are either given by the
 user or are synthesized  through appropriate mechanisms. 

%Given a control flow graph of the program --which is computed by the verification environment, a function that gives for each class the memory required to create an object of that class, which is provided by the user, and a variant for each loop and recursive method, which can either be provided by the user, or synthesized through appropriate mechanisms. 

Based on this information, the annotation assistant 
inserts the ghost assignments on appropriate places, and then computes
recursively the memory allocated on each loop and method. 
A pseudo-code of the algorithm for inferring an upper bound for method allocations is given in Fig. \ref{methodAlloc}.
Essentially, it finds the maximal memory that can be allocated in a method by exploring all its possible execution paths.  
The algorithm for exploring an execution path starts from the last instruction in the path (i.e. a \return{} instruction). 
The algorithm use standars techniques to detect the loop entry instructions. For each loop entry instruction it also finds  
the set  of the corresponding ``loop end'' instructions i.e. the  instructions  that target   to  and which are dominated by
the loop entry instruction; you are not entering in detail in the loop detection
algorithm as it is standard and the reader may  see Section 10 of 
 \cite{ASU86cpt} for a description of the algorithms.

\begin{figure}[!htp]
function $\allocMethod{.}$\\
\textbf{Input:} Bytecode of a method \methodd{} . \\
\textbf{Output:} Upper bound of the memory allocated by \methodd{} . \\
\textbf{Body:}
\begin{enumerate}
   \item Detect all the loops in \methodd{}; for every loop $l$ determine $\loopSet{l}$, $\loopEntry{l}$ and $\loopEndsSet{l}$;
   \item Apply the function \allocatedOnly \  to each instruction $\instrAt{k}$, such that $\instrAt{k} = \return$;
   \item Take the maximum of the results given in the previous step: \\
 $max_{\instrAt{k} = \return } \allocated{\instrAt{k}}$.
\end{enumerate}
\caption{\sc Inference algorithm}
\label{methodAlloc}
\end{figure}

The auxiliary function $allocPath$, which infers the maximal allocations done by the set of execution paths ending with the same \return{} instruction,
 is given in Fig. \ref{fig:allocMethod}.
Inferring the memory allocated inside loops is done by the function $\allocLoopWithEnd{\cdot}{\cdot}$, which is invoked by \allocatedOnly{} whenever the 
current instruction belong to a loop. The specification of the function is shown in Fig. \ref{fig:loopPath}.

\begin{figure}[!hbp]
%\centerline{
$\allocated{\instrAt{s}}$ = 
$$ \left\{ \begin{array}{ll}
\allocIns{\instrAt{s} }   &  if \ \instrAt{s} \ has \ no \ predecessors \\
& \\
  \begin{array}{l}
            \allocLoop{\loopEntry{l}} \\
             + \\
            max_{\instrAt{k} \in \preds(\instrAt{s} )-\loopEndsSet{\progLoop{l}}}( \allocated{\instrAt{k}} ) \\
                   \end{array}      & if \  \instrAt{s} \in \loopSet{\progLoop{l}} \\
& \\
\begin{array}{l}
\allocIns{\instrAt{s}} \\
 + \\
max_{\instrAt{k} \in \preds(\instrAt{s} )}
 ( \allocated{\instrAt{k}} )
                       \end{array} & else 
\end{array}
\right.
$$
\caption{\sc Definition of the function $\allocated{\instrAt{s}}$} 
\label{fig:allocMethod}
\end{figure}


\begin{figure}[!hbp]
$\allocLoop{\loopEntry{l}} = $
$$ \begin{array}{l}
    \maxIter{l} * max_{ \instrAt{e} \in \loopEndsSet{l}  } (\allocLoopWithEnd{\loopEntry{l}}{\instrAt{e}} )
   \end{array}$$



$\allocLoopWithEnd{\loopEntry{l}}{\instrAt{s}} = $
$$ 
\left\{\begin{array}{ll}

 \allocIns{\loopEntry{l}}  & if \  \instrAt{s} = \loopEntry{l} \\
  & \\
 \begin{array}{l}
           \allocLoop{\loopEntry{l'}} \\
          + \\
      max_{\instrAt{k} \in \preds(\loopEntry{l'} ) - \loopEndsSet{\progLoop{l'}}}
       ( \allocLoopWithEnd{\loopEntry{l}}{\instrAt{k}} )
    \end{array} &  \begin{array}{l}
                                        if \  \instrAt{s} \in  \loopSet{\progLoop{l'}} \\
                                          \progLoop{l'} \ is \  nested \ in \ \progLoop{l}
                                    \end{array} \\
  & \\
  \begin{array}{l}
     \allocIns{\instrAt{s}} \\
     + \\
     max_{\instrAt{k} \in \preds(\instrAt{s} )}
     ( \allocLoopWithEnd{\loopEntry{l}}{\instrAt{k}} )
                       \end{array} & else \\

\end{array} \right.
$$
 \caption{\sc Definition of the function $\allocLoop{\cdot}$ and  $\allocLoopWithEnd{\cdot}{\cdot}$ }
\label{fig:loopPath}
\end{figure}


%In essence, it first analyses the method's bytecode by identifying the entry loop instruction, the instructions
% that are inside the loop and the set of instructions by which the loop terminates. 
%Then  it finds the maximal number of allocations that can be done per execution path: 
%it starts from all \return \ instructions and ``inspects'' the execution paths upto the entry program
%instruction (in backwards direction  );  if an instruction in the path belongs to a loop then the allocations done per iteration in the loop are 
%calculated and the analysis proceeds from the instructions that target the entry of the loop and that do not belong to the loop; the assistant computes the memory required for %each loop using the memory required for each iteration of the loop, and the variant of the loop, which provides information about the number of
%iterations; upper bound of the allocations done per iteration in a loop are calculated also in a backwards direction 
%(starting from instructions that belong to the loop and whose next instruction is the unique loop entry) by finding the 
%maximal number of allocations per iteration path and if an instruction in an iteration path appears to be instruction that belongs to an inner loop 
%then first an upper bound for the allocations done in the inner loop are inferred and the analysis continues from instructions that target the entry of the nested loop and do not belong to the nested loop.    

The annotation assistant currently synthesize only simple
memory policies (i.e., whenever the memory consumption policy does not depend on
the values of inputs).
%and could be significantly improved in this respect. 
Furthermore, it does not deal with arrays, subroutines, nor exceptions. Our approach may be extended to treat such cases (see the discussion in Section \ref{sec:conc} about how to include arrays in our analysis). For sake of simplicity,
we have also restricted the loop analysis only to those with a unique
entry point, which is the case for code produced by non-optimizing
compilers.
Dealing with loops with a multiple entry points is left for a future work. %A pre-analysis could give us all the entry points of more
%general loops, for instance by the algorithms given in \cite{CJPS05cmu}; our approach may be thus applied straightforwardly.


%\subsection{Conclusion}\label{sec:conc}
%
\section{Achievements}
% done. summary
We have  presented an infrastructure for verification of Java bytecode programs   which allows to reason about potentially
sophisticated  functional and security properties and
which benefits from verification over Java source programs. We have also 
introduced the bytecode specification language BML tailored to Java bytecode, a compiler
from the Java source specification language JML to BML and a verification 
condition generator for Java bytecode programs. 
We have shown that the verification procedure is correct w.r.t. a big step  operational semantics of Java bytecode programs. 
Moreover, we have
proven that the verification procedure for Java like programs
and Java like bytecode are syntactically equivalent (modulo names and types). 
%This scheme is actually part of the PCC architecture of the
%European project Mobius\footnote{the site name} which aims to resolve the problems
%of mobile and ubicuous computing via PCC. 
We have developed a prototype of a verification condition generator based on the weakest precondition calculus presented in this thesis, as well 
as a compiler from the corresponding subset of JML to BML.
These two components have been integrated in the JACK \cite{BRL-JACK} verification framework 
developed and supported by our research team Everest at INRIA Sophia Antipolis which has been initially designed for
 the verification of Java source programs annotated with JML specification.

We would like to give a brief description of the implementation of the verification condition generator.
 The extension of the tool to bytecode programs which we added also interfaces these theorem provers. The bytecode 
verification condition generator works as follows. For the verification of a class file containing BML specification, it will generate verification conditions for every
 method of this class including the constructors. For generating the verification conditions concerning a method implementation, first the control flow
 graph corresponding to the bytecode instruction is built. The latter is transformed into an acyclic control flow graph where the backedges are 
removed.
 Then the verification procedure proceeds by generating over every execution path in the control flow graph its corresponding verification conditions. 
For every path which terminates by throwing an uncaught exception, the postcondition is the specified exceptional postcondition for this case. For the paths which terminate normally, 
the normal postcondition is taken. For every path which terminates with an instruction which is dominated by a loop entry and whose direct successor is the same loop entry, the postcondition 
is the corresponding loop invariant. The bytecode verification in Jack uses the intermediate language for the verification conditions and thus, bytecode verification conditions 
 can be translated to several different theorem provers - Simplify \cite{Simpl05DNS} which is an automatic decision procedure, 
the Atelier B and the Coq interactive theorem prover assistants. 

The bytecode verification condition generator benefits also from the original user friendly interface of the JACK tool.  In particular, 
the user can see the verification conditions in his favorite language - Java, Simplify, Coq or B. The lemmas are classified 
to what part of the annotation they refer to, as for instance, a lemma which refers to the establishment of the postcondition, or the preservation of the loop invariant.
The hypothesis in the lemma also hold the index of the instruction from which they originate. 
We have used the prototype of the bytecode verification condition generator for the case studies presented in Chapter \ref{applications:optimComp}.

% JACK (short for Java Applet Correctness Kit) is designed as a plugin for the Java interface development
% environment eclipse. 
%% It was originally tailored to the verification of Java source programs 
%w.r.t. their JML specifications. The tool has an intermediate proof obligation language which allows to extend it easily to interface more 
% theorem provers. Thus, the tool interfaces several theorem provers - Simplify \cite{Simpl05DNS} which is an automatic decision procedure, 
%%the Atelier B and the Coq interactive
%theorem prover assistant. 

\section{Future work}
In the following, we identify the directions for extending the work presented in this thesis

\subsection{Language coverage of the verification condition generator}
The bytecode verification condition generator works only for the sequential fragment of Java. But realistic applications 
rely often on multi - threading which is difficult to verify against a functional specifications or security policies.
One of the important aspects of the correctness of multi - threaded programs is the absence of deadlocks, 
and race conditions. Such properties can be ensured  by type systems \cite{FA99TSL,flanagan00typebased} or static verification based on program logic \cite{FLL02ESC}.  
The absence of deadlock and race conditions is a first step in the verification of the functional correctness of multi threaded programs. In order to build a full 
verification scheme for checking functional correctness more has to be done.
The earliest work for  verification of  parallel programs is  the Owicki and Gries approach   
\cite{nipkow99owickigries}  and the rely - guarantee approach. However, 
the first approach is not modular and requires a large amount of verification conditions while for the second, the annotation procedure can not be automatised.

% Such techniques for reasoning over the correctness of parallel programs  exist.
% One of the first logic - based verification techniques for parallel programs is due to Owicki and Gries 
%\cite{nipkow99owickigries}  in which every point of parallel interference is annotated and then the verification consists in establishing that
% all the possible inter leavings of all the threads respect the annotation. This technique is on one hand not modular as the verification process 
%needs the implementation of every program component and on other hand the number of verification conditions may be very big.
% Another approach is the rely guarantee technique which uses a Hoare style verification conditions \cite{nieto03relyguarantee}.
%There, the program points of interference are annotated not only with the predicate that must hold
%at the point but also with rely and guarantee  conditions which express what conditions the program guarantees to the other threads and what 
%the program requires from the other threads. This technique although tempting because of its modularity and the smaller number of verification conditions is difficult to apply
%as for guessing the rely and guarantee conditions requires an in - depth understanding of the program to be verified.  
Extending our verification scheme for bytecode will certainly be based on a more recent work  where one of the basic concerns is to establish method atomicity  \cite{TES03CF}. 
The notion of a statement atomicity states  that however a statement is interleaved with other parallel programs, the result of its execution will not change.
The atomicity can be  detected via static checking \cite{TES03CF} using type systems. Thus, the program verification process is separated in two parts
- first checking for program atomicity  \cite{TES03CF} are done  
and then verifying the functional correctness using  methodologies for sequential programs as Hoare style reasoning. 
In this last approach in the case of Java, the basic concern is to establish the atomicity of method bodies, i.e. method 
execution does not depend on the possible interleaving with threads.
Recently, E.Rodriguez and al. in \cite{RodriguezDFHLR05} proposed an extension for JML for multi threaded
 programs. Their proposal introduces  new specification keywords which allow to express that a variable is locked or
 that a method is atomic.% Giving the semantics of these keywords is still an ongoing work but we consider that the meaning of these specification constructs does not differ on source and bytecode. 
    
 



\subsection{Property coverage for the specification language}
Another direction which may be pursued as a future work of the thesis  is the extension of the expressiveness of the specification language BML. 
So far, BML supports method contracts - method pre and post  conditions, frame conditions, intermediate annotations as for instance
loop invariants, class specifications as well as special specification operators.
These are very useful aspects which allow for dealing with complex properties and 
gives a semantics on bytecode level  to a relatively small subset of the 
high specification language JML which corresponds to JML Level 0 \footnote{ http://www.cs.iastate.edu/~leavens/JML/jmlrefman/jmlrefman\_2.html\#SEC19}. 
 But it is certainly of interest to support more features of JML in BML
as this will turn the latter language richer. However, the meaning  of JML constructs 
(at least from our experience up to  now) is the same as the meaning of their corresponding part in BML.  

 An important example is the  JML construct for pure methods which has been  identified as  a challenge in the position paper \cite{LeavensLeinoMueller06}. 
 These methods does not modify the program state and thus, pure methods can be used in specifications 
 (only side effect free  expressions may occur in expressions).
 This gives more expressive  specifications as with them, for instance, specification can talk about the result of method invocation or use pure methods
 as a predicate relating their  initial and final state. 
 Formalizing and establishing the meaning of pure methods is difficult and a literature exists for this problem \cite{DarvasMueller06}.
 As we said above, the treatment of pure methods is the same on source and bytecode.

Also, support for specification constructions for alias control is certainly useful  especially because it allows for a modular verification 
of class invariants and frame conditions.
The alias control is guaranteed through ownership type systems which check that only an owner of a reference can modify its contents.
 This can considerably improve the current implementation for the verification of object invariants  \cite{DietlMueller05}.
In particular, our way of proving object invariants is non modular - at every method call the invariants of all visible \todo{say what does it mean visibility}
objects must be valid and they are assumed to hold when the call is terminated; similarly, when a method body is verified in its precondition the invariants of all visible
objects are assumed to hold and at the end of the method body all these invariants must be established. 
In practice, it is very difficult to verify that all the invariants for the all visible objects in a method  hold.
In order to keep the number of the verification conditions reasonable, we check the invariants only for the current object this and the 
objects received as parameters which is not sound.

 
\subsection{Preservation of verification conditions}

So far, we have shown that non-optimizing Java compilation
 preserves the  form of the verification conditions on source and
 bytecode.  We identify two basic directions for future work:
\begin{description}
 \item[Source and non optimized bytecode verification conditions equivalent modulo] % implement the compiler from Java source pogs to bytecode pogs
We have experimented with the verification conditions on source and
 bytecode in JACK and saw that in practice they are almost equivalent
 syntactically. From one part, there are the difference in the types 
 supported on bytecode and source level. For instance, the JVM does not
 provide support for boolean type values which are basically encoded as
 integer values. The same is true for byte and short values.  Another
 difference is the identifiers for variables and fields. For instance, in Java
 names for fields, method local variables and parameters are their identifiers which are given by the
 program developer. On bytecode method local variables and parameters are encoded as elements of the
 method register table and field names are encoded as numbers of the constant
 pool table of the class. A  simple but useful extension to the prototype for
 bytecode verification is a compiler from source proof obligations to bytecode proof obligations
 which overcomes those differences. This can be considered also as a step
 towards the  building a PCC architecture where the certificate generation benefits from
 the source level verification and thus allows for treating sophisticated
 security policies.

\item[Relation between verification conditions on Java source and optimized Java bytecode]
 The equivalence  between verification conditions on source and the corresponding non optimized bytecode is important as it
 allows that bytecode programs  benefit from source verification. In particular, it makes feasible Proof Carrying Code
 for sophisticated client requirements.
 However, a step further in this direction is to investigate the 
 relation between source programs and their bytecode counterpart produced by an optimizing compiler.
 This is interesting for the following reasons.
 It is a fact that interpretation of bytecode on the JVM is slower than execution by its corresponding assembly code. 
 In order to speed up the execution time for a Java bytecode program, one might use 
 a just-in-time compilation which  translates on the fly the bytecode into the machine specific language. However, JIT compilation can potentially slow
 the execution exactly because it does compilation on the fly.  Another possibility is to perform 
 optimizations on the bytecode. Currently, most of the  Java compilers do not support much optimizations.
 However, there do already exist Java optimizing compilers, for instance the Soot optimization framework\footnote{http://www.sable.mcgill.ca/soot/} 
 and most probably the number of the Java optimizing compilers will increase with the evolution of the Java language.
 A first step in the latter direction is the work of C. Kunz et al.\cite{BGKRsas06} who give an algorithm for translating 
 certificates and annotations over a non optimized program into a certificate  and annotation for its optimized version.
 Their work addresses  optimizations like constant propagation, loop induction, dead register elimination etc. 
\end{description}
\subsection{Towards a PCC architecture}

The bytecode verification condition generator and the BML compiler is the first step towards a PCC framework. 
The missing  part is  the certificate format which comes along with the bytecode and which  is the evidence for 
that the bytecode respects the client requirements. Defining an encoding of the certificate should take into account several factors:
\begin{itemize} 
  \item certificate size must be reasonably small. This is important, for instance,  if the certified program comes over a network with a limitted bandwith
  \item certificates must be easily checked. This means that the certificate checker is  small and simple.
	       Of course, the code consumer might not want to spend all of its computation 
	      resources for checking that the certificate guarantees the program conformence to its policies.     
\end{itemize}

Note that the certificate size and its checking complexity are dual: the bigger the certificate is more manageable is the checking process and viceversa. 
The problem becomes even more difficult if the certificate must be checked on the device because of the computational and space constraints.
 


% towards.PCC
% For building a PCC framework from the components cited above 
% % there is still missing the proof certificate, the decision procedure
% that will be used by the producer for the certificate generation and the type checker used by the code
% client for checking the certificate. Important problems in this direction are
% \begin{itemize}
%  \item light weight verification condition generators. In particular, we refer 
%        to verification condition generation techniques which are simple and do not need
%	much computational resources. Because a verification condition generator always
%	form part of the trusted computing base on the client side, building such verification 
%	condition generators is important for on - device checking which rely on limitted computational 
%	resources  
  
%   \item generation of certificates. This is important for several reasons.
%         The certificate may certainly  arrive via the network and should not corrupt the performance 
%  
% 
% %  \item efficient type checker on the client site. This is in particular important 
%         if the device is with limitted resources where a complex certificate checking procedure
%         may corrupt the performance of the device
%        
%     
% \end{itemize}


 %To do this,  it is still missing the proof
%certificate, the decision procedure used by the code producer 
%for building the certificate  as well as the type checker used by the code
%client for checking the certificate. 

% to do. type systems
Another perspective in this direction is how   to encode type systems into the bytecode logic. 
Type systems provide a high level of automation. 
Their encoding in the logic can be useful as the certificate can be generated
 automatically and thus, avoids the user interaction. However, type systems are  conservative in the sense 
that they tend to reject a large amount of correct programs. A possible solution to this problem are hybrid certificates which combine both type systems and program 
logic. In this approach, the unknown code comes supplied with a  derivation in the logic generated potentially with the help of user interaction 
for the parts of the  code which can not be inferred by the type system.   The client side then applies a type inference procedure over  the
 unknown code and once it gets to the place in the parts of the code where the 
type inference does not work but for which there is a derivation in the certificate, he will type check that derivation.   
This is actually an approach which will be adopted in the Mobius project. 


The objective of this thesis was to 
give the basis for the a bytecode verification framework and to show that it is feasible. A further objective, pursued in the European project
 Mobius (short for Ubiquity, Mobility and Security) 
is to build basis for guaranteeing security and trust in program application in the presence of mobile and ubicuous computing. We hope that we have convinced
the reader for the importance of such techniques and in particular of the evolution from source verification to
 low level verification  and the necessity of an interactive verification process for building evidence for the security of unknown applications. 

%\section{Test coverage}
