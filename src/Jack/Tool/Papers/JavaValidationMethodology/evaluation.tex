\section{Evaluations}
\subsection{Industrial evaluations}
Two industrial evaluations have been carried internally by industrial
partners Axalto and Oberthur. The purpose of the present section is to
summarize the outcomes of these evaluations, that respectively focused
on the PayFlex case study and on a file system. More details can be
found in internal documents by the partners.

\subsubsection{Verification of banking case studies}
As a first test for Jack, we have studied a little banking
application.  This section presents different metrics concerning the
evaluation of the tool on this package.
\begin{table}
 \begin{tabular}{|l|c|c|c|c|c|c|c|} \hline
 Classes & Java & JavaDoc & JML & Proof & Automatic & Time to PO & Time to \\
  & lines & lines & lines & obligations & proof & generate (s) & prove (s)\\  \hline
 Transfert\_src  & 116 & 34 & 150 & 359 & 91\% & 22,5 & 238 \\
 AccountMan\_src & 105 & 51 & 236 & 269 & 82\% & 12,7 & 195 \\
 Currency\_src   &  93 & 20 &  28 &  50 & 96\% &  7,6 &  17 \\
 Balance\_src    &  64 & 38 &  58 & 335 & 95\% & 16,5 & 191 \\
 Spending\_rule  &  40 & 33 &  62 &  42 & 67\% & 13,6 & 217 \\ \hline
 \end{tabular}
\caption{banking applet metrics}
\label{MetricsTable}
\end{table}
Different remarks can be made from Table \ref{MetricsTable},
concerning the cost of adding JML annotations, the performance 
of the tool, as well as the cost associated to the proof.
The case study was also used to evaluate the following points:
\begin{itemize}
\item \emph{Cost of the annotations}
A first remark concerns the cost of the annotations.  The metrics
given here only concern the number of lines but one can see that the
documentation size (JavaDoc and JML) is one and half greater than the
code size.  So, writing the JML specification seems to be a costly
activity.  This remark can be moderated by two points: this
development was the first that we made, and annotations were added to
already existing code. So it suffers from its lack of abstraction, and
the annotations are really verbose. Moreover the time to specify is to
be compared to the time to test.

\item \emph{Interactive phase}
The automatic phases are quite responsive with some seconds to
generate proof obligations and attempt to discharge them
automatically.  Furthermore, the automatic proof rating is reasonable.
It is quite greater than the usual value for a B development (around
80\%).

Nevertheless, after automatic proof step, there remain 111 lemmas to
prove using the Atelier B interface.  An expert needs between 4 and 5
days to prove them.
\end{itemize}
Then, the new version of Jack has been evaluated on the Payflex case
study with the automatic mode using Simplify combined with the
interactive mode using Coq. Relevant changes on the Coq mode include:
\begin{itemize}
\item the transitivity relation for subtypes is now explicit:
      as a result, proofs about the payflex file system are better handled
      when one must decide whether a file of type T is also of type T' 
or not.
  
\item  the display of proof obligation hypothesis is clearer and more 
readable than on previous version

  
\item an prototype editor for coq inside Eclipse has been
  developed. It can be used as an alternative to ProofGeneral or
  CoqIDE.

  
\item the use of pure (i.e. side-effect free) method calls in JML
  annotations is better handled: the method call is no longer
  automatically replaced by its postcondition, instead it is displayed
  as a Java method call and the user can unfold it when desired (as is
  done in the Krakatoa tool for instance)
  
\item JML model variables (i.e. used for specification purpose only)
  are now well supported, and one can also use some methods declared
  with the keyword 'native' (it is not standard JML) in
  annotations.These methods are native with respect to a particular
  prover, that is they must be defined directly in the prover. That is
  a way to develop some 'specification libraries' to reuse when
  needed.
\end{itemize}
In order to reduce the overhead of writing annotations, Axalto has
also worked on the JML specification of the JC applet in general (as
the tool uses as input the JML specification). Some theoretical
results on the verification of specific properties have been obtained,
and a paper has been issued on the subject\cite{Rousset}. We start the
development of a tool with graphical interface to assist the writing
of annotations: from an UML class diagram or the code itself of an
application, it proposes some specification patterns to the user
(non-null by default for references, ranges for variable of integer
type and arrays length), then translation to adequate class invariants
and method preconditions is performed. In practice, annotations
generated in that manner reveal sufficient to prove that each method
'does not go wrong' (i.e. is runtime error free) and all the
verification conditions are filled in automatic mode with Simplify
using tools like Krakatoa or Jack.

\subsubsection{Verification of a file system}
This section describes the evaluation of the JACK product by Oberthur
Card Systems to specify Java Card programs in a smart card technology
context.

\subsubsection{File system specification}
For our evaluation, we choose a case study in line with the objectives
of the INSPIRED project. We specify a large part of a file system on
smart card. The file system stores user data and manages access
control on these data. It is for example an important security feature
of e-passports.

In our evaluation, we consider a realistic file system containing
three kinds of files:

\begin{itemize}
\item Elementary files (EF): contain user data. There are several types of EF:
\begin{itemize}
\item Binary files: are files without internal structure
\item Record files: where data are stored in records. These files can be:
\begin{itemize}
\item With records of variable length
\item With record of fixed length
\item With a cyclic structure
\end{itemize}
\end{itemize}
\item Dedicated files (DF): are similar to \lq\lq folders\rq\rq\; they contain
other EF or DF

\item Master file (MF): it is a particular DF that is the root of the
file system

\end{itemize}
We specified a complete set of commands that are:\\
\begin{tabular}{l}
CREATE FILE \\
SELECT FILE \\
READ BINARY \\
UPDATE BINARY \\
READ RECORD \\
UPDATE RECORD \\
APPEND RECORD \\
DELETE FILE
\end{tabular}\\
These commands are specified in ISO-IEC 7816 as APDU (Application
Protocol Data Unit). We recall the APDU structure. It comprises:
\begin{itemize}
\item CLA (1 byte): define the command class 
\item  INS (1byte): command instruction
\item  P1 (1 byte) and P2 (1 byte): parameters of the command
\item  LC (1 byte): command length
\item  Data field (N bytes): data of the command
\item LE (1 byte): expected length of the response
\end{itemize}
The response APDU contains two fields:
\begin{itemize} 
\item Body: containing the data returned by the card
\item SW1||SW2: status of the command execution (9000 means \lq\lq correct
execution\rq\rq  and others values are reserved for error and warning
status).

\end{itemize}
Our specification uses heavily an existing Java Card API specification
developed by Erik Poll in the VERIFICARD project. Other important
ingredients are: 
\begin{itemize}
\item Specification of basic operations on APDU (offset
computation, slicing\ldots) 
\item Arithmetic operation: this point is quite difficult because we
  use heavily the cast between the types short (signed) and byte
  (unsigned) Operations on Bits: ({\tt \&\&}, {\tt <<}, \ldots) : to
  avoid a complex specification, we put them as axioms.
\end{itemize}
Another important ingredient is the access condition control
that is performed by the following method:

 checkAccessCondition(oper,fichier)

that take two parameters:
\begin{itemize}
\item
oper: describes the requested operation. Possible values are:
\\
\begin{tabular}{l}
AMB\_CREATE\_DF \\
AMB\_CREATE\_EF \\
AMB\_EF\_READ  \\
AMB\_EF\_APPEND\_RECORD \\
AMB\_DF\_DELETE  \\
AMB\_EF\_DELETE \\
AMB\_EF\_WRITE 
\end{tabular}\\
\item fichier: is the reference on the file on which the operation will act. 
\end{itemize}.
The result of the method has two possible values:  ACCESS GRANTED 
or ACCESS DENIED.

\paragraph{Complete example of specification}

We present in this section the JML specification of he READ BINARY
command. The corresponding method has the following signature:

readBinary(byte p1,byte p2,byte Lc,byte[] buffer, short offset)
We distinguish three different steps for the specification development: 
\begin{itemize}
\item Lookup the file
\item Offset computation
\item Copy of the found data in the output buffer
\end{itemize}

\subparagraph{Lookup the file}
The parameters p1, p2 are used to identify the file. A private method
is especially defined for this operation.

\begin{lstlisting}
ElementaryFile getBinary(byte P1,byte p2)}.

/*@ requires true;
@ ensures ((byte)(p1 & (byte)(0x80) == 0)
@               ==> \result == m_oCurrentFile;
@ ensures ((byte)(p1 & (byte)(0x80) != 0)
@               ==> (\exists ElementaryFile ef;
@                       ef.m_bSFI == (p2 & 0x7F);
@                       \result == ef)
@ ensures \typeof(\result) <: \type(Elementaryfile)
@*/
\end{lstlisting}
This file research is based on a of a linked list that we completely specified, for operations: create, insert, remove. 

\subparagraph{Offset computation}
This part is performed by the following method:
\begin{lstlisting}
getOffset(byte p1,byte p2)

/*@ requires true;
@ modifies \nothing;
@ ensures ((byte)(p1 & 0x80)== 0x80)==>
@               \result == Util.makeShort((byte)0,P2);//(1)
@ ensures ((byte)(p1 & 0x80)!= 0x80)==>
@               \result == Util.makeShort(P1,P2);//(2)
@ ensures \result >= 0;//(3)
@*/
\end{lstlisting}
\subparagraph{Copy data}
This step is performed without difficulties in JML. 

We put together all the previous pieces of specification to build the specification of the READ BINARY command.
\begin{lstlisting} 
/*@ requires apdu != null;
@ requires apduR != null;
@ modifies m_oCurrentFile;
@ modifies ISOException.systemInstance.theSw[0],
ISOException.systemInstance.systemInstance._reason;
@ ensures (\exists TransparentFile ef;
@               ef == getBinaryFile(p1, p2);
@               ef.checkfFileStateAccessRead() ==>
@               checkAccessCondition(File.AMB_DEF_READ, ef)
@                       == ACCESS_GRANTED
@               ==> (
@                       (((byte)0x00 & length)+getUpdateOffset(p1, p2)<= ef.length) ==>
@                       (
@                               (\forall short s;
@                                       0 <=s
@                                       && s<((byte)0x00 & length);
@                                            ef.data[s+getUpdateOffset(p1,p2)] ==apduR[s])
@                               &&
@                               \result == ((byte)0x00 & length)
@                       ))
@               &&
@               (
@                       (((byte)0x00 & length)+getUpdateOffset(p1, p2)> ef.length) ==>
@                       (
@                               (\forall short s;
@                                       0 <=s
@                                       && s < (ef.length -getUpdateOffset(p1, p2));
@                                               ef.data[s+getUpdateOffset(p1, p2)]== apduR[s])
@                               &&
@                               \result == (ef.length -getUpdateOffset(p1, p2))
@                       )
@               )
@               &&
@               ((byte)(p1 & (byte)0x80) == (byte)0x80
@               ==> ef == currentFile
@       )));
@       signals (ISOException ex) true;
@*/
\end{lstlisting}
A complete example of proof As example, we extract from our
development a proof from the APPEND RECORD command. It concerns 
the creation of a record and constraints on its size.

The proof is composed of the following sections: first, it contains
a header to import all the necessary libraries:
\begin{lstlisting}
Add LoadPath "c:\coq\lib\theories\bool" as Coq.Bool.
Require Import Bool.
Add LoadPath "c:\coq\lib\theories\ZArith" as Coq.ZArith.
Require Import ZArith.
Add LoadPath "c:\coq\lib\theories\Logic" as Coq.Logic.
Require Import Classical.
Require Import "C:\eclipse\workspace\IDONE\JPOs
\com_oberthurcs_javacard_common_filesystem_VariableRecordElementaryFile".

Load "C:\eclipse\workspace\IDONE\JPOs\localTactics.v".

Open Scope Z_scope.
Open Scope J_Scope.
\end{lstlisting}
Then, it contains a declaration of Java Card variables:
\begin{lstlisting}
Section JackProof.
Variable Result_setData_3: bool.
Variable byteelements_0: REFERENCES -> t_int -> t_byte.
Variable filesystem_Record_m_sLength_0: 
REFERENCES -> t_short.
Variable filesystem_Record_m_baData_0: 
REFERENCES -> REFERENCES.
Variable filesystem_Record_index_0: 
REFERENCES -> t_short.
Variable filesystem_Record_state_0:
 REFERENCES -> t_short.
Variable util_LinkedObject_next_0: 
REFERENCES -> REFERENCES.
Variable util_LinkedObject_indice_0: 
REFERENCES -> t_short.
Variable util_LinkedObject_previous_0: 
REFERENCES -> REFERENCES.
Variable util_LinkedObject_jmlPrevious_0: 
REFERENCES -> REFERENCES.
Variable util_LinkedObject_jmlNext_0: 
REFERENCES -> REFERENCES.
Variable f_java_lang_Object_owner_0: REFERENCES -> REFERENCES.
Variable newObject_8: REFERENCES.
Variable this: REFERENCES.
Variable l_p_baSource: REFERENCES.
Variable l_p_sOffset: t_short.
Variable l_p_sLength: t_short.
Variable l_p_sRecordSize: t_short.
\end{lstlisting}
Then one finds hypotheses coming from the specification of called methods:
\begin{lstlisting}
Variable hyp1 : (Result_setData_3 = true).
Variable hyp2 : (not ((newObject_8 = null))).
Variable hyp3 : (not ((newObject_8 = null))).
Variable hyp4 : (not ((newObject_8 = null))).
Variable hyp5 : (not ((instances newObject_8))).
Variable hyp6 : (newObject_8 <> null).
Variable hyp7 : (((j_lt 255 l_p_sRecordSize)) ->
((filesystem_Record_m_sLength_0 newObject_8) = 255)).
Variable hyp8 : (((j_le l_p_sRecordSize 0)) ->
((filesystem_Record_m_sLength_0 newObject_8) = 255)).
Variable hyp9 : (((((j_lt 0 l_p_sRecordSize)) /\ ((j_le l_p_sRecordSize 255))))
 -> ((filesystem_Record_m_sLength_0 newObject_8) = l_p_sRecordSize)).
Variable hyp10 : forall x1258, (x1258 <> newObject_8) ->((f_m_sLength x1258) 
= (filesystem_Record_m_sLength_0 x1258)).
Variable hyp11 : (forall (x1259:REFERENCES), 
(~((singleton REFERENCES (filesystem_Record_m_baData_0 newObject_8)) x1259)) 
-> ((byteelements_0 x1259) = (byteelements x1259))).
Variable hyp12 : forall (Result_checkData_4: bool),
        ((((((((((((j_le 0 l_p_sLength)) /\ 
((j_le l_p_sLength (arraylength (filesystem_Record_m_baData_0 newObject_8)))))) /\ 
((j_le (j_add l_p_sLength l_p_sOffset) (arraylength l_p_baSource))))) ->
(Result_checkData_4 = true))) /\ 
((((((((j_lt l_p_sLength 0)) \/ 
((j_lt (arraylength (filesystem_Record_m_baData_0 newObject_8)) l_p_sLength)))) \/ 
((j_lt (arraylength l_p_baSource) (j_add l_p_sLength l_p_sOffset))))) ->
(Result_checkData_4 = false))))) ->
(Result_setData_3 = Result_checkData_4))).
Variable hyp13 : forall (Result_checkData_5: bool),
        ((((((((((((j_le 0 l_p_sLength)) /\ 
((j_le l_p_sLength (arraylength (filesystem_Record_m_baData_0 newObject_8)))))) /\ 
((j_le (j_add l_p_sLength l_p_sOffset) (arraylength l_p_baSource))))) ->
(Result_checkData_5 = true))) /\ 
((((((((j_lt l_p_sLength 0)) \/ 
((j_lt (arraylength (filesystem_Record_m_baData_0 newObject_8)) l_p_sLength))))
 \/ ((j_lt (arraylength l_p_baSource) (j_add l_p_sLength l_p_sOffset))))) ->
(Result_checkData_5 = false))))) ->
(((Result_checkData_5 = true)) ->
(forall (l_s171: t_short), 
(((((j_le 0 l_s171)) /\ ((j_lt l_s171 l_p_sLength)))) ->
(((byteelements_0 (filesystem_Record_m_baData_0 newObject_8)) l_s171) 
= ((byteelements_0 l_p_baSource) (j_add l_s171 l_p_sOffset)))))))).
Variable hyp14 : forall x1260, (x1260 <> newObject_8) ->
((f_m_baData x1260) = (filesystem_Record_m_baData_0 x1260)).
\end{lstlisting}
Statements concerning variable that are not modified in this method
\begin{lstlisting}
Variable hyp15 : forall x1261, (x1261 <> newObject_8) ->
((f_index x1261) = (filesystem_Record_index_0 x1261)).
Variable hyp16 : forall x1262, (x1262 <> newObject_8) ->
((f_state x1262) = (filesystem_Record_state_0 x1262)).
Variable hyp17 : forall x1263, (x1263 <> newObject_8) ->
((f_next x1263) = (util_LinkedObject_next_0 x1263)).
Variable hyp18 : forall x1264, (x1264 <> newObject_8) ->
((f_indice x1264) = (util_LinkedObject_indice_0 x1264)).
Variable hyp19 : forall x1265, (x1265 <> newObject_8) ->
((f_previous x1265) = (util_LinkedObject_previous_0 x1265)).
Variable hyp20 : forall x1266, (x1266 <> newObject_8) ->
((f_jmlPrevious x1266) = (util_LinkedObject_jmlPrevious_0 x1266)).
Variable hyp21 : forall x1267, (x1267 <> newObject_8) ->
((f_jmlNext x1267) = (util_LinkedObject_jmlNext_0 x1267)).
Variable hyp22 : forall x1268, (x1268 <> newObject_8) ->
((f_owner x1268) = (f_java_lang_Object_owner_0 x1268)).
Variable hyp23 : (instances this).
Variable hyp24 : (subtypes (typeof this) (class c_VariableRecordElementaryFile)).
\end{lstlisting}
Hypotheses extracted from the class where is performed the proof:
\begin{lstlisting}
Variable hyp25 : 
((union REFERENCES instances (singleton REFERENCES null)) l_p_baSource).
Variable hyp26 : (((l_p_baSource <> null)) ->
(subtypes (typeof l_p_baSource) (array (class c_byte) 1))).
Variable hyp29 : (l_p_baSource <> null).
Variable hyp30 : (j_le 0 l_p_sOffset).
\end{lstlisting}
The statement of the lemma to prove:
\begin{lstlisting}
Lemma l:
forall (Result_checkData_0: bool),
        ((((((((((((j_le 0 l_p_sLength)) /\ 
((j_le l_p_sLength (arraylength (filesystem_Record_m_baData_0 newObject_8)))))) /\ 
((j_le (j_add l_p_sLength l_p_sOffset) (arraylength l_p_baSource))))) ->
(Result_checkData_0 = true))) /\ 
((((((((j_lt l_p_sLength 0)) \/ 
((j_lt (arraylength (filesystem_Record_m_baData_0 newObject_8)) l_p_sLength)))) \/ 
((j_lt (arraylength l_p_baSource) (j_add l_p_sLength l_p_sOffset))))) ->
(Result_checkData_0 = false))))) ->
(((Result_checkData_0 = true)) ->
((((((filesystem_Record_m_sLength_0 newObject_8) = l_p_sRecordSize)) \/ 
(((filesystem_Record_m_sLength_0 newObject_8) = 255)))) /\ 
((forall (l_s128: t_short), (((((j_le 0 l_s128)) /\ ((j_lt l_s128 l_p_sLength)))) ->
(((byteelements_0 (filesystem_Record_m_baData_0 newObject_8)) l_s128)
 = ((byteelements_0 l_p_baSource) (j_add l_s128 l_p_sOffset)))))))))).
\end{lstlisting}
The proof:
\begin{lstlisting}
Proof with autoJack; arrtac.
intros.
split.
generalize (dec_Zle  l_p_sRecordSize 0).
intro.
inversion_clear H1.
right.
apply (hyp8 H2).
generalize (Znot_le_gt l_p_sRecordSize 0 H2).
intro.
generalize (dec_Zle  l_p_sRecordSize 255).
unfold Decidable.decidable.
intro.
inversion_clear H3.
generalize (Zgt_lt l_p_sRecordSize 0 H1).
intro.
left.
auto.
generalize (Znot_le_gt l_p_sRecordSize 255  H4).
intro.
generalize (Zgt_lt l_p_sRecordSize 255 H3).
intro.
right.
apply (hyp7 H5).
intros.
rewrite H0 in H.
generalize (hyp13 true H).
intros.
cut (true = true).
intro.
apply (H2 H3 l_s128 H1).
reflexivity.
Qed.
End JackProof.
\end{lstlisting}
\paragraph{Conclusion}
We experiment the JML specification on a non-trivial example, showing that this language is suitable for smart card applications. Especially because we use a specification style that is very close from the implementation. However we remark that in some cases, we are obliged to use complex techniques like: auxiliary functions, sequence memorisation\ldots


Furthermore, Jack is based on an old JML syntax and unfortunately it
does not support some very interesting functionalities:
\begin{itemize}
\item Model methods: that are methods exclusively used for specification and that can be used as a toolbox to solve common problems

\item Refinement: with the file jml-refined one can refine JML
specification at several level of precision
\end{itemize}
In our experiment, we tried to used JML in a Common Criteria
certification framework. Our conclusion is that it fulfil partially
the requirements of the development activity (ADV): FSP (functional
specifications), HLD (high level design) and LLD (low level design),
because it permits the complete description of the interfaces at each
of these levels.

\begin{quote}
ADV FSP.3.3C The functional specification shall describe the
purpose and method of use of all external TSF
interfaces, providing complete details of all effects,
exceptions and error messages.

ADV HLD.3.8C The high-level design shall describe the purpose
and method of use of all interfaces to the subsystems
of the TSF, providing complete details
of all effects, exceptions and error messages.

ADV LLD.2.5C The low-level design shall describe the purpose
and method of use of all interfaces to the modules
of the TSF, providing complete details of
all effects, exceptions and error messages.
\end{quote}
However it is not possible to formally prove the correspondence between these levels (FSP, HLD and LLD). A semi-formal correspondence based on matrix is possible. To have a formal proof of correspondence, the jml-refined functionality is necessary.    





\subsection{Results}  \label{results}


We have an implementation of the JML compiler \ref{comJML} and the bytecode verification condition generator based on the weakest precondition calculus (Section \ref{wp}) 
which are integrated in JACK. 

The performed tests show that JML compilation augments around twice the file size. 
For the example given at fig.~\ref{replaceSrc}, the class file without the specification extensions is 548 bytes, 
and the class with the BCSL extension BCSL is 954 bytes. 
Another important point about the size of the bytecode specification is that it is proportional to the source specification: 
the more specific is the specification, the greater will be the size of the class file. 


We studied the relationship between the source code proof obligations generated 
by the standard feature of JACK and the bytecode proof obligations generated by our implementation over the corresponding bytecode produced by a nonoptimizing compiler.  
Basically, they are the same modulo the names of the program variables.

 We illustrate these differences by giving the proof obligations on source and bytecode level respectively of the example program from the previous sections. In particular, Fig. \ref{vcLoopPreserv} gives a comparison between the proof obligation for invariant preservation and  Fig. \ref{vcEnsures} compares the source and bytecode proof obligation
 for one of the cases of the postcondition correctness (as there are several return instructions).

The verification conditions on bytecode and source level for the postcondition  correctness given in Fig. \ref{vcEnsures}
 have the same shape modulo names (see Section \ref{comJML} for how method local variables and field names are compiled).
In Section \ref{comJML} we showed that the postcondition for our example was compiled by performing structural transformations 
in an equivalent expression. Despite those changes, the goals (which is actually the postcondition) on bytecode and source level are not only equivalent
but structurally the same. Still, in the bytecode proof obligation we have one more hypothesis than on source level. The extra hypothesis in the bytecode
version of the proof obligation is related to the fact that the result type is boolean but the JVM encodes boolean expressions as integers.

Fig. \ref{vcLoopPreserv} shows the proof obligations for the loop preservation. As you can see the hypothesis and the goal have the same ``shape'' on bytecode and source code and the differences are due to the variable names.



\begin{figure}{!h}

$$\begin{array}{ll}
Hypothesis \ on \ bytecode:  & Hypothesis \ on \ source \ level:  \\
 & \\


\begin{array}{l}
 \register{1} \neq \\
\#19 (\register{0})[\register{2}\_at\_ins\_22]
\end{array}  

&  
\begin{array}{l}
 \srcVar{obj} \neq \\
  ListArray.list(\this)[\srcVar{i}\_at\_ins\_26] 
\end{array}   \\



 & \\

\#19(\register{0}) \neq \Mynull &  ListArray.list( \this) \neq \Mynull\\


& \\

\begin{array}{l}
  len(\#19 (\register{0})) > \\
 \register{2}\_at\_ins\_22 
\end{array}
& 
\begin{array}{l}
  len(ListArray.list(\this)) > \\
\srcVar{i}\_at\_ins\_26
\end{array}         \\ 



 & \\

 \register{2}\_at\_ins\_22 \geq 0 &   \srcVar{i}\_at\_ins\_26    \geq 0    \\


 & \\
\begin{array}{l}
  \register{2}\_at\_ins\_22 < \\
  len(\#19(\register{0}))
\end{array} &

\begin{array}{l}
  \srcVar{i}\_at\_ins\_26 <\\
  len(ListArray.list(\this))
\end{array}   \\


 & \\
\begin{array}{l}
  \register{2}\_at\_ins\_22 \leq \\
  len( \#19(\register{0}))
\end{array} 
&  
\begin{array}{l} 
  \srcVar{i}\_at\_ins\_26 \leq \\
  len(ListArray.list(\this))
\end{array}   \\


 &\\
 \register{2}\_at\_ins\_22 \geq 0 &   \srcVar{i}\_at\_ins\_26 \geq 0 \\



 &\\
 \begin{array}{l} 
         \forall  var(0). \ 0 \leq var(0) \wedge var(0) < (\register{2}\_at\_ins\_22) \Rightarrow \\
                \Myspace    \#19(\register{0})[var(0)] \neq \register{1}
      \end{array} &        
      \begin{array}{l} 
             \forall  var(0). \ 0 \leq var(0) \wedge var(0) < (\srcVar{i}\_at\_ins\_26) \Rightarrow \\
                 \Myspace       ListArray.list(\this)[var(0)] \neq \srcVar{obj}
      \end{array}  \\

 typeof(\register{0}) <: ListArray &    typeof(this) <: ListArray     \\

& \\
& \\
Goal \ on \ bytecode: & Goal \ on \ source \ level: \\

& \\

  \begin{array}{l}
               1 + \register{2}\_at\_ins\_22 \leq  len(ListArray.list(\register{0}))  \\

               1 + \register{2}\_at\_ins\_22 \geq 0 \\

               \forall  var(0). 0 \leq var(0) \wedge var(0) < 1 + \register{2}\_at\_ins\_22 \Rightarrow \\
                   \Myspace  ListArray.list(\register{0})[var(0)] \neq \register{1} 

       \end{array}
& 

       \begin{array}{l}
             1 + \srcVar{i}\_at\_ins\_26 \leq  len(ListArray.list(this))  \\
	     \\
             1 + \srcVar{i}\_at\_ins\_26 \geq 0 \\
	     \\
             \forall  var(0). 0 \leq var(0) \wedge\\
	     \Myspace  var(0) < 1 + \srcVar{i}\_at\_ins\_26 \Rightarrow \\
                  \Myspace  ListArray.list(this)[var(0)] \neq \srcVar{obj} 
       \end{array}   

 
\end{array}$$



\caption{Source and Bytecode verification condition for loop preservation for method \texttt{ListArray.isElem} }
\label{vcLoopPreserv}
\end{figure}








\begin{figure}[!h]
$$\begin{array}{ll}
Hypothesis \ on \ bytecode:  & Hypothesis \ on \ source \ level:  \\
 & \\

\begin{array}{l}
   \register{2}\_at\_ins\_22 \geq \\
   len(\#19(\register{0}))
\end{array} &

\begin{array}{l}
   \srcVar{i}\_at\_ins\_26 \geq \\
   len(ListArray.list(\this))
\end{array} \\
& \\


\#19(\register{0}) \neq \Mynull & ListArray.list(\this) \neq \Mynull \\

 & \\
\register{2}\_at\_ins\_22) \leq len(\#19(\register{0})) &    \srcVar{i}\_at\_ins\_26  \leq  len(ListArray.list(\this)) \\
& \\

\register{2}\_at\_ins\_22 \geq 0 &    \srcVar{i}\_at\_ins\_26  \geq 0  \\
& \\

\begin{array}{l}
   \forall  var(0). \  0 \leq var(0) \wedge  \\
   \Myspace var(0) < \register{2}\_at\_ins\_22 \Rightarrow \\
   \#19(\register{0})[var(0)] = \register{1} 
\end{array} 
& 
\begin{array}{l}
   \forall  var(0). \  0 \leq var(0) \wedge \\
        \Myspace var(0) < \srcVar{i}\_at\_ins\_26 \Rightarrow \\
   ListArray.list(\this)[var(0)] = \srcVar{obj} 
\end{array} 

\\

& \\
 typeof(\register{0}) <: ListArray & typeof( \this) <:  ListArray  \\

& \\

0=0 \vee 0=1 & \\
& \\
& \\
Goal \ on \ bytecode: & Goal \ on \ source \ level: \\

& \\
   \begin{array}{l}
       \Myfalse  \iff \exists  var(0) . \ 0 \leq var(0) \wedge \\
       var(0) < len(Binc.\#19(\register{0})) \wedge\\
       \#19(\register{0})[var(0)] = \register{1}  
   \end{array}

&
  \begin{array}{l}
       \Myfalse \iff \exists  var(0) . \ 0 \leq var(0) \wedge \\
       var(0) < len(Binc.ListArray.List(this)) \wedge\\
       \#19(this)[var(0)] = \srcVar{obj}  

  \end{array}
\end{array}
$$

\caption{Source and Bytecode verification condition for one case of the postcondition correctness }
\label{vcEnsures}
\end{figure}
 















%\subsubsection{Example}
% We give a simple example of how the \wpi \ works. Block $\blockm{6}$ (starts at instr. \texttt{6}) in Fig.~\ref{blockBC} ends with a branching instruction and in the case when the condition is true (the current element of the array is not equal to the first parameter of the method \texttt{replace}) the execution will continue at $\blockm{19}$. Below we give the part of the weakest precondition for block $\blockm{6}$ in case the control flows to block $\blockm{19}$( the condition of its last instruction holds and in this case 
%the predicate $pre(b^{6}, b^{19})$ is $\wpi(\blockm{19})$).  The implications with conclusion \Myfalse \ stand for the possible exceptions \texttt{NullPointer} and \texttt{ArrayIndexOutOfBound} exceptions that may be thrown (as no postcondition is specified explicitely for these cases of abnormal termination, the one by default is taken). 

%\input wpExample.tex



\subsection{Bytecode Verifier}
The tools have also been tested on a bytecode verifier java implementation. A termination proof has been provided.
A specific implementation has been coded with on one hand the main loop which remain unchanged whatever the specifications of the virtual machine Java chosen, and other instructions and memory states which depends on selected model.
\subsubsection{Implementation and Modelisation}
The main loop is in a package which contains abstract classes: 
the instructions and the states are implemented in a more generic way.
The package containing the implementation is composed from the instructions for the standard Java types and of the states of memory typing. 
\paragraph {Memory states}
The memory states are represented by the State class, which is an abstract class.  It does not contain any precise definition of the memory: 
one has no information on the stack or on the local variables table. 
The implementation is relatively simple: 
it is a class which contains a type stack and a table of the types of the local variables. 
Functions allowing to read simply these structures and to generate verification error in the cases of misuse are defined.   
\paragraph{Instructions}
The instructions are also represented by an abstract class: 
the class Instruction.  
Since in the Kildall algorithm each instruction is associated to a memory state,  the Instruction class has a field of the State type. 
An instruction can also have one or more successors. 
This relation is represented by a field which is the list of the successors of the instruction.  
One of the other aspects is the fact that on associate to each instruction a boolean field to determine if it has been modified or not.

Several properties of the bytecode verifier are formalised in this class.
First of all one verifies that the successors of the instruction are well included in the others instructions of the program. 
If these successors pointed towards external instructions, an verification error would be returned. 

The others important properties concern the pure function {\tt
buildNewState}.  This function builds the typing state of the
execution of an instruction on the current state.  This construction
can fail if the instruction tries for example to pop an element when
the stack is empty.  If it succeeds, a new non null state is built.


Around ten instructions have been implemented: {\tt load} and {\tt
blind} for the access to local variables, {\tt push} and {\tt pop} to
obtain or put element on the stack, {\tt op1} and {\tt op2} which is
two operators who consume both the two top element of the stack and
which replaces them by a result of a certain type, {\tt ifle} and {\tt
jump} instructions of jump towards another instruction successor, {\tt
nop} the instruction which does not do anything and finally {\tt stop}
which is an instruction which does not have a successor.  These
instructions have an associated type in the OperandType class, who can
be None, Type1 or Type2.  Those are the minimal instructions to have a
Java-like program.

\paragraph {The main loop}
The main loop is implemented in the Verifier class.  It is not an
abstract class because it uses the properties of the State and
Instruction abstract classes to verify an instruction set on
particular states.  This class provides two functions, the function
{\tt verify} in which the loop is written and the function {\tt check}
which verify an instruction.

%The m \ 'ethode check ensures that all \ 'states of the successors of an instruction donn \ 'ee,    are larger or \ 'equal that the \ 'states before the ex \ 'ecution of the m \ 'ethode. This   propri \ 'and \ 'E seems simple \ `has to express but it implies several Pr \ 'erequis.  First of all it should be guaranteed that the successors of the instructions point all worms of   valid instructions. Then that all the instructions are diff \ 'erentes of no one and that theirs  \ 'states are too diff \ 'erents of no one them.    The Pr \ calculation weaker 'econdition of Jack forces us \ `has to add these propri \ 'and \ 'be    Li \ 'ees \ `with the S \ 'emantic of the language Java.  .   %Pour to facilitate the evidence I have \ 'and \ 'E oblig \ 'E  %de to add a certain number of assertions.    

The {\ tt verify} method is the main loop of the bytecode verifier.
It contains two nested loops. The internal one is a {\tt for} loop
which iterates on the instructions and verify all the quoted
instructions (as described in the Kildall algorithm). The termination
of the internal loop is easy to prove.  The {\tt for} loop executes as
many time as there are numbers in the table.  The external {\tt while}
loop stops the algorithm when no more instruction typing state is
modified.  This termination is not obvious to prove, especially with
JML, since it only allow to prove loop termination by giving an
integer variant.

Since the states have to be used to show the algorithm termination,
one has to make correspond each state with an integer. Thus at each
loop iteration, the integer associated with the state either increase
or preserve the same value; and it exists a maximum value.


\begin{figure}[ht]  
\begin{center}    
\begin{tabular}{p {0.4 \textwidth} c c c c}  
{\bf Classes:} & State & Instruction & Verifier \\  
{\bf Lines of code:} & 14 & 47 & 66 \\  
{\bf Lines of annotations:} & 20 & 54 & 81 \\  \raggedright 
{\bf Proof obligations:} & 26 & 129 & 627 \\  \raggedright 
{\bf Automatically proved proof obligations:} & 17 & 93 & 112 \\  
{\bf Average length of a non-automatic proof:} & 3 & 6 & 12 \\    
\end{tabular}  
\end{center}  
\caption{Some statistics on proof}  
\label{stats}  
\end{figure}    
\subsubsection{Proofs}
The first proofs are relatively easy.  The State class is proved
almost automatically; these proofs are not due to the code of the
methods but some standard verification Jack adds to each methods,
mostly to verify that all the public invariants are not broken after
the execution of each method.  Since the methods in the State class
are quite simple (and do not break the invariants), the automation is
good for these kind of proof obligations.  The only special case is
for the constructor where it is necessary to break a disjunction ({\tt
instance S $\vee$ S = null}) to prove the invariants.


The Instruction class has also been relatively easy to prove.  A
significant number of proof was done automatically (approximately 90
\%) and as for the State class this was mainly the verifications of
invariants; then majority of proof could be trivially resolved as most
of the methods are observer or accessors to private fields.  Two
methods verify some properties over the instructions, namely {\tt
checkDomain} which checks if the successors of the instruction are
contained in the program and {\tt isSuccessor} which test if the
instruction passed as a parameter is in the list of the successors of
the current instruction.  Thes two methods contains loops, so the
proof obligations generated are quite differents. We have to use some
arithmetic to prove their termination, which is not automated since it
does not appear often.  Finally the Verifier class was harder to
prove. One of the main reason is that the main method contains 2
loops; the inner one, easy to verify (a bit more difficult than the
ones contained in the Instruction class) because it simply consult
each intruction of the instruction array representing the program, but
with much more properties expressed on it. The proof obligations
generated for the main loop containing the inner loop are harder to
prove because the loop terminates only if it reaches a fixpoint, so it
cannot be expressed by simple arithmetic like the previous ones.  The
lemmas of this class were containing too many hypotheses to be
automatically proved.  In fact for each class some properties are
brought from the previous ones, this adds lots of hypothesis but it
adds too to the complexity of the proofs.  That's why there was some
kind of exponentional growth in the size of the proof obligations for
each class when they were defined.  Around 500 proof obligations had
to be solved manually.  Some of them were obvious and were resolved
with quite the same script, but the script cannot be automated.  Some
of them were complex: the proof script became little large (an average
of 30 steps).  As one should expect the lemmas concerning the loop
invariant of the verify method and its initialization were the most
difficult to prove.


%The first proofs are relatively easy. 
%The State class is proved almost automatically; 
%except for the constructor where it is necessary to break a disjunction ({\tt instance S $\vee$ S = null}) to prove the invariants.

%The Instruction class has also been relatively easy to prove. 
%A significant number of proof was done automatically (approximately 90 \%); 
%then majority of proof could be trivially resolved, except some lemma concerning  a loop termination.    

%Finally the Verifier class was harder to prove.
%The lemmas were containing too many hypotheses to be automatically proved.
%Around 500 proof obligations have to be resolved manually.
%Some of them was obvious and were resolved with quite the same script, but the script cannot be automated.
%Some of them was complex: the proof script became little large (an average of 30 steps).
%The lemmas concerning the loop invariant of the verify method and its initialization were the most difficult.


\subsection{Low-Footprint Java-to-Native Compilation}


% Keywords: embedded devices, Java, Java card, exceptions, ahead of time compilation






\subsection{Generation of the Verification Conditions}
\label{vcGen}
For generating the verification conditions, we use a bytecode verification condition generator (vcGen) based on a bytecode weakest precondition calculus~\cite{JBL05MP}. The weakest precondition function $wp$ returns for every instruction \texttt{ins}, normal postcondition $\psi$, and exceptional function $\excPost$ the weakest predicate \\ $\wpi( \texttt{ins} ,\psi ,\excPost)$ such that if it holds in the pre-state of the instruction \texttt{ins} and if the instruction terminates normally, then the normal postcondition $\psi$ holds in the poststate and if \texttt{ins} terminates on an exception \texttt{Exc}, then the predicate $\excPost(\texttt{Exc})$ holds. From the annotated bytecode the vcGen calculates a set of verification conditions for every method of the application. The verification conditions for a method are generated by tracing all the execution paths in it starting at every \texttt{return}, \texttt{athrow} and loop end instruction up to reaching the method entry point. During the process of generation of the verification conditions, for every instruction that may throw a \verb!Runtime! exception a new verification condition is generated.

In figure \ref{fig:wpRule}, we show the weakest precondition rule for the \texttt{getfield} instruction. As the virtual machine is stack-based, the rule mentions the stack \texttt{stack} and the stack counter \texttt{cntr}, thus the stack top element is referred as \stack{\counter}. If the top stack element \stack{\counter} is not null, \texttt{getfield} pops \stack{\counter} which is an object reference and pushes the value of the referenced field onto the operand stack in \stack{\counter}. If the stack top element is null, the Java Virtual Machine specification says that the \texttt{getfield} instruction throws a \texttt{NullPointerException}.

When the verification condition generator works over a method, it labels the formula related to the exceptional termination of every instruction with the index of the instruction in the bytecode array of the method. For example, if a \texttt{getField} instruction is met in the bytecode of a method, a conjunction is generated and the conjunct related to the exception is labeled as shown by figure \ref{fig:wpRule}. Finally, indexing the verification conditions allows to identify later in the proof phase which instructions can be optimized.



\begin{figure}
$$
\begin{array}{l}
\wpi(ind : \texttt{getfield} \ \texttt{Cl.f}, \ \psi, \ \excPost) = \\
\biggl( 
\begin{array}{l}
	%\begin{array}{l}
   		\stack{\counter} \not= \Mynull\Rightarrow   \\
	\Myspace \psi\begin{array}{l} \subst{\stack{\counter}}{\texttt{Cl.f} (\stack{ \counter}) } \\[0 mm]

		\end{array}\\
	%\end{array}
   \wedge \\
    ind : \stack{\counter} = \Mynull 	\Rightarrow\\
   \Myspace	 \excPost(\texttt{NullPointerException})
        \begin{array}{l}
          \subst{ \counter }{ 0} \\
          \subst{\stack{0}}{ \texttt{ref}_{NullPointer} }
	\end{array}
    \end{array} \biggr)
\end{array}
 $$
\caption{The weakest precondition rule for the \texttt{putfield} instruction}
\label{fig:wpRule}
\end{figure}



%\nocite{Adl-Tabatabai1998, Cierniak2000}


\subsection{Memory consumption}

%\input BML/cmdBML.tex

\newcommand{\code}{\textit{code}}
\newcommand{\indexComp}{\textit{index}}





%\section{Introduction} \label{bcsl}
This chapter presents the bytecode level specification language, called for short BML and a compiler from a
 subset of the high level Java specification language JML to BML which from now we shall call \JMLtoBML. 
The chapter is organized as follows.
 In section \ref{BCSLprelim}, we give an overview of the main features of JML. A detailed overview of BML is given in section \ref{BCSLgrammar}.  
  As we stated before, we support also a compiler from the high level specification language JML into BML. The 
 compilation process from JML to BML is discussed in section  \ref{BCSLcompile}.
 The full specification of the new user defined Java attributes in which the JML specification is compiled is given in the appendix.





\subsubsection{Modeling Memory Consumption}\label{sec:verif}
The objective of this section is to demonstrate how the user can
annotate and verify programs in order to obtain an upper bound on
memory consumption. We begin by describing the principles of our
approach, then turn to establish its soundness, and finally show
how it can be applied to non-trivial examples involving recursive
methods and exceptions.


\subsection{Principles}
Let us begin with a very simple memory consumption policy which aims
at enforcing that  programs do not consume more than
some fixed amount of memory \Max . To enforce this policy, we first
introduce a ghost variable \Mem\ that represents at any given point of
the program the memory used so far. Then, we annotate the program both
with the policy and with additional statements that will be used to
check that the application respects the policy.



\paragraph{The precondition} of the method $\method$ should ensure that
there must be enough free memory for the method execution. Suppose
that we know an upper bound of the allocations done by method $\method$
in any execution. We will denote this upper bound by
\allocMethod{\method}. Thus there must be at least
\allocMethod{\method}\ free memory units from the allowed \Max\ when
method $\method$ starts execution. Thus the precondition for the method
$\method$ is:
$$
\requires \ \Mem + \allocMethod{\method}  \leq \Max.
$$
%\todo{ to leave this paragraph or not. It is about the initialization of the variable \Mem} 
The precondition of the
program entry point (i.e., the method from which an application
may start its execution) should state that the program has not
allocated any memory, i.e. require that variable \Mem \ is  0:
$$
\requires \ \Mem == 0.
$$
\paragraph{The normal postcondition} of the method $\method$ must
guarantee that the memory allocated during a normal execution of
$\method$ is not more than some fixed number \allocMethod{\method}\
of memory units. Thus for the method $\method$ the postcondition is:
$$
\ensures \  \Mem \leq \old(\Mem) + \allocMethod{\method}.
$$

\paragraph{The exceptional postcondition} of the method $\method$ must
say that the memory allocated during an execution of $\method$ that
terminates by throwing an exception \texttt{Exception} is not more
than \allocMethod{\method}\ units. Thus for the method $\method$ the
exceptional postcondition is:

$$
\exsures{Exception} \  \Mem \leq \old(\Mem) + \allocMethod{\method}.
$$


\paragraph{Loops} must also be annotated with appropriate invariants. 
%Assuming that we know that loop $\progLoop{l}$ iterates no more than $\maxIter{l}$ as well as an upper bound  $\allocLoop{l}$ of the allocations done per iteration in $l$. 
Let us assume that loop $\progLoop{l}$ iterates no more than $\maxIter{l}$ and let $\allocLoop{l}$ be an upper bound of the memory allocated per iteration in $l$.
Below we give a general form of loop specification w.r.t. the property for constraint memory consumption. The loop invariant of a loop $\progLoop{l}$ states that at every iteration the loop body is not going to allocate more than $\allocLoop{l}$ memory units and that the iterations are no more than $\maxIter{l}$. We also declare an expression which guarantees loop termination, i.e. a variant (here an integer expression whose values decrease at every iteration  and is always bigger or equal to 0).
$$\begin{array}{ll}
\modifies &  \ i, \Mem \\
\invariant: & \ \Mem \le \atState{\Mem}{Before_{l}} + i * \allocLoop{l} \\
                & \wedge \\
                & i \le \maxIter{l}\\
\variant: & \maxIter{l} - i \\
\end{array}$$
 A special variable appears in the invariant, $\atState{\Mem}{Before_{l}}$. It denotes the value of the consumed memory just before entering for the first time the loop \progLoop{l}. At every iteration the consumed memory must not go beyond the upper bound given for the body of loop.

\paragraph{For every instruction that allocates memory} the ghost
variable \Mem\ must also be updated accordingly. For the purpose of
this paper, we only consider dynamic object creation with the bytecode
\new; arrays are left for future work and briefly discussed in the
conclusion. 

The function $allocInstance: Class \rightarrow int$ gives an estimation of the memory used by an instance of a class.
%In order to perform the update for \new\ bytecodes, we must assume given a function $allocInstance: Class \rightarrow int$ which maps classes to an estimation of the memory that any instance of the class may occupy. 
At every program point where a bytecode \srcCode{\new \ A} is found, the ghost variable \Mem\ must be incremented by $\allocInstance{A}$. This
is achieved by inserting a ghost assignment immediately after any \new\ instruction, as shown below:
$$
\begin{array}{l}
\srcCode{\new \ A} \\
 // \set \ \Mem = \Mem + $\allocInstance{A}$.
\end{array}
$$

\subsection{Correctness}
%An important question is if the annotations that we prescribe here guarantees that the memory used in a program is not more than a fixed upper bound \Max. 
We want to guarantee that the memory allocated by a given program is bounded by a constant \Max.
We can prove that our annotation is correct w.r.t. to the policy for constraint memory use, by instrumenting the operational semantics of the bytecode language given in
 Section \ref{subsec:sound}. The instrumented operational semantics
manipulates states as before, but it is extended with the special variable \Mem. Thus, states in the new semantics have the form:

$$\configMem{h,\fram{m,\pc,l,s},\stf,\Mem}$$

%The variable \Mem \ changes its value only for instructions that allocate space in the heap, i.e. \new\ instructions:

%$$\small{\frac{
%\begin{array}[c]{c}
%\ \InstAt(m,\pc)=\new \ A ,
%\end{array}}
%{\begin{array}[t]{c} \config{h,\fram{m,\pc,l,v::s},\stf, \Mem} \to_{\new\ A} \\ \config{h + \allocInstance{A},\fram{m,\pc+1,l,s},\stf ,\Mem + \allocInstance{A}}
%\end{array}}}$$



The other instructions do not affect \Mem, so the corresponding rules of the operational semantics are as before. As we saw in the previous section to every
instruction of the form $\new\ A$ we attach the annotation $\set\ \Mem = \Mem + \allocInstance(A)$. The proof obligation generator converts this annotation into new value for the variable \Mem:

$$
\begin{array}{l}
wp(\set \ \Mem = \Mem + \allocInstance{A}, \psi) = \\
\ \ \ \ \ \ \ \ \ \ \ \ \psi[ \Mem \leftarrow \Mem + \allocInstance{A} ]
\end{array}
$$

We can prove that whenever the allocated space in the heap increments, 
the ghost variable \Mem\ also increments, which is a sufficient condition to guarantee the correctness of the annotations. 
So far we do not deal with garbage collection (see discussion in Section \ref{sec:conc}).

\subsection{Examples}
We illustrate hereafter our approach by several examples. 
%\alarm{talk about number of proof obligations, which are discharged automatically in Coq, etc}

\subsubsection{Inheritance and overridden methods} Overriding methods are treated as follows: whenever a call is performed to a method \method,
we require that there is enough free memory space for the maximal
consumption by all the  methods that override or are overridden by
\method. In Fig. \ref{classExt} we show a class \verb!A! and its
extending class \verb!B!, where \verb!B! overrides the method \method\ from class \verb!A!. Method \method\ is invoked by $n$. Given that the dynamic type of the parameter passed to $n$ is not known, we cannot know which of the two
methods will be invoked. This is the reason for requiring enough memory space for the execution of any of these methods.
%After the method execution we consider the extreme case where there is executed the method \method\ that consumes the most.

\begin{figure}[!htp]
Specification of method $m$ in class A:
$$
\begin{array}{ll}
\requires & \Mem + k  \leq \Max \\
\modifies & \Mem \\
\ensures & \Mem  \leq \old(\Mem) + k
\end{array}
$$

Specification for method $m$ in class B:
$$
\begin{array}{ll}
\requires & \Mem + l  \leq \Max \\
\modifies & \Mem \\
\ensures & \Mem  \leq \old(\Mem) + l
\end{array}
$$

\begin{verbatim}
method n(A a)
...
//{ prove Mem <= Mem +max(l,k) }
invokevirtual m <A>
//{ assume Mem <= \old(Mem) + max(l,k)}
...
\end{verbatim}
\caption{\sc Example of overridden methods}
\label{classExt}
\end{figure}


\subsubsection{Recursive Methods} In Fig. \ref{recMeth} the bytecode of the recursive method $m$ and its specification is shown. For sake of space we show only a simplified version of the bytecode; we assume that the constructors for the class \srcCode{A} and \srcCode{C} do not allocate memory. Besides the precondition and the postcondition, the specification also includes information about the termination of the method: \variant\ $\local{1}$, meaning that the local variable $\local{1}$ decreases on every recursive call down to and no more than $0$, guaranteeing that the execution of the method will terminate.
 
%Now we explain why such a precondition is required for method \textbf{m} in order to specify the property for constraint memory consumption. 

We explain first the precondition. If the condition of line \srcCode{1} is not true, the execution continues at line \srcCode{2}.

\begin{figure}[!hbp]
\begin{alltt}
public class D \{
  public void m( int i) \{
    if (i > 0) \{
      new A();
      m(i - 1);
      new A();
    \} else \{
      new C();
      new A();
   \}
  \}
\}
\end{alltt}

$$
\begin{array}{ll}
 \requires & ( \Mem + \local{1}*2*\allocInstance{A} + \\
           &  \allocInstance{A} + \allocInstance{C}) \le \Max \\
 \variant  & \local{1} \\
 \ensures  & \local{1} \ge 0 \\
           & \wedge \\
           & \Mem <= \old(\Mem) +  \old(\local{1})*2*\allocInstance{A} + \allocInstance{A}\\
           &  +  \allocInstance{C})
\end{array}$$

\begin{alltt}
\srcCode{\textbf{public void m()}}
//\small{\textit{local variable loaded on} }
//\small{\textit{the operand stack of method \textbf{m}}}
\srcCode{0 \load\_1}
//\small{ \textit{ if \local{1} <= 0 jump}}
\srcCode{1 ifle 12}
\srcCode{2 new <A>} //\small{ \textit{ here \local{1} > 0  } }
//set \Mem = \Mem +  \allocInstance{A}
\srcCode{3 invokespecial <A.<init>>}
\srcCode{4 aload\_0}
\srcCode{5 iload\_1}
\srcCode{6 iconst\_1}
//\small{\textit{\local{1} decremented with 1}}
\srcCode{7 isub}
//\small{ \textit{ recursive call with the new value of \local{1}}}
\srcCode{8 invokevirtual <D.m>}//
\srcCode{9 new <A>}
//set \Mem = \Mem +  \allocInstance{A}
\srcCode{10 invokespecial <A.<init>>}
\srcCode{11 goto 16}
//\small{\textit{target of the jump at \srcCode{1}}}
\srcCode{12 new <A>}
//set \Mem = \Mem +  \allocInstance{A}
\srcCode{13 invokespecial <A.<init>>}
\srcCode{14 new  <C>}
//set \Mem = \Mem +  \allocInstance{C}
\srcCode{15 invokespecial <C.<init>>}
\srcCode{16 return}
\end{alltt}

\caption{\sc Example of a recursive method}
 \label{recMeth}
\end{figure}

In the sequential execution up to line \srcCode{7}, the program allocates at most $\allocInstance{A}$ memory units and decrements by $1$ the value of $\local{1}$. The instruction at line \srcCode{8} is a recursive call to $m$, which either will take the same branch if $\local{1} > 0 $ or will jump to line \srcCode{12} otherwise, where it allocates at most $\allocInstance{A} +  \allocInstance{C}$ memory units. On returning from the recursive call one more allocation will be performed at line \srcCode{9}.
 Thus $m$ will execute, $\local{1}$ times, the instructions from lines \srcCode{4} to \srcCode{35},  and it finally will execute all the instructions from lines  \srcCode{12} to \srcCode{16}.
The postcondition states that the method will perform no more
than $\old(\local{1})$ recursive calls (i.e., the value of the register variable in the pre-state of the method) and that on every recursive call it allocates no more than two instances of class \texttt{A} and that it will finally allocate one instance of class \texttt{A} and another of class \texttt{C}.


\subsubsection{More precise specification} We can be more precise in specifying the precondition of a method by considering what are the field values of an instance, for example. Suppose that we have the method \method\ as shown in Fig. \ref{excMeth}. We assume that in the constructor of the class \texttt{A} no allocations are done. The first line of the method \method\ initializes one of the fields of field \texttt{b}. Since nothing guarantees that field \texttt{b} is not \Mynull, the execution may terminate with
\texttt{NullPointerException}. Depending on the values of the parameters passed to \method, the memory allocated will be different. The precondition establishes what is the expected space of free resources depending on if the field
\texttt{b} is \Mynull  or not. In particular we do not require anything for
the free memory space in the case when \texttt{b} is \Mynull. In the
normal postcondition we state that the method has allocated an
object of class \texttt{A}. The exceptional postcondition states
that no allocation is performed if \texttt{NullpointerException} causes the execution termination.

\begin{figure}[!hbp]
$$
\begin{array}{ll}
 \requires &  \local{1} != \Mynull \Rightarrow  \\
           & \phantom{\local{1}} \Mem +  \allocInstance{A} \le \Max \\
       %& \wedge \\
       %&  \local{1} == \Mynull \Rightarrow  \\
           %& \phantom{\local{1}} \Mem +  \allocInstance{B} + \allocInstance{A}   \le \Max \\
  \modifies & \Mem \\
  \ensures  & \Mem \le \old(\Mem) +  \allocInstance{A} \\
  \exsures{NullPointerException}  & \Mem == \old(\Mem)   \\
\end{array}$$

\begin{tabular}{lr}
\begin{minipage}[t]{170pt}
\begin{alltt}
\srcCode{0 aload\_0}
\srcCode{1 getfield<C.b>}
\srcCode{2 iload\_2}
\srcCode{3 putfield <B.i>}
\srcCode{4 new <A>}
//set \Mem = \Mem +
      \allocInstance{A}
\srcCode{5 dup}
\srcCode{6 invokespecial <A.<init>>}
\srcCode{7 astore\_1}
\srcCode{8 return}
\end{alltt}
\end{minipage}
 &
\begin{minipage}[t]{170pt}
\begin{alltt}
public class C \{
  B b;
  public void m(A a, int i) \{
    b.i = i ;
    a = new A();
  \}
\}
\end{alltt}
\end{minipage}
\end{tabular}
\caption{\sc Example of a method with possible exceptional termination}
\label{excMeth}
\end{figure}

\subsubsection{Inferring Memory Allocation}\label{sec:infer}
In the previous section, we have described how the memory consumption
of a program can be modeled in BML and verified using an appropriate
verification environment. While our examples illustrate the benefits
of our approach, especially regarding the precision of the analysis,
the applicability of our method is hampered by the cost of providing
the annotations manually. In order to reduce the burden of manually
annotating the program, one can rely on annotation assistants that
infer automatically some of the program annotations (indeed such
assistants already exist for loop invariants, loop variants, or
class invariants). In this section, we describe an implementation of
an annotation assistant dedicated to the analysis of memory consumption,
and illustrate its working on an example.

\subsection{Annotation assistant}
The inference algorithm proposed here works on programs
 without recursive methods, exception throwing and handling.  


The user must provide information about the memory required to create objects of the given classes,
 i.e. he must give the definition  of the function $\allocInstance{\cdot}$.
 The variant and the maximum number of iterations $\maxIter{l}$ for every loop $l$  are either given by the
 user or are synthesized  through appropriate mechanisms. 

%Given a control flow graph of the program --which is computed by the verification environment, a function that gives for each class the memory required to create an object of that class, which is provided by the user, and a variant for each loop and recursive method, which can either be provided by the user, or synthesized through appropriate mechanisms. 

Based on this information, the annotation assistant 
inserts the ghost assignments on appropriate places, and then computes
recursively the memory allocated on each loop and method. 
A pseudo-code of the algorithm for inferring an upper bound for method allocations is given in Fig. \ref{methodAlloc}.
Essentially, it finds the maximal memory that can be allocated in a method by exploring all its possible execution paths.  
The algorithm for exploring an execution path starts from the last instruction in the path (i.e. a \return{} instruction). 
The algorithm use standars techniques to detect the loop entry instructions. For each loop entry instruction it also finds  
the set  of the corresponding ``loop end'' instructions i.e. the  instructions  that target   to  and which are dominated by
the loop entry instruction; you are not entering in detail in the loop detection
algorithm as it is standard and the reader may  see Section 10 of 
 \cite{ASU86cpt} for a description of the algorithms.

\begin{figure}[!htp]
function $\allocMethod{.}$\\
\textbf{Input:} Bytecode of a method \methodd{} . \\
\textbf{Output:} Upper bound of the memory allocated by \methodd{} . \\
\textbf{Body:}
\begin{enumerate}
   \item Detect all the loops in \methodd{}; for every loop $l$ determine $\loopSet{l}$, $\loopEntry{l}$ and $\loopEndsSet{l}$;
   \item Apply the function \allocatedOnly \  to each instruction $\instrAt{k}$, such that $\instrAt{k} = \return$;
   \item Take the maximum of the results given in the previous step: \\
 $max_{\instrAt{k} = \return } \allocated{\instrAt{k}}$.
\end{enumerate}
\caption{\sc Inference algorithm}
\label{methodAlloc}
\end{figure}

The auxiliary function $allocPath$, which infers the maximal allocations done by the set of execution paths ending with the same \return{} instruction,
 is given in Fig. \ref{fig:allocMethod}.
Inferring the memory allocated inside loops is done by the function $\allocLoopWithEnd{\cdot}{\cdot}$, which is invoked by \allocatedOnly{} whenever the 
current instruction belong to a loop. The specification of the function is shown in Fig. \ref{fig:loopPath}.

\begin{figure}[!hbp]
%\centerline{
$\allocated{\instrAt{s}}$ = 
$$ \left\{ \begin{array}{ll}
\allocIns{\instrAt{s} }   &  if \ \instrAt{s} \ has \ no \ predecessors \\
& \\
  \begin{array}{l}
            \allocLoop{\loopEntry{l}} \\
             + \\
            max_{\instrAt{k} \in \preds(\instrAt{s} )-\loopEndsSet{\progLoop{l}}}( \allocated{\instrAt{k}} ) \\
                   \end{array}      & if \  \instrAt{s} \in \loopSet{\progLoop{l}} \\
& \\
\begin{array}{l}
\allocIns{\instrAt{s}} \\
 + \\
max_{\instrAt{k} \in \preds(\instrAt{s} )}
 ( \allocated{\instrAt{k}} )
                       \end{array} & else 
\end{array}
\right.
$$
\caption{\sc Definition of the function $\allocated{\instrAt{s}}$} 
\label{fig:allocMethod}
\end{figure}


\begin{figure}[!hbp]
$\allocLoop{\loopEntry{l}} = $
$$ \begin{array}{l}
    \maxIter{l} * max_{ \instrAt{e} \in \loopEndsSet{l}  } (\allocLoopWithEnd{\loopEntry{l}}{\instrAt{e}} )
   \end{array}$$



$\allocLoopWithEnd{\loopEntry{l}}{\instrAt{s}} = $
$$ 
\left\{\begin{array}{ll}

 \allocIns{\loopEntry{l}}  & if \  \instrAt{s} = \loopEntry{l} \\
  & \\
 \begin{array}{l}
           \allocLoop{\loopEntry{l'}} \\
          + \\
      max_{\instrAt{k} \in \preds(\loopEntry{l'} ) - \loopEndsSet{\progLoop{l'}}}
       ( \allocLoopWithEnd{\loopEntry{l}}{\instrAt{k}} )
    \end{array} &  \begin{array}{l}
                                        if \  \instrAt{s} \in  \loopSet{\progLoop{l'}} \\
                                          \progLoop{l'} \ is \  nested \ in \ \progLoop{l}
                                    \end{array} \\
  & \\
  \begin{array}{l}
     \allocIns{\instrAt{s}} \\
     + \\
     max_{\instrAt{k} \in \preds(\instrAt{s} )}
     ( \allocLoopWithEnd{\loopEntry{l}}{\instrAt{k}} )
                       \end{array} & else \\

\end{array} \right.
$$
 \caption{\sc Definition of the function $\allocLoop{\cdot}$ and  $\allocLoopWithEnd{\cdot}{\cdot}$ }
\label{fig:loopPath}
\end{figure}


%In essence, it first analyses the method's bytecode by identifying the entry loop instruction, the instructions
% that are inside the loop and the set of instructions by which the loop terminates. 
%Then  it finds the maximal number of allocations that can be done per execution path: 
%it starts from all \return \ instructions and ``inspects'' the execution paths upto the entry program
%instruction (in backwards direction  );  if an instruction in the path belongs to a loop then the allocations done per iteration in the loop are 
%calculated and the analysis proceeds from the instructions that target the entry of the loop and that do not belong to the loop; the assistant computes the memory required for %each loop using the memory required for each iteration of the loop, and the variant of the loop, which provides information about the number of
%iterations; upper bound of the allocations done per iteration in a loop are calculated also in a backwards direction 
%(starting from instructions that belong to the loop and whose next instruction is the unique loop entry) by finding the 
%maximal number of allocations per iteration path and if an instruction in an iteration path appears to be instruction that belongs to an inner loop 
%then first an upper bound for the allocations done in the inner loop are inferred and the analysis continues from instructions that target the entry of the nested loop and do not belong to the nested loop.    

The annotation assistant currently synthesize only simple
memory policies (i.e., whenever the memory consumption policy does not depend on
the values of inputs).
%and could be significantly improved in this respect. 
Furthermore, it does not deal with arrays, subroutines, nor exceptions. Our approach may be extended to treat such cases (see the discussion in Section \ref{sec:conc} about how to include arrays in our analysis). For sake of simplicity,
we have also restricted the loop analysis only to those with a unique
entry point, which is the case for code produced by non-optimizing
compilers.
Dealing with loops with a multiple entry points is left for a future work. %A pre-analysis could give us all the entry points of more
%general loops, for instance by the algorithms given in \cite{CJPS05cmu}; our approach may be thus applied straightforwardly.



