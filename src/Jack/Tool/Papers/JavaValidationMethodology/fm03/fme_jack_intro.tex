 Providing high quality on applet development is becoming a crucial
issue, especially when those applets are aimed to be loaded and executed in smart cards.  Actually, the card
remains a specific domain where post issuance corrections are very expensive due to the deployment process and
the mass production. Currently, the quality is ensured by costly test campaigns, whenever tests are technically
possible. We consider that using formal techniques is a solution that allows us to increase the quality, but
also to reduce validation costs.

 Formal validation of Java programs is a growing research
 field.  As Java has become a reference language, many technologies are
 emerging to help Java program validation.  Java can also be
 considered as a good support for formal techniques, as it has precise semantics \cite{Gosl00a}.

 Nevertheless, proving program correctness, and more generally using
formal methods, is traditionally an activity reserved for experts.  This restriction is usually caused by the
mathematical nature of the concepts involved.  This explains why formal techniques are difficult to introduce in
industrial processes, even if they are now widely used in research and teaching activities.  However, we believe
that this restriction can be reduced by providing notations and tools hiding the mathematical formalisms.
Therefore, formal tools should be developed to fit into classical developers environment.  We strongly believe
that efforts should be done to allow users to benefit from formal techniques without having to learn new
formalisms and to become experts. Java developers should be able to validate their code, or at least to get a
good assurance on its correctness.

 This paper presents such a tool: the Java Applet Correctness Kit
 (or \JACK).  This tool, already briefly described in
 \cite{BR-gdc2002}, is a formal tool that allows one to prove properties on
 Java programs using the Java Modeling Language \cite{LBR00} (JML).
 Its application domain is, at the moment, smart card applets, but one can consider that it can be useful in
many development contexts.
 It generates proof obligations
 allowing to prove that the Java code conforms to its JML
 specification.  The lemmas are translated into the B language \cite{bbook},
 allowing to use the automatic prover developed within the B method.

 But the tool is not yet another lemma generator for Java, since it also provides a lemma
 viewer integrated in the eclipse
 IDE\footnote{\texttt{http://www.eclipse.org}}.  This allows to
 hide the formalisms used behind a graphical interface.  Lemmas are
 presented to users in a way they can understand them easier, by using the Java syntax and highlighting code portions to help
 the understanding. Using \JACK, one does not have to learn a formal language to be convience on
code correctness.

 The remainder of the paper is organized as follow.  Section
 \ref{JavaModellingLanguage} describes JML and the different tools
 supporting it. Section \ref{JavaAppletCorrectnessKit} presents the
 architecture and the main principles of the tool we have
 developed. Section \ref{Industrialisation} describes more precisely
 the innovative parts of the tool and explains why we consider it as
 accessible to any developers. Section \ref{Case Study} describes
 experiments on an applet and the metrics that have been
 collected.  Section \ref{Perspectives} presents research perspectives
 and Section \ref{Conclusion} concludes.
