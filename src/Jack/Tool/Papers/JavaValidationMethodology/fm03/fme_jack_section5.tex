\label{Case Study}
% Tableau avec chiffres, etc...
To test \JACK, we have developed a little banking application.  This section presents different metrics
concerning the evaluation of the tool on this package.
\begin{table}
 \begin{tabular}{|l|c|c|c|c|c|c|c|} \hline
 Classes & Java & JavaDoc & JML & Proof & Automatic & Time to PO & Time to \\
  & lines & lines & lines & obligations & proof & generate (s) & prove (s)\\  \hline
 Transfert\_src  & 116 & 34 & 150 & 359 & 91\% & 22,5 & 238 \\
 AccountMan\_src & 105 & 51 & 236 & 269 & 82\% & 12,7 & 195 \\
 Currency\_src   &  93 & 20 &  28 &  50 & 96\% &  7,6 &  17 \\
 Balance\_src    &  64 & 38 &  58 & 335 & 95\% & 16,5 & 191 \\
 Spending\_rule  &  40 & 33 &  62 &  42 & 67\% & 13,6 & 217 \\ \hline
 \end{tabular}
\caption{banking applet metrics}
\label{MetricsTable}
\end{table}
Different remarks can be made from Table \ref{MetricsTable}, concerning the cost of adding JML annotations, the
performance of the tool, as well as the cost associated to the proof.
\paragraph{Cost of the annotations}
A first remark concerns the cost of the annotations.  The metrics given here only concern the number of lines
but one can see that the documentation size (JavaDoc and JML) is one and half greater than the code size.  So,
writing  the JML specification seems to be a costly activity.  This remark can be moderated by two points: this
development was the first that we made, and annotations were added to already existing code. So it suffers from
its lack of abstraction, and the annotations are really verbose. Moreover the time to specify is to be compared
to the time to develop test data.
\paragraph{Responsiveness}
The automatic phases are quite responsive with some seconds to generate proof obligations.  The automatic proof
is not as cheap as the proof obligation generation and takes a few minutes. However, this still remains
acceptable, since it is a non-blocking task that does not require users to wait for its completion.

For larger applets, it is however expected that the time required for the automatic proof will significantly
increase.
\paragraph{Proof}
The automatic proof rating is much good.  It is quite greater than the usual value for a B development (around
80\%).  This is mainly due to the fact that we used this applet as a test applet for extending the prover. So,
as a side effect, the automatic prover is customized for this special applet.

Nevertheless, after automatic proof step, there remain 111 lemmas to prove using the Atelier B interface.  An
expert needs between 4 and 5 days to prove them.
