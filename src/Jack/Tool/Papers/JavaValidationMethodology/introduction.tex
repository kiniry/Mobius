\chapter{Introduction}
Providing high quality on applet development is becoming a crucial
issue, especially when those applets are aimed to be loaded and executed in smart cards.  Actually, the card
remains a specific domain where post issuance corrections are very expensive due to the deployment process and
the mass production. Currently, the quality is ensured by costly test campaigns, whenever tests are technically
possible. We consider that using formal techniques is a solution that allows us to increase the quality, but
also to reduce validation costs.

 Formal validation of Java programs is a growing research
 field.  As Java has become a reference language, many technologies are
 emerging to help Java program validation.  Java can also be
 considered as a good support for formal techniques, as it has precise semantics \cite{Gosl00a}.

 Nevertheless, proving program correctness, and more generally using
formal methods, is traditionally an activity reserved for experts.  This restriction is usually caused by the
mathematical nature of the concepts involved.  This explains why formal techniques are difficult to introduce in
industrial processes, even if they are now widely used in research and teaching activities.  However, we believe
that this restriction can be reduced by providing notations and tools hiding the mathematical formalisms.
Therefore, formal tools should be developed to fit into classical developers environment.  We strongly believe
that efforts should be done to allow users to benefit from formal techniques without having to learn new
formalisms and to become experts. Java developers should be able to validate their code, or at least to get a
good assurance on its correctness.

