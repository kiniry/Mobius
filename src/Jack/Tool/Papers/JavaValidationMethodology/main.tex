\documentclass[fullpage,10pt]{article}
\usepackage{times}
\usepackage{epsfig}
\usepackage{enumerate}
\usepackage{paralist}
\usepackage{amsmath}
\usepackage{amssymb}
\usepackage{amscd}
\usepackage{array}
\usepackage{tabularx}
\usepackage{xspace}
\usepackage{url}
\usepackage{float}
\usepackage{multirow}
\usepackage{listings}
\usepackage{latexsym,pstricks,pst-node,pst-tree,boxedminipage,stmaryrd}
\usepackage{alltt}
\usepackage{fancyhdr}
\usepackage{fullpage}

%\markboth{INSPIRED}{INSPIRED}

%\pagestyle{myheadings}
\fancypagestyle{plain}{
\fancyhf{}
\renewcommand{\headwidth}{\textwidth}
\renewcommand{\headrulewidth}{0pt}
\lhead{{\scriptsize \begin{tabular}{l}
DEVELOPER ORIENTED APPROACH TO APPLET CORRECTNESS \\
INSPIRED\_INR\_WP2.6\_D9.1\_R1.0
\end{tabular}}}

\rhead{{\scriptsize \begin{tabular}{r}
INSPIRED IST-2003-507894 \\
30 JUNE 2006
\end{tabular}
}}
\fancyfoot[LE,RO]{\thepage}
}
\pagestyle{plain}

\lstset{language=Java,flexiblecolumns=false,mathescape=true}
\lstset{commentstyle=\sl,basicstyle=\normalfont\ttfamily\footnotesize}
\newcommand{\mjvm}[1]{{\sf #1}}
\parindent 1cm
\parskip 0.2cm
\topmargin 0.2cm
\oddsidemargin 1cm
\evensidemargin 0.5cm
\textwidth 15cm
\textheight 21cm

\def\lstlanguagefiles{lstlangjml.sty}
\lstloadlanguages{Jml}

\newcommand{\benchname}[1]{\texttt{#1}}

\newcommand{\comment}[1]{{\sf #1}}
\newcommand{\alarm}[1]{\marginpar{#1}}

\newtheorem{definition}{Definition}
\newtheorem{lemma}{Lemma}
\newtheorem{theorem}{Theorem}


\def \bsl       {\symbol{92}}
\def \unsc      {\symbol{95}}

\newcommand{\todo}[1]{ \textbf{#1}}
\newcommand{\fig}[1]{ Fig.}
\newcommand{\jmlKey}[1]{\texttt{#1}}% wrapping jml keywords
\newcommand{\java}[1]{\texttt{#1}}
\newcommand{\stack}[1]{\texttt{st(#1)}}% element on top stack 
\newcommand{\counter}{\texttt{ct}}

\newcommand{\true}{\texttt{true}}
\newcommand{\false}{\texttt{false}}

\newcommand{\wpi}{\textit{wp}}

\newcommand{\instr}[1]{\texttt{#1}}

\newcommand{\substitution}[2]{[\tt{#1} \leftarrow \tt{#2}]}
% thegrammar for the bytecode specification language
\newcommand{\ClassSpec}{\rm{ClassSpec}}
\newcommand{\MethodSpec}{\rm{MethodSpec}}
\newcommand{\SpecCase}{\textrm{SpecCase}}
\newcommand{\jmlStmt}[1]{\textrm{#1}}
\newcommand{\interMethodSpec}{\rm{InterMethodSpec}}
\newcommand{\loopSpec}{\rm{loopSpec}}
%\newcommand{\assert}{\rm{assertSpec}}

\newcommand{\ArithExpr}{\texttt{Arithmetic\_Expr}}
\newcommand{\expression}{\mathcal{E} }

\newcommand{\integer}{\texttt{int} }
\newcommand{\register}[1]{\texttt{lv[#1]} }
\newcommand{\reference}{\texttt{ref} }
\newcommand{\intLiteral}{\texttt{int\_literal} }
\newcommand{\Mynull}{\texttt{null}}
\newcommand{\this}{\texttt{this}}
\newcommand{\fieldAccess}[1]{\texttt{field\_cp\_index(}#1 \texttt{)}}
\newcommand{\arrayAccess}[2]{#1[#2] }

\newcommand{\result}{\jmlKey{$\backslash$result}}
\newcommand{\oldp}[1]{\jmlKey{$\backslash$old(}#1\jmlKey{)}}
\newcommand{\typeof}[1]{\jmlKey{$\backslash$typeof(}#1 \jmlKey{)}}
\newcommand{\EXC}{\texttt{EXC}}

\newcommand{\excPost}{\psi^{exc}}

\newcommand{\Myspace}{\phantom{aa}}
\newcommand{\predicate}{ \mathcal{P}} 
\newcommand{\Myfalse}{\textit{false}}
\newcommand{\Mytrue}{ \textit{true} }
\newtheorem{defn}{Definition} 



% abstractCtrlFlow.tex
\newcommand{\execRel}{\rightarrow} % the execution relation
\newcommand{\blockm}[1]{ \tt{b^{#1}} }
\newcommand{\blockSeq}[1]{ \tt{b_{seq}^{#1}} }
%\newcommand{\pathm}[2]{\blockm{#1} \execRel^{*} \blockm{#2} }

\newcommand{\blockPost}[1]{ \it{post(}\tt{b_{seq}^{#1}}\it{)}}

\newcommand{\invariant}{\textit{I}}

\newcommand{\srcVar}[1]{\texttt{#1} }

\newcommand{\method}{\texttt{m} }

% recuperé dans le prelude du code généré par l'Atelier B.
% Doit permettre d'écrire à peu-près correctement <+
\def\famletter#1{\ifcase #1 0\or 1\or 2\or 3\or 4\or 5\or 6\or 7\or
    8\or 9\or A\or B\or C\or D\or E\or F\fi}
\font\msx=msam10
\newfam\msxfam \textfont\msxfam=\msx
\edef\fx{\famletter\msxfam}
\mathchardef    \dres       "2\fx43
\def    \lover      {\mathbin{{\dres} \llap{$-\!\!\!\!-\!$}}}

\def\keywords{\noindent\bf Keywords: \vspace{0pt}
\it\normalsize\normalsize}
\def\endkeywords{\par}

\def\acknowledgement{\vspace{4pt} \noindent{\bf Acknowledgements} \\ \vspace{1pt}
\normalsize\normalsize}
\def\endkeywords{\par}

\newcommand{\JACK}{\texttt{JACK}}
\newcommand{\ESC}{\texttt{ESC/Java}}
\newcommand{\LOOP}{\texttt{LOOP}}


\title{Methodology for Java Application Validation}

\author{INRIA Sophia Antipolis\\Everest team}

\date{}
\newcommand{\bbb}{\mathbb{B}}
% moved here from carmel.tex
\newcommand{\spp}{\hspace{1.5cm}}


\newcommand{\semb}{\llbracket}
\newcommand{\seme}{\rrbracket}
\newcommand{\sem}[1]{ \semb #1 \seme}
\newcommand{\ia}[1]{ \semb #1 \seme}
\newcommand{\fia}[1]{ \mathcal{F}\semb #1 \seme}
\newcommand{\ria}[1]{ \mathcal{R}\semb #1 \seme}

\newcommand{\Coq}{{\sf Coq}}
\newcommand{\ocaml}{\textsc{ocaml}}

\newcommand{\memvar}{\text{Mem}}

\newcommand{\AbVal}{\widehat{\text{Val}}}
\newcommand{\Val}{{\text{Val}}}
%\newcommand{\Stack}{{\text{Stack}}}
\newcommand{\AbStack}{{\widehat{\text{Stack}}}}
\newcommand{\SafeCallStack}{{\text{SafeCallStack}}}
\newcommand{\OneCall}{{\text{OneCall}}}
\newcommand{\Var}{{\text{Var}}}
\newcommand{\LocalVar}{{\text{LocalVar}}}
\newcommand{\AbLocalVar}{{\widehat{\text{LocalVar}}}}
\newcommand{\State}{{\text{State}}}
\newcommand{\Trace}{{\text{Trace}}}
\newcommand{\AbState}{{\widehat{\text{State}}}}
\newcommand{\progCount}{{\text{progCount}}}
\newcommand{\fieldName}{{\text{fieldName}}}
\newcommand{\methodName}{{\mathrm{methodName}}}
\newcommand{\MM}{{\methodName}}
\newcommand{\PP}{{\progCount}}
\newcommand{\className}{{\text{className}}}
\newcommand{\varName}{{\text{varName}}}
\newcommand{\Adress}{{\text{Adress}}}
\newcommand{\Instruction}{{\text{Instruction}}}
\newcommand{\InstAt}{{\text{InstAt}}}
\newcommand{\Constraint}{{\text{Constraint}}}
\newcommand{\St}{\widehat{\text{St}}}

\newcommand{\config}[1]{{\langle\!\langle #1 \rangle\!\rangle}}
\newcommand{\fram}[1]{{\left\langle #1 \right\rangle}}
\newcommand{\num}{{\text{num}}}
\newcommand{\reff}{{\text{ref}}}
\newcommand{\nul}{{\text{null}}}
\newcommand{\some}{{\text{some}}}
\newcommand{\nameClass}{{\text{nameClass}}}
\newcommand{\class}{{\text{class}}}
\newcommand{\newObject}{{\text{newObject}}}
\newcommand{\newArray}{{\text{newArray}}}
\newcommand{\lengthArray}{{\text{lengthArray}}}
\newcommand{\fieldValue}{{\text{fieldValue}}}
\newcommand{\Value}{{\text{Value}}}
\newcommand{\classes}{{\text{classes}}}
\newcommand{\RefValue}{{\text{RefValue}}}
\newcommand{\Location}{{\text{Location}}}
\newcommand{\methodLookup}{{\text{methodLookup}}}
\newcommand{\nbArgument}{{\text{nbArgument}}}
\newcommand{\nameMethod}{{\text{nameMethod}}}

\newcommand{\ClassName}{{{\text{ClassName}}}}
\newcommand{\FieldName}{{{\text{FieldName}}}}
\newcommand{\default}{{{\text{default}}}}
\newcommand{\END}{{{\text{END}}}}
\newcommand{\AbHeap}{{\widehat{\text{Heap}}}}

\newcommand{\Sinit}{{\mathcal{S}_{\mathit{init}}}}

\newcommand{\instructionAt}{{\text{instructionAt}}}


%\newcommand{\abs}{\mbox{${\cal S}\!\mathit{t}$}}
\newcommand{\abs}{\mbox{$\Sigma$}}
\newcommand{\addr}{\mbox{\em addr}}
\newcommand{\pc}{\mathit{pc}}
\newcommand{\stf}{{\mathit{sf}}}
\newcommand{\cl}{{\mathit{cl}}}

\newcommand{\analyze}{\text{\tt analyse}}
\newcommand{\Analyze}{\ensuremath{\mathit{Unbounded}(P)}}
\newcommand{\Program}{\text{Program}}

\newenvironment{constraint}{%
\noindent
\hspace{.3mm}$
\begin{array}[t]{l}}{%
\end{array}$\\[0.2cm]
}

\newenvironment{contraint}{%
\noindent
\hspace{1cm}$
\begin{array}[t]{l}}{%
\end{array}$\\[0.5cm]
}




\newcommand{\AbPop}{\widehat{\text{pop}}}
\newcommand{\AbPush}{\widehat{\text{push}}}
\newcommand{\AbTop}{\widehat{\text{top}}}
\newcommand{\AbBinop}{\widehat{\text{binop}}}
\newcommand{\AbApply}{\widehat{\text{apply}}}
\newcommand{\AbSubst}{\widehat{\text{subst}}}
\newcommand{\Num}{{{\text{Num}}}}
\newcommand{\AbNum}{{\widehat{\text{Num}}}}
\newcommand{\AbRef}{{\widehat{\text{Ref}}}}
\newcommand{\AbObject}{{\widehat{\text{Object}}}}
%\newcommand{\AbHeap}{{\widehat{\text{Heap}}}}
\newcommand{\Flow}{\text{\tt Flow}}

\newenvironment{myalltt}{\vspace*{-3pt}\begin{alltt}}{\end{alltt}\vspace*{-3pt}}

%%
%% Added by Gerardo (28/09/2004)
%%

\newcommand{\Rule}[2]
{
\frac{#1}
{
\begin{array}{l}
#2
\end{array}
}
}

\newcommand\Loop{\ensuremath{\mathit{Loop}}}
\newcommand\Pred{\ensuremath{\mathit{Pred}}}
\newcommand\BC{\ensuremath{\mathit{BC}}}
\newcommand\MutRec{\ensuremath{\mathit{MutRecR}}}
\newcommand\Rec{\ensuremath{\mathit{Rec}}}
\newcommand\Anc{\ensuremath{\mathit{Anc}}}
\newcommand\LoopCall{\ensuremath{\mathit{LoopCall}}}
\newcommand\Call{\ensuremath{\mathit{Call}}}
\newcommand\Q{\ensuremath{\mathit{Q}}}
\newcommand\tr{\ensuremath{\mathit{tr}}}

\newcommand\End{\ensuremath{\mathtt{end}}}
\newcommand\nop{\ensuremath{\mathtt{nop}}}
\newcommand\push{\ensuremath{\mathtt{push}}}
\newcommand\pop{\ensuremath{\mathtt{pop}}}
\newcommand\dup{\ensuremath{\mathtt{dup}}}
\newcommand\swap{\ensuremath{\mathtt{swap}}}
\newcommand\numop{\ensuremath{\mathtt{numop}}}
\newcommand\load{\ensuremath{\mathtt{load}}}
\newcommand\store{\ensuremath{\mathtt{store}}}
\newcommand\inc{\ensuremath{\mathtt{inc}}}
\newcommand\add{\ensuremath{\mathtt{add}}}
\newcommand\sub{\ensuremath{\mathtt{sub}}}
\newcommand\goto{\ensuremath{\mathtt{goto}}}
\newcommand\If{\ensuremath{\mathtt{if}}}
\newcommand\looksw{\ensuremath{\mathtt{lookupswitch}}}
\newcommand\tabsw{\ensuremath{\mathtt{tableswitch}}}
\newcommand\newarray{\ensuremath{\mathtt{newarray}}}
%\newcommand\default{\ensuremath{\mathtt{default}}}
\newcommand\new{\ensuremath{\mathtt{new}}}
\newcommand\checkc{\ensuremath{\mathtt{checkcast}}}
\newcommand\getst{\ensuremath{\mathtt{getstatic}}}
\newcommand\putst{\ensuremath{\mathtt{putstatic}}}
\newcommand\instof{\ensuremath{\mathtt{instanceof}}}
\newcommand\getfd{\ensuremath{\mathtt{getfield}}}
\newcommand\getfdt{\ensuremath{\mathtt{getfield this}}}
\newcommand\putfd{\ensuremath{\mathtt{putfield}}}
\newcommand\invdef{\ensuremath{\mathtt{invokedefinite}}}
\newcommand\invvir{\ensuremath{\mathtt{invokevirtual}}}
\newcommand\invint{\ensuremath{\mathtt{invokeinterface}}}
\newcommand\return{\ensuremath{\mathtt{return}}}
\newcommand\arrlh{\ensuremath{\mathtt{arraylength}}}
\newcommand\arrld{\ensuremath{\mathtt{arrayload}}}
\newcommand\arrst{\ensuremath{\mathtt{arraystore}}}
\newcommand\throw{\ensuremath{\mathtt{throw}}}
\newcommand\jsr{\ensuremath{\mathtt{jsr}}}
\newcommand\ret{\ensuremath{\mathtt{ret}}}
\newcommand\thiss{\ensuremath{\mathtt{this}}}
%\newcommand\instr{\ensuremath{\mathtt{instr}}}
\newcommand\const{\ensuremath{\mathtt{const}}}
\newcommand\Array{\ensuremath{\mathsf{array}}}
\newcommand\Int{\ensuremath{\mathsf{int}}}
\newcommand\byte{\ensuremath{\mathsf{byte}}}
\newcommand\short{\ensuremath{\mathsf{short}}}
\newcommand\bool{\ensuremath{\mathsf{bool}}}
\newcommand\Ctxt{\ensuremath{\mathsf{Ctxt}}}

\newcommand\warn{\ensuremath{<!>}}
\newcommand\trace{\overline{s}}
\newcommand\defi{\stackrel{\mathrm{def}}{=}}
\newcommand\lub{\sqcup}

\newcommand\Size{\ensuremath{\mathit{Size}}}
\newcommand\Pre{\ensuremath{\mathrm{Pre}}}
\newcommand\Post{\ensuremath{\mathrm{Post}}}
\newcommand\old{\ensuremath{\backslash\mathrm{old}}}

%%%%%%%%%%%%%%%%%%%%%%%%%%%%%%%%%%%%%%%%%%%%%%%%%%%%%%%
%%
%% Mariela's definitions
%%
%%%%%%%%%%%%%%%%%%%%%%%%%%%%%%%%%%%%%%%%%%%%%%%%%%%%%%%


%\newcommand{\pathm}[2]{\blockm{#1} $\ll^{*}$ \blockm{#2} }



%%%%%%%%%Specification commands
\newcommand{\annotation}{BML}
\newcommand{\Apredicate}{\textit{P}}


\newcommand{\requires}{\texttt{requires}}
\newcommand{\ensures}{\texttt{ensures}}
\newcommand{\exsures}[1]{\texttt{exsures(#1)}}
\newcommand{\assert}{\texttt{assert}}

\newcommand{\variant}{\texttt{variant}}

\newcommand{\declare}{\texttt{declare}}
\newcommand{\ghost}{\texttt{Model}}
\newcommand{\ghostSet}{\texttt{set}}
\newcommand{\modifies}{\texttt{modifies}}
\newcommand{\ensemble}[2]{#1 .. #2}
\newcommand{\maxIter}[1]{ iter^#1 }
\newcommand{\progLoop}[1]{\textit{#1}}
\newcommand{\memConsAt}[1]{\Mem^l}

\newcommand{\atState}[2]{#1^{#2} }

%\newcommand{\Mem}{\texttt{MemUsed}}
\newcommand{\Mem}{\texttt{Mem}}
%\newcommand{\old}{\texttt{old}}
\newcommand\Max{\ensuremath{\texttt{Max}}}
\newcommand{\allocated}[1]{allocPath(#1)}
\newcommand{\srcCode}[1]{\texttt{#1}}
\newcommand{\local}[1]{\texttt{localVar}(#1)}



%%%%%%%%%%%% allocation function
\newcommand{\visited}{\texttt{visited}}



%\newcommand{\instrAt}[1]{i_{#1}}
%\newcommand{\instanceOfAlloc}[1]{instanceOfAllocates( #1 )}

\newcommand{\allocInstance}[1]{\texttt{allocInst(#1)}}
\newcommand{\allocLoop}[1]{\texttt{loopCon(#1)}} % function that returns directly the allocations done by a loop : multiplied by the max iterations it can do
\newcommand{\allocMethod}[1]{\texttt{mthdCon(#1)}} % returns the allocations done in a method
\newcommand{\allocLoopWithEnd}[2]{alloc\_loop\_path(#1 , #2)} % returns the allocations done in a loop for a particular path that starts at the start instruction of a loop and that % ends with an insstruction that leads back to the start instructions

\newcommand{\allocIns}[1]{alloc\_instr(#1)} % function that returns that the allocation by the argument

\newcommand{\numLoop}[1]{\textit{numberLoop}(#1)}
\newcommand{\loopEndsSet}[1]{loopEndSet(#1)}
\newcommand{\loopSet}[1]{loopSet(#1)}
\newcommand{\loopEntry}[1]{entry(#1)} % predicate that says that the instruction is an entry to a loop

\newcommand{\backedge}[2]{backedge(#1,#2)} % the start of the backedge  and the end of the backedge


%\newcommand{\wpi}[3]{ \rm{wp}( \srcCode{#1}, #2, #3) } % wp for instructions
\newcommand{\wpExe}[1]{ \rm{wp}(#1) } % wp for blocks
\newcommand{\normalPost}{\psi^{n}}


%\newcommand{\stack}[1]{St(#1)}
\newcommand{\topStack}{c}
\newcommand{\javaNull}{null}
\newcommand{\Ref}[1]{ref_{#1} }
\newcommand{\NULL}{\texttt{null}}



\newcommand{\prevIns}[1]{prev(#1 )}
\newcommand{\nextIns}[1]{next(#1 )}
\newcommand{\targetIns}[1]{target(#1)}



%%%%%%%%%%%%%%%%%%%%%%%%%%%%%%%%%%%%%%%%%%%%%%%%%%%%%


          % input user-defined commands

\begin{document}
%\maketitle      
\tableofcontents
\pagebreak
%\listoffigures
%\listoftables
%\section*{Preface}\normalsize
% \addcontentsline{toc}{chapter}{Preface}
%This document corresponds to the Dxx.xx deliverable of the Inspired project.




 
%%%\mainmatter
\documentclass[a4paper, 11pt]{article}

\begin{document}
toto
\end{document}
    
\section{Introduction}


Smart cards are trusted personal devices whose characteristics are
regulated by the ISO 7816 standard. As other trusted personal devices,
smartcards are designed to store and process confidential data, and
can act as tokens to provide users with a secure electronic
representation in a large network. They are widely deployed and used
in application areas such as mobile telecommunications, banking,
transportation, electronic identity, and digital rights management
(DRM). Further, they hold the promise to play a key role in the
e-society, especially as a means to guarantee users a personalized,
global, and secure access to applications and services.


The prominent role played by trusted personal devices in security
sensitive applications make them an ideal target for
attacks. Traditionally, the main concern with smartcards has been with
hardware attacks in which the attacker gains access to confidential
information or disturbs the functioning of the card through
observation (e.g. of power or electro-magnetic radiations) or invasion
(e.g. overriding sensors or attaching probes). This issue is studied in
Deliverable D8.1.

The trusted personal device remains a specific domain where post
issuance corrections are very expensive due to the deployment process
and the mass production. Furthermore, the emergence of new generation
trusted personal devices increasingly connected to networks and
providing execution support for complex programs and the prospect of
logical attacks has urged the trusted personal devices industry to
improve the quality of their software, as logical attacks are
potentially easier to launch than physical attacks (for example they
do not require physical access to the device, and are easier to
replicate from one device to the other), and may have a huge impact.
In particular, a malicious attacker spreading over the network and
disconnecting or disrupting devices massively could have deep consequences.

This deliverable reports on the development of methodologies and tools
that increase confidence in applications.  For concreteness, we focus 
on Java applications that can be executed on devices that embed Java
Virtual Machines (JVM) or their variants, in particular Java Card
Virtual Machines (JCVM). Java enabled devices are a natural choice for
formal methods because:
\begin{inparaenum}[i)]
\item they are widely deployed in the field;
\item they feature mechanisms that contribute to the security of the
platform and the applications that execute over it;
\item detailed informal specifications of the Java platform are publicly
available, and can be scrutinized.
\end{inparaenum}
However, it should be clear that the methods presented in this document
are relevant to other execution platforms for trusted personal devices.


\subsection{Security issues}
While the focus of the deliverable is on application validation,
security is a holistic property of a system, and formal techniques
must therefore be employed at different levels to provide strong
guarantees about the security of a TPD and its applications.
Essentially, the levels are: the hardware, platform, the libraries,
and the applications.

The need to consider security at those levels is illustrated for
example by the case study described on Page 55 in Deliverable D8.2,
which is concerned with secure platforms. The development of secure
API is discussed in Deliverables D7.1 and D7.2. As previously mentioned,
hardware security is discussed in Deliverable D8.1.


\paragraph*{Platform} The TPD security architecture guarantees that
downloaded applications are innocuous and comply with some basic
policies related to typing, initialization or access control. Such
basic policies are the cornerstones upon which the overall security of
the smartcard will rely. Therefore it is important to verify that the
security architecture does enforce these basic policies as
intended. Thus, an important application of formal methods to TPD
security is platform verification, which aims at providing an abstract
model of the Java platform and security architecture, and at proving
that the security functions play their expected role.

\paragraph*{Libraries} However, it is not sufficient to show that
security functions are correctly designed. In particular, one also has
to ensure that other components of the infrastructure, in particular
API, are correctly designed and implemented. For the purpose of this
deliverable, where the focus is on Java based TPD, the Java API and
the Global Platform API constitute two prominent components of the
infrastructure whose correct design is central to security. 


\paragraph*{Applications} 
Platform and libraries verification is a fundamental step towards
guaranteeing the security of smartcards, and a prerequisite for Common
Criteria evaluations at the highest levels. Nevertheless, the
guarantees offered by the Java security architecture are limited, and
further verifications must be performed to verify that applications
make a legitimate use of the infrastructure, and do not attempt any
hostile action.

Thus, application validation is another important application of
formal methods to TPD security. To date, testing campaigns remain the
primary means to ensure the quality of applications. However, testing
campaigns are expensive and only provide partial guarantees with
regard to the reliability of software. Therefore, it is important to
develop other advanced techniques for applet validation.

There are many facets to applet validation, each with its own
objectives and techniques:
\begin{itemize}
\item one can enhance existing security architectures to enforce
security properties not addressed by current architectures, in
particular confidentiality and availability.  Verification can be
performed by enhanced bytecode verification mechanisms;


\item one can abandon the realm of type systems and its associated
benefits and choose develop logical methods for specifying and
verifying either automatically or efficiently a specific class of
security properties. Verification can be performed by (possibly
efficient and hence incomplete) logic-based proof inference
mechanisms;




\item one can exploit the expressive power of logical methods to
require that applications, or at least sensitive fragments of
applications, are subjected to functional verification, i.e. to
verifications that establish their correctness in terms of
functionality as well as security.
\end{itemize}



\subsection{Logical verification of security properties using JML}
In order to provide precise analyzes with a limited overhead, we
advocate an integrated approach where validation techniques of
increasing strength are used, starting from automated techniques such
as testing and moving towards formal validation using a combination of
automated and interactive tools. In addition, we aim at overcoming the
difficulty of introducing formal techniques in industrial processes by
providing notations and tools hiding the mathematical formalisms and
by integrating formal techniques into classical developers environment
so as to allow users to benefit from formal techniques without having
to learn new formalisms and to become experts.

All the tools and results presented in this document were developed
with this goal in mind, notably the choice of JML as assertion
language and the development of JACK and its associated feature.
Using those techniques, Java developers should be able to validate
their code, or at least to get a good assurance on its correctness.


\subsubsection{JML}
JML~\cite{Leavens-Baker-Ruby99b,Leavens-Baker-Ruby03}, the ``Java
Modeling Language'', is a behavioral interface specification language
for Java; that is, it specifies both the behavior and the syntactic
interface of Java code.  The syntactic interface of a Java class or
interface consists of its method signatures, the names and types of
its fields, etc.  This is what is commonly meant by an application
programming interface (API).  The behavior of such an API can be
precisely documented in JML annotations; these describe the intended
way that programmers should use the API.  In terms of behavior, JML
can detail, for example, the preconditions and postconditions for
methods as well as class invariants. These specifications are given as
annotations of the Java source file. More precisely, they are included
as special Java comments, either after the symbols \lstinline!//@! or
enclosed between \lstinline!/*@! and
\lstinline[basicstyle=\normalfont\ttfamily\small\sl]!@*/!. For example,
the general schema for the annotation of a method is the following:
\begin{lstlisting}
/*@ behavior
  @   requires <precondition>;
  @   ensures <postcondition if no exception raised>;
  @   signals(E) <postcondition when exception E raised>;
  @   assignable <modified fields and variables>;
  @*/
\end{lstlisting}
where \lstinline!requires! specifies the conditions on variables, fields
and method parameters at the beginning of the method call so that the
conditions after \lstinline!ensures! hold at the end of the method
call and the conditions after \lstinline!signals(E)! hold if an
exception is raised and not caught inside the analyzed method.  The
underlying model is a an extension of Hoare-Floyd logic: if the
precondition holds at the beginning of the method call, then
postconditions (with and without exceptions) will hold after the
call. The \lstinline!assignable! clause specifies side-effect affected
variables and is used during the weakest precondition calculus for
method invocations.

An important goal for the design of JML is that it should be easily
understandable by Java programmers. This is achieved by staying as
close as possible to Java syntax and semantics.  Another important
design goal is that JML {\em not} impose any particular design method
on users; instead, JML should be able to document Java programs
designed in any manner \cite{Leavens-Baker-Ruby03}.

JML uses Java's expression syntax in assertions,
thus JML's notation is easy for programmers to learn.  
Because JML supports quantifiers such as
\verb_\forall_ and \verb_\exists_, and because JML allows ``model''
(i.e., specification-only) fields and methods, specifications can
easily be made precise and complete.
JML assertions are written as special
annotation comments in Java code,
so that they are ignored by Java compilers but can be used
by tools that support JML\@.  Within annotation comments JML extends the
Java syntax with several keywords.  It also extends Java's expression syntax with several
operators.
The central ingredients of a JML specification are preconditions
(given in {\tt requires} clauses), postconditions (given in {\tt
  ensures} clauses), and (class and interface) invariants.  These are
all expressed as boolean expressions in JML's extension to Java's
expression syntax.
In addition to ``normal'' postconditions, the language also supports
``exceptional'' postconditions, specified in {\tt signals} clauses.
These can be used to specify what must be true when a method throws an
exception. 

\paragraph*{Styles of specification}
Due to its expressiveness and versatility, the JML specification
language supports several styles of specifications; the choice of one
style of specification over the others depends on the purpose of the
verification effort. In a nutshell, one can either opt for lightweight
specifications in which one introduces enough annotations to reason
about some specific safety property, such as the absence of
exceptions, or heavyweight specifications where functional behavior is
considered. There is of course a great liberty in how \lq\lq
lightweight\rq\rq\ or \lq\lq heavyweight\rq\rq\ a specification should
be, and different styles can be used in different parts of an
application.

In addition, one may opt for defensive specifications, in which methods
are annotated with preconditions that prevent exceptions to occur, or
offensive specifications, which use appropriate clauses to specify 
exceptional postconditions.

\subsubsection{Verification techniques and tools}

JML specifications correctness can be verified either during runtime
or statically~\cite{BurdyCCEKLLP03}. To be verified during runtime, the
source code must have been compiled using \texttt{jmlc}, which is a
enhanced Java compiler for JML annotated code. This compiler adds to
the generated program assertions checking instructions corresponding
to the JML specifications of the program: preconditions, postconditions 
and loop or class invariants. An exception is raised during the execution 
if a JML condition fails. The JML runtime assertion checker can be used
for unit testing~\cite{CL02:ecoop}.


For the static verification of Java programs, several tools are
available using (variations of) JML as specification language. These
tools adopt different compromises between soundness and automation,
and thus it is useful to use them in combination, starting from
automatic but unsound tools, and pursuing with sound but interactive
tools.  Among these tools, ESC/Java2~\cite{CK04:cassis} offers the higher
level of automation as it does not require any user interaction and
relies on the Simplify automatic prover. It is particularly useful for
checking null pointers or array bounds limits; however it is unsound
and incomplete.

In order to further increase the level of reliability of applications,
we propose a methodology based on static verification using
JACK~\cite{BRL-JACK}, a tool that generates proof obligations that
can be discharged using proof assistants or automatic provers.


\subsection{Main contributions}
The work reported in this deliverable builds upon the JACK tool,
that was initially developed within Gemplus. The tool development
was transferred to INRIA at the beginning of the project, with
the objective to improve and increase its functionalities so as
to address the needs of INSPIRED. The main contributions of the
work carried within INSPIRED are:
\begin{itemize}
\item support for verifying high-level security properties
\item support for verifying bytecode programs
\item validation of the methodology for estimating resource
usage and optimizing code for low-footprint.
\end{itemize}
In order to carry the above tasks and evaluations, it has also
been necessary to make many improvements to the tool itself.
It is not in the scope of the present document to describe these 
improvements.


\subsection{Contents of the deliverable}
This document is organized as follows, the next chapter introduce the
assertion language JML, chapter 2 describes the JACK tool with its
extension feature, chapter 3 presents some evaluations done with the
tools and the last chapter concludes.
        
%\chapter{Java Validation}
This chapter gives an overview of the JML language and the tools that have been developed to deal with the it.

\section{Java and JavaCard}
 Formal validation of Java programs is a growing research
 field.  As Java has become a reference language, many technologies are
 emerging to help Java program validation.  Java can also be
 considered as a good support for formal techniques, as it has precise 
semantics \cite{Gosl00a}.

JavaCard is a popular programming language for multiple
application smart cards.  According to the JavaCard Forum \footnote{http://www.javacardforum.org},
which involves key players in the field of smart cards, 
including smart card manufacturers and banks, the JavaCard language has two
important features that make it the ideal choice for smart cards: 
\begin{itemize}
\item JavaCard programs are written in a subset of Java, using
the JavaCard APIs (Application Programming Interfaces). JavaCard
developers can therefore benefit from the well-established Java technology; 

\item the JavaCard security model enables multiple applications to
coexist  on the same card and communicate securely, and in principle,
enables new applications to be loaded on the card after its issuance.
\end{itemize}
Yet recent research has unveiled several problems in the JavaCard
security model, most notably with object sharing and the associated
mechanism of shareable interfaces.
This has  emphasized the necessity to develop environments for
verifying the security of the JavaCard platform and of JavaCard
programs.  Thus far JavaCard security (and also Java security) has
been studied  mainly at two levels:    
\begin{itemize}
\item  platform level: here the goal is to prove safety properties of
the language, in particular type safety and properties related to
memory management; 
\item  application level: here the goal is to prove that a specific
program obeys a given property, and in particular that it satisfies a
security policy, for example based on information flow. 
\end{itemize}
We are focusing at the application level, developing tools and methodologies based on JML to reach this goal.
\section{JML}
JML~\cite{Leavens-Baker-Ruby99b,Leavens-Baker-Ruby03}, the
``Java Modeling Language'', is a behavioral interface
specification language for Java; that is, it specifies both the behavior
and the syntactic interface of Java code.  The syntactic interface of
a Java class or interface consists of its method signatures,
the names and types of its fields, etc.
This is what is commonly meant by an application programming
interface (API).
The behavior of such an API can be precisely documented in JML annotations;
these describe the intended way that programmers should
use the API.  In terms of behavior, JML can detail, for example, the
preconditions and postconditions for methods as well as class
invariants.

An important goal for the design of JML is that it should be easily
understandable by Java programmers. This is achieved by staying as
close as possible to Java syntax and semantics.  Another important
design goal is that JML {\em not} impose any particular design method
on users; instead, JML should be able to document Java programs
designed in any manner \cite{Leavens-Baker-Ruby03}.

JML uses Java's expression syntax in assertions,
thus JML's notation is easy for programmers to learn.  
Because JML supports quantifiers such as
\verb_\forall_ and \verb_\exists_, and because JML allows ``model''
(i.e., specification-only) fields and methods, specifications can
easily be made precise and complete.
JML assertions are written as special
annotation comments in Java code,
so that they are ignored by Java compilers but can be used
by tools that support JML\@.  Within annotation comments JML extends the
Java syntax with several keywords.  It also extends Java's expression syntax with several
operators.
The central ingredients of a JML specification are preconditions
(given in {\tt requires} clauses), postconditions (given in {\tt
  ensures} clauses), and (class and interface) invariants.  These are
all expressed as boolean expressions in JML's extension to Java's
expression syntax.
In addition to ``normal'' postconditions, the language also supports
``exceptional'' postconditions, specified in {\tt signals} clauses.
These can be used to specify what must be true when a method throws an
exception. 

%=====================================================================
%=====================================================================
\subsection{The JML tool suite}
\label{tools}

Since JML specifications are meant to be read and written by ordinary
Java programmers, it is important to support the conventional ways
that these programmers create and use documentation.  Consequently,
the {\tt jmldoc} tool
produces browsable HTML pages containing both the
API and the specifications for Java code, in the style of pages
generated by Javadoc~\cite{Friendly95}.

%=====================================================================
The JML compiler (\texttt{jmlc}), developed at Iowa State University,
is an extension to a Java compiler and compiles Java programs
annotated with JML specifications into Java
bytecode~\cite{Cheon03,Cheon-Leavens02b}.  The compiled bytecode includes
runtime assertion checking instructions that check JML specifications
such as preconditions, normal and exceptional postconditions,
invariants, and history constraints.  The execution of such assertion
checks is transparent in that, unless an assertion is violated, and
except for performance measures (time and space), the behavior of the
original program is unchanged.  The transparency of runtime assertion
checking is guaranteed, as JML assertions are not allowed to have any
side-effects~\cite{Leavens-etal03a}.




\section{ESC/Java2}
\label{escjava}

ESC/Java2 tool~\cite{Flanagan-Et-Al02}, originally developed at Compaq Research,
performs what is called ``extended static
checking''~\cite{ESC:Overview,10yearsESC},
compile-time checking that goes well beyond type checking.  It can
check relatively simple assertions and can check for certain kinds of
common errors in Java code, such as dereferencing \texttt{null},
indexing an array outside its bounds, or casting a reference to an
impermissible type.  ESC/Java2 supports a subset of JML and also checks
the consistency between the code and the given JML annotations.  The
user's interaction with ESC/Java2 is quite similar to the interaction
with the compiler's type checker: the user includes JML annotations in
the code and runs the tool, and the tool responds with a list of
possible errors in the program.

JML annotations affect ESC/Java2 in two ways.  First, the given JML
annotations help ESC/Java2 suppress spurious warning messages.   Second,
annotations make ESC/\-Java2 do additional checks.  
In these two ways, the use of JML annotations enables ESC/Java2 to
produce warnings not at the source locations where errors manifest
themselves at runtime, but at the source locations where the errors
are committed.

%=====================================================================
%\section{Applications of JML to Java Card}
%\label{applications}

%Although JML is able to specify arbitrary sequential Java programs,
%most of the serious applications of JML and JML tools up to now
%have targeted Java Card.  Java Card$^{TM}$ is a dialect of Java specifically
%designed for the programming of the latest generation of smartcards.
%Java Card is adapted to the hardware limitations of smartcards; for
%instance, it does not support floating point numbers, strings, object
%cloning, or threads.

%Java Card is a well-suited target for the application of formal
%methods.  It is a relatively simple language with a restricted API\@.
%Moreover, Java Card programs, called ``applets'', are small, typically
%on the order of several KBytes of bytecode.  Additionally, correctness
%of Java Card programs is of crucial importance, since they are used in
%sensitive applications such as bank cards and mobile phone SIMs.  (An
%interesting overview of security properties that are relevant for Java
%Card applications is available~\cite{MarletLM01}.)

%JML, and several tools for JML, have been used for Java Card,
%especially in the context of the EU-supported project VerifiCard
%(www.verificard.org).  JML has been used to write a formal
%specification of almost the entire Java Card
%API~\cite{PollBergJacobs01}.  This experience has shown that JML is
%expressive enough to specify a non-trivial existing API\@.  The
%runtime assertion checker has been used to specify and verify a
%component of a smartcard operating system~\cite{PollHarteldeJong02}.




\section{Tools and Methodologies}
Traditionally, applications for trusted personal devices undergo
extensive testing and code review in order to avoid programming 
errors and security flaws. However, test campaigns and code review
can be expensive and do not guarantee the absence of programming
errors and security flaws. Thus, we propose to use validation
techniques based on annotations and verification conditions,
that do guarantee the lack of programming mistakes and that
all security properties that have been specified and verified
do indeed hold.

This section describes some tools that can be handled to validate Java applications
and methodologies that can be applied depending on different
validation purposes.  All those tools are part a Java application
validation workshop called JACK.

\subsection{JACK}
The Java Applet Correctness Kit (or \JACK), already briefly described in \cite{BRL-JACK}, is
a formal tool that allows one to prove properties on Java programs
using the Java Modeling Language \cite{Leavens-Baker-Ruby03} (JML).

It generates proof obligations (also called verification condition) allowing to prove that the Java code
conforms to its JML specification.  The lemmas issued from the generation are are described into an
internal formula language called JPOL (Java/Jack Proof Obligation
Language). Then JPOL verification conditions are translated into
different prover language, namely Coq, PVS, Simplify and the B
language \cite{bbook}, allowing to use the automatic provers Simplify
and the provers developed within the B method and interactive provers
like the Coq proof assistant, PVS or Click'n'Prove.

 But the tool is not yet another lemma generator for Java, since it
 also provides a lemma viewer integrated in the eclipse
 IDE\footnote{\texttt{http://www.eclipse.org}}, which is one of the
 most commonly used IDE for Java developers.  This allows to hide the
 formalisms used behind a graphical interface.  Lemmas are presented
 to users in a way they can understand them easier, by using the Java
 syntax and highlighting code portions to help the
 understanding. Using \JACK, one does not have to learn a formal
 language to be confident on code correctness.

\subsubsection{Jack Proof Obligation Language}
 \label{JPOL}
The Java/Jack Proof Obligation Language is an internal language used in Jack to represent verification conditions issued from the weakest precondition calculus. It can be considered as a melting-pot high level language based on first order logic with some specific features like:
\begin{itemize}
\item some basic set theory constructions (coming from the B notation)
\item some JML keywords
\item some Java constructions
\end{itemize}
The language is typed but has no specific syntax. It corresponds to an internal representation in the tool. Its semantic is given by the translation into the different theorem provers. Adding a theorem prover in JACK corresponds mainly to convert expressions of the JPOL language into the specific theorem prover language.

\subsubsection{JACK Plugins}
Different plugins have been developped around the JACK workshop benefeting from its extensible architecture. A front-head for Java bytecode has been developped. This tool generates verification condition from Java bytecode in the JPOL language. This allows it to benefit from the JACK features like verification condition viewer and multi-provers. This tool is described in a section~\ref{bytecodesection}. Provers are also plugin that have been successively added to JACK. And finally tools allowing to help in writing the assertions into the code has been developped. The first one generates precondition to prevent against runtime exception and is based on the in-built JACK weakest precondition calculus, the second one aims to propagate core-anotations allowing to automatize the proof of some security property: this tool is described in the next section.


\subsection{Security property propagation}\label{sec:highlevel}

While JML is easily accessible to Java developers and tools exist
to manage the annotations, actually writing
the specifications of a Java application is labor-intensive and
error-prone, as it is easy to forget some annotations. There
exist tools which assist in writing these annotations,
\emph{e.g.}~Daikon~\cite{ErnstCGN2001:TSE} and Houdini~\cite{FlanaganL01}
use heuristic methods to produce annotations for simple safety and
functional invariants.  However, these tools cannot be guided by the
user---they do not require any user input---and in particular cannot
be used to synthesize annotations from realistic security policies.

We describe here, a method that, given a security policy,
automatically annotates a Java (Card) application, in such a way that
if the application respects the annotations then it also respects the
security policy. The generation of annotations proceeds in two phases:
synthesizing and weaving.
\begin{enumerate}
\item Based on the security policy we \emph{synthesize} core annotations, 
specifying the behavior of the methods directly involved.
\item Next we propagate these annotations to all methods directly or
indirectly invoking the methods that form the core of the security
policy, thus \emph{weaving} the security policy throughout the
application. 
\end{enumerate} 

To show the usefulness of our approach, we applied the algorithm to
several realistic examples of industrial applications. When doing
this, we actually found violations against the security policies
documented for some of these applications. To help understanding, we present an example of some security property before explaining the weaving algorithm. Results are reported in Section~\ref{SecResults}.

\subsubsection{Atomicity}\label{SecHighLevelSecProp}
In order in explain the weaving algorithm, we fist give an example of security property taken from credit card security requirements.  

A smart card does not include a power supply, thus a brutal retrieval
from the terminal could interrupt a computation and bring the system in
an incoherent state. To avoid this, the Java Card
specification prescribes the use of a transaction mechanism to
control synchronized updates of sensitive data. A 
statement block surrounded by the methods \texttt{beginTransaction()} and
\texttt{commitTransaction()} can be considered atomic.
If something happens while executing the transaction (or if
\texttt{abortTransaction()} is executed), the card will
roll back its internal state to the state before the transaction was
begun.

To ensure the proper functioning and prevent abuse of this mechanism,
several security properties can be specified.

\begin{quote}
\textbf{No nested transactions} Only one level of transactions
is allowed.\smallskip\\
\textbf{No exception in transaction} All exceptions that may be thrown
inside a transaction, should also be caught inside the
transaction.\smallskip\\
\textbf{Bounded retries}
No pin verification may happen within a transaction.
\end{quote} 
The second property ensures that the \texttt{commitTransaction} will
always be executed. If the exception is not caught, the
\texttt{commitTransaction} would be ignored and the transaction would
not be finished. The last property excludes pin verification within a
transaction. If this would be allowed, one could abort the transaction
every time a wrong pin code has been entered. As this rolls
back the internal state to the state before the transaction was
started, this would also reset the retry counter, thus allowing an
unbounded number of retries. Even though the specification of the Java
Card API prescribes that the retry counter for pin verification cannot
be rolled back, in general one has to check this kind of properties.



\subsubsection{Automatic Verification of Security Properties}\label{SecVerif}
As explained above, we are interested in the verification of
high-level security properties that are not directly related to a
single method or class, but that guarantee the overall
well-functioning of an application. Writing appropriate JML
annotations for such properties is tedious and error-prone, as they
have to be spread all over the application. Therefore, we propose a
way to construct such annotations automatically. First we synthesize
core-annotations for methods directly involved in the property.  For
example, when specifying that no nested transactions are allowed, we
annotate the methods \texttt{beginTransaction},
\texttt{commitTransaction} and
\texttt{abortTransaction}. Subsequently, we propagate the necessary 
annotations to all methods (directly or indirectly) invoking these
core-methods.  The generated annotations are sufficient to respect the
security properties, \emph{i.e.}~if the applet does not violate the
annotations, it respects the corresponding high-level security
property.

Whether the applet respects its annotations can be established with
JACK~\cite{BRL-JACK}. Since for most security properties the
annotations are relatively simple---but there are many---it is
important that these verifications are done automatically, without any
user interaction. The results in Section~\ref{SecResults} show that
for the generated annotations all correct proof obligations can indeed
be automatically discharged.

\paragraph*{Architecture}

%\begin{figure}[ht]
%\begin{center}
%\epsfig{file=isaac.eps, width=9cm}
%\end{center}
%\caption{\sc Tool set for verifying high-level security properties}\label{FigArch}
%\end{figure}



%Figure~\ref{FigArch} shows the general architecture of the tool set
%for verifying high-level security properties. 
Our annotation generator
can be used as a front-end for any tool accepting JML-annotated Java
(Card) applications. As input we have a security property and a Java
Card applet. The output is a JML Abstract Syntax Tree (AST), using the
format as defined for the standard JML parser. When pretty-printed,
this AST corresponds to a JML-annotated Java file. From this
annotated file, JACK generates appropriate proof obligations to check
whether the applet respects the security property.

\paragraph*{Automatic Generation of Annotations}
We provide here a brief description of the
weaving phase, \emph{i.e.}~how the core-annotations are propagated
throughout the applet. Functions \textsf{mod}, \textsf{pre},
\textsf{post} and \textsf{exc\-post} have been defined, propagating assignable clauses,
preconditions, postconditions and exceptional postconditions,
respectively. These functions have been implemented for
the full Java Card language, but to present our ideas, we only give
the definitions of  \textsf{pre} for a representative subset of statements: statement
composition, method calls, conditional and \texttt{try}-\texttt{catch}
statements and special set-annotations. We assume the existence of
domains \textsf{MethName} of method names, \textsf{Stmt} of Java Card
statements, \textsf{Expr} of Java Card expressions, and \textsf{Var}
of static ghost variables, and functions \textsf{call} and
\textsf{body}, denoting a method call and body, respectively.

All functions are defined as mutual recursive functions on method
names, statements and expressions. When a method call is encountered,
the implementation will check whether annotations already have been
generated for this method (either by synthesizing or weaving). If not
it will recursively generate appropriate annotations. Java Card
applets typically do not contain (mutually) recursive method calls,
therefore this does not cause any problems. Generating appropriate
annotations for recursive methods would require more care (and in
general it might not be possible to do without any user interaction).


%\paragraph{Propagation of assignable clauses}
%First we define a function \textsf{mod} that propagates
%assignable clauses for static ghost variables.
%\begin{definition}
%\(\mathsf{mod} \colon \mathsf{MethName} \rightarrow
%\mathcal{P}(\mathsf{Var})\), 
%\(
%\mathsf{mod}  \colon  \mathsf{Stmt}   \rightarrow
%\mathcal{P}(\mathsf{Var}) \), and 
%\(\mathsf{mod}  \colon  \mathsf{Expr}   \rightarrow  \mathcal{P}(\mathsf{Var})\)
%by rules like (where \(m,n\,\colon\mathsf{MethName}\),
%\(s_1,s_2\,\colon\mathsf{Stmt}\), \(c\,\colon\mathsf{Expr}\) and \(x\colon\mathsf{Var}\)):
%\[
%\begin{array}{rcl}
%\mathsf{mod}(m) & = & \mathsf{mod}(\mathsf{body}(m)) \smallskip \\
%\mathsf{mod}(s_1 \mathtt{;} s_2) & = & \mathsf{mod}(s_1) \cup \mathsf{mod}(s_2)\\
%\mathsf{mod}(\mathsf{call}(n)) & = &  \mathsf{mod}(n) \\
%\mathsf{mod}(\mathtt{if\:(} c \mathtt{)\:} s_1 \mathtt{\:else\:} s_2)
%&=& \mathsf{mod}(c) \cup \mathsf{mod}(s_1) \cup \mathsf{mod}(s_2)\\
%\mathsf{mod}(\mathtt{try\:} s_1 \mathtt{\:catch\:(} E \mathtt{)\:}
%s_2) & = & \mathsf{mod}(s_1) \cup \mathsf{mod}(s_2)\\
%\mathsf{mod}(\mathtt{set\:}x \mathtt{\:=\:} c) & = &  \{ x \} 
%\end{array}
%\]
%\end{definition}

We define the function \textsf{pre} for
propagating preconditions. This function analyses a
method body in a sequential way---from beginning to end---computing
which preconditions of the methods called within the body have to be
propagated. To understand the reasoning behind the definition, we will
first look at an example. Suppose we are checking the \textbf{No
nested transactions} property for an application, which contains a
method \texttt{m}, whose only method calls are those shown, and which
does not contain any set annotations.
\begin{verbatim}
void m() { ... // some internal computations
           JCSystem.beginTransaction();
           ... // computations within transaction
           JCSystem.commitTransaction(); }
\end{verbatim}
Core-annotations are synthesised for \texttt{beginTransaction}
and \texttt{commit\-Transaction}. The annotations for
\texttt{beginTransaction} are shown below
\begin{verbatim}
/*@ requires TRANS == 0;
  @ assignable TRANS;
  @ ensures TRANS == 1; @*/
public static native void beginTransaction() 
                          throws TransactionException;
\end{verbatim}
Since the method is native, one cannot describe its body. However, if
it had been non-native, an annotation \texttt{//@ set TRANS = 1;}
would have been generated, to ensure that the method satisfies its
specification.

Likewise \texttt{commitTransaction} requires \texttt{TRANS == 1}
and ensures \texttt{TRANS == 0}. As we assume that \texttt{TRANS} is
not modified by the code that precedes the call to
\texttt{beginTransaction},  the only way the precondition of this method
can hold, is by requiring that it already holds at the moment
\texttt{m} is called. Thus, the precondition of
\texttt{beginTransaction} has to be propagated. In contrast, the
precondition for \texttt{commitTransaction} (\texttt{TRANS == 1})
has to be established by the postcondition of
\texttt{begin\-Transaction}, because the variable \texttt{TRANS} is
modified by this method. 
Thus, preconditions containing only unmodified variables should be
propagated.  Propagating pre- or postconditions can be considered as
passing on a method contract. Method bodies can only pass on contracts
for variables they do not modify; once they modify a variable it is
their duty to ensure that the necessary conditions are satisfied.

We assume the existence of a domain \textsf{Pred} of predicates using
static ghost variables only, and function
\textsf{fv}, returning the set of free variables.

\begin{definition}[\textsf{pre}]
We define
\(\mathsf{pre} \colon \mathsf{MethName} \rightarrow
\mathcal{P}(\mathsf{Pred})\),
\( \mathsf{pre}  \colon  \mathsf{Stmt} \rightarrow
\mathcal{P}(\mathsf{Var}) \rightarrow \mathcal{P}(\mathsf{Pred}) \), and
\( \mathsf{pre}  \colon  \mathsf{Expr} \rightarrow
\mathcal{P}(\mathsf{Var}) \rightarrow \mathcal{P}(\mathsf{Pred}) \)
(where \(m,n\,\colon\mathsf{MethName}\),
\(s_1,s_2\,\colon\mathsf{Stmt}\), \(c\,\colon\mathsf{Expr}\),
\(V\colon\mathcal{P}(\mathsf{Var})\) and \(x\colon\mathsf{Var}\)): 
\[
\begin{array}{lcl}
\mathsf{pre}(m) & = & \mathsf{pre}(\mathsf{body}(m), \emptyset) \smallskip\\
\mathsf{pre}(s_1 \mathtt{;} s_2, V) & = & \mathsf{pre}(s_1, V) \cup 
                                          \mathsf{pre}(s_2, V \cup \mathsf{mod}(s_1))\\

\mathsf{pre}(\mathsf{call}(n),V) & = & 
                \{ p \mid p \in \mathsf{pre}(n) \wedge 
                          (\mathsf{fv}(p) \cap V) = \emptyset\}\\
\mathsf{pre}(\mathtt{if\:(} c \mathtt{)\:} s_1 \mathtt{\:else\:} s_2, V) & = &
   \mathsf{pre}(c, V) \cup 
   \mathsf{pre}(s_1, V \cup \mathsf{mod}(c)) \cup
   \mathsf{pre}(s_2, V \cup \mathsf{mod}(c))\\
\mathsf{pre}(\mathtt{try\:} s_1 \mathtt{\:catch\:(} E \mathtt{)\:} s_2, V) & = & 
   \mathsf{pre}(s_1, V) \cup 
   \mathsf{pre}(s_2, V \cup \mathsf{mod}(s1))\\
\mathsf{pre}( \mathtt{set\:} x \mathtt{\:=\:} c) & = & \{\: \}
\end{array}
\]
\end{definition}

In the rules defining \textsf{pre} on \textsf{Stmt} and \textsf{Expr},
the second argument denotes the set of static ghost variables that have been
modified so far. When calculating the precondition for a method, we
calculate the precondition of its body, assuming that so far no variables
have been modified. For a statement composition, we first
propagate the preconditions for the first sub-statement, and then for
the second sub-statement, but taking into account the variables
modified by the first sub-statement. When propagating the preconditions
for a method call, we propagate all preconditions of the called method
that do not contain modified variables.  Since we are restricting our
annotations to expressions containing static ghost variables only, in
the rule for the conditional statement we cannot take the outcome of
the conditional expression into account. As a consequence, we
sometimes generate too strong annotations, but in practice this does
not cause problems. Moreover, it should be emphasised that this
only can make us reject correct applets, but it will never make us
accept incorrect ones.  Similarly, for the
\texttt{try}-\texttt{catch} statement, we always propagate the
precondition for the \texttt{catch} clause, without checking whether it
actually can get executed. Again, this will only make us reject
correct applets, but it will never make us accept incorrect
ones. Finally, a set annotation does not give rise to any propagated
precondition. 

Notice that by definition, we have the following property for the
function \textsf{pre} (where \(s\) is either in \textsf{Stmt} or
\textsf{Expr}, and \(V\) is a set of static ghost variables).
\[
p \in \textsf{pre}(s, V) \Leftrightarrow (p \in \textsf{pre}(s,
\emptyset) \wedge (\textsf{fv}(p) \cap V) = \emptyset)
\]

%\paragraph{Propagation of postconditions}
In a similar way, we define functions \textsf{post} and
\textsf{exc\-post},  computing the
set of postconditions and exceptional postconditions that have to be
propagated for method names, statements and expressions. The main
difference with the definition of \textsf{pre} is that these functions run
through a method from the end to the beginning. Moreover, they have to
take into account the different paths through the method. For
each of these possible paths, we calculate the appropriate
(exceptional) postcondition. The overall (exceptional) postcondition
is then defined as the disjunction of the postconditions related to
the different paths through the method. 

\paragraph{Example}
For the example discussed above, our functions compute the following
annotations.

\begin{verbatim}
/*@ requires TRANS == 0;
  @ assignable TRANS;
  @ ensures TRANS == 0; @*/
void m() { 
   ... // some internal computations
  JCSystem.beginTransaction();
  ... // computations within transaction
  JCSystem.commitTransaction(); }
\end{verbatim}
This might seem trivial, but it is important to realise that similar
annotations will be generated for all methods calling
\texttt{m}, and transitively for all methods calling the methods
calling \texttt{m} \emph{etc.}
Having an algorithm to generate such annotations enables to check
automatically a large class of high-level security properties.
Results on real JavaCard applications are presented in Section~\ref{SecResults}.

\subsection{Bytecode validation}\label{bytecodesection}
We propose a bytecode verification framework with the following components: a bytecode specification language, a compiler from source
 program annotations into bytecode annotations and a verification condition generator over Java bytecode.

In a client-producer scenario, these features bring to the producer means to supply the sufficient specification information 
which will allow the client to establish trust in the code, especially when the client policy is potentially complex and a fully automatic specification inference
will fail. On the other hand, the client is supplied with a procedure to check the untrusted annotated code. 

Our approach is tailored to Java bytecode.
The Java technology is widely applied to mobile and embedded components because of its portability across platforms. 
For instance, its dialect JavaCard is largely used in smart card applications and the J2ME Mobile Information Device Profile 
(MIDP) finds application in GSM mobile components. 


The proposed scheme is composed of several components.
 We define a bytecode specification language, called BML, and supply a compiler from 
 the high level Java specification language JML~\cite{JMLRefMan} to BML. 
 BML supports a JML subset which is expressive enough to specify rich functional properties. 
The specification is inserted in the class file format in newly defined attributes and thus makes not
 only the code mobile but also its specification. These class
 file extensions do not affect the JVM performance.
We define a bytecode logic in terms of weakest precondition calculus for the sequential Java bytecode language. 
The logic gives rules for almost all Java bytecode instructions and supports the Java specific features such as
exceptions, references, method calls and subroutines.  
 We have implementations of a verification condition generator based on the weakest precondition calculus and of
 the JML specification compiler. Both are part of JACK.

  The full specifications of the JML compiler, the weakest precondition predicate transformer definition and its proof of correctness can be found in~\cite{JBL05MP}.
  
The remainder of this section is organized as follows: 
Subsection~\ref{architecture_s} reviews scenarios in which the architecture may be appropriate to use; 
 Subsection~\ref{bcSpecLg} presents the bytecode specification language BML and the JML compiler; Subsection~\ref{wpbc} discusses the main
features of the \wpi (short for weakest precondition calculus); Subsection~\ref{pogEquiv} discusses the relationship between the verification conditions for JML annotated source and BML annotated bytecode.


\subsubsection{Applications}
\label{architecture_s}  


The overall objective is to allow a client to trust a code produced by an untrusted code producer. Our approach is especially suitable
 in cases where the client policy involves non trivial functional or safety requirements and thus, a full automatization of the verification 
process is impossible. To this end, we propose a PCC technique that exploits the JML compiler and the weakest predicate function presented in the article. 
 
%The framework is presented in Fig.~\ref{architecture}; note that certificates and their checking are not yet implemented
% and thus are in oblique font.
  


%\begin{figure}[!ht]
%\centering
%\epsfig{ file=sac.eps, width=10cm}
%\caption{\sc The overall architecture for client producer scenarios }
%\label{architecture}
%\end{figure}

In the first stage of the process the client provides the functional
and (or) security requirements to the producer.  The requirements can
be in different form:
\begin{itemize}
\item Typical functional requirements can be a specified interface
describing the application to be developed. In that case, the client
specifies in JML the features that have to be implemented by the code
producer.
\item Client security requirements can be a restricted access to some
method from the API expressed as a finite state machine.  For example,
suppose that the client API provides transaction management facilities
- the API method \texttt{open} for opening and method \texttt{close}
for closing transactions. In this case, a requirement can be for no
nested transactions.  This means that the methods \texttt{open} and
\texttt{close} can be annotated to ensure that the method
\texttt{close} should not be called if there is no transaction running
and the method \texttt{open} should not be called if there is already
a running transaction. In this scenario, we can apply results of
previous work \cite{m+04:cardis}.  
\end{itemize}
Usually, the development process involves annotating the source code
with JML specification, generating verification conditions, using
proof obligation generator over the source code and discharging proofs
which represent the program safety certificate and finally, the
producer sends the certificate to the client along with the annotated
class files.  Yielding certificates over the source code is based on
the observation that proof obligations on the source code and
non-optimized bytecode respectively are syntactically the same modulo
names and basic types. Every Java file of the untrusted code is
normally compiled with a Java compiler to obtain a class file. Every
class file is extended with user defined attributes that contain the
BML specification, resulting from the compilation of the JML
specification of the corresponding Java source file.


We have extended the Jack tool with a compiler from
 JML to BML and a bytecode verification condition generator. In the next sections, we introduce
 the BML language, the JML compiler and the bytecode \wpi calculus which underlines the bytecode verification condition generator.
 

\subsubsection{Bytecode Specification Language}\label{bcSpecLg}

In this section, we introduce a bytecode specification language which we call BML (short for ByteCode Specification Language).
 BML is based on the design principles of JML (Java Modeling Language)~\cite{JMLRefMan}, which is a behavioral interface specification 
language following the design by contract approach \cite{M97oos}.

In the following, we give the grammar of BML and sketch the compiler from JML to BML. 

\paragraph{Grammar} \label{grammar}


BML corresponds to a representative subset of JML and is expressive enough for most purposes including the description of non trivial functional and security properties.


 Specification clauses in BML that are taken from JML and inherit their semantics directly from JML include:
class specification, i.e. class invariants and history constraints, 
  method preconditions, normal and exceptional postconditions, method frame conditions (the locations that may be modified by the method), inter method specification, as for instance loop invariants and loop frame conditions(this is not a standard feature of JML but we were inspired for this by the JML extensions in JACK ~\cite{BRL-JACK}). 
We also support specification inheritance and thus behavioral subtyping as described in \cite{Dhara-Leavens96}. Most of the Java expressions like field access expressions, local variables, etc can be mentioned in the BML specification.
BML supports the standard JML specification operators as for example, $\old{\expression}$ which is used in method postconditions and
 designates the value of the expression $\expression$ in the prestate of a method, $ \result$ which stands for the value the method
returns if it is not void,  $\typeof{\expression}$ which stands for type of $\expression$ etc.  

\subsubsection{Compiling JML into BML}\label{comJML}


We now turn to explaining how JML specifications are compiled into user defined attributes for Java class files. Recall that a class file defines
a single class or interface and contains information about  the class name, interfaces implemented by the class, super class, methods and fields declared in the class and references. The Java Virtual Machine Specification (JVM) \cite{VMSpec} mandates that the class file contains data structure usually referred as the \textbf{constant\_pool} table which is used to construct the runtime constant pool upon class or interface creation. The runtime constant pool serves for loading, linking and resolution of references used in the class. The JVM allows to add to the class file user specific information(\cite{VMSpec}, ch.4.7.1). This is done by defining user specific attributes  (their structure is predefined by JVM).

Thus the ``JML compiler'' \footnote{not to be confused, Gary Leavens also calls his tool jmlc JML compiler, which transforms JML into runtime checks and thus generates input for the jmlrac tool  } compiles the JML source specification into user defined attributes. The compilation process has three stages:
\begin{enumerate}
\item Compilation of the Java source file. This can be done by any Java compiler that supplies for every method in the generated class file 
the \textbf{Line\_Number\_Table} and \textbf{Local\_Variable\_Table}  attributes. The presence in the Java class file format of 
these attribute is optional \cite{VMSpec}, yet almost all standard non optimizing compilers can generate these data. 
The \textbf{Line\_Number\_Table} describes the link between the source line and the bytecode of a method.  
The \textbf{Local\_Variable\_Table} describes the local variables that appear in a method. 
Those attributes are important for the next phase of the JML compilation.
\item Compilation of the JML specification from the source file and the resulting class file. In this phase, Java and JML source identifiers are 
linked with their identifiers on bytecode level, namely with the corresponding indexes either from the cp (short for constant pool) or the array of 
local variables described in the \textbf{Local\_Variable\_Table} attribute. If a field
identifier, for which no cp index exists, appears in the JML specification, a new index is added in the cp and the field identifier in question
is compiled to the new cp index. It is also in this phase that the specification parts like the loop invariants and the assertions which should hold at a certain point in the source program must be associated to the respective program point on bytecode level. The specification
is compiled in binary form using tags in the standard way. The compilation of an expression is a tag followed by the compilation of its subexpressions. 

Another interesting point in this stage of the JML compilation is how the type differences on source and bytecode level are treated. 
The JVM does not provide a direct support for integral types like byte, short, char, neither for boolean.
 Those types are rather encoded as integers in the bytecode. The JML compiler performs transformation on specifications that involve Java boolean values and variables.

\item add the result of the compiled specifications components in
newly defined attributes in the class file.
 For example, the specifications of all the loops in a method are compiled to a unique method attribute.% whose syntax is given in Fig.~\ref{loopAttribute}. 
This attribute is an array of data structures each describing a single loop from the method source code. 
More precisely, every element contains information about the instruction where the loop starts as specified in the
\textbf{Line\_Number\_Table}, the locations that can be modified in a loop iteration, 
 the invariant associated to this loop and the decreasing expression in case of total correctness, 

\end{enumerate}

%\begin{figure}[ht]
%\textbf{
%\begin{tabbing}
%JML\=Loop\_specification\_attribute \{ \\
%\> ...\\
%\> \{\hspace{3 mm}\=  index;\\
%\> \>  modifies\_count;\\
%\> \> formula modifies[modifies\_count];\\
%\> \> formula invariant;\\
%\> \> expression decreases;\\
%\> \} loop[loop\_count];\\
%\}
%\end{tabbing}
%}
%
%\caption{\sc Structure of the loop specification attribute}
%\label{loopAttribute}
%
%\end{figure}
 The most problematic part of the specification compilation is the identification of
 which loop in the source corresponds to which bytecode loop in the control flow
 graph. To do this, we assume that the control flow graph is reducible, 
i.e. there are no jumps into the middle of the loops from outside; graph reducibility allows to establish the same order between loops in the
 bytecode and source code level and to find the right places in the bytecode where the loop invariants must hold.


\subsubsection{Weakest Precondition Calculus For Java
Bytecode}\label{wpbc} In this section, we define a bytecode logic in
terms of a weakest precondition calculus. The proposed weakest
precondition \wpi \ supports all Java bytecode sequential instructions
except for floating point arithmetic instructions and 64 bit data
(\java{long} and \java{double} types), including exceptions, object
creation, references and subroutines. The calculus is defined over the
method control flow graph and supports BML annotation, i.e. bytecode
method's specification like preconditions, normal and exceptional
postconditions, class invariants, assertions at particular program
point among which loop invariants. The verification condition
generator applied to a method bytecode generates a proof obligation
for every execution path by applying first the weakest predicate
transformer to every \instr{return} instruction, \instr{athrow}
instruction and end of a loop instruction and then following in a
backwards direction the control flow up to reaching the entry point
instruction. In a related document \cite{JBL05MP}, we show that the
$\wpi$ function is correct.


 In Fig.~\ref{instrWP}, we show the \wpi \ rule for the \instr{Type\_load i} instruction.
 As the example shows the \wpi \ function takes three arguments:
the instruction for which we calculate the precondition, 
the instruction's postcondition $\psi$ and the exceptional postcondition function $\excPost$ which for any exception \texttt{Exc} and 
instruction index \texttt{ind} returns the
corresponding exceptional postcondition $\excPost(\texttt{Exc}, \tt{ind})$. One can also notice that the rule involves the stack expressions \counter 
(stands for the counter of the method execution stack) \ and \stack{ i } (stands for the element at ind \texttt{i} from the stack top).
 This is because the JVM is stack based and the instructions take their arguments from the method execution stack and 
 put the result on the stack.
 The \wpi \ rule for  \instr{Type\_load i} increments the stack counter \counter \ and loads on the stack top the contents
 of the local variable $\register{i}$. 




\begin{figure}[ht]
\[
\begin{array}{l}
\wpi(\instr{Type\_load \ i}, \ \psi, \ \excPost)  = \\
\begin{array}{l}  \psi\substitution{\counter }{\counter+1} \substitution{\stack{\counter+1}}{\register{i}} \end{array} \\
\\
\\
 \wpi(\instr{putField} \ \texttt{Cl.f}, \ \psi, \ \excPost)  = \\
\begin{array}{l}
                \stack{\counter -1} \not= \Mynull\Rightarrow   
         \psi\begin{array}{l} \substitution{\counter}{ \counter-2 } \\[0 mm] 
                           \substitution{\texttt{Cl.f} }{ \texttt{Cl.f}\oplus [\stack{\counter -1} \rightarrow \stack{\counter}] } \\
                \end{array}\\

   \wedge \\
        \stack{\counter-1} = \Mynull    \Rightarrow \excPost(\tt{NullPointerExc})
        \begin{array}{l}
          \substitution{ \counter }{ 0} \\
          \substitution{\stack{0}}{ \stack{\counter}} 
        \end{array}
    \end{array} %\biggr. & \\
\end{array}
 \]       
\caption{\sc Examples for bytecode wp rules}
 \label{instrWP}

\end{figure}

In the following, we consider how instance fields, 
loops exception handling and subroutines are treated. We omit here aspects like method invocation and object creation because of space limitations but a detailed explanation can be found in~\cite{JBL05MP}. 

\paragraph{Manipulating object fields}
Instance fields are treated as functions, where the domain of a field \texttt{f} 
declared in the class \texttt{Cl} is the set of objects of class \texttt{Cl} and its subclasses.
We are using function updates when assigning a value to a field reference as, for instance in~\cite{B00ppp}.
In Fig.\ref{instrWP}, we give the \wpi \ rule for the
instruction \instr{putfield} \texttt{Cl.f}, which updates the field \texttt{Cl.f}\footnote{ \texttt{Cl.f} stands for the field \texttt{f} declared in class 
\texttt{Cl}} of the object referenced by the reference stored in the stack below the stack top \stack{\counter-1} with the value on the stack top \stack{\counter}.
Note that the rule takes in account the possible exceptional termination of the instruction execution.


\paragraph*{Loops}

Identifying loops on bytecode and source programs is different because of their different nature --- 
the first one lacks while the second has structure. While on source level loops correspond to loop statements,  
on bytecode level we have to analyze the control flow graph in order to find them.
 The analysis consists in looking for the backedges in the control flow graph using standard compiler techniques. 
  
 We assume that a method's bytecode is provided with sufficient specification and in particular loop invariants.
 Under this assumption, we build an abstract control flow graph where the backedges are replaced by
 the corresponding invariant. We apply the \wpi \ function over the abstract version of the control flow graph which generates verification conditions for the 
preservation and initialization of every invariant in the abstraction graph. 


     

\paragraph*{Exceptions and Subroutines}
Exception handlers are treated by identifying the instruction at which the handler compilation starts. The JVM specification mandates 
that a Java compiler must supply for every method an \textbf{Exception\_Table} attribute that contains data structures describing the compilation of every implicit (in the presence of subroutines) or explicit exception handler: the instruction at which the compiled exception handler starts,
 the protected region (its start and end instruction indexes), and the exception type the exception handler protects from. Thus, 
for every instruction \instr{ins} in method \method~ which may terminate exceptionally on exception \texttt{Exc} the exceptional function
 $\excPost$  returns the \wpi \ predicate of the exceptional handler protecting \instr{ins} from \texttt{Exc} if such a handler exists.
Otherwise, $\excPost$ returns the specified exceptional postcondition for exception \texttt{Exc} as specified in the specification of
method \method.

Subroutines are treated by abstract inlining\footnote{NB: we do not transform the bytecode. It is rather the \wpi \
 function that treats subroutines as if the subroutines were inlined}. First, the instructions of every subroutine
 are identified. 
To this end, we assume that the code has passed the bytecode verification and that every subroutine terminates with a \instr{ret} 
instruction(usually, the compilation of subroutines ends with a \instr{ret} instruction but it is not always the case). Thus, by abstract inlining, we mean that
 whenever the \wpi~function is applied to an instruction \instr{jsr}  \texttt{ind}, a postcondition $\psi$ and an exceptional postcondition function $\excPost$, its precondition  $\wpi^{\instr{jsr} \ \tt{ind}}$ is calculated as follows: the \wpi \ is applied to the bytecode instructions that represent the subroutine which starts at instruction \texttt{ind},
 the postcondition $\psi$ and the exceptional postcondition function  $\excPost$.





 \subsubsection{Relation between verification conditions on source and
 bytecode level } \label{pogEquiv} 

In order to establish the validity of source level verification, we
must relate proof obligations at source level to proof obligations at
bytecode level. We have shown that compilation (almost) preserves
proof obligations, in the sense that the set of proof obligations
generated for an annotated source code program is equal to the set of
proof obligations generated for the corresponding annotated compiled
program. Formal developments may be found
in~\cite{gta05:fast,BP06:sac}.


At a pratical level, the relationship between the source code proof
obligations generated by the standard feature of JACK and the bytecode
proof obligations generated by our implementation over the
corresponding bytecode produced by a non optimizing compiler over the
examples given in \cite{JPVC03JKM}. The proof obligations were the
same modulo names and short, byte and boolean values as well as
hypothesis names. The proof obligations on bytecode and source level
that we proved interactively in Coq produced proof scripts which were
also equal modulo those values and the hypothesis's names. This means
that an appropriate encoding of proof obligations on source and
bytecode level can be found, where the names of the source and
bytecode hypothesis are the same and where the source obligations can
be transformed in such a way that the produced proof script for the
transformed source proof obligation can be applied to the
corresponding one on bytecode level. This shall be useful in future
scenarios for TPD with dynamic loading of program and components.





        
\chapter{Evaluations}
\section{Industrial evaluation: Oberthur}
\section{Industrial evaluation: Axalto}
\section{Bytecode Verifier}
The tools have also been tested on a bytecode verifier java implementation. A termination proof has been provided.
A specific implementation has been coded with on one hand the main loop which remain unchanged whatever the specifications of the virtual machine Java chosen, and other instructions and memory states which depends on selected model.
The main loop is in a package which contains abstract classes: 
the instructions and the states are implemented in a more generic way.
The package containing the implementation is composed from the instructions for the standard Java types and of the states of memory typing. 
\subsection {Memory states}
The memory states are represented by the State class, which is an abstract class.  It does not contain any precise definition of the memory: 
one has no information on the stack or on the local variables table. 
The implementation is relatively simple: 
it is a class which contains a type stack and a table of the types of the local variables. 
Functions allowing to read simply these structures and to generate verification error in the cases of misuse are defined.   
\subsection{Instructions}
The instructions are also represented by an abstract class: 
the class Instruction.  
Since in the Kildall algorithm each instruction is associated to a memory state,  the Instruction class has a field of the State type. 
An instruction can also have one or more successors. 
This relation is represented by a field which is the list of the successors of the instruction.  
One of the other aspects is the fact that on associate to each instruction a boolean field to determine if it has been modified or not.

Several propreties of the bytecode verifier are for;alised in this class.
First of all one verifies that the successors of the instruction are well included in the others instructions of the program. 
If these successors pointed towards external instructions, an verification error would be returned. 

The others important properties concern the pure function {\tt buildNewState}.
This function builds the typing state of the execution of an instruction  on the current state.
This construction can fail if the instruction tries for example to pop an element when the stack is empty.
If it succeeds, a new non null state is build.

Around ten instructions have been implementeds:  {\tt load} and {\tt blind} for the access to local variables, {\tt push} and {\ ttpop}   to obtain or put element on the stack, {\tt op1} and {\ ttop2} which is two operators who consume both the two top element of the stack and which replaces them by a eesultat of a certain type, {\tt ifle} and {\tt jump} instructions of jump towards another instruction successor, {\tt nop}  the instruction which does not do anything   and finally {\tt stop} which is an instruction which does not have a successor.
These instructions have an associated type in the OperandType class,
who can be None, Type1 or Type2.  Those are the minimal instructions to have a Java-like program.  
\subsection {The main loop}
The main loop is implementes in the Verifier class.
It is not an abstract class because it uses the properties of the State and Instruction abstract classes  to verify an instruction set on particular states.
This class provides two functions, the function  {\ tt verify} in which the loop is written and the function {\ tt check} which verify an instruction.

%The m \ 'ethode check ensures that all \ 'states of the successors of an instruction donn \ 'ee,    are larger or \ 'equal that the \ 'states before the ex \ 'ecution of the m \ 'ethode. This   propri \ 'and \ 'E seems simple \ `has to express but it implies several Pr \ 'erequis.  First of all it should be guaranteed that the successors of the instructions point all worms of   valid instructions. Then that all the instructions are diff \ 'erentes of no one and that theirs  \ 'states are too diff \ 'erents of no one them.    The Pr \ calculation weaker 'econdition of Jack forces us \ `has to add these propri \ 'and \ 'be    Li \ 'ees \ `with the S \ 'emantic of the language Java.  .   %Pour to facilitate the evidence I have \ 'and \ 'E oblig \ 'E  %de to add a certain number of assertions.    

The {\ tt verify} method is the main loop of the bytecode verifier.
It contains two nested loops. 
The internal one is a {\tt for} loop which iterates on the instructions  and verify all the quoted instructions (as described in the Kildall algorithm). 
The termination of the internal loop is easy to prove.
The {\tt for} loop executes as many time as there are numbers in the table.
The external {\tt while} loop stops the algorithm when no more instruction typing state is modified.
This termination is not obvious to prove, especially with JML, since it only allow to prove loop termination by giving an integer variant.

Since the states have to be used to show the algorithm termination,  
one has to make correspond each state  with an integer. 
Thus at each loop iteration, the integer associated with the state either increase or preserve the same value; and it exists a maximum value.


\begin{figure}[ht]  
\begin{center}    
\begin{tabular}{p {0.4 \textwidth} c c c c}  
{\bf Classes:} & State & Instruction & Verifier \\  
{\bf Lines of code:} & 14 & 47 & 66 \\  
{\bf Lines of annotations:} & 20 & 54 & 81 \\  \raggedright 
{\bf Proof obligations:} & 26 & 129 & 627 \\  \raggedright 
{\bf Automatically proved proof obligations:} & 17 & 93 & 112 \\  
{\bf Average length of a non-automatic proof:} & 3 & 6 & 12 \\    
\end{tabular}  
\end{center}  
\caption{Some statistics on proof}  
\label{stats}  
\end{figure}    
\subsection{Proofs}
The first proofs are relatively easy. 
The State class is proven almost automatically; 
except for the constructor where it is necessary to break a disjunction ({\tt instance S $\vee$ S = null}) to prove the invariants.

The Instruction class has also been relatively easy to prove. 
A significant number of proof was done automatically (approximately 90 \%); 
then majority of proof could be trivially resolved, except some lemma concerning  a loop termination.    

Finally the Verifier class was harder to prove.
The lemmas were containing too many hypotheses to be automatically proved.
Around 500 proof obligations have to be resolved manually.
Some of them was obvious and were resolved with quite the same script, but the script cannot be automated.
Some of them was complex: the proof script became little large (an average of 30 steps).
The lemmas conerning the loop invariant of the verify method and its initialisation were the most difficult.
\section{Low-Footprint Java-to-Native Compilation}


% Keywords: embedded devices, Java, Java card, exceptions, ahead of time compilation






\subsection{Generation of the Verification Conditions}
\label{vcGen}
For generating the verification conditions, we use a bytecode verification condition generator (vcGen) based on a bytecode weakest precondition calculus~\cite{JBL05MP}. The weakest precondition function $wp$ returns for every instruction \texttt{ins}, normal postcondition $\psi$, and exceptional function $\excPost$ the weakest predicate \\ $\wpi( \texttt{ins} ,\psi ,\excPost)$ such that if it holds in the pre-state of the instruction \texttt{ins} and if the instruction terminates normally, then the normal postcondition $\psi$ holds in the poststate and if \texttt{ins} terminates on an exception \texttt{Exc}, then the predicate $\excPost(\texttt{Exc})$ holds. From the annotated bytecode the vcGen calculates a set of verification conditions for every method of the application. The verification conditions for a method are generated by tracing all the execution paths in it starting at every \texttt{return}, \texttt{athrow} and loop end instruction up to reaching the method entry point. During the process of generation of the verification conditions, for every instruction that may throw a \verb!Runtime! exception a new verification condition is generated.

In figure \ref{fig:wpRule}, we show the weakest precondition rule for the \texttt{getfield} instruction. As the virtual machine is stack-based, the rule mentions the stack \texttt{stack} and the stack counter \texttt{cntr}, thus the stack top element is referred as \stack{\counter}. If the top stack element \stack{\counter} is not null, \texttt{getfield} pops \stack{\counter} which is an object reference and pushes the value of the referenced field onto the operand stack in \stack{\counter}. If the stack top element is null, the Java Virtual Machine specification says that the \texttt{getfield} instruction throws a \texttt{NullPointerException}.

When the verification condition generator works over a method, it labels the formula related to the exceptional termination of every instruction with the index of the instruction in the bytecode array of the method. For example, if a \texttt{getField} instruction is met in the bytecode of a method, a conjunction is generated and the conjunct related to the exception is labeled as shown by figure \ref{fig:wpRule}. Finally, indexing the verification conditions allows to identify later in the proof phase which instructions can be optimized.



\begin{figure}
$$
\begin{array}{l}
\wpi(ind : \texttt{getfield} \ \texttt{Cl.f}, \ \psi, \ \excPost) = \\
\biggl( 
\begin{array}{l}
	%\begin{array}{l}
   		\stack{\counter} \not= \Mynull\Rightarrow   \\
	\Myspace \psi\begin{array}{l} \subst{\stack{\counter}}{\texttt{Cl.f} (\stack{ \counter}) } \\[0 mm]

		\end{array}\\
	%\end{array}
   \wedge \\
    ind : \stack{\counter} = \Mynull 	\Rightarrow\\
   \Myspace	 \excPost(\texttt{NullPointerException})
        \begin{array}{l}
          \subst{ \counter }{ 0} \\
          \subst{\stack{0}}{ \texttt{ref}_{NullPointer} }
	\end{array}
    \end{array} \biggr)
\end{array}
 $$
\caption{The weakest precondition rule for the \texttt{putfield} instruction}
\label{fig:wpRule}
\end{figure}



%\nocite{Adl-Tabatabai1998, Cierniak2000}


\section{Memory consumption}
%\subsection{Introduction}

%\input BML/cmdBML.tex

\newcommand{\code}{\textit{code}}
\newcommand{\indexComp}{\textit{index}}





%\section{Introduction} \label{bcsl}
This chapter presents the bytecode level specification language, called for short BML and a compiler from a
 subset of the high level Java specification language JML to BML which from now we shall call \JMLtoBML. 
The chapter is organized as follows.
 In section \ref{BCSLprelim}, we give an overview of the main features of JML. A detailed overview of BML is given in section \ref{BCSLgrammar}.  
  As we stated before, we support also a compiler from the high level specification language JML into BML. The 
 compilation process from JML to BML is discussed in section  \ref{BCSLcompile}.
 The full specification of the new user defined Java attributes in which the JML specification is compiled is given in the appendix.





%\subsection{Preliminaries}\label{sec:prelim}
%\subsubsection{Java class files} \label{classFileFormat}
The standard format for Java bytecode programs is the so-called class
file format which is specified in the Java Virtual Machine
Specification~\cite{VMSpec}. For the purpose of this paper, it is
sufficient to know that class files contain the definition of a single
class or interface, and are structured into a hierarchy of different
attributes that contain information such as the class name, the name
of its superclass or the interfaces it implements, a table of the
methods declared in the class. Moreover an attribute may contain other
attributes. For example the attribute that describes a single method
contains a \verb!Local_Variable_Table! attribute that describes the
method parameters and its local variables.
%; further in this section we will denote the table of local variables
%by $l$ and the $i^{th}$ variable by $l[i]$.

In addition to these attributes which provide all the information
required by a standard implementation of the JVM, class files can
accommodate user-defined attributes.  We take advantage of this
possibility and introduce additional attributes given in the Bytecode
Specification Language, described below.


\subsubsection{The Bytecode Specification Language}
The {\it Bytecode Specification Language} (BCSL) \cite{LM05:acc} is a
variant of the Java Modelling Language (JML) \cite{JMLRefMan} tailored
to Java bytecode. For our purposes, we only need to consider a
restricted fragment of BCSL, which is given in Fig.~\ref{fig:bml}; we
let $\expression$ and $\predicate$ denote respectively the set of BCSL
expressions and predicates. As for JML, BCSL specifications contain
different forms of statements, in the form of predicates tagged with
appropriate keywords. BCSL predicates are built from expressions using
standard predicate logic; furthermore BCSL expressions are bytecode
programs that correspond to effect-free Java expressions, or BCSL
specific expressions.  The latter include expressions of the form
\verb!\oldp(exp)! which refers to the value of the expression
\verb!exp! at the beginning of the method, or $\mbox{\tt
exp}^{\mbox{{\tt pc}}}$ which refers to the value of the expression
\verb!expr! at program point \verb!pc!. Note that the latter is not
standard in JML but can be emulated introducing a ghost variable
$\mbox{\tt exp}^{\mbox{{\tt pc}}}$ and performing the ghost assignment
\verb!set exp!$\mbox{}^{\mbox{{\tt pc}}}$\verb!= exp! at program point
\verb!pc!.


Statements can be used for the following purposes:
\begin{itemize}
\item Specifying method preconditions, which following the design by
contract principles, must be satisfied upon method invocation. They
are formulated using statements of the form $\requires \ \predicate$;


\item Specifying method postconditions, which must be guaranteed upon
returning normally from the method. Such postconditions are formulated
using statements of the form $\ensures\ \predicate$;

\item Specifying method exceptional postconditions, which must be
guaranteed upon returning exceptionally from the method. Such
postconditions are formulated using statements of the form \\
$\exsures{Exception} \predicate$, that record the reason for
exceptional termination;

\item Stating loop invariants, which are predicates that must hold
every time the program enters the loop: $\invariant\ \predicate$;

\item Guaranteeing termination of loops and recursive methods, using
statements of the form $\variant\ \expression$ which provide a measure (in
the case of BCSL, a positive number) that strictly decreases at each
iteration of the loop/recursive call;


\item Local assertions, using $\assert \ \predicate$, which asserts
that $\predicate$ holds at the program point immediately after the
assertion;

\item Declaring and updating ghost variables, using statements of the
form $\declare \ \ghost \ Type \ name$ and $ \ghostSet \ \expression =
\expression$;


\item Keeping track of variables that are modified by a method or in a
loop, using declarations of the form $\modifies \ var$. During the
generation of verification conditions, one checks that variables that
are not declared as modifiable by the clause above will not be
modified during the execution of the method/loop. This information is
also used to generate the verification conditions.
\end{itemize}

\begin{figure}
%\begin{frameit}
$$
\begin{array}{lll} 
\mbox{\annotation}-{\sf stmt} & = &
                                       \requires \ \predicate \\
                              & \mid & \ensures  \ \predicate  \\
                           & \mid  & \exsures{Exception} \ \predicate  \\
                               & \mid  &  \assert  \ \predicate  \\
                                & \mid & \invariant \  \predicate  \\
                                & \mid & \variant \  \expression  \\
                                & \mid &  \declare \ \ghost \ Type \ name \\
                                 & \mid & \modifies  \ var  \\
                                 & \mid & \ghostSet \ \expression = \expression

\end{array}
$$
\caption{{\sc Specification language}}\label{fig:bml}
%\end{frameit}
\end{figure}

Note that, as alluded above, annotations are not
inserted directly into bytecode; instead they are gathered into
appropriate user defined attributes of an extended class file. Such
extended class files can be obtained either through direct
manipulation of standard class files, or using an extended compiler
that outputs extended class files from JML annotated programs,
see~\cite{LM05:acc}.

\subsubsection{Verification of annotated bytecode}
In order to validate annotated Java bytecode programs, we resort to a
verification environment for Java bytecode, which is an adaptation by
L.~Burdy and the second author~\cite{LM05:acc} of
JACK~\cite{BRL-JACK}. The environment consists of two main
components:
\begin{itemize}
\item A verification condition generator, which takes as input an annotated
applet and generates a set of verification conditions which are sufficient
to guarantee that the applet meets its specification;

\item A proof engine that attempts to discharge the verification
conditions automatically, and then sends the remaining verification
conditions to proof assistants where they can be discharged
interactively by the user.

\end{itemize}


\paragraph{Generating the Verification Conditions}\label{subsec:verification}
The verification condition generator (VCGen) takes as input an
extended class file and returns as output a set of proof obligations,
whose validity guarantees that the program satisfies its
annotations. The VCGen proceeds in a modular fashion in the sense that
it addresses each method separately, and is based on computing weakest
preconditions. More precisely, for every method $\method$,
postcondition $\psi$ that must hold after normal termination of
$\method$, and exceptional postcondition $\psi'$ that must hold after
exceptional termination of $\method$ (for simplicity we consider only
one exception in our informal discussion), the VCGen computes a
predicate $\phi$ whose validity at the onset of method execution
guarantees that $\psi$ will hold upon normal termination, and $\psi'$
will hold upon exceptional termination. The VCGen will then return
several proof obligations that correspond, among other things, to the
fact that the precondition of $\method$ given by the specification
entails the predicate $\phi$ that has been computed, and to the fact
that variants and invariants are correct.


The procedure for computing weakest preconditions is described in
detail in~\cite{LM05:acc}. In a nutshell, one first defines for each
bytecode a predicate transformer that takes as input the
postconditions of the bytecode, i.e. the predicates to be satisfied
upon execution of the bytecode (different predicates can be provided
in case the bytecode is a branching instruction), and returns a
predicate whose validity prior to the execution of bytecode guarantees
the postconditions of the bytecode. The definition of such functions
is based on a single instruction, so the next step is to use these
functions to compute weakest preconditions for programs.  This is done
by building the control flow graph of the program, and then by
computing the weakest preconditions of the program, using the graph.

Note that the verification condition generator operates on BCSL
statements which are built from extended BCSL expressions. Indeed,
predicate transformers for instructions need to refer to the operand
stack and must therefore consider expressions of the form
\verb!st(i)! which represent the \verb!i!-element of the stack \verb!st!.

%$$wp( \store \ l(i) , \psi , \psi') = \psi[\verb!top! \leftarrow \verb!top-1!][l[i] \leftarrow \verb!st(top)!].$$


\paragraph{Discharging verification conditions}
Verification conditions are expressed in an intermediate language and
then translated to automatic theorem provers and proof assistants.  In
our examples, we have used Simplify~\cite{simplify} as automatic
prover and Coq~\cite{coq} as proof assistant. The Coq plug-in for Jack
was developed by J.~Charles, and adapted to Java bytecode by L.~Burdy.


\subsubsection{Correctness of the method}\label{subsec:sound}
The verification method is correct in the sense that one can prove
that for all methods $\method$ of the program the (exceptional)
postcondition of the method holds upon (exceptional) termination of
the method provided the method is called in a state satisfying the
method precondition and provided all verification conditions can be
shown to be valid.


The correctness of the verification method is established relative to
an operational semantics that describes the transitions to be taken by
the virtual machine depending upon the state in which the machine is
executed. There are many formalisations of the operational semantics
of the JVM, see
e.g.~\cite{FM03:jar,KN02:tcs,siv04:jlap,BSS:jbook}. 

%Such semantics manipulate states of the form
%$\config{h,\fram{m,\pc,l,s},\stf}$, where $h$ is the heap of objects,
%$\fram{m,\pc,l,s}$ is the current \emph{frame} and $\stf$ is the
%current call stack (a list of frames). A frame $\fram{m,\pc,l,s}$
%contains a method name $m$ and a program point $\pc$ within $m$, a set
%of local variables $l$, and a local operand stack~$s$.
%%The rule for the generic instruction \instr\ is formalized as a
%The operational semantics for each instruction is formalised as rules specifying transition between states, or between a state and some tag that
%indicates abnormal termination. For example, the semantics of
% the instruction $\store$ is given by the transition
%rule below, where  $\InstAt(m,\pc)$ is the function that extracts
%the $\pc$-th instruction from the body of method $\method$:

%$$\frac{
%%\begin{array}[c]{c}
%\InstAt(m,\pc)=\store \ i
%%\end{array}}%
%}
%{\begin{array}[t]{c}
%\config{h,\fram{m,\pc,l,v::s},\stf} \to_{\store\ i} \\
%\ \ \ \ \ \ \ \ \ \ \ \ \ \ \ \ \ \ \ \ 
%\config{h,\fram{m,\pc+1,l[i \mapsto v],s},\stf}
%\end{array}}$$
%In order to establish the correctness of our method, one first needs
%to establish the correctness of the predicate transformer for each
%bytecode. For example for the instruction $\store$ we show that:
%$$\begin{array}[t]{c} 
%wp(\store \ i , \psi )( \config{h,\fram{m,\pc,l,v::s},\stf} ) \ \ \Rightarrow  
%\\
%\psi( \config{h,\fram{m,\pc+1,l[i \mapsto v],s},\stf})
%\end{array}$$
%In the above $\psi(\config{h,\fram{m,\pc,l,v::s},\stf} )$ is to be
%understood as the instance of the formula $\psi$ in which all local
%variables $l$ and field references are substituted with their
%corresponding values in state $\config{h,\fram{m,\pc,l,v::s},\stf} $.


%The proof proceeds by a case analysis on the instruction to be
%executed, and makes an intensive use of auxiliary substitution
%lemmas that relate e.g. the stack of the pre-state with the stack
%of the post-state of executing an instruction. Then one proves the
%correctness of the method by induction on the length of the
%execution sequence.

We have proved the correctness of our method for a fragment of the JVM
that includes the following constructs: Stack manipulation: \push,
\pop, \dup, \dup 2, \swap, \numop, etc; Arithmetic instructions:
type\_\add, type\_\sub, etc; Local variables manipulation:
type\_\load, type\_\store, etc; Jump instructions: \If, \goto; Object
creation and object manipulation: \new, \putfd, \getfd, \newarray,
etc; Array instructions: \arrst, \arrld, etc; Method calls and return:
\invvir, \return; Subroutines: \jsr\ and \ret.

Note however that our method imposes some mild restrictions on the
structure of programs: for example, we require that $\jsr$ and
$\throw$ instructions are not entry for loops in the control flow
graph in order to prevent pathological recursion.  Lifting such
restrictions is left for future work.


\subsection{Modelling memory consumption}\label{sec:verif}
The objective of this section is to demonstrate how the user can
annotate and verify programs in order to obtain an upper bound on
memory consumption. We begin by describing the principles of our
approach, then turn to establish its soundness, and finally show
how it can be applied to non-trivial examples involving recursive
methods and exceptions.


\subsection{Principles}
Let us begin with a very simple memory consumption policy which aims
at enforcing that  programs do not consume more than
some fixed amount of memory \Max . To enforce this policy, we first
introduce a ghost variable \Mem\ that represents at any given point of
the program the memory used so far. Then, we annotate the program both
with the policy and with additional statements that will be used to
check that the application respects the policy.



\paragraph{The precondition} of the method $\method$ should ensure that
there must be enough free memory for the method execution. Suppose
that we know an upper bound of the allocations done by method $\method$
in any execution. We will denote this upper bound by
\allocMethod{\method}. Thus there must be at least
\allocMethod{\method}\ free memory units from the allowed \Max\ when
method $\method$ starts execution. Thus the precondition for the method
$\method$ is:
$$
\requires \ \Mem + \allocMethod{\method}  \leq \Max.
$$
%\todo{ to leave this paragraph or not. It is about the initialization of the variable \Mem} 
The precondition of the
program entry point (i.e., the method from which an application
may start its execution) should state that the program has not
allocated any memory, i.e. require that variable \Mem \ is  0:
$$
\requires \ \Mem == 0.
$$
\paragraph{The normal postcondition} of the method $\method$ must
guarantee that the memory allocated during a normal execution of
$\method$ is not more than some fixed number \allocMethod{\method}\
of memory units. Thus for the method $\method$ the postcondition is:
$$
\ensures \  \Mem \leq \old(\Mem) + \allocMethod{\method}.
$$

\paragraph{The exceptional postcondition} of the method $\method$ must
say that the memory allocated during an execution of $\method$ that
terminates by throwing an exception \texttt{Exception} is not more
than \allocMethod{\method}\ units. Thus for the method $\method$ the
exceptional postcondition is:

$$
\exsures{Exception} \  \Mem \leq \old(\Mem) + \allocMethod{\method}.
$$


\paragraph{Loops} must also be annotated with appropriate invariants. 
%Assuming that we know that loop $\progLoop{l}$ iterates no more than $\maxIter{l}$ as well as an upper bound  $\allocLoop{l}$ of the allocations done per iteration in $l$. 
Let us assume that loop $\progLoop{l}$ iterates no more than $\maxIter{l}$ and let $\allocLoop{l}$ be an upper bound of the memory allocated per iteration in $l$.
Below we give a general form of loop specification w.r.t. the property for constraint memory consumption. The loop invariant of a loop $\progLoop{l}$ states that at every iteration the loop body is not going to allocate more than $\allocLoop{l}$ memory units and that the iterations are no more than $\maxIter{l}$. We also declare an expression which guarantees loop termination, i.e. a variant (here an integer expression whose values decrease at every iteration  and is always bigger or equal to 0).
$$\begin{array}{ll}
\modifies &  \ i, \Mem \\
\invariant: & \ \Mem \le \atState{\Mem}{Before_{l}} + i * \allocLoop{l} \\
                & \wedge \\
                & i \le \maxIter{l}\\
\variant: & \maxIter{l} - i \\
\end{array}$$
 A special variable appears in the invariant, $\atState{\Mem}{Before_{l}}$. It denotes the value of the consumed memory just before entering for the first time the loop \progLoop{l}. At every iteration the consumed memory must not go beyond the upper bound given for the body of loop.

\paragraph{For every instruction that allocates memory} the ghost
variable \Mem\ must also be updated accordingly. For the purpose of
this paper, we only consider dynamic object creation with the bytecode
\new; arrays are left for future work and briefly discussed in the
conclusion. 

The function $allocInstance: Class \rightarrow int$ gives an estimation of the memory used by an instance of a class.
%In order to perform the update for \new\ bytecodes, we must assume given a function $allocInstance: Class \rightarrow int$ which maps classes to an estimation of the memory that any instance of the class may occupy. 
At every program point where a bytecode \srcCode{\new \ A} is found, the ghost variable \Mem\ must be incremented by $\allocInstance{A}$. This
is achieved by inserting a ghost assignment immediately after any \new\ instruction, as shown below:
$$
\begin{array}{l}
\srcCode{\new \ A} \\
 // \set \ \Mem = \Mem + $\allocInstance{A}$.
\end{array}
$$

\subsection{Correctness}
%An important question is if the annotations that we prescribe here guarantees that the memory used in a program is not more than a fixed upper bound \Max. 
We want to guarantee that the memory allocated by a given program is bounded by a constant \Max.
We can prove that our annotation is correct w.r.t. to the policy for constraint memory use, by instrumenting the operational semantics of the bytecode language given in
 Section \ref{subsec:sound}. The instrumented operational semantics
manipulates states as before, but it is extended with the special variable \Mem. Thus, states in the new semantics have the form:

$$\configMem{h,\fram{m,\pc,l,s},\stf,\Mem}$$

%The variable \Mem \ changes its value only for instructions that allocate space in the heap, i.e. \new\ instructions:

%$$\small{\frac{
%\begin{array}[c]{c}
%\ \InstAt(m,\pc)=\new \ A ,
%\end{array}}
%{\begin{array}[t]{c} \config{h,\fram{m,\pc,l,v::s},\stf, \Mem} \to_{\new\ A} \\ \config{h + \allocInstance{A},\fram{m,\pc+1,l,s},\stf ,\Mem + \allocInstance{A}}
%\end{array}}}$$



The other instructions do not affect \Mem, so the corresponding rules of the operational semantics are as before. As we saw in the previous section to every
instruction of the form $\new\ A$ we attach the annotation $\set\ \Mem = \Mem + \allocInstance(A)$. The proof obligation generator converts this annotation into new value for the variable \Mem:

$$
\begin{array}{l}
wp(\set \ \Mem = \Mem + \allocInstance{A}, \psi) = \\
\ \ \ \ \ \ \ \ \ \ \ \ \psi[ \Mem \leftarrow \Mem + \allocInstance{A} ]
\end{array}
$$

We can prove that whenever the allocated space in the heap increments, 
the ghost variable \Mem\ also increments, which is a sufficient condition to guarantee the correctness of the annotations. 
So far we do not deal with garbage collection (see discussion in Section \ref{sec:conc}).

\subsection{Examples}
We illustrate hereafter our approach by several examples. 
%\alarm{talk about number of proof obligations, which are discharged automatically in Coq, etc}

\subsubsection{Inheritance and overridden methods} Overriding methods are treated as follows: whenever a call is performed to a method \method,
we require that there is enough free memory space for the maximal
consumption by all the  methods that override or are overridden by
\method. In Fig. \ref{classExt} we show a class \verb!A! and its
extending class \verb!B!, where \verb!B! overrides the method \method\ from class \verb!A!. Method \method\ is invoked by $n$. Given that the dynamic type of the parameter passed to $n$ is not known, we cannot know which of the two
methods will be invoked. This is the reason for requiring enough memory space for the execution of any of these methods.
%After the method execution we consider the extreme case where there is executed the method \method\ that consumes the most.

\begin{figure}[!htp]
Specification of method $m$ in class A:
$$
\begin{array}{ll}
\requires & \Mem + k  \leq \Max \\
\modifies & \Mem \\
\ensures & \Mem  \leq \old(\Mem) + k
\end{array}
$$

Specification for method $m$ in class B:
$$
\begin{array}{ll}
\requires & \Mem + l  \leq \Max \\
\modifies & \Mem \\
\ensures & \Mem  \leq \old(\Mem) + l
\end{array}
$$

\begin{verbatim}
method n(A a)
...
//{ prove Mem <= Mem +max(l,k) }
invokevirtual m <A>
//{ assume Mem <= \old(Mem) + max(l,k)}
...
\end{verbatim}
\caption{\sc Example of overridden methods}
\label{classExt}
\end{figure}


\subsubsection{Recursive Methods} In Fig. \ref{recMeth} the bytecode of the recursive method $m$ and its specification is shown. For sake of space we show only a simplified version of the bytecode; we assume that the constructors for the class \srcCode{A} and \srcCode{C} do not allocate memory. Besides the precondition and the postcondition, the specification also includes information about the termination of the method: \variant\ $\local{1}$, meaning that the local variable $\local{1}$ decreases on every recursive call down to and no more than $0$, guaranteeing that the execution of the method will terminate.
 
%Now we explain why such a precondition is required for method \textbf{m} in order to specify the property for constraint memory consumption. 

We explain first the precondition. If the condition of line \srcCode{1} is not true, the execution continues at line \srcCode{2}.

\begin{figure}[!hbp]
\begin{alltt}
public class D \{
  public void m( int i) \{
    if (i > 0) \{
      new A();
      m(i - 1);
      new A();
    \} else \{
      new C();
      new A();
   \}
  \}
\}
\end{alltt}

$$
\begin{array}{ll}
 \requires & ( \Mem + \local{1}*2*\allocInstance{A} + \\
           &  \allocInstance{A} + \allocInstance{C}) \le \Max \\
 \variant  & \local{1} \\
 \ensures  & \local{1} \ge 0 \\
           & \wedge \\
           & \Mem <= \old(\Mem) +  \old(\local{1})*2*\allocInstance{A} + \allocInstance{A}\\
           &  +  \allocInstance{C})
\end{array}$$

\begin{alltt}
\srcCode{\textbf{public void m()}}
//\small{\textit{local variable loaded on} }
//\small{\textit{the operand stack of method \textbf{m}}}
\srcCode{0 \load\_1}
//\small{ \textit{ if \local{1} <= 0 jump}}
\srcCode{1 ifle 12}
\srcCode{2 new <A>} //\small{ \textit{ here \local{1} > 0  } }
//set \Mem = \Mem +  \allocInstance{A}
\srcCode{3 invokespecial <A.<init>>}
\srcCode{4 aload\_0}
\srcCode{5 iload\_1}
\srcCode{6 iconst\_1}
//\small{\textit{\local{1} decremented with 1}}
\srcCode{7 isub}
//\small{ \textit{ recursive call with the new value of \local{1}}}
\srcCode{8 invokevirtual <D.m>}//
\srcCode{9 new <A>}
//set \Mem = \Mem +  \allocInstance{A}
\srcCode{10 invokespecial <A.<init>>}
\srcCode{11 goto 16}
//\small{\textit{target of the jump at \srcCode{1}}}
\srcCode{12 new <A>}
//set \Mem = \Mem +  \allocInstance{A}
\srcCode{13 invokespecial <A.<init>>}
\srcCode{14 new  <C>}
//set \Mem = \Mem +  \allocInstance{C}
\srcCode{15 invokespecial <C.<init>>}
\srcCode{16 return}
\end{alltt}

\caption{\sc Example of a recursive method}
 \label{recMeth}
\end{figure}

In the sequential execution up to line \srcCode{7}, the program allocates at most $\allocInstance{A}$ memory units and decrements by $1$ the value of $\local{1}$. The instruction at line \srcCode{8} is a recursive call to $m$, which either will take the same branch if $\local{1} > 0 $ or will jump to line \srcCode{12} otherwise, where it allocates at most $\allocInstance{A} +  \allocInstance{C}$ memory units. On returning from the recursive call one more allocation will be performed at line \srcCode{9}.
 Thus $m$ will execute, $\local{1}$ times, the instructions from lines \srcCode{4} to \srcCode{35},  and it finally will execute all the instructions from lines  \srcCode{12} to \srcCode{16}.
The postcondition states that the method will perform no more
than $\old(\local{1})$ recursive calls (i.e., the value of the register variable in the pre-state of the method) and that on every recursive call it allocates no more than two instances of class \texttt{A} and that it will finally allocate one instance of class \texttt{A} and another of class \texttt{C}.


\subsubsection{More precise specification} We can be more precise in specifying the precondition of a method by considering what are the field values of an instance, for example. Suppose that we have the method \method\ as shown in Fig. \ref{excMeth}. We assume that in the constructor of the class \texttt{A} no allocations are done. The first line of the method \method\ initializes one of the fields of field \texttt{b}. Since nothing guarantees that field \texttt{b} is not \Mynull, the execution may terminate with
\texttt{NullPointerException}. Depending on the values of the parameters passed to \method, the memory allocated will be different. The precondition establishes what is the expected space of free resources depending on if the field
\texttt{b} is \Mynull  or not. In particular we do not require anything for
the free memory space in the case when \texttt{b} is \Mynull. In the
normal postcondition we state that the method has allocated an
object of class \texttt{A}. The exceptional postcondition states
that no allocation is performed if \texttt{NullpointerException} causes the execution termination.

\begin{figure}[!hbp]
$$
\begin{array}{ll}
 \requires &  \local{1} != \Mynull \Rightarrow  \\
           & \phantom{\local{1}} \Mem +  \allocInstance{A} \le \Max \\
       %& \wedge \\
       %&  \local{1} == \Mynull \Rightarrow  \\
           %& \phantom{\local{1}} \Mem +  \allocInstance{B} + \allocInstance{A}   \le \Max \\
  \modifies & \Mem \\
  \ensures  & \Mem \le \old(\Mem) +  \allocInstance{A} \\
  \exsures{NullPointerException}  & \Mem == \old(\Mem)   \\
\end{array}$$

\begin{tabular}{lr}
\begin{minipage}[t]{170pt}
\begin{alltt}
\srcCode{0 aload\_0}
\srcCode{1 getfield<C.b>}
\srcCode{2 iload\_2}
\srcCode{3 putfield <B.i>}
\srcCode{4 new <A>}
//set \Mem = \Mem +
      \allocInstance{A}
\srcCode{5 dup}
\srcCode{6 invokespecial <A.<init>>}
\srcCode{7 astore\_1}
\srcCode{8 return}
\end{alltt}
\end{minipage}
 &
\begin{minipage}[t]{170pt}
\begin{alltt}
public class C \{
  B b;
  public void m(A a, int i) \{
    b.i = i ;
    a = new A();
  \}
\}
\end{alltt}
\end{minipage}
\end{tabular}
\caption{\sc Example of a method with possible exceptional termination}
\label{excMeth}
\end{figure}

\subsection{Inferring memory allocation}\label{sec:infer}
In the previous section, we have described how the memory consumption
of a program can be modeled in BML and verified using an appropriate
verification environment. While our examples illustrate the benefits
of our approach, especially regarding the precision of the analysis,
the applicability of our method is hampered by the cost of providing
the annotations manually. In order to reduce the burden of manually
annotating the program, one can rely on annotation assistants that
infer automatically some of the program annotations (indeed such
assistants already exist for loop invariants, loop variants, or
class invariants). In this section, we describe an implementation of
an annotation assistant dedicated to the analysis of memory consumption,
and illustrate its working on an example.

\subsection{Annotation assistant}
The inference algorithm proposed here works on programs
 without recursive methods, exception throwing and handling.  


The user must provide information about the memory required to create objects of the given classes,
 i.e. he must give the definition  of the function $\allocInstance{\cdot}$.
 The variant and the maximum number of iterations $\maxIter{l}$ for every loop $l$  are either given by the
 user or are synthesized  through appropriate mechanisms. 

%Given a control flow graph of the program --which is computed by the verification environment, a function that gives for each class the memory required to create an object of that class, which is provided by the user, and a variant for each loop and recursive method, which can either be provided by the user, or synthesized through appropriate mechanisms. 

Based on this information, the annotation assistant 
inserts the ghost assignments on appropriate places, and then computes
recursively the memory allocated on each loop and method. 
A pseudo-code of the algorithm for inferring an upper bound for method allocations is given in Fig. \ref{methodAlloc}.
Essentially, it finds the maximal memory that can be allocated in a method by exploring all its possible execution paths.  
The algorithm for exploring an execution path starts from the last instruction in the path (i.e. a \return{} instruction). 
The algorithm use standars techniques to detect the loop entry instructions. For each loop entry instruction it also finds  
the set  of the corresponding ``loop end'' instructions i.e. the  instructions  that target   to  and which are dominated by
the loop entry instruction; you are not entering in detail in the loop detection
algorithm as it is standard and the reader may  see Section 10 of 
 \cite{ASU86cpt} for a description of the algorithms.

\begin{figure}[!htp]
function $\allocMethod{.}$\\
\textbf{Input:} Bytecode of a method \methodd{} . \\
\textbf{Output:} Upper bound of the memory allocated by \methodd{} . \\
\textbf{Body:}
\begin{enumerate}
   \item Detect all the loops in \methodd{}; for every loop $l$ determine $\loopSet{l}$, $\loopEntry{l}$ and $\loopEndsSet{l}$;
   \item Apply the function \allocatedOnly \  to each instruction $\instrAt{k}$, such that $\instrAt{k} = \return$;
   \item Take the maximum of the results given in the previous step: \\
 $max_{\instrAt{k} = \return } \allocated{\instrAt{k}}$.
\end{enumerate}
\caption{\sc Inference algorithm}
\label{methodAlloc}
\end{figure}

The auxiliary function $allocPath$, which infers the maximal allocations done by the set of execution paths ending with the same \return{} instruction,
 is given in Fig. \ref{fig:allocMethod}.
Inferring the memory allocated inside loops is done by the function $\allocLoopWithEnd{\cdot}{\cdot}$, which is invoked by \allocatedOnly{} whenever the 
current instruction belong to a loop. The specification of the function is shown in Fig. \ref{fig:loopPath}.

\begin{figure}[!hbp]
%\centerline{
$\allocated{\instrAt{s}}$ = 
$$ \left\{ \begin{array}{ll}
\allocIns{\instrAt{s} }   &  if \ \instrAt{s} \ has \ no \ predecessors \\
& \\
  \begin{array}{l}
            \allocLoop{\loopEntry{l}} \\
             + \\
            max_{\instrAt{k} \in \preds(\instrAt{s} )-\loopEndsSet{\progLoop{l}}}( \allocated{\instrAt{k}} ) \\
                   \end{array}      & if \  \instrAt{s} \in \loopSet{\progLoop{l}} \\
& \\
\begin{array}{l}
\allocIns{\instrAt{s}} \\
 + \\
max_{\instrAt{k} \in \preds(\instrAt{s} )}
 ( \allocated{\instrAt{k}} )
                       \end{array} & else 
\end{array}
\right.
$$
\caption{\sc Definition of the function $\allocated{\instrAt{s}}$} 
\label{fig:allocMethod}
\end{figure}


\begin{figure}[!hbp]
$\allocLoop{\loopEntry{l}} = $
$$ \begin{array}{l}
    \maxIter{l} * max_{ \instrAt{e} \in \loopEndsSet{l}  } (\allocLoopWithEnd{\loopEntry{l}}{\instrAt{e}} )
   \end{array}$$



$\allocLoopWithEnd{\loopEntry{l}}{\instrAt{s}} = $
$$ 
\left\{\begin{array}{ll}

 \allocIns{\loopEntry{l}}  & if \  \instrAt{s} = \loopEntry{l} \\
  & \\
 \begin{array}{l}
           \allocLoop{\loopEntry{l'}} \\
          + \\
      max_{\instrAt{k} \in \preds(\loopEntry{l'} ) - \loopEndsSet{\progLoop{l'}}}
       ( \allocLoopWithEnd{\loopEntry{l}}{\instrAt{k}} )
    \end{array} &  \begin{array}{l}
                                        if \  \instrAt{s} \in  \loopSet{\progLoop{l'}} \\
                                          \progLoop{l'} \ is \  nested \ in \ \progLoop{l}
                                    \end{array} \\
  & \\
  \begin{array}{l}
     \allocIns{\instrAt{s}} \\
     + \\
     max_{\instrAt{k} \in \preds(\instrAt{s} )}
     ( \allocLoopWithEnd{\loopEntry{l}}{\instrAt{k}} )
                       \end{array} & else \\

\end{array} \right.
$$
 \caption{\sc Definition of the function $\allocLoop{\cdot}$ and  $\allocLoopWithEnd{\cdot}{\cdot}$ }
\label{fig:loopPath}
\end{figure}


%In essence, it first analyses the method's bytecode by identifying the entry loop instruction, the instructions
% that are inside the loop and the set of instructions by which the loop terminates. 
%Then  it finds the maximal number of allocations that can be done per execution path: 
%it starts from all \return \ instructions and ``inspects'' the execution paths upto the entry program
%instruction (in backwards direction  );  if an instruction in the path belongs to a loop then the allocations done per iteration in the loop are 
%calculated and the analysis proceeds from the instructions that target the entry of the loop and that do not belong to the loop; the assistant computes the memory required for %each loop using the memory required for each iteration of the loop, and the variant of the loop, which provides information about the number of
%iterations; upper bound of the allocations done per iteration in a loop are calculated also in a backwards direction 
%(starting from instructions that belong to the loop and whose next instruction is the unique loop entry) by finding the 
%maximal number of allocations per iteration path and if an instruction in an iteration path appears to be instruction that belongs to an inner loop 
%then first an upper bound for the allocations done in the inner loop are inferred and the analysis continues from instructions that target the entry of the nested loop and do not belong to the nested loop.    

The annotation assistant currently synthesize only simple
memory policies (i.e., whenever the memory consumption policy does not depend on
the values of inputs).
%and could be significantly improved in this respect. 
Furthermore, it does not deal with arrays, subroutines, nor exceptions. Our approach may be extended to treat such cases (see the discussion in Section \ref{sec:conc} about how to include arrays in our analysis). For sake of simplicity,
we have also restricted the loop analysis only to those with a unique
entry point, which is the case for code produced by non-optimizing
compilers.
Dealing with loops with a multiple entry points is left for a future work. %A pre-analysis could give us all the entry points of more
%general loops, for instance by the algorithms given in \cite{CJPS05cmu}; our approach may be thus applied straightforwardly.


%\subsection{Conclusion}\label{sec:conc}
%
\section{Achievements}
% done. summary
We have  presented an infrastructure for verification of Java bytecode programs   which allows to reason about potentially
sophisticated  functional and security properties and
which benefits from verification over Java source programs. We have also 
introduced the bytecode specification language BML tailored to Java bytecode, a compiler
from the Java source specification language JML to BML and a verification 
condition generator for Java bytecode programs. 
We have shown that the verification procedure is correct w.r.t. a big step  operational semantics of Java bytecode programs. 
Moreover, we have
proven that the verification procedure for Java like programs
and Java like bytecode are syntactically equivalent (modulo names and types). 
%This scheme is actually part of the PCC architecture of the
%European project Mobius\footnote{the site name} which aims to resolve the problems
%of mobile and ubicuous computing via PCC. 
We have developed a prototype of a verification condition generator based on the weakest precondition calculus presented in this thesis, as well 
as a compiler from the corresponding subset of JML to BML.
These two components have been integrated in the JACK \cite{BRL-JACK} verification framework 
developed and supported by our research team Everest at INRIA Sophia Antipolis which has been initially designed for
 the verification of Java source programs annotated with JML specification.

We would like to give a brief description of the implementation of the verification condition generator.
 The extension of the tool to bytecode programs which we added also interfaces these theorem provers. The bytecode 
verification condition generator works as follows. For the verification of a class file containing BML specification, it will generate verification conditions for every
 method of this class including the constructors. For generating the verification conditions concerning a method implementation, first the control flow
 graph corresponding to the bytecode instruction is built. The latter is transformed into an acyclic control flow graph where the backedges are 
removed.
 Then the verification procedure proceeds by generating over every execution path in the control flow graph its corresponding verification conditions. 
For every path which terminates by throwing an uncaught exception, the postcondition is the specified exceptional postcondition for this case. For the paths which terminate normally, 
the normal postcondition is taken. For every path which terminates with an instruction which is dominated by a loop entry and whose direct successor is the same loop entry, the postcondition 
is the corresponding loop invariant. The bytecode verification in Jack uses the intermediate language for the verification conditions and thus, bytecode verification conditions 
 can be translated to several different theorem provers - Simplify \cite{Simpl05DNS} which is an automatic decision procedure, 
the Atelier B and the Coq interactive theorem prover assistants. 

The bytecode verification condition generator benefits also from the original user friendly interface of the JACK tool.  In particular, 
the user can see the verification conditions in his favorite language - Java, Simplify, Coq or B. The lemmas are classified 
to what part of the annotation they refer to, as for instance, a lemma which refers to the establishment of the postcondition, or the preservation of the loop invariant.
The hypothesis in the lemma also hold the index of the instruction from which they originate. 
We have used the prototype of the bytecode verification condition generator for the case studies presented in Chapter \ref{applications:optimComp}.

% JACK (short for Java Applet Correctness Kit) is designed as a plugin for the Java interface development
% environment eclipse. 
%% It was originally tailored to the verification of Java source programs 
%w.r.t. their JML specifications. The tool has an intermediate proof obligation language which allows to extend it easily to interface more 
% theorem provers. Thus, the tool interfaces several theorem provers - Simplify \cite{Simpl05DNS} which is an automatic decision procedure, 
%%the Atelier B and the Coq interactive
%theorem prover assistant. 

\section{Future work}
In the following, we identify the directions for extending the work presented in this thesis

\subsection{Language coverage of the verification condition generator}
The bytecode verification condition generator works only for the sequential fragment of Java. But realistic applications 
rely often on multi - threading which is difficult to verify against a functional specifications or security policies.
One of the important aspects of the correctness of multi - threaded programs is the absence of deadlocks, 
and race conditions. Such properties can be ensured  by type systems \cite{FA99TSL,flanagan00typebased} or static verification based on program logic \cite{FLL02ESC}.  
The absence of deadlock and race conditions is a first step in the verification of the functional correctness of multi threaded programs. In order to build a full 
verification scheme for checking functional correctness more has to be done.
The earliest work for  verification of  parallel programs is  the Owicki and Gries approach   
\cite{nipkow99owickigries}  and the rely - guarantee approach. However, 
the first approach is not modular and requires a large amount of verification conditions while for the second, the annotation procedure can not be automatised.

% Such techniques for reasoning over the correctness of parallel programs  exist.
% One of the first logic - based verification techniques for parallel programs is due to Owicki and Gries 
%\cite{nipkow99owickigries}  in which every point of parallel interference is annotated and then the verification consists in establishing that
% all the possible inter leavings of all the threads respect the annotation. This technique is on one hand not modular as the verification process 
%needs the implementation of every program component and on other hand the number of verification conditions may be very big.
% Another approach is the rely guarantee technique which uses a Hoare style verification conditions \cite{nieto03relyguarantee}.
%There, the program points of interference are annotated not only with the predicate that must hold
%at the point but also with rely and guarantee  conditions which express what conditions the program guarantees to the other threads and what 
%the program requires from the other threads. This technique although tempting because of its modularity and the smaller number of verification conditions is difficult to apply
%as for guessing the rely and guarantee conditions requires an in - depth understanding of the program to be verified.  
Extending our verification scheme for bytecode will certainly be based on a more recent work  where one of the basic concerns is to establish method atomicity  \cite{TES03CF}. 
The notion of a statement atomicity states  that however a statement is interleaved with other parallel programs, the result of its execution will not change.
The atomicity can be  detected via static checking \cite{TES03CF} using type systems. Thus, the program verification process is separated in two parts
- first checking for program atomicity  \cite{TES03CF} are done  
and then verifying the functional correctness using  methodologies for sequential programs as Hoare style reasoning. 
In this last approach in the case of Java, the basic concern is to establish the atomicity of method bodies, i.e. method 
execution does not depend on the possible interleaving with threads.
Recently, E.Rodriguez and al. in \cite{RodriguezDFHLR05} proposed an extension for JML for multi threaded
 programs. Their proposal introduces  new specification keywords which allow to express that a variable is locked or
 that a method is atomic.% Giving the semantics of these keywords is still an ongoing work but we consider that the meaning of these specification constructs does not differ on source and bytecode. 
    
 



\subsection{Property coverage for the specification language}
Another direction which may be pursued as a future work of the thesis  is the extension of the expressiveness of the specification language BML. 
So far, BML supports method contracts - method pre and post  conditions, frame conditions, intermediate annotations as for instance
loop invariants, class specifications as well as special specification operators.
These are very useful aspects which allow for dealing with complex properties and 
gives a semantics on bytecode level  to a relatively small subset of the 
high specification language JML which corresponds to JML Level 0 \footnote{ http://www.cs.iastate.edu/~leavens/JML/jmlrefman/jmlrefman\_2.html\#SEC19}. 
 But it is certainly of interest to support more features of JML in BML
as this will turn the latter language richer. However, the meaning  of JML constructs 
(at least from our experience up to  now) is the same as the meaning of their corresponding part in BML.  

 An important example is the  JML construct for pure methods which has been  identified as  a challenge in the position paper \cite{LeavensLeinoMueller06}. 
 These methods does not modify the program state and thus, pure methods can be used in specifications 
 (only side effect free  expressions may occur in expressions).
 This gives more expressive  specifications as with them, for instance, specification can talk about the result of method invocation or use pure methods
 as a predicate relating their  initial and final state. 
 Formalizing and establishing the meaning of pure methods is difficult and a literature exists for this problem \cite{DarvasMueller06}.
 As we said above, the treatment of pure methods is the same on source and bytecode.

Also, support for specification constructions for alias control is certainly useful  especially because it allows for a modular verification 
of class invariants and frame conditions.
The alias control is guaranteed through ownership type systems which check that only an owner of a reference can modify its contents.
 This can considerably improve the current implementation for the verification of object invariants  \cite{DietlMueller05}.
In particular, our way of proving object invariants is non modular - at every method call the invariants of all visible \todo{say what does it mean visibility}
objects must be valid and they are assumed to hold when the call is terminated; similarly, when a method body is verified in its precondition the invariants of all visible
objects are assumed to hold and at the end of the method body all these invariants must be established. 
In practice, it is very difficult to verify that all the invariants for the all visible objects in a method  hold.
In order to keep the number of the verification conditions reasonable, we check the invariants only for the current object this and the 
objects received as parameters which is not sound.

 
\subsection{Preservation of verification conditions}

So far, we have shown that non-optimizing Java compilation
 preserves the  form of the verification conditions on source and
 bytecode.  We identify two basic directions for future work:
\begin{description}
 \item[Source and non optimized bytecode verification conditions equivalent modulo] % implement the compiler from Java source pogs to bytecode pogs
We have experimented with the verification conditions on source and
 bytecode in JACK and saw that in practice they are almost equivalent
 syntactically. From one part, there are the difference in the types 
 supported on bytecode and source level. For instance, the JVM does not
 provide support for boolean type values which are basically encoded as
 integer values. The same is true for byte and short values.  Another
 difference is the identifiers for variables and fields. For instance, in Java
 names for fields, method local variables and parameters are their identifiers which are given by the
 program developer. On bytecode method local variables and parameters are encoded as elements of the
 method register table and field names are encoded as numbers of the constant
 pool table of the class. A  simple but useful extension to the prototype for
 bytecode verification is a compiler from source proof obligations to bytecode proof obligations
 which overcomes those differences. This can be considered also as a step
 towards the  building a PCC architecture where the certificate generation benefits from
 the source level verification and thus allows for treating sophisticated
 security policies.

\item[Relation between verification conditions on Java source and optimized Java bytecode]
 The equivalence  between verification conditions on source and the corresponding non optimized bytecode is important as it
 allows that bytecode programs  benefit from source verification. In particular, it makes feasible Proof Carrying Code
 for sophisticated client requirements.
 However, a step further in this direction is to investigate the 
 relation between source programs and their bytecode counterpart produced by an optimizing compiler.
 This is interesting for the following reasons.
 It is a fact that interpretation of bytecode on the JVM is slower than execution by its corresponding assembly code. 
 In order to speed up the execution time for a Java bytecode program, one might use 
 a just-in-time compilation which  translates on the fly the bytecode into the machine specific language. However, JIT compilation can potentially slow
 the execution exactly because it does compilation on the fly.  Another possibility is to perform 
 optimizations on the bytecode. Currently, most of the  Java compilers do not support much optimizations.
 However, there do already exist Java optimizing compilers, for instance the Soot optimization framework\footnote{http://www.sable.mcgill.ca/soot/} 
 and most probably the number of the Java optimizing compilers will increase with the evolution of the Java language.
 A first step in the latter direction is the work of C. Kunz et al.\cite{BGKRsas06} who give an algorithm for translating 
 certificates and annotations over a non optimized program into a certificate  and annotation for its optimized version.
 Their work addresses  optimizations like constant propagation, loop induction, dead register elimination etc. 
\end{description}
\subsection{Towards a PCC architecture}

The bytecode verification condition generator and the BML compiler is the first step towards a PCC framework. 
The missing  part is  the certificate format which comes along with the bytecode and which  is the evidence for 
that the bytecode respects the client requirements. Defining an encoding of the certificate should take into account several factors:
\begin{itemize} 
  \item certificate size must be reasonably small. This is important, for instance,  if the certified program comes over a network with a limitted bandwith
  \item certificates must be easily checked. This means that the certificate checker is  small and simple.
	       Of course, the code consumer might not want to spend all of its computation 
	      resources for checking that the certificate guarantees the program conformence to its policies.     
\end{itemize}

Note that the certificate size and its checking complexity are dual: the bigger the certificate is more manageable is the checking process and viceversa. 
The problem becomes even more difficult if the certificate must be checked on the device because of the computational and space constraints.
 


% towards.PCC
% For building a PCC framework from the components cited above 
% % there is still missing the proof certificate, the decision procedure
% that will be used by the producer for the certificate generation and the type checker used by the code
% client for checking the certificate. Important problems in this direction are
% \begin{itemize}
%  \item light weight verification condition generators. In particular, we refer 
%        to verification condition generation techniques which are simple and do not need
%	much computational resources. Because a verification condition generator always
%	form part of the trusted computing base on the client side, building such verification 
%	condition generators is important for on - device checking which rely on limitted computational 
%	resources  
  
%   \item generation of certificates. This is important for several reasons.
%         The certificate may certainly  arrive via the network and should not corrupt the performance 
%  
% 
% %  \item efficient type checker on the client site. This is in particular important 
%         if the device is with limitted resources where a complex certificate checking procedure
%         may corrupt the performance of the device
%        
%     
% \end{itemize}


 %To do this,  it is still missing the proof
%certificate, the decision procedure used by the code producer 
%for building the certificate  as well as the type checker used by the code
%client for checking the certificate. 

% to do. type systems
Another perspective in this direction is how   to encode type systems into the bytecode logic. 
Type systems provide a high level of automation. 
Their encoding in the logic can be useful as the certificate can be generated
 automatically and thus, avoids the user interaction. However, type systems are  conservative in the sense 
that they tend to reject a large amount of correct programs. A possible solution to this problem are hybrid certificates which combine both type systems and program 
logic. In this approach, the unknown code comes supplied with a  derivation in the logic generated potentially with the help of user interaction 
for the parts of the  code which can not be inferred by the type system.   The client side then applies a type inference procedure over  the
 unknown code and once it gets to the place in the parts of the code where the 
type inference does not work but for which there is a derivation in the certificate, he will type check that derivation.   
This is actually an approach which will be adopted in the Mobius project. 


The objective of this thesis was to 
give the basis for the a bytecode verification framework and to show that it is feasible. A further objective, pursued in the European project
 Mobius (short for Ubiquity, Mobility and Security) 
is to build basis for guaranteeing security and trust in program application in the presence of mobile and ubicuous computing. We hope that we have convinced
the reader for the importance of such techniques and in particular of the evolution from source verification to
 low level verification  and the necessity of an interactive verification process for building evidence for the security of unknown applications. 

\section{Test coverage}


\section{Achievements}
% done. summary
We have  presented an infrastructure for verification of Java bytecode programs   which allows to reason about potentially
sophisticated  functional and security properties and
which benefits from verification over Java source programs. We have also 
introduced the bytecode specification language BML tailored to Java bytecode, a compiler
from the Java source specification language JML to BML and a verification 
condition generator for Java bytecode programs. 
We have shown that the verification procedure is correct w.r.t. a big step  operational semantics of Java bytecode programs. 
Moreover, we have
proven that the verification procedure for Java like programs
and Java like bytecode are syntactically equivalent (modulo names and types). 
%This scheme is actually part of the PCC architecture of the
%European project Mobius\footnote{the site name} which aims to resolve the problems
%of mobile and ubicuous computing via PCC. 
We have developed a prototype of a verification condition generator based on the weakest precondition calculus presented in this thesis, as well 
as a compiler from the corresponding subset of JML to BML.
These two components have been integrated in the JACK \cite{BRL-JACK} verification framework 
developed and supported by our research team Everest at INRIA Sophia Antipolis which has been initially designed for
 the verification of Java source programs annotated with JML specification.

We would like to give a brief description of the implementation of the verification condition generator.
 The extension of the tool to bytecode programs which we added also interfaces these theorem provers. The bytecode 
verification condition generator works as follows. For the verification of a class file containing BML specification, it will generate verification conditions for every
 method of this class including the constructors. For generating the verification conditions concerning a method implementation, first the control flow
 graph corresponding to the bytecode instruction is built. The latter is transformed into an acyclic control flow graph where the backedges are 
removed.
 Then the verification procedure proceeds by generating over every execution path in the control flow graph its corresponding verification conditions. 
For every path which terminates by throwing an uncaught exception, the postcondition is the specified exceptional postcondition for this case. For the paths which terminate normally, 
the normal postcondition is taken. For every path which terminates with an instruction which is dominated by a loop entry and whose direct successor is the same loop entry, the postcondition 
is the corresponding loop invariant. The bytecode verification in Jack uses the intermediate language for the verification conditions and thus, bytecode verification conditions 
 can be translated to several different theorem provers - Simplify \cite{Simpl05DNS} which is an automatic decision procedure, 
the Atelier B and the Coq interactive theorem prover assistants. 

The bytecode verification condition generator benefits also from the original user friendly interface of the JACK tool.  In particular, 
the user can see the verification conditions in his favorite language - Java, Simplify, Coq or B. The lemmas are classified 
to what part of the annotation they refer to, as for instance, a lemma which refers to the establishment of the postcondition, or the preservation of the loop invariant.
The hypothesis in the lemma also hold the index of the instruction from which they originate. 
We have used the prototype of the bytecode verification condition generator for the case studies presented in Chapter \ref{applications:optimComp}.

% JACK (short for Java Applet Correctness Kit) is designed as a plugin for the Java interface development
% environment eclipse. 
%% It was originally tailored to the verification of Java source programs 
%w.r.t. their JML specifications. The tool has an intermediate proof obligation language which allows to extend it easily to interface more 
% theorem provers. Thus, the tool interfaces several theorem provers - Simplify \cite{Simpl05DNS} which is an automatic decision procedure, 
%%the Atelier B and the Coq interactive
%theorem prover assistant. 

\section{Future work}
In the following, we identify the directions for extending the work presented in this thesis

\subsection{Language coverage of the verification condition generator}
The bytecode verification condition generator works only for the sequential fragment of Java. But realistic applications 
rely often on multi - threading which is difficult to verify against a functional specifications or security policies.
One of the important aspects of the correctness of multi - threaded programs is the absence of deadlocks, 
and race conditions. Such properties can be ensured  by type systems \cite{FA99TSL,flanagan00typebased} or static verification based on program logic \cite{FLL02ESC}.  
The absence of deadlock and race conditions is a first step in the verification of the functional correctness of multi threaded programs. In order to build a full 
verification scheme for checking functional correctness more has to be done.
The earliest work for  verification of  parallel programs is  the Owicki and Gries approach   
\cite{nipkow99owickigries}  and the rely - guarantee approach. However, 
the first approach is not modular and requires a large amount of verification conditions while for the second, the annotation procedure can not be automatised.

% Such techniques for reasoning over the correctness of parallel programs  exist.
% One of the first logic - based verification techniques for parallel programs is due to Owicki and Gries 
%\cite{nipkow99owickigries}  in which every point of parallel interference is annotated and then the verification consists in establishing that
% all the possible inter leavings of all the threads respect the annotation. This technique is on one hand not modular as the verification process 
%needs the implementation of every program component and on other hand the number of verification conditions may be very big.
% Another approach is the rely guarantee technique which uses a Hoare style verification conditions \cite{nieto03relyguarantee}.
%There, the program points of interference are annotated not only with the predicate that must hold
%at the point but also with rely and guarantee  conditions which express what conditions the program guarantees to the other threads and what 
%the program requires from the other threads. This technique although tempting because of its modularity and the smaller number of verification conditions is difficult to apply
%as for guessing the rely and guarantee conditions requires an in - depth understanding of the program to be verified.  
Extending our verification scheme for bytecode will certainly be based on a more recent work  where one of the basic concerns is to establish method atomicity  \cite{TES03CF}. 
The notion of a statement atomicity states  that however a statement is interleaved with other parallel programs, the result of its execution will not change.
The atomicity can be  detected via static checking \cite{TES03CF} using type systems. Thus, the program verification process is separated in two parts
- first checking for program atomicity  \cite{TES03CF} are done  
and then verifying the functional correctness using  methodologies for sequential programs as Hoare style reasoning. 
In this last approach in the case of Java, the basic concern is to establish the atomicity of method bodies, i.e. method 
execution does not depend on the possible interleaving with threads.
Recently, E.Rodriguez and al. in \cite{RodriguezDFHLR05} proposed an extension for JML for multi threaded
 programs. Their proposal introduces  new specification keywords which allow to express that a variable is locked or
 that a method is atomic.% Giving the semantics of these keywords is still an ongoing work but we consider that the meaning of these specification constructs does not differ on source and bytecode. 
    
 



\subsection{Property coverage for the specification language}
Another direction which may be pursued as a future work of the thesis  is the extension of the expressiveness of the specification language BML. 
So far, BML supports method contracts - method pre and post  conditions, frame conditions, intermediate annotations as for instance
loop invariants, class specifications as well as special specification operators.
These are very useful aspects which allow for dealing with complex properties and 
gives a semantics on bytecode level  to a relatively small subset of the 
high specification language JML which corresponds to JML Level 0 \footnote{ http://www.cs.iastate.edu/~leavens/JML/jmlrefman/jmlrefman\_2.html\#SEC19}. 
 But it is certainly of interest to support more features of JML in BML
as this will turn the latter language richer. However, the meaning  of JML constructs 
(at least from our experience up to  now) is the same as the meaning of their corresponding part in BML.  

 An important example is the  JML construct for pure methods which has been  identified as  a challenge in the position paper \cite{LeavensLeinoMueller06}. 
 These methods does not modify the program state and thus, pure methods can be used in specifications 
 (only side effect free  expressions may occur in expressions).
 This gives more expressive  specifications as with them, for instance, specification can talk about the result of method invocation or use pure methods
 as a predicate relating their  initial and final state. 
 Formalizing and establishing the meaning of pure methods is difficult and a literature exists for this problem \cite{DarvasMueller06}.
 As we said above, the treatment of pure methods is the same on source and bytecode.

Also, support for specification constructions for alias control is certainly useful  especially because it allows for a modular verification 
of class invariants and frame conditions.
The alias control is guaranteed through ownership type systems which check that only an owner of a reference can modify its contents.
 This can considerably improve the current implementation for the verification of object invariants  \cite{DietlMueller05}.
In particular, our way of proving object invariants is non modular - at every method call the invariants of all visible \todo{say what does it mean visibility}
objects must be valid and they are assumed to hold when the call is terminated; similarly, when a method body is verified in its precondition the invariants of all visible
objects are assumed to hold and at the end of the method body all these invariants must be established. 
In practice, it is very difficult to verify that all the invariants for the all visible objects in a method  hold.
In order to keep the number of the verification conditions reasonable, we check the invariants only for the current object this and the 
objects received as parameters which is not sound.

 
\subsection{Preservation of verification conditions}

So far, we have shown that non-optimizing Java compilation
 preserves the  form of the verification conditions on source and
 bytecode.  We identify two basic directions for future work:
\begin{description}
 \item[Source and non optimized bytecode verification conditions equivalent modulo] % implement the compiler from Java source pogs to bytecode pogs
We have experimented with the verification conditions on source and
 bytecode in JACK and saw that in practice they are almost equivalent
 syntactically. From one part, there are the difference in the types 
 supported on bytecode and source level. For instance, the JVM does not
 provide support for boolean type values which are basically encoded as
 integer values. The same is true for byte and short values.  Another
 difference is the identifiers for variables and fields. For instance, in Java
 names for fields, method local variables and parameters are their identifiers which are given by the
 program developer. On bytecode method local variables and parameters are encoded as elements of the
 method register table and field names are encoded as numbers of the constant
 pool table of the class. A  simple but useful extension to the prototype for
 bytecode verification is a compiler from source proof obligations to bytecode proof obligations
 which overcomes those differences. This can be considered also as a step
 towards the  building a PCC architecture where the certificate generation benefits from
 the source level verification and thus allows for treating sophisticated
 security policies.

\item[Relation between verification conditions on Java source and optimized Java bytecode]
 The equivalence  between verification conditions on source and the corresponding non optimized bytecode is important as it
 allows that bytecode programs  benefit from source verification. In particular, it makes feasible Proof Carrying Code
 for sophisticated client requirements.
 However, a step further in this direction is to investigate the 
 relation between source programs and their bytecode counterpart produced by an optimizing compiler.
 This is interesting for the following reasons.
 It is a fact that interpretation of bytecode on the JVM is slower than execution by its corresponding assembly code. 
 In order to speed up the execution time for a Java bytecode program, one might use 
 a just-in-time compilation which  translates on the fly the bytecode into the machine specific language. However, JIT compilation can potentially slow
 the execution exactly because it does compilation on the fly.  Another possibility is to perform 
 optimizations on the bytecode. Currently, most of the  Java compilers do not support much optimizations.
 However, there do already exist Java optimizing compilers, for instance the Soot optimization framework\footnote{http://www.sable.mcgill.ca/soot/} 
 and most probably the number of the Java optimizing compilers will increase with the evolution of the Java language.
 A first step in the latter direction is the work of C. Kunz et al.\cite{BGKRsas06} who give an algorithm for translating 
 certificates and annotations over a non optimized program into a certificate  and annotation for its optimized version.
 Their work addresses  optimizations like constant propagation, loop induction, dead register elimination etc. 
\end{description}
\subsection{Towards a PCC architecture}

The bytecode verification condition generator and the BML compiler is the first step towards a PCC framework. 
The missing  part is  the certificate format which comes along with the bytecode and which  is the evidence for 
that the bytecode respects the client requirements. Defining an encoding of the certificate should take into account several factors:
\begin{itemize} 
  \item certificate size must be reasonably small. This is important, for instance,  if the certified program comes over a network with a limitted bandwith
  \item certificates must be easily checked. This means that the certificate checker is  small and simple.
	       Of course, the code consumer might not want to spend all of its computation 
	      resources for checking that the certificate guarantees the program conformence to its policies.     
\end{itemize}

Note that the certificate size and its checking complexity are dual: the bigger the certificate is more manageable is the checking process and viceversa. 
The problem becomes even more difficult if the certificate must be checked on the device because of the computational and space constraints.
 


% towards.PCC
% For building a PCC framework from the components cited above 
% % there is still missing the proof certificate, the decision procedure
% that will be used by the producer for the certificate generation and the type checker used by the code
% client for checking the certificate. Important problems in this direction are
% \begin{itemize}
%  \item light weight verification condition generators. In particular, we refer 
%        to verification condition generation techniques which are simple and do not need
%	much computational resources. Because a verification condition generator always
%	form part of the trusted computing base on the client side, building such verification 
%	condition generators is important for on - device checking which rely on limitted computational 
%	resources  
  
%   \item generation of certificates. This is important for several reasons.
%         The certificate may certainly  arrive via the network and should not corrupt the performance 
%  
% 
% %  \item efficient type checker on the client site. This is in particular important 
%         if the device is with limitted resources where a complex certificate checking procedure
%         may corrupt the performance of the device
%        
%     
% \end{itemize}


 %To do this,  it is still missing the proof
%certificate, the decision procedure used by the code producer 
%for building the certificate  as well as the type checker used by the code
%client for checking the certificate. 

% to do. type systems
Another perspective in this direction is how   to encode type systems into the bytecode logic. 
Type systems provide a high level of automation. 
Their encoding in the logic can be useful as the certificate can be generated
 automatically and thus, avoids the user interaction. However, type systems are  conservative in the sense 
that they tend to reject a large amount of correct programs. A possible solution to this problem are hybrid certificates which combine both type systems and program 
logic. In this approach, the unknown code comes supplied with a  derivation in the logic generated potentially with the help of user interaction 
for the parts of the  code which can not be inferred by the type system.   The client side then applies a type inference procedure over  the
 unknown code and once it gets to the place in the parts of the code where the 
type inference does not work but for which there is a derivation in the certificate, he will type check that derivation.   
This is actually an approach which will be adopted in the Mobius project. 


The objective of this thesis was to 
give the basis for the a bytecode verification framework and to show that it is feasible. A further objective, pursued in the European project
 Mobius (short for Ubiquity, Mobility and Security) 
is to build basis for guaranteeing security and trust in program application in the presence of mobile and ubicuous computing. We hope that we have convinced
the reader for the importance of such techniques and in particular of the evolution from source verification to
 low level verification  and the necessity of an interactive verification process for building evidence for the security of unknown applications. 

%\appendix
%


\begin{appendix} 
\section{Specification of the Bytecode Specification Compiler} \label{appendix1}

%%%%%%%%%%%%%%%%%%%%%%%%%%%%% 

\subsection{Class annotation} 
\label{Classspecification}
The following attributes can be added (if needed) only to the array of attributes of  the \texttt{class\_info } structure.

\subsubsection{Ghost variables} \label{modelvar}
\textbf{
\begin{tabbing}
Gho\=st\_Field\_attribute \{\\
\\
\> u2 attribute\_name\_index; \\
\> u4 attribute\_length;\\
\> u2 fields\_count;\\
\> \{\hspace{3 mm}\= u2 access\_flags; \\  
\> \> u2 name\_index;\\
\> \> u2 descriptor\_index;\\
\> \} fields[fields\_count];\\
\}
\end{tabbing}
}

\textbf{  attribute\_name\_index}\\
    The value of the attribute\_name\_index item must be a valid index into the constant\_pool table  . The constant\_pool entry at that index must be a CONSTANT\_Utf8\_info structure representing the string "Ghost\_Field".\\

\textbf{ attribute\_length }\\
   the length of the attribute in bytes = 2 + 6*fields\_count.\\

\textbf{ access\_flags} \\
  The value of the access\_flags item is a mask of modifiers used to describe access permission to and properties of a field. \\

\textbf{ name\_index} \\
The value of the name\_index item must be a valid index into the constant\_pool table. The constant\_pool entry at that index must be a CONSTANT\_Utf8\_info structure which must represent a valid Java field name stored as a simple (not fully qualified) name, that is, as a Java identifier. \\

\textbf{ descriptor\_index} \\
   The value of the descriptor\_index item must be a valid index into the constant\_pool table. The constant\_pool entry at that index must be a CONSTANT\_Utf8 structure which must represent a valid Java field descriptor.

% \subsubsection{Model(pure)  methods} \label{modelvar}
%\textbf{
%\begin{tabbing}
%Pur\=e\_Method\_attribute \{\\
%\\
%%\> u2 attribute\_name\_index; \\
%\> u4 attribute\_length;\\
%\> u2 methods\_count;\\
%\> \{\hspace{3 mm}\= u2 access\_flags; \\
%\> \> u2 name\_index;\\
%\> \> u2 descriptor\_index;\\
%\> \> u2 attributes\_count;\\
%\> \> attribute\_info attributes[attributes\_count];\\
%\> \} method\_info\_structure[methods\_count];\\
%\}
%\end{tabbing}
%}

%\textbf{  attribute\_name\_index}\\
%    The value of the attribute\_name\_index item must be a valid index into the constant\_pool table  . The constant\_pool entry at that index must be a CONSTANT\_Utf8\_info structure representing the string "Model\_Method".\\

%\textbf{ attribute\_length }\\
%   the length of the attribute in bytes.\\

%\textbf{ method\_info\_structure } \\
% a structure where the name\_index, descriptor\_index are indexes in the constant pool.

\subsubsection{Class invariant}
\textbf{
\begin{tabbing}
JML\=ClassInvariant\_attribute \{ \\ 
\> u2 attribute\_name\_index;\\ 
\> u4 attribute\_length;\\ 
\> formula attribute\_formula;\\ 
\}  
\end{tabbing}
}

\textbf{   attribute\_name\_index}\\
  The value of the attribute\_name\_index item must be a valid index into the  constant\_pool table. The constant\_pool entry at that index must be a CONSTANT\_Utf8\_info structure representing the string "ClassInvariant".\\

\textbf{   attribute\_length}\\
   the length of the attribute in bytes - 6.\\

\textbf{   attribute\_formula} \\
   code of the formula that represents the invariant, see  ~(\ref{formula}) for formula grammar \\\\

\subsubsection{History Constraints}
\textbf{
\begin{tabbing}
JML\=HistoryConstraints\_attribute \{ \\ 
\> u2 attribute\_name\_index;\\ 
\> u4 attribute\_length;\\ 
\> formula attribute\_formula;\\ 
\}
\end{tabbing}
}

\textbf{   attribute\_name\_index}\\
  The value of the attribute\_name\_index item must be a valid index into the constant\_pool table. The constant\_pool entry at that index must be a CONSTANT\_Utf8\_info structure representing the string "Constraint".\\

\textbf{   attribute\_length}\\
   the length of the attribute in bytes - 6.\\

\textbf{   attribute\_formula} \\
   code of the formula that is a predicate of the form \texttt{P\( state, old(state)\)} that establishes relation between the prestate and the postate  of a method execution. see  ~(\ref{formula}) for formula grammar \\\\

%%%%%%%%%%%%%%%%%%%%%%%%%%%%%%
\subsection{Method annotation}

\subsubsection{Method specification}
The \rm{JML} keywords \texttt{requires, ensures, exsures} will be defined in a newly attribute in Java VM bytecode that can be inserted into the structure \texttt{method\_info}  as elements of the array \texttt{attributes}.


\textbf{    
\begin{tabbing}
JML\=Method\_attribute \{ \\ 
\> u2 attribute\_name\_index;\\ 
\> u4 attribute\_length;\\ 
\> formula requires\_formula;\\
\> u2 spec\_count;\\
\> \{\hspace{3 mm}\= formula spec\_requires\_formula; \\
\> \> u2 modifies\_count;\\
\> \> formula modifies[modifies\_count];\\
\> \> formula ensures\_formula;\\
\> \> u2 exsures\_count;\\
\> \> \{\hspace{3 mm}\= u2 exception\_index; \\
\> \> \> formula exsures\_formula;\\
\> \> \} exsures[exsures\_count];\\
\> \} spec[spec\_count];   \\
\}
\end{tabbing}
}

\textbf{   attribute\_name\_index}\\
  The value of the attribute\_name\_index item must be a valid index into the constant\_pool table. The constant\_pool entry at that index must be a CONSTANT\_Utf8\_info structure representing the string "MethodSpecification".\\

\textbf{   attribute\_length}\\
   The length of the attribute in bytes.\\

\textbf{   requires\_formula} \\
   The formula that represents the precondition (in the subsection see Formulas )\\

\textbf{   spec\_count} \\
   The number of specification case.\\

\textbf{   spec[]} \\
   Each entry in the spec array represents a case specification. Each entry must contain the following items:\\

\textbf{   spec\_requires\_formula} \\
   The formula that represents the precondition (in the subsection see Formulas )\\

\textbf{   modifies\_count} \\
   The number of modified variable.\\

\textbf{   modifies[]} \\
   The array of modified formula.\\

\textbf{  ensures\_formula } \\
   The formula that represents the postcondition (in the subsection see Formulas )\\\\

\textbf{   exsures\_count} \\
   The number of exsures clause.\\

\textbf{   exsures[] } \\
   Each entry in the exsures array represents an exsures clause. Each entry must contain the following items:\\

\textbf{   exception\_index} \\
   The index must be a valid index into the constant\_pool table. The constant\_pool entry at this index must be a CONSTANT\_Class\_info structure representing a class type that this clause is declared to catch.\\

\textbf{  exsures\_formula } \\
   The formula that represents the exceptional postcondition (in the subsection see Formulas )\\
\textit{Note:}\\
if the exsures clause is of the form: \\  
\texttt{ exsures} (\texttt{Exception\_name}  e) P(e) 
it is first  transformed in : 
\texttt{exsures} \texttt{Exception\_name} P(e)[e $\leftarrow$ \textrm{EXCEPTION}], 
where \textrm{ EXCEPTION} is a special keyword for the specification language, for which in JML there is no correspondent one.

\subsubsection{Set}
These are particular assertions that assign to model fields. 
 %Asserts must bve relied the proper place where the loop starts. To identify where exactly the bytecode for a loop starts, the assumption that the \texttt{code } attribute   \texttt{LineNumberTable} must be present.    
\textbf{  
\begin{tabbing}
Ass\=ert\_attribute \{\\
\> u2 attribute\_name\_index;\\
\> u4 attribute\_length;\\
\> u2 set\_count;\\
\> \{\hspace{3 mm}\= u2 index; \\
\> \> expression e1; \\
\> \> expression e2; \\
\> \} set[set\_count];\\
\}
\end{tabbing}
}

\textbf{  attribute\_name\_index}\\
    The value of the attribute\_name\_index item must be a valid index into the \texttt{ constant\_pool table}  . The \texttt{constant\_pool} entry at that index must be a\texttt{ CONSTANT\_Utf8\_info} structure representing the string "Set".\\

\textbf{  attribute\_length} \\
   The length of the attribute in bytes.

\textbf{  set\_count} \\
   The number of set statement.\\

\textbf{  set[]} \\
   Each entry in the set array represents a set statement. Each entry must contain the following items:\\

\textbf{  index} \\
   The index in the bytecode where the \textbf{ assignment} is done.\\

\textbf{  e1 } \\
the expression to which is assigned a value. It must be a JML expression, i.e. a JML field, or a dereferencing a field of JML reference object
  an assignment expression see ~(\ref{jmlExprs})\\

\textbf{  e2 } \\
the expression that is assigned as value to the JML expression\\ 


\subsubsection{Assert}
 %Asserts must bve relied the proper place where the loop starts. To identify where exactly the bytecode for a loop starts, the assumption that the \texttt{code } attribute   \texttt{LineNumberTable} must be present.    
\textbf{  
\begin{tabbing}
Ass\=ert\_attribute \{\\
\> u2 attribute\_name\_index;\\
\> u4 attribute\_length;\\
\> u2 assert\_count;\\
\> \{\hspace{3 mm}\= u2 index; \\
\> \> formula predicate; \\
\> \} assert[assert\_count];\\
\}
\end{tabbing}
}

\textbf{  attribute\_name\_index}\\
    The value of the attribute\_name\_index item must be a valid index into the \texttt{ constant\_pool table}  . The \texttt{constant\_pool} entry at that index must be a\texttt{ CONSTANT\_Utf8\_info} structure representing the string "Assert".\\

\textbf{  attribute\_length} \\
   The length of the attribute in bytes.

\textbf{  assert\_count} \\
   The number of assert statement.\\

\textbf{  assert[]} \\
   Each entry in the assert array represents an assert statement. Each entry must contain the following items:\\

\textbf{  index} \\
   The index in the bytecode where the \textbf{ predicate} must hold

\textbf{  predicate } \\
  the predicate that must hold at index  \textbf{index} in the bytecode ,see  ~(\ref{formula})


\subsubsection{Loop specification}
%Loops are ordered by giving to every loop entry a different number. An inner loop has always a number bigger than the loop that contains it. A loop that is after another loop has always a bigger number than the loop that is  executed before it. One can find the same order of loops on source and bytecode level. 
\textbf{     
\begin{tabbing}
JML\=Loop\_specification\_attribute \{\\
\> u2 attribute\_name\_index;\\
\> u4 attribute\_length;\\
\> u2 loop\_count;\\
\> \{\hspace{3 mm}\= u2 index;\\
\> \> u2 modifies\_count;\\
\> \> formula modifies[modifies\_count];\\
\> \> formula invariant;\\
\> \> expression decreases;\\
\> \} loop[loop\_count];\\
\}
\end{tabbing}
}

\textbf{   attribute\_name\_index }\\
    The value of the attribute\_name\_index item must be a valid index into the  \texttt{ constant\_pool table}. The \texttt{constant\_pool} entry at that index must be a\texttt{ CONSTANT\_Utf8\_info} structure representing the string "Loop\_Specification''.

\textbf{   attribute\_length }\\
    The length of the attribute in bytes\\

\textbf{   loop\_count }\\
    The length of the array of loop specifications\\

\textbf{   index }\\  
    The index of the instruction in the bytecode array that corresponds to the  entry of the loop 
\\ %in the  \texttt{LineNumberTable } where the beginning of the corresponding loop is described\\

\textbf{   modifies\_count}\\
    The number of modified variable.\\

\textbf{   modifies[]} \\
    The array of modified expressions.\\

\textbf{  invariant } \\
    The predicate that is the loop invariant. It is a formula written in the grammar specified in the section Formula ,see  ~(\ref{formula}) \\

\textbf{  decreases } \\
    The expression whose decreasing after every loop execution will guarantee loop termination 

\subsubsection{Block specification}
Here also the   \texttt{LineNumberTable}  attribute must be present.    

\textbf{  
\begin{tabbing}
Blo\=ck\_attribute \{\\
\> u2 attribute\_name\_index;\\
\> u4 attribute\_length;     \\
\> u2 start\_index;          \\
\> u2 end\_index;            \\
\> formula precondition;\\
\> u2 modifies\_count;\\
\> formula modifies[modifies\_count];\\
\> formula postcondition; \\ 
\}
\end{tabbing}
}

\textbf{  attribute\_name\_index }\\
    The value of the attribute\_name\_index item must be a valid index into the \texttt{ constant\_pool table}. The \texttt{constant\_pool} entry at that index must be a\texttt{ CONSTANT\_Utf8\_info} structure representing the string "\_specification ".\\

\textbf{  attribute\_length } \\
   The length of the attribute in bytes - 6, i.e. equals \texttt{n+m}.\\

\textbf{start\_index} \\
   The index in the  \texttt{LineNumberTable } where the beginning of the block  is described

\textbf{end\_index} \\
   The index in the  \texttt{LineNumberTable } where the end of the block  is described

\textbf{  precondition} \\
  The predicate that is the precondition of the block ,see  ~(\ref{formula})

\textbf{   modifies\_count} \\
   The number of modified variable.\\

\textbf{   modifies[]} \\
   The array of modified formula.\\

\textbf{  postcondition} \\
  the predicate that is the postcondition of the block ,see  ~(\ref{formula})

\subsection{Formula and Expression compiler function }
The compiler function is denoted with  $ \ulcorner \urcorner_{\tt{context}}$. It is defined inductively over the grammar of the specification language as defined in Section 
\ref{grammar} and more particularly on Fig. \ref{bclGrammar}. 
The compiling function depends on the context $\tt{context}$ and in particular it is important for compiling field references and method call expressions.
 The context is the class where the method or field is declared. For example when  compiling the fully qualified name \texttt{a.b } the two subexpressions \texttt{a} and \texttt{b}
are compiled one after another. The subexpression \texttt{a} will be compiled in the context of \texttt{this} type ,i.e. in the context of the class where this expression appears and the subexpression \texttt{b} will be compiled in the context of the class of the subexpression \texttt{a} as it is a field of the class of the
 subexpression \texttt{a}.  

\subsection{Formulas} \label{formula}

%\textit{Notation} : $...$ - means that the preceding expression may be repeated 0 or more times.  
\subsubsection{Translation of formulas}
\begin{tabbing}
$\ulcorner \tt{Formula} \urcorner_{\tt{context}}$ ::= \= $\ulcorner \tt{Connector} \urcorner \ \ulcorner \tt{Formula_1} \urcorner_{\tt{context}} \  ... \  \ \ulcorner \tt{Formula_n} \urcorner_{\tt{context}} \mid $\\
\> $\ulcorner \tt{Quantifier} \urcorner \ulcorner \tt{Formula_1} \urcorner_{\tt{context}}$\\
\> $\ulcorner \tt{PredicateSymbol} \urcorner \ \ulcorner \tt{Expression} \urcorner_{\tt{context}} \  ... \ulcorner \tt{Expression } \urcorner_{\tt{context}}  \mid$ \\
\> $\ulcorner \tt{True} \urcorner \mid $ \\
\> $\ulcorner \tt{False} \urcorner $
\end{tabbing}

\textit{Remark:}
$n$ is coded in 1 byte.\\
Every quantification that contains a range , i.e. every formula of the form : $\forall \  A \  a; P(a); Q(a)$ should be  transformed into  $\forall \ A \ a; P(a) \Rightarrow Q(a)$

\subsubsection{Predicate constants}

\begin{center}
\begin{tabular}[t]{|c|l|}
\hline
\texttt{Predicate} & \code\\
\hline
$\tt{True} $ & 0x00 \\
$\tt{False}$ & 0x01\\
\hline
\end{tabular}\\[2 mm]
Codes for the predicate constants \texttt{True}, \texttt{False}
\end{center}
$ \ulcorner$ \texttt{True}$\urcorner_{\tt{context}} $  ::= $ \code(\tt{True})   $ \\
$ \ulcorner$ \texttt{False}$\urcorner_{\tt{context}} $  ::= $ \code(\tt{False}) $ \\

\subsubsection{Logical connectors}
\begin{center}
\begin{tabular}[t]{|c|l|}
\hline
\texttt{Connector} & \code \\
\hline
$\wedge$ & 0x02 \\
$\vee$   & 0x03  \\
$\Rightarrow$ & 0x04 \\
$!$ & 0x05  \\
\hline
\end{tabular}\\[2 mm]
Codes for the \texttt{Connector} symbols  
\end{center}
\begin{tabbing}
$\ulcorner \tt{Connector} \urcorner_{\tt{context}} $ ::= \= \code $(\tt{Connector}) $ \\
\end{tabbing}

\subsubsection{Quantifiers}

$ \ulcorner \tt{Quantifier} \urcorner ::= \ulcorner \tt{Quantificator symbol} \urcorner \ (\ulcorner \rm{Type} \urcorner_{\tt{context}} \ \ulcorner \rm{BoundVar} \urcorner_{\tt{context}})_n $, where $n$ is the number of bound variables. \\

\subsubsection{Bound Variables}
$\ulcorner \tt{BoundVar} \urcorner_{\tt{context}}  $ = code(\texttt{BoundVar} ) \texttt{int}\\
where \texttt{int} is a fresh integer value.


\subsubsection{Quantificator symbols }
\begin{center}  
\begin{tabular}[t]{|c|l|}
\hline
\texttt{Quantificator symbol} &\code\\
\hline
$\forall$ & 0x06  \\
$\exists$  & 0x07 \\
\hline
\end{tabular}\\[2 mm]
Codes for \texttt{Quantification symbols}
\end{center}
$\ulcorner$\texttt{ Quantification symbol} $\urcorner_{\tt{context}} $ ::= code(\texttt{ Quantification symbol} )

The code of any bound variable $ident$ is a fresh variable coded in \emph{1 byte} that must replace any occurrence of $ident$ in the  predicate coming after the quantification expression\\ 
The type \textrm{Type} is a \texttt{ fully qualified  name} expression.


\subsubsection{Predicate symbols }
\begin{center}
\begin{tabular}{|c|l|}
\hline 
\texttt{PredicateSymbol } & \code \\
\hline
$ \tt{==} $ &  0x10 \\ 
$ \tt{> } $ &  0x11 \\
$ \tt{<}  $ &  0x12 \\
$ \tt{<=} $ &  0x13 \\
$ \tt{>= }$ &  0x14 \\
$ \tt{instanceof }$ &  0x15 \\
$ \tt{<:}$ &  0x16 \\
\hline
\end{tabular}\\[2 mm]
Codes for the \texttt{Predicate Symbols} symbols
\end{center}
\begin{tabbing}
$\ulcorner \tt{PredicateSymbol} \urcorner_{\tt{context}} $ ::= \= $ \code(\tt{PredicateSymbol}) $ \\
\end{tabbing}



\subsection{Expressions}

Here the grammar for well formed expressions is described. We use the prefixed representation of expressions , e.g    $ + \ \tt{Arithmetic\_Expression \ Arithmetic\_Expression}  $ which stands for the infix representation    $  \tt{Arithmetic\_Expression \ +  \ Arithmetic\_Expression}  $


\subsubsection{Arithmetic Expressions}

\begin{center}
\begin{tabular}{|c|c|}
  \hline
  Operator Symbol  & \code  \\
  \hline
  + & 0x20  \\
  - & 0x21  \\
  * & 0x22 \\
  / & 0x23 \\
  \% & 0x24 \\
  - & 0x25  \\
  int Literal i & 0x40i \\
  char Literal i & 0x41i \\
  \hline

\end{tabular}\\[2 mm]
Codes for Arithmetic operations
\end{center}
binary operations : \\
$\ulcorner \tt{ op \ Expression_1 \ Expression_2 } \urcorner_{\tt{context}} $ = \\
\phantom{..............................................} \code \texttt{$($op$ ) $} \\
\phantom{..............................................} $ \ulcorner  \tt{  Expression_1 }  \urcorner_{\tt{context}}$ \\
\phantom{..............................................} $\ulcorner \tt{ Expression_2 } \urcorner_{\tt{context}}$ \\\\\\
%\phantom{..............................................} 0x00 \\\\\\


unary operations : \\
$\ulcorner \tt{ op \ Expression }   \urcorner_{\tt{context}}$ = \\
\phantom{..................................... }\code \texttt{$($op$)$} \\
\phantom{.....................................} $\ulcorner \tt{Expression}  \urcorner_{\tt{context}} $\\
% \phantom{.....................................} 0x00


\subsubsection{JML expressions}

\begin{center}
\begin{tabular}{|c|l|}
\hline
   JML constant & \code  \\
\hline
  $\backslash$ \texttt{typeof}   & 0x50 \\
  $\backslash$ \texttt{elemtype} & 0x51 \\
  $\backslash$ \texttt{result} & 0x52 \\
  $\backslash$ \texttt{old} &  \\
  $\ast$ &  0x53 \\
 $\backslash$ \texttt{type} & 0x54  \\ 
 $\backslash$ \texttt{Type} & 0x55  \\ 
\hline
\end{tabular}\\[2 mm]
Codes of JML constant
\end{center}
$\ulcorner \tt{ \backslash typeof(Expression) } \urcorner_{\tt{context}}$ = \\
\phantom{.................... .........} $  \tt{code(\backslash typeof) }  $\\
\phantom{.................... ........} $\  \ulcorner \tt{ Expression } \urcorner_{\tt{context}} $ \\
%\phantom{.................... ........}0x00 \\\\

$\ulcorner \tt{ \backslash elemtype(Expression) } \urcorner_{\tt{context}}$ = \\
\phantom{.................... ........} $ \tt{code(\backslash elemtype) } $ \\
\phantom{.................... ........}$\ulcorner \tt{ Expression } \urcorner_{\tt{context}}$ \\
%\phantom{.................... ........}0x00 \\\\

$\ulcorner \tt{ \backslash result } \urcorner_{\tt{context}}$ =  \texttt{code( $\backslash$ \texttt{result})} \\\\ % 0x00 \\\\

$\ulcorner \tt{ \lbrack } \ \tt{ Expression} \ \ast \ \urcorner_{\tt{context}}$  = $ \ulcorner \tt{\lbrack} \urcorner_{\tt{context}} $  $ \ulcorner \tt{Expression } \urcorner_{\tt{context}} \ \ulcorner \ast \urcorner_{\tt{context}} $ \\\\ %0x00 \\\\

$\ulcorner \tt{\backslash old (Expression ) } \urcorner_{\tt{context}} $ = $\tt{code(\backslash old )} \ulcorner \tt{Expression} \urcorner_{\tt{context}}$ \\
%\textit{Note:} $\ulcorner . \urcorner_{\tt{context}}^{old}$ is the same as $\ulcorner . \urcorner_{\tt{context}}$ with the exception that for  identifiers code it is using the function \textrm{old}, defined in  ~\ref{Codesvariables}\\\\

$\ulcorner \backslash \tt{type}( Expression) \urcorner_{\tt{context}} $ = $\ulcorner \backslash \tt{type} \urcorner_{\tt{context}} $ \\
\phantom{$\ulcorner \backslash \tt{type}( Expression) \urcorner $ = } $ \ulcorner Expression \urcorner_{\tt{context}} $\\
%\phantom{$\ulcorner \backslash \tt{type}( Expression) \urcorner $ = } 0x00 \\\\




$\ulcorner \backslash \tt{TYPE} \urcorner_{\tt{context}} $ = \code($\tt{TYPE} $ ) \\ %0x00 \\
see  ~\ref{Codesvariables}, etc. see  ~\ref{CodesJMLkw}, ~\ref{CodesJavakw} \\ 

% \subsubsection{ Calls to Pure Methods}
% \begin{center}
% \begin{tabular}{|c|c|}
%    \hline
%    symbol & \code \\
%    \hline
%    \texttt{( } &  0x60  \\
%    \hline
% \end{tabular}\\[2 mm]
% Code for method call symbol
% % \end{center}
% $\ulcorner \tt{(}  \urcorner_{\tt{context}} $ = \code( $ \tt{ ( \phantom{)} }$ )

% $\ulcorner \tt{ (  \ Expression \ Expression_1 ...... Expression_n  } \urcorner_{\tt{context}}$ = \\
% $ \phantom{................................................} $ $\ulcorner \tt{(} \urcorner_{\tt{context}} $  \\
% $ \phantom{................................................} $ $\ulcorner \tt{Expression} \urcorner_{\tt{context}}$  \\
% $ \phantom{................................................} $ $ n$  \\
% $ \phantom{................................................} $ $\ulcorner \tt{Expression_1} \urcorner_{\tt{type(this)}} $ \\
% $ \phantom{................................................} $ ...\\
% $ \phantom{................................................} $ $\ulcorner \tt{Expression_n} \urcorner_{\tt{type(this)}} $ \\
%$ \phantom{................................................}$  0x00 

\subsubsection{Array access}
\begin{center}

\begin{tabular}{|c|c|}
\hline
symbol & \code \\
  \hline
  % after \\: \hline or \cline{col1-col2} \cline{col3-col4} ...

   \texttt{ [ } &  0x61 \\
  \hline
\end{tabular}\\[2 mm]
Code for array access symbol
\end{center}
$\ulcorner \tt{[}  \urcorner_{\tt{context}} $ = \code(\texttt{[})

$\ulcorner \tt{ [  \ Expression \ Arithmetic\_Expression   } \urcorner_{\tt{context}}$ = \\
$ \phantom{................................................}$  $\ulcorner \tt{[} \urcorner_{\tt{context}}$  \\
$ \phantom{................................................}$  $\ulcorner \tt{Expression} \urcorner_{\tt{context}}$\\ 
$ \phantom{................................................}$  $\ulcorner \tt{Arithmetic\_Expression} \urcorner_{\tt{context}}$ \\ 
%$ \phantom{................................................}$  0x00 

\subsubsection{Cast expression}

\begin{center}
\begin{tabular}{|c|c|}
   \hline
  symbol & \code \\
  \hline
  % after \\: \hline or \cline{col1-col2} \cline{col3-col4} ...
   \texttt{cast } &  0x62  \\
  \hline
\end{tabular}\\[2 mm]
Codes for cast symbol
\end{center}
$\ulcorner \tt{cast }\urcorner_{\tt{context}} $ = \code(\texttt{cast})

$\ulcorner \tt{cast} \ \tt{Expression} \ \tt{Expression} \urcorner_{\tt{context}}$ = \\
$ \phantom{................................................}$  $\ulcorner \tt{cast} \urcorner_{context} $ \\
$ \phantom{................................................}$  $\ulcorner \tt{Expression} \urcorner_{context}$\\ 
$ \phantom{................................................}$  $\ulcorner \tt{Expression} \urcorner_{context}$ \\ 
%$ \phantom{................................................}$  0x00 

\subsubsection{References} \label{Codesvariables}


\subsubsection{Variable Names}
Variable names denote either local variables (parameters) , class or instance fields, either JML ghost fields. 


\begin{tabular}{|c|l|}
\hline
kind of name   & compile name    \\
\hline 
Field name   & 0x80 \indexComp ( Field Name)    \\
& \\
 Local Variable &  0x90 \indexComp(  Local Variable )    \\
& \\
JML ghost Field name   & 0xA0 \indexComp (JML ghost Field name  )   \\
 ) \\ 
\hline
\end{tabular}\\[2 mm]

The function \indexComp is defined as follows : \\ \\ \\
\begin{tabular}{|c|l|}
\hline
Variable Identifier & \indexComp ( Name)\\
\hline
Field name  & the constant pool index at which a  ConstantFieldReference attribute \\
& describes the  field  \\
& \\
JML field name & the constant pool index at which a  ConstantFieldReference attribute \\
& describes the  field \\
& \\
Local Variable & the index of the registers of the method that represents \\
&  this variable(  + start\_ind + length )  \\
\hline
\end{tabular}

Two remarks :
\begin{enumerate}
\item the function \texttt{index } has the same definition for JML ghost fields and Java fields. Note that Java compiler adds constant fields data structures in the constant\_pool only for fields that are dereferenced. For any field that is mentioned in the specification but not dereferenced in the Java code a new constant field refernence will be added  on JML compilation time.
\item Note that Java compilers may generate code that uses the same register to store values of different types at different states of execution (and consequently at different points in the bytecode ). In the present specification we consider that any register contains exactly one type of values at any point in the code and that it hold not more than one method parameter at any point in the bytecode.
\end{enumerate}

\subsubsection{Java keywords}\label{CodesJavakw}
\begin{center}
\begin{tabular}{|c|l|}
\hline
Java keyword  & \code  \\
\hline
\texttt{this} & \texttt{0x8000} \\
\texttt{null} & \texttt{0x06}   \\
\hline
\end{tabular}\\[2 mm]
Codes for Java keywords
\end{center}
$\ulcorner$ keyword $\urcorner_{context}$ = \code(keyword)\\


\textit{Note:} for the reserved Java keyword \texttt{this}, the JVMS always puts the reference to the \texttt{this} object at position 0 in the array of local variables for any non static method. 

\subsubsection{Fully qualified names} \label{References}

\begin{center}
\begin{tabular}{|c|c|}
\hline \\
symbol & \code \\ 
\hline \\
\texttt{.} & 0x63 \\
\hline
\end{tabular}\\[2 mm]
\end{center}


$\ulcorner \tt{.} \urcorner_{context} $ = \code( \texttt{.})  \\

$\ulcorner \tt{.} \ \tt{Expression_1}  \tt{Expression_2}  \urcorner_{context}$ =  \[ \left \{ \begin{array}{ll} 
   \ulcorner\tt{.} \urcorner_{\tt{context}}   \ulcorner \tt{Expression_2} \urcorner_{type(\tt{this})}  \  \ulcorner \tt{this}\urcorner_{\tt{context}}  & if \ \tt{Expression_1} == \tt{this} \\
    & \\
    \ulcorner \tt{Expression_2} \urcorner_{type(\tt{Expression_1})}                                             & if \ \tt{Expression_1} == \tt{super} \\
    & \\
    \ulcorner \tt{Expression_2} \urcorner_{\tt{Expression_1}}                                      & if \ \tt{Expression_1} \ is \  a  \ class \  name \\
    & \\
   \ulcorner\tt{.} \urcorner_{\tt{context}} \ulcorner \tt{Expression_2} \urcorner_{typelocal(s)} \ \ulcorner \tt{local(s)}\urcorner_{\tt{context}}  &  if \ \tt{Expression_1} \\
                                                                      &  is \ a \ local \ variable \\
                                                                      & \wedge \\
                                                                      & index\_in\_local\_array(\tt{Expression_1}  ) \\ 
                                                                      & == s \\
                                                                     
    & \\
   \ulcorner\tt{.} \urcorner_{\tt{context}} \ulcorner  \tt{Expression_2} \urcorner_{ret\_type(expr)} \ \ulcorner \tt{Expression_1}  \urcorner_{\tt{context}}  & if \ \tt{Expression_1}   = \\
                                                                                                                   & \phantom{...} \tt{ ( } \\
                                                                                                                   & \phantom{...} \tt{expr}\\
                                                                                                                   & \phantom{...} length(list\_expr)\\
                                                                                                                   & \phantom{...} list\_expr \\ 
                                                                                                                  
     & \\
    \ulcorner\tt{.} \urcorner_{\tt{context}} \ulcorner \tt{Expression_2} \urcorner_{elem\_type(expr_1)} \  \ulcorner  \tt{Expression_1} \urcorner_{\tt{context}}   
      & if \ \tt{Expression_1} = \\ 
                                                                                                                   & \phantom{...} \tt{ [ expr_1 \ expr_2} \\ 
   
   
    & \\   
   \ulcorner\tt{.} \urcorner_{\tt{context}} \ulcorner  \tt{Expression_2} \urcorner_{type(\tt{Expression_1})} \  \rm{code}(context, \tt{Expression_1})   & if \ \tt{Expression_1} \ is \\
   &  a \ field \ name \\
   
   & \\  
   \ulcorner\tt{.} \urcorner_{\tt{context}} \ulcorner  \tt{Expression_2} \urcorner_{type(\tt{Expression_1})} \  \ulcorner Expression_1 \urcorner_{\tt{context}}    & else  \\
                                       
                                            \end{array}
                                            \right .
                                            \]





\subsubsection{Specific keywords for the language} 
We introduce the keyword  \textrm{EXCEPTION} that may appear only in exceptional postconditions. It stands for the thrown exception  object
\begin{center}
\begin{tabular}{|c|c|}
\hline \\
\textrm{EXCEPTION} & 0xB5 \\
\hline
\end{tabular}\\[2 mm]
\subsubsection{Codes}
\end{center}
$\ulcorner \rm{EXCEPTION} \urcorner_{\tt{context}} $ = code(\textrm{EXCEPTION}) \\\\\\\\\\\\\\
 




\subsection{Codes}
\begin{center}
\begin{tabular}[t]{|c|c|l|}
\hline
\texttt{Code}&\texttt{Symbol}&\texttt{Grammar}\\
\hline
0x00 & $\tt{True}$  & \\
0x01 & $\tt{False}$ &\\
0x02 & $\wedge$ & Formula Formula \\
0x03 & $\vee$ & Formula Formula \\
0x04 & $\Rightarrow$ & Formula Formula \\
0x05 & $!$ & Formula\\
0x06 & $\forall$ & n $($ Type $)_n$ Formula \\
0x07 & $\exists$ & n $($ Type $)_n$ Formula \\
0x10 & $\tt{==} $ & Expression Expression \\ 
0x11 & $\tt{> } $ & Expression Expression \\
0x12 & $\tt{<}  $ & Expression Expression \\
0x13 & $\tt{<=} $ & Expression Expression \\
0x14 & $\tt{>= }$ & Expression Expression \\
0x15 & $\tt{instanceof }$ & Expression Type \\
0x16 & $\tt{<:}$ & Type Type \\
0x20 & $+$ & Expression Expression \\
0x21 & $-$ & Expression Expression \\
0x22 & $*$ & Expression Expression \\
0x23 & $/$ & Expression Expression \\
0x24 & $\%$ & Expression Expression \\
0x25 & $-$ & Expression \\
0x30 & $and$ & Expression Expression \\
0x31 & $or$ & Expression Expression \\
0x32 & $xor$ & Expression Expression \\
0x33 & $<<$ & Expression Expression \\
0x34 & $>>$ & Expression Expression \\
0x35 & $>>>$ & Expression Expression \\
0x40 & $\tt{int \ constant}$ & i \\
0x41 & $\tt{char \ constant}$ & i \\
0x50 & $\backslash$ \texttt{typeof} & Expression \\
0x51 & $\backslash$ \texttt{elemtype} & Type \\
0x52 & $\backslash$ \texttt{result} & \\
0x53 & $\ast$ & Expression \\
0x54 & $\backslash$ \texttt{type} & Expression \\ 
0x55 & $\backslash$ \texttt{Type} & \\ 
0x56 & $\backslash$ \texttt{old} & \\ 
0x60 & \texttt{(} & Expression n $($ Expression $)_n$\\
0x61 & \texttt{[} & Expression Expression \\
0x62 & \texttt{cast} & Type Expression \\
0x63 & \texttt{.}& Expression Expression \\
0x64 & \texttt{? :} & Formula Formula Formula \\
0x70 & \texttt{this} & \\
0x72 & \texttt{null} & \\
0x80 & Fieldref & i \\
0x90 & Local variable & i \\
0xA0 & JML ghost field& i\\
0xB0 & Methodref & i \\
0xC0 & Type & i \\
0xE0 & BoundVar & i \\
0xF0 & Stack \\
0xF1 & Counter \\
\hline
\end{tabular}\\[2 mm]
\end{center}

\end{appendix}



%%\backmatter

\bibliographystyle{plain}
\bibliography{cardis05/biblio,memory/bib_mem-wpc,sttt/sttt,sac/specification,sac/bibliography,fm03/jack,../specification.bib,/net/home/gbarthe/bib/string,/net/home/gbarthe/bib/gilles,/net/home/gbarthe/bib/article,/net/home/gbarthe/bib/book,/net/home/gbarthe/bib/lncs,/net/home/gbarthe/bib/misc,/net/home/gbarthe/bib/proceedings,/net/home/gbarthe/bib/techrep,/net/home/gbarthe/bib/thesis,/net/home/gbarthe/bib/software,/net/home/gbarthe/bib/web}

\end{document}


% Articles :
%  - JML
%  - Jack FME
%  - Cardis
%  - Bytecode
% + rapport de Thibaut


% 1 Introduction 1-2
% 3 T & M
% 3.1 JACK 7-17
% 3.2 Sec prop propag 18-24
% 3.3 Bytecode 25-32
% 4 Eval
% 4.1 Ober   33-34
% 4.2 Axal   35-36
% 4.3 Byte Ver 37-38
% 4.4 Footprint 39-44
% 4.5 Memory  45-49
% 5 Concl 51

% TODO API specification
%      Pile IP

% TODO
% Citer Inspired
% Intro : Existing approach
% 2.1.5 : A completer
% 2.3 : Motivation
% 2.3.5 : A completer
% Conclusion : Related work + on va travailler ensemble Javacard oui mais midp..
