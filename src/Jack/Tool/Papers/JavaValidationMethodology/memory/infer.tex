In the previous section, we have described how the memory consumption
of a program can be modeled in BML and verified using an appropriate
verification environment. While our examples illustrate the benefits
of our approach, especially regarding the precision of the analysis,
the applicability of our method is hampered by the cost of providing
the annotations manually. In order to reduce the burden of manually
annotating the program, one can rely on annotation assistants that
infer automatically some of the program annotations (indeed such
assistants already exist for loop invariants~\cite{NimmerE02:ISSTA} and class
invariants~\cite{log04:vmcai}). In this section, we describe an
implementation of an annotation assistant dedicated to the analysis of
memory consumption, and illustrate its functioning on an example.
 
The annotation assistant performs two tasks. First, it inserts the
ghost assignments on appropriate places; for this task, the user must
provide annotations about the memory required to create objects of the
given classes. 

Second, it inserts pre- and postconditions for each method. In this case, variants for loops and recursive methods may be given by the user or be
synthesized through appropriate mechanisms.  Based on this
information, the annotation assistant recursively computes the memory
allocated on each loop and method. Essentially, it finds the maximal
memory that can be allocated in a method by exploring all its possible
execution paths.

The function $\allocMethod{.}$ is defined as follows:
\begin{itemize}
\item \textbf{Input:} Annotated bytecode of a method \method, and memory
policies for methods that are called by \method;

\item \textbf{Output:} Upper bound of the memory allocated by \method;

\item \textbf{Body:} The first step is to compute the loop structure
of the method, then to compute an upper bound to the memory allocated
by each loop using its variant, and then to compute an upper bound to
the memory allocated along each execution path.
\end{itemize}



%A pseudo-code of the algorithm for inferring an upper bound for method
%allocations is given in Fig.~\ref{methodAlloc}.  Essentially, it finds
%the maximal memory that can be allocated in a method by exploring all
%its possible execution paths. in Fig.~\ref{methodAlloc} the auxiliary
%function $allocPath(\cdot)$ infers the allocations done by the set of
%execution paths ending with the same \return\ instruction.

%\begin{figure}[t]
%function $\allocMethod{.}$\\
%\textbf{Input:} Bytecode of a method $m$. \\
%\textbf{Output:} Upper bound of the memory allocated by $m$. \\
%\textbf{Body:}
%\begin{enumerate}
%   \item Detect all the loops in $m$;
%  \item For every loop $l$ determine $\loopSet{l}$, $\loopEntry{l}$ and $\loopEndsSet$;
%   \item Apply the function $\allocated{\cdot}$ to each instruction $i_k$, such that $i_k = \return$;
%  \item Take the maximum of the results given in the previous step:  $max_{i_k = \return } \allocated{i_k}$.
%\end{enumerate}
%\caption{\sc Inference algorithm}
%\label{methodAlloc}
%\end{figure}

%Inferring the memory allocated inside loops is done by the function $\allocLoopWithEnd{\cdot}{\cdot}$, which is invoked by $allocPath$ whenever the current instruction belongs to a loop. The specification of the function is shown in Fig. \ref{fig:loopPath} (where $P = max_{\instrAt{k} \in preds(\loopEntry{l'} ) - \loopEndsSet{\progLoop{l'}}}$).

%\begin{figure}[!ht]
%$\allocated{\instrAt{s}}$ = 
%$$ \left\{ \begin{array}{l}
%\allocIns{\instrAt{s} } \hspace*{1.8cm}  \mbox{if  $\instrAt{s}$  has  no  predecessors} \\
%            \allocLoop{\loopEntry{l}} \ + \\
%\ \ \ \ \            max_{\instrAt{k} \in preds(\instrAt{s} )-\loopEndsSet{\progLoop{l}}}( \allocated{\instrAt{k}} ) \\
%\hspace*{4cm}  \mbox{if  $\instrAt{s}\in \loopSet{\progLoop{l}}$} \\
%\allocIns{\instrAt{s}} \ + \ max_{\instrAt{k} \in preds(\instrAt{s} )}( \allocated{\instrAt{k}} ) \\
%\hspace*{4cm} \mbox{otherwise}
%\end{array}
%\right.
%$$
%\caption{\sc Definition of the function $\allocated{\instrAt{s}}$} 
%\label{fig:allocMethod}
%\end{figure}


%\begin{figure}[!ht]
%$\allocLoopWithEnd{\loopEntry{l}}{\instrAt{s}} = $
%$$ 
%\left\{\begin{array}{l}

% \allocIns{\loopEntry{l}}   \hspace*{1.8cm} \mbox{if $\instrAt{s} = \loopEntry{l}$} \\
%  \allocLoop{\loopEntry{l'}} \ + \\
%\ \ \ \ \      P(\allocLoopWithEnd{\loopEntry{l}}{\instrAt{k}}) \\
%\hspace*{2cm}  \mbox{if $\instrAt{s} \in  \loopSet{\progLoop{l'}} \ \land \ \progLoop{l'}$ is  nested in $\progLoop{l}$} \\

%     \allocIns{\instrAt{s}} \ + \\
%\ \ \ \ \     max_{\instrAt{k} \in preds(\instrAt{s} )}(\allocLoopWithEnd{\loopEntry{l}}{\instrAt{k}}) \\
% \hspace*{5cm} \mbox{otherwise}
%\end{array} \right.
%$$
% \caption{\sc Definition of the function $\allocLoopWithEnd{\loopEntry{l}}{\instrAt{s}}$}
%\label{fig:loopPath}
%\end{figure}

The annotation assistant currently synthesizes only simple memory
policies (i.e., whenever the memory consumption policy does not depend
on the input).  Furthermore, it does not deal with arrays,
subroutines, nor exceptions, and is restricted to loops with a unique
entry point. The latter restriction is not critical because it
accommodates code produced by non-optimizing compilers. However, a
pre-analysis could give us all the entry points of more general loops,
for instance by the algorithms given in \cite{CJPS05cmu}; our approach
may be thus applied straightforwardly. 


%\paragraph{Example}

%Let us consider the bytecode given in Fig. \ref{inf:src}, which is a
%simplified version of the bytecode corresponding to the source code
%given in the right of the figure. For simplicity of presentation, we
%do not show all the instructions (the result of the inference
%procedure is not affected). Method \method\ has two branching
%instructions, where two objects are created: one instance of class \texttt{A}
%and another of class \texttt{B}. Our inference algorithm gives that
%$\allocMethod{\method} =$ $\allocInstance{A} +$ $\allocMethod{A.init}
%+ \allocInstance{B} + \allocMethod{B.init}$.

%%Due to limitation on space, we do not explain the details of such inference, which is given in Fig. \ref{inf:ex} ($\instrAt{k}$ refers to the bytecode instruction at position $k$).

%\begin{figure}[!hbp]
%\begin{tabular}{lr}
%\begin{minipage}[t]{4.3cm}
%\begin{alltt}
%\begin{small}
%\srcCode{0 aload\_1} 
%\srcCode{1 ifnonnull 6 } 
%\srcCode{2 new <A>}
%... 
%\srcCode{4 invokespecial <A.<init>>} 
%\srcCode{6 aload\_2}
%\srcCode{7 ifnonnull 12}
%\srcCode{8 new <B>} 
%... 
%\srcCode{10 invokespecial <B.<init>>}
%...
%\srcCode{12 return}
%\end{small}
%\end{alltt}
%\end{minipage} &

%\begin{minipage}[t]{4cm}
%\begin{alltt}
%\small{
%public void 
% m (A a , B b )   \{
%  if (a == null) \{
%    a = new A(); \}
%  if (b == null) \{
%    b = new B(); \}\}
%}
%\end{alltt}
%\end{minipage}
%\end{tabular}
%\caption{\sc Example}
%\label{inf:src}
%\end{figure}

%%The procedure presented above terminates as an acyclic
%%representation of the control flow graph is used.
