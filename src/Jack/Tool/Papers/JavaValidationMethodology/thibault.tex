A step forward to lower the gap between specification and annotations is to describe properties at a more abstract level. This move towards high level specification is deeply facilitated by the frequent use of tools such as UML to specify software components.

\paragraph{The Automaton model}
A minimal set of constraint can be defined concerning the choice of the abstract model with the hope of keeping most of JML's expressiveness. First of all, the chosen model needs to be able to generate JML annotation for native methods. This requirement is mainly due to the fact that Java Card API is not provided together with JML specification. As a result, the model being chosen must be able to fit with both native and non-native methods. Secondly, the abstract model should consider methods' sequences as well as their pre and postconditions : this level of description surely is not only the most adequate to connect implementation to its high level specification, but also very crucial to smart verification. Furthermore, this choice is of great importance to connect the present work with the existing propagation tool. Finally, all possible improvement such as the support of invariants would be definitely a plus to keep as much as possible of the JML expressiveness.
For many reasons, the choice of finite state machine as a model to describe these properties is about to fulfill the previous expectations. FSM have been historically employed to model check software execution. Thereof, it is possible to assert that it is capable of specifying valid and invalid sequences in an intuitive way to people of the verification field. Nevertheless, the semantic of this automaton would require to be adapted to the expression of properties. It will be in particular compulsory to make an automaton stick to a property, by defining what could be the properties' states and what could make it switch from one to another. However, automatons make trivial the possibility to express pre and postcondition as they can be seen as methods' use cases, which is possible to describe through automatons' conditional evolution. As a consequence, the model used for this present study will be the FSM model.

\paragraph{Translating automatons to Properties}
Starting from the use of FSM, the initial problem was to correlate the semantic of states and transitions with a program execution to characterize properties. On one hand, the state should be characterized by a unique set of properties from which a given set of evolution is possible. Hence, a program state could be defined by a set of values attached to program variables, or possibly by method's status (i.e. active state would represent the method currently executed). On the other hand, the evolution between those states is expressible with the automaton model through guards, update, and message passing. Once again, several possibilities are acceptable. In the case where the active state represents the method currently executing, transitions might be use to express pre and postconditions through guards and even specify some entry/exit action in the update field. Nevertheless, it is as well acceptable to assert that program's evolution is conditioned by the sequent call and returns of method, so that transition should be attached to methods, and pre and postconditions could be specified as state invariants.

Even though several formalisms would be suitable, many of them could be discarded due to the limited nature of the properties expressible. First, correlating the active state with the method currently executing happened to be a bad idea. Such a representation imposes transition to define the pre and postcondition with the meaning that a method would neither be executed without respecting the preconditions of the incoming transitions nor terminate without fulfilling the postcondition specified as guard. For a unique transition plays both the roles of pre and post condition, choosing this modeling scheme would be perfectly appropriate to for expressing exact sequences of method call for which postcondition stands also as the precondition of the next method being called. This could be practical for describing completely known sequences which would be the case for protocols specification. However, considering the annotation of the {\tt beginTransaction} and {\tt commitTransaction} methods (which are involved in the transaction process of smart cards) demonstrates in the general case the need to partially specify sequences as methods have to be called in between those two methods. Moreover, this representation is not suitable to express properties dealing with recursion.

As a result, methods will be attached to transition through the definition of method's events. Because transitions are meant to express a logical condition for going from the active state to the next, the FSM model supposes that the transition to take no time for switching. Therefore, events on methods should be define so that to be expressible in transitions. Fortunately, these events appear obvious for they are exactly the one considered in specifying pre and postcondition. First is the method call, which will cause the method to be executed. Second and third are the normal termination invoke by a simple return statement or an exceptional termination triggered by the throw statement. As a consequence, the whole systems evolution will be conditioned by 3 types of event which could stand for any language supporting exceptions.

From what has been defined so far, transitions' role could finally be defined to carrying method's contract and the contribution of states to the model could be clarified. Yet, only two possibilities are offered to specify the contract: either states or transitions have to carry the pre and postconditions. The choice of state invariant to carry this information was discarded right away for the simple reason that a unique description would again stand for sequent post and preconditions. Guards appeared to be more likely to hold the contract for it would link methods' events to their triggering conditions. In other words, taking a transition would mean that the associated method call or return was done with respect to the pre or postcondition. This solution presents the huge advantage of keeping state free to specify invariants in accordance with the meaning a user would give to it. 


