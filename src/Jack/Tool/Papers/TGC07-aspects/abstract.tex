The Security Manager is an important system component in the Java
Virtual Machine (JVM).  It forbids access to some specific methods
depending on a given security policy.  It modifies both the behavior
of the Java system libraries and of the JVM, for it throws exceptions
in case of security policy violation.  The difficulty for the
verification of a environment containing it lies in the fact that its
use mixes library calls (easy to specify) as well as direct VM calls
(hard to take in account because they modifies directly the Java
semantic of the execution).  Moreover its use could be modelled solely
as a JVM level component, which would make the use of this component
more homogenous as well as the application of its security policies it
applies more trustable.

In this article we use a JVM-level modelisation of a SecurityManager
seen as an aspect component, and we define a verification framework to
verify statically programs that takes in account the presence of this
component, and more generally aspects.