To make the weaving lighter then the weaving done by {\tt ajc}, only
the specification of the advices will be weaved. Therefore we need a form
for the bytecode that can handle directly the weaving of the specification.
For this purpose we use a guarded command language. The language of 
choice is BoogiePL, because work has been done on the bytecode level for it.
There is a formal definition of Java bytecode translation to BoogiePL 
\cite{LehnerM07} as well as existing implementations of the translation
\cite{javatrans07,coqtrans06}.

\paragraph{description of the language}

\paragraph{bodyless BML model methods}

\begin{itemize} 
\item 
an assert of the require clause, followed by
\item 
an instruction to state that all the variable assignable have new
values: it can be represented by JML keyword {\tt fresh} which is
usually used for method specifications, but here could be used to
annotate the method, and finally
\item the assume of the method's postcondition.
\end{itemize}
Nevertheless, the exceptional cases being difficult to desugar, it is
interesting to have the model method calls in JML specification.

\begin{figure}
\begin{center}\begin{minipage}{4cm}
The program code:
\bcode
...\\
//@ m();\\
//@ assert(i == 1);\\
...
\ecode
Could be desugared to:
\bcode
...\\
//@ assert (i > 0);\\
//@ fresh(i);\\
//@ assume (i == 1);\\
//@ assert(i == 1);\\
...
\ecode
\end{minipage}\end{center}
\caption{A non exceptional model method call desugared}
\label{model_meth}
\end{figure}
