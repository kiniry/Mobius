% This is LLNCS.DEM the demonstration file of
% the LaTeX macro package from Springer-Verlag
% for Lecture Notes in Computer Science,
% version 2.3 for LaTeX2e
%
\documentclass[draft]{llncs}
%
\usepackage{makeidx}  % allows for indexgeneration
\usepackage[inline, nomargin]{fixme}
\usepackage{listings}
\usepackage{float}
\usepackage{macros}
%
\begin{document}

\newcommand{\rarrow}{$\rightarrow$}
\newcommand{\conj}{$\wedge$}
\newcommand{\disjonc}{$\vee$}
\newcommand{\s}{\,}
\newcommand{\btab}{\begin{tt}\begin{tabbing}}
\newcommand{\etab}{\end{tabbing}\end{tt}}
\newcommand{\bcode}{\begin{tt}\begin{small}\begin{tabbing}}
\newcommand{\ecode}{\end{tabbing}\end{small}\end{tt}}
%
\frontmatter          % for the preliminaries
%

\mainmatter              % start of the contributions
%
\title{Taking into account Java's Security Manager for static verification}
%
\titlerunning{Taking in account Java's Security Manager}  % abbreviated title (for running head)
%                                     also used for the TOC unless
%                                     \toctitle is used
%
\author{Julien Charles \and C\'esar Kunz}
%
\authorrunning{Julien Charles and C\'esar Kunz}   % abbreviated author list (for running head)
%
%%%% list of authors for the TOC (use if author list has to be modified)
\tocauthor{Julien Charles, C\'esar Kunz}
%
\institute{INRIA Sophia-Antipolis\\
\email{\{julien.charles, cesar.kunz\}@inria.fr}}

\maketitle              % typeset the title of the contribution

\begin{abstract}
\documentclass[]{llncs}

\title{FMCO Tutorial Proposal:\\
JACK~---~a tool for validation of security and behaviour of Java applications}

\author{Gilles Barthe\inst{1} \and
        Lilian Burdy\inst{1} \and
        Julien Charles\inst{1} \and
        Benjamin Gr\'egoire\inst{1} \and
        Marieke Huisman\inst{1} \and
        Jean-Louis Lanet\inst{2} \and
        Mariela Pavlova\inst{1} \and
        Antoine Requet\inst{2}}
\institute{INRIA Sophia Antipolis, France \and Gemplus, France}

\begin{document}
\maketitle

%\begin{abstract}
%JACK is a tool set for the validation of Java applications. The
%security requirements and behaviour of the application can be
%specified with JML. JACK computes the verification conditions that
%ensure correctness of the application. The verification conditions can
%be verified with different automatic and interactive provers.

%To support the whole development process of secure applications, JACK
%allows to do the verification both at source code and at bytecode
%level. Specifications of source code programs can be automatically
%compiled into bytecode specifications. 

%JACK also provides support for writing specifications, as it
%implements several algorithms to automatically generate annotations.
%\end{abstract}

\paragraph{Motivation}
Over the last years, the use of trusted personal devices, such as
mobile phones, PDAs and smart cards, has become more and more
widespread. As they are often used with security-sensitive
applications, they are an ideal target for attacks. Traditionally,
research has been focusing on avoiding hardware attacks, but the
emergence of new generation trusted personal devices that are
increasingly connected to networks and provide execution support for
complex programs has put forward the need to improve the quality of
software to avoid logical attacks.  Logical attacks are potentially
easier to launch than physical attacks (for example they do not
require physical access to the device, and are easier to replicate
from one device to the other), and may have a huge impact.  In
particular, a malicious attacker spreading over the network and
disconnecting or disrupting devices massively could have deep
consequences.

\paragraph{A tool for application validation}
We report here on the development of the tool JACK that can be used
throughout the application development process to increase confidence
in it. JACK allows to verify Java applications that are annotated with
JML (Java Modeling Language\footnote{See
\texttt{http://www.jmlspecs.org}.}). An advantage of using JML is that
there is wide range of tools and techniques available that use JML as
input language, \emph{i.e.}\ JML specifications can be used to improve
testing, they can be simulated, and in particular they can be
verified. We distinguish two kinds of verification: at runtime, using
\texttt{jmlc} or statically. Several tools provide
static verification of JML-annotated programs, adopting different
compromises between soundness and automation.  Among these tools, JACK
aims at achieving a high level of soundness, while keeping
automation. Therefore, JACK automatically generates proof obligations
that can be discharged both by automatic and by interactive theorem
provers. The automatic prover that is used is Simplify, that is also
used by several other tools for program verification. The interactive
theorem prover that is used by JACK is Coq. 

The tutorial will focus on the main characteristics of JACK that
distinguish it from other tools for the static verification of
JML-annotated Java programs is that it provides support for the whole
development process of secure applications:
\begin{itemize}
\item it is integrated with the widely-used IDE Eclipse\footnote{See
\texttt{http://www.eclipse.org}.};
\item it provides means to generate annotations; and
\item it allows verification of source code and bytecode level
programs, and source code specifications can be compiled into bytecode
specifications.
\end{itemize}
Moreover, JACK allows to verify complex functional behaviour
specifications, by providing advanced support for interactive
verification. 


\paragraph{Integration with Eclipse}
Because of the tight integration with Eclipse, the developer does not
have to change tools to validate the application. A special JACK view
has been implemented, that allow to inspect the generated proof
obligations in different views (in a Java-like syntax, or in the
language of the prover). Moreover, syntax colouring of the original
source code allows to see to which parts of the application and
specification the proof obligation relates.

\paragraph{Generation of annotations}
One of the drawbacks of using JML-like annotations for specification
is that it is labour-intensive and error-prone to write them, as it is
easy to forget some annotations. We provide two different annotation
generators: the first generator computes frame conditions and simple
preconditions that are sufficient to avoid runtime exceptions, the
second generator takes as input a security policy and generates and
propagates annotations in such a way that if the application respects
the annotations, then it also respects the security policy.

\paragraph{Verification of bytecode programs}
A possible approach to the secure loading of applications on trusted
personal devices is the use of proof-carrying code. In such a
framework, the applications come equipped with a specification and a
proof that allow the client to establish trust in the application. To
achieve this, we need the ability to reason about bytecode. JACK
provides a verification condition generator for BML, the Bytecode
Modeling Language, a bytecode variation of BML. In addition, it
provides a compiler from JML to BML, and we have shown that this
compiler basically preserves proof obligations (and thus source code
level proofs can be reused at bytecode level).

\paragraph{Support for interactive verification}
As interactive program verification is known to be labour-intensive,
several advanced Coq tactics have been developed that help with this
verification. To be able to write expressive specifications, a
\texttt{native} construct has been proposed for JML, that allows to
link JML constructs directly with the logic of the underlying
prover. This allows to develop the theory about these constructs
directly in the logic of the theorem prover, which makes verification
much simpler.

\paragraph{Literature}
Several papers have been published around JACK; here we list a few.

\nocite{BRL03:fme,Charles06,m+04:cardis,BurdyP06}

\bibliographystyle{plain}
\bibliography{../specification.bib,/net/home/gbarthe/bib/string,/net/home/gbarthe/bib/gilles,/net/home/gbarthe/bib/article,/net/home/gbarthe/bib/book,/net/home/gbarthe/bib/lncs,/net/home/gbarthe/bib/misc,/net/home/gbarthe/bib/proceedings,/net/home/gbarthe/bib/techrep,/net/home/gbarthe/bib/thesis,/net/home/gbarthe/bib/software,/net/home/gbarthe/bib/web}

\end{document}

\end{abstract}
%
\section{Introduction}
\section{Introduction}


Smart cards are trusted personal devices whose characteristics are
regulated by the ISO 7816 standard. As other trusted personal devices,
smartcards are designed to store and process confidential data, and
can act as tokens to provide users with a secure electronic
representation in a large network. They are widely deployed and used
in application areas such as mobile telecommunications, banking,
transportation, electronic identity, and digital rights management
(DRM). Further, they hold the promise to play a key role in the
e-society, especially as a means to guarantee users a personalized,
global, and secure access to applications and services.


The prominent role played by trusted personal devices in security
sensitive applications make them an ideal target for
attacks. Traditionally, the main concern with smartcards has been with
hardware attacks in which the attacker gains access to confidential
information or disturbs the functioning of the card through
observation (e.g. of power or electro-magnetic radiations) or invasion
(e.g. overriding sensors or attaching probes). This issue is studied in
Deliverable D8.1.

The trusted personal device remains a specific domain where post
issuance corrections are very expensive due to the deployment process
and the mass production. Furthermore, the emergence of new generation
trusted personal devices increasingly connected to networks and
providing execution support for complex programs and the prospect of
logical attacks has urged the trusted personal devices industry to
improve the quality of their software, as logical attacks are
potentially easier to launch than physical attacks (for example they
do not require physical access to the device, and are easier to
replicate from one device to the other), and may have a huge impact.
In particular, a malicious attacker spreading over the network and
disconnecting or disrupting devices massively could have deep consequences.

This deliverable reports on the development of methodologies and tools
that increase confidence in applications.  For concreteness, we focus 
on Java applications that can be executed on devices that embed Java
Virtual Machines (JVM) or their variants, in particular Java Card
Virtual Machines (JCVM). Java enabled devices are a natural choice for
formal methods because:
\begin{inparaenum}[i)]
\item they are widely deployed in the field;
\item they feature mechanisms that contribute to the security of the
platform and the applications that execute over it;
\item detailed informal specifications of the Java platform are publicly
available, and can be scrutinized.
\end{inparaenum}
However, it should be clear that the methods presented in this document
are relevant to other execution platforms for trusted personal devices.


\subsection{Security issues}
While the focus of the deliverable is on application validation,
security is a holistic property of a system, and formal techniques
must therefore be employed at different levels to provide strong
guarantees about the security of a TPD and its applications.
Essentially, the levels are: the hardware, platform, the libraries,
and the applications.

The need to consider security at those levels is illustrated for
example by the case study described on Page 55 in Deliverable D8.2,
which is concerned with secure platforms. The development of secure
API is discussed in Deliverables D7.1 and D7.2. As previously mentioned,
hardware security is discussed in Deliverable D8.1.


\paragraph*{Platform} The TPD security architecture guarantees that
downloaded applications are innocuous and comply with some basic
policies related to typing, initialization or access control. Such
basic policies are the cornerstones upon which the overall security of
the smartcard will rely. Therefore it is important to verify that the
security architecture does enforce these basic policies as
intended. Thus, an important application of formal methods to TPD
security is platform verification, which aims at providing an abstract
model of the Java platform and security architecture, and at proving
that the security functions play their expected role.

\paragraph*{Libraries} However, it is not sufficient to show that
security functions are correctly designed. In particular, one also has
to ensure that other components of the infrastructure, in particular
API, are correctly designed and implemented. For the purpose of this
deliverable, where the focus is on Java based TPD, the Java API and
the Global Platform API constitute two prominent components of the
infrastructure whose correct design is central to security. 


\paragraph*{Applications} 
Platform and libraries verification is a fundamental step towards
guaranteeing the security of smartcards, and a prerequisite for Common
Criteria evaluations at the highest levels. Nevertheless, the
guarantees offered by the Java security architecture are limited, and
further verifications must be performed to verify that applications
make a legitimate use of the infrastructure, and do not attempt any
hostile action.

Thus, application validation is another important application of
formal methods to TPD security. To date, testing campaigns remain the
primary means to ensure the quality of applications. However, testing
campaigns are expensive and only provide partial guarantees with
regard to the reliability of software. Therefore, it is important to
develop other advanced techniques for applet validation.

There are many facets to applet validation, each with its own
objectives and techniques:
\begin{itemize}
\item one can enhance existing security architectures to enforce
security properties not addressed by current architectures, in
particular confidentiality and availability.  Verification can be
performed by enhanced bytecode verification mechanisms;


\item one can abandon the realm of type systems and its associated
benefits and choose develop logical methods for specifying and
verifying either automatically or efficiently a specific class of
security properties. Verification can be performed by (possibly
efficient and hence incomplete) logic-based proof inference
mechanisms;




\item one can exploit the expressive power of logical methods to
require that applications, or at least sensitive fragments of
applications, are subjected to functional verification, i.e. to
verifications that establish their correctness in terms of
functionality as well as security.
\end{itemize}



\subsection{Logical verification of security properties using JML}
In order to provide precise analyzes with a limited overhead, we
advocate an integrated approach where validation techniques of
increasing strength are used, starting from automated techniques such
as testing and moving towards formal validation using a combination of
automated and interactive tools. In addition, we aim at overcoming the
difficulty of introducing formal techniques in industrial processes by
providing notations and tools hiding the mathematical formalisms and
by integrating formal techniques into classical developers environment
so as to allow users to benefit from formal techniques without having
to learn new formalisms and to become experts.

All the tools and results presented in this document were developed
with this goal in mind, notably the choice of JML as assertion
language and the development of JACK and its associated feature.
Using those techniques, Java developers should be able to validate
their code, or at least to get a good assurance on its correctness.


\subsubsection{JML}
JML~\cite{Leavens-Baker-Ruby99b,Leavens-Baker-Ruby03}, the ``Java
Modeling Language'', is a behavioral interface specification language
for Java; that is, it specifies both the behavior and the syntactic
interface of Java code.  The syntactic interface of a Java class or
interface consists of its method signatures, the names and types of
its fields, etc.  This is what is commonly meant by an application
programming interface (API).  The behavior of such an API can be
precisely documented in JML annotations; these describe the intended
way that programmers should use the API.  In terms of behavior, JML
can detail, for example, the preconditions and postconditions for
methods as well as class invariants. These specifications are given as
annotations of the Java source file. More precisely, they are included
as special Java comments, either after the symbols \lstinline!//@! or
enclosed between \lstinline!/*@! and
\lstinline[basicstyle=\normalfont\ttfamily\small\sl]!@*/!. For example,
the general schema for the annotation of a method is the following:
\begin{lstlisting}
/*@ behavior
  @   requires <precondition>;
  @   ensures <postcondition if no exception raised>;
  @   signals(E) <postcondition when exception E raised>;
  @   assignable <modified fields and variables>;
  @*/
\end{lstlisting}
where \lstinline!requires! specifies the conditions on variables, fields
and method parameters at the beginning of the method call so that the
conditions after \lstinline!ensures! hold at the end of the method
call and the conditions after \lstinline!signals(E)! hold if an
exception is raised and not caught inside the analyzed method.  The
underlying model is a an extension of Hoare-Floyd logic: if the
precondition holds at the beginning of the method call, then
postconditions (with and without exceptions) will hold after the
call. The \lstinline!assignable! clause specifies side-effect affected
variables and is used during the weakest precondition calculus for
method invocations.

An important goal for the design of JML is that it should be easily
understandable by Java programmers. This is achieved by staying as
close as possible to Java syntax and semantics.  Another important
design goal is that JML {\em not} impose any particular design method
on users; instead, JML should be able to document Java programs
designed in any manner \cite{Leavens-Baker-Ruby03}.

JML uses Java's expression syntax in assertions,
thus JML's notation is easy for programmers to learn.  
Because JML supports quantifiers such as
\verb_\forall_ and \verb_\exists_, and because JML allows ``model''
(i.e., specification-only) fields and methods, specifications can
easily be made precise and complete.
JML assertions are written as special
annotation comments in Java code,
so that they are ignored by Java compilers but can be used
by tools that support JML\@.  Within annotation comments JML extends the
Java syntax with several keywords.  It also extends Java's expression syntax with several
operators.
The central ingredients of a JML specification are preconditions
(given in {\tt requires} clauses), postconditions (given in {\tt
  ensures} clauses), and (class and interface) invariants.  These are
all expressed as boolean expressions in JML's extension to Java's
expression syntax.
In addition to ``normal'' postconditions, the language also supports
``exceptional'' postconditions, specified in {\tt signals} clauses.
These can be used to specify what must be true when a method throws an
exception. 

\paragraph*{Styles of specification}
Due to its expressiveness and versatility, the JML specification
language supports several styles of specifications; the choice of one
style of specification over the others depends on the purpose of the
verification effort. In a nutshell, one can either opt for lightweight
specifications in which one introduces enough annotations to reason
about some specific safety property, such as the absence of
exceptions, or heavyweight specifications where functional behavior is
considered. There is of course a great liberty in how \lq\lq
lightweight\rq\rq\ or \lq\lq heavyweight\rq\rq\ a specification should
be, and different styles can be used in different parts of an
application.

In addition, one may opt for defensive specifications, in which methods
are annotated with preconditions that prevent exceptions to occur, or
offensive specifications, which use appropriate clauses to specify 
exceptional postconditions.

\subsubsection{Verification techniques and tools}

JML specifications correctness can be verified either during runtime
or statically~\cite{BurdyCCEKLLP03}. To be verified during runtime, the
source code must have been compiled using \texttt{jmlc}, which is a
enhanced Java compiler for JML annotated code. This compiler adds to
the generated program assertions checking instructions corresponding
to the JML specifications of the program: preconditions, postconditions 
and loop or class invariants. An exception is raised during the execution 
if a JML condition fails. The JML runtime assertion checker can be used
for unit testing~\cite{CL02:ecoop}.


For the static verification of Java programs, several tools are
available using (variations of) JML as specification language. These
tools adopt different compromises between soundness and automation,
and thus it is useful to use them in combination, starting from
automatic but unsound tools, and pursuing with sound but interactive
tools.  Among these tools, ESC/Java2~\cite{CK04:cassis} offers the higher
level of automation as it does not require any user interaction and
relies on the Simplify automatic prover. It is particularly useful for
checking null pointers or array bounds limits; however it is unsound
and incomplete.

In order to further increase the level of reliability of applications,
we propose a methodology based on static verification using
JACK~\cite{BRL-JACK}, a tool that generates proof obligations that
can be discharged using proof assistants or automatic provers.


\subsection{Main contributions}
The work reported in this deliverable builds upon the JACK tool,
that was initially developed within Gemplus. The tool development
was transferred to INRIA at the beginning of the project, with
the objective to improve and increase its functionalities so as
to address the needs of INSPIRED. The main contributions of the
work carried within INSPIRED are:
\begin{itemize}
\item support for verifying high-level security properties
\item support for verifying bytecode programs
\item validation of the methodology for estimating resource
usage and optimizing code for low-footprint.
\end{itemize}
In order to carry the above tasks and evaluations, it has also
been necessary to make many improvements to the tool itself.
It is not in the scope of the present document to describe these 
improvements.


\subsection{Contents of the deliverable}
This document is organized as follows, the next chapter introduce the
assertion language JML, chapter 2 describes the JACK tool with its
extension feature, chapter 3 presents some evaluations done with the
tools and the last chapter concludes.




\section{The verification framework}
The verification framework we are building in this paper 
is an adaptation of static 
verification based on weakest precondition calculus for aspects. 
Here we will consider 
the SecurityManager as an invasive aspect, 
so we must take into the weaving process. 

The verification is made of several steps:
\begin{enumerate}
\item the program and it's aspects are annotated using Pipa,
\item the behavioural specifications are desugared
\item the specifications of the aspects are compiled into model methods
\item the model methods are weaved to the program
\item a weakest precondition calculus is made on the transformed program, and verification
conditions are generated, and finally,
\item the verification conditions can be discharged automatically or interactively.
\end{enumerate}
This verification procedure is indeed made of 3 main parts: the annotation
part (step 1 and 2), the compilation and program transformation part (step 3 and 4)
and the verification part (steps 5 and 6).
The crucial part here being the second, since the third can be done with any verification tool
supporting static verification with Java and JML, and part 1 is only simple desugaring, and has been
extensively described in \cite{RaghavanL00}.


\subsection{Specifications}
The weakest precondition calculus we are targetting relies on program and aspects specifications that are
precise enough to make the verification conditions generated provable.
The annotation language we use is like Pipa \cite{ZhaoR03} (the extension
of JML for Aspects), but in a limited form. The annotation language is a JML level 0 type language 
\cite{Leavens-etal07}, to which
we remove the universe types, but to which we add model methods as well as purity, and some constructs
to handle aspects found in Pipa. The other restriction we add is that we don't treat behavioural specifications
because they can be desugared \cite{RaghavanL00}.

\subsubsection{JML} JML is a behavioural specification language for Java. It is made of numerous keyword: you 
can express methods pre and post conditions with it (the keywords {\tt requires} and {\tt ensures}), exceptional
postcondition ({\tt exsures}). There
are notions of frame conditions (the {\tt assignable} clauses). One can also express assertions ({\tt assert} 
construct), and loop invariants.

\paragraph{Invasive model methods} 
Model methods are an interesting construct of JML: these are specification
methods that can be with or without side-effects, their effects being determined by their specifications.
They can have a body, but in our framework we are interested by bodyless model methods. They are used to 
assume an effect over the program, and so state that if their requires are satisfied, the given effect (given by
the ensure clause) will be satisfied.
In this paper we won't talk about pure model methods which could be use in specifications. Here we rather talk
about invasive ones: just like invasive aspects they are meant to state that the program has been modified 
non explicitly.
\begin{figure}
\begin{center}\begin{minipage}{3cm}
\bcode
ghost int i;\\
...\\
requires i > 0;\\
assignable i;\\
ensures i == 1;\\
model void m();
\ecode
\end{minipage}\end{center}
\caption{A model method definition}
\label{model_meth_def}
\end{figure}


\paragraph{Model methods call}
To be able to use model methods with the full expressivity desired, we add a new annotation to JML, the simple 
annotation method call. Its syntax would be simply the model method call in the middle of annotations as shown 
on figure \ref{model_meth}. This construct is natural from a JML point of view because for most of the non 
exceptional cases  it could be desugared easily into 
more primitive JML. Hence a normal model call could be desugared into 3 construct (it is similar to what is done
in guarded commands \cite{BarnettL05}):
\begin{itemize} 
\item an assert of the require clause, followed by 
\item an  instruction to state that all the variable assignable have new values: it can be represented by JML 
keyword {\tt fresh} which is usually used for method specifications, but here could be used to annotate the method, 
and finally 
\item the assume of the method's postcondition.
\end{itemize}
Nevertheless, the exceptional cases being difficult to desugar, it is interesting to have the model method calls in
JML specification.

\begin{figure}
\begin{center}\begin{minipage}{4cm}
The program code:
\bcode
...\\
//@ m();\\
//@ assert(i == 1);\\
...
\ecode
Could be desugared to:
\bcode
...\\
//@ assert (i > 0);\\
//@ fresh(i);\\
//@ assume (i == 1);\\
//@ assert(i == 1);\\
...
\ecode
\end{minipage}\end{center}
\caption{A non exceptional model method call desugared}
\label{model_meth}
\end{figure}
\subsubsection{Pipa}
Pipa is a behavioural specification language based on JML. As was previously stated, we don't treat behaviours 
which was it seems the main feature the authors of Pipa were trying to emphasize in their paper.
The specification of an advice is similar to the one of a method call. It has a pre and post condition,
an exceptional postcondition and a frame condition. The specification are not different for JML's method's 
specification, being applied on advices.
Pipa mainly adds two construct plus a specific semantic we are interested in:
the specific constructs for around advices, and their semantic of the ensure clause.

\paragraph{Around advice specifications} The difficulty of specification of around advices is due to the presence
of the proceed construct. It has to be stated in the method specification and in JML there are no keywords to
represent it. To represent the proceed in specification, Pipa uses the predicate {\tt proceed}
as well as the {\tt then} construct as what was stated in Clifton and Leavens paper~\cite{clifton02spectators}.
The specification of an around advice is divided in two the part before the proceeds (the part before the 
execution of the target instruction) and the part after the proceeds. These two parts have both pre and post 
conditions as well as a frame condition. There is also for the first part a new keyword proceeds which take a 
boolean condition which is an invariant for an around advice which determined if the instruction must be executed.

\begin{figure}
\begin{center}\begin{minipage}{3cm}\bcode
int f;\\
...\\
/*\=@ \ \ \=requires f > 0;\+ \\
@ \>ensures f == 1;\\
@ \>proceeds true;\\
@ then\\
@ \> requires f == 2;\\
@ \> ensures f == 3;\\
@*/\-\\
void \= around() : call (void m()) \{\\
\>...\\
\} \ecode\end{minipage}\end{center}
\caption{Around annotations}
\label{arround_annot}
\end{figure}
\paragraph{Semantic of the ensure clause}
The other interesting element in Pipa is the semantic of the ensure clause which slightly differs from the one in
JML. It enables to express properties about the caller variables which were passed to the advice while 
weaving. For instance if you have a variable {\tt x} inside a method, 
and the advice is weaved on a method call that uses {\tt x}, you can express properties over {\tt x} in the 
postcondition. In normal JML it makes no sense to express a property like that, and if {\tt x} is mentionned
in the postcondition, it would have been the old value of {\tt x}. It is especially useful to express real 
invasive aspects. In our framework we will use this semantic for {\it any} postcondition, {\it e.g.} 
for normal method calls as well.

\subsection{Compilation to invasive model methods}
The main idea is to compile aspects to their effects, which would be modeled by methods that are defined
solely by their specifications. 
They are specifically
\subsubsection{Translation of the specifications}
The annotations are translated directly to the model methods. The method has the same signature
that of the aspect, and the specifications are copied entirely to the model method. It works smoothly
for most of the case except for the around advice which is split into 2 model methods: the one corresponding
to the before part, with added the proceed instruction as conjuncted in the ensures.
\subsubsection{Translation of pointcuts}
The pointcuts are not directly translated to the model methods, though the conditions over the pointcut undecidable
at compile time are added to the require clause.

\subsection{Weaving of the models}

First all the pointcut are abstracted as a syntactical pointcut. Then they are ordered using the compilation
order.
They are ordered in a list made of couples (instr, (model\_list)) where model\_list is the list of model methods 
that correspond to a specific syntactical match.

%
\begin{figure}
\bcode
weave \=(instr; Linstr, (instr, (model :: model\_list)) :: Lmatch) -> \\
\>model:: weave (instr; Linstr, (instr, (model\_list)) :: Lmatch);\\
weave \=(instr; Linstr, (instr, (instr :: model\_list)) :: Lmatch) -> \\
\>instr:: weave (instr; Linstr, (instr, (model\_list)) :: Lmatch);\\
weave (instr:: Linstr, (instr', (model :: model\_list)) :: Lmatch) -> \\
\>instr <> instr',  weave (instr; Linstr, Lmatch);\\
weave (instr:: Linstr, (instr, (nil)) :: Lmatch) -> \\
\>weave (Linstr, Lmatch);\\
weave (instr:: Linstr, nil) -> \\
\>instr :: weave (Linstr, Lmatch).
\ecode
\caption{The algorithm to weave method}
\label{weaving_algo}
\end{figure}
Then for each program point that matches

\subsection{The weakest precondition calculus}

In our framework aspects are considered as methods calls that are
transferred the whole control flow, and are permitted to modify local
variables in the caller methods. Therefore when aspects are weaved
there wp is just like the one of a method call.  For instance if an
advice is weaved before an instruction i, the effective execution of
the program will be of

The aspect are compiled into a list containing the.  The predicate
transformers associated with an advice is similar to the weakest
precondition clause toward an invasive method call i.e. a method call
that would be able to modify local variables as well as the other
program variables.

One of the amusing fact about this approach is that the weakest
precondition calculus is parametized by a new semantic which is given
by the aspects.

General form of the final wp rule:
\bcode

\\
\ecode
where befores, arounds and afters are the compiled predicate

%
\section{Verifying a security manager}
\subsection{Intro/Definition/related work: sys component verif}
\subsection{An aspect modelisation}
\subsection{An instanciation example}
\section{Conclusion}

\section{Achievements}
% done. summary
We have  presented an infrastructure for verification of Java bytecode programs   which allows to reason about potentially
sophisticated  functional and security properties and
which benefits from verification over Java source programs. We have also 
introduced the bytecode specification language BML tailored to Java bytecode, a compiler
from the Java source specification language JML to BML and a verification 
condition generator for Java bytecode programs. 
We have shown that the verification procedure is correct w.r.t. a big step  operational semantics of Java bytecode programs. 
Moreover, we have
proven that the verification procedure for Java like programs
and Java like bytecode are syntactically equivalent (modulo names and types). 
%This scheme is actually part of the PCC architecture of the
%European project Mobius\footnote{the site name} which aims to resolve the problems
%of mobile and ubicuous computing via PCC. 
We have developed a prototype of a verification condition generator based on the weakest precondition calculus presented in this thesis, as well 
as a compiler from the corresponding subset of JML to BML.
These two components have been integrated in the JACK \cite{BRL-JACK} verification framework 
developed and supported by our research team Everest at INRIA Sophia Antipolis which has been initially designed for
 the verification of Java source programs annotated with JML specification.

We would like to give a brief description of the implementation of the verification condition generator.
 The extension of the tool to bytecode programs which we added also interfaces these theorem provers. The bytecode 
verification condition generator works as follows. For the verification of a class file containing BML specification, it will generate verification conditions for every
 method of this class including the constructors. For generating the verification conditions concerning a method implementation, first the control flow
 graph corresponding to the bytecode instruction is built. The latter is transformed into an acyclic control flow graph where the backedges are 
removed.
 Then the verification procedure proceeds by generating over every execution path in the control flow graph its corresponding verification conditions. 
For every path which terminates by throwing an uncaught exception, the postcondition is the specified exceptional postcondition for this case. For the paths which terminate normally, 
the normal postcondition is taken. For every path which terminates with an instruction which is dominated by a loop entry and whose direct successor is the same loop entry, the postcondition 
is the corresponding loop invariant. The bytecode verification in Jack uses the intermediate language for the verification conditions and thus, bytecode verification conditions 
 can be translated to several different theorem provers - Simplify \cite{Simpl05DNS} which is an automatic decision procedure, 
the Atelier B and the Coq interactive theorem prover assistants. 

The bytecode verification condition generator benefits also from the original user friendly interface of the JACK tool.  In particular, 
the user can see the verification conditions in his favorite language - Java, Simplify, Coq or B. The lemmas are classified 
to what part of the annotation they refer to, as for instance, a lemma which refers to the establishment of the postcondition, or the preservation of the loop invariant.
The hypothesis in the lemma also hold the index of the instruction from which they originate. 
We have used the prototype of the bytecode verification condition generator for the case studies presented in Chapter \ref{applications:optimComp}.

% JACK (short for Java Applet Correctness Kit) is designed as a plugin for the Java interface development
% environment eclipse. 
%% It was originally tailored to the verification of Java source programs 
%w.r.t. their JML specifications. The tool has an intermediate proof obligation language which allows to extend it easily to interface more 
% theorem provers. Thus, the tool interfaces several theorem provers - Simplify \cite{Simpl05DNS} which is an automatic decision procedure, 
%%the Atelier B and the Coq interactive
%theorem prover assistant. 

\section{Future work}
In the following, we identify the directions for extending the work presented in this thesis

\subsection{Language coverage of the verification condition generator}
The bytecode verification condition generator works only for the sequential fragment of Java. But realistic applications 
rely often on multi - threading which is difficult to verify against a functional specifications or security policies.
One of the important aspects of the correctness of multi - threaded programs is the absence of deadlocks, 
and race conditions. Such properties can be ensured  by type systems \cite{FA99TSL,flanagan00typebased} or static verification based on program logic \cite{FLL02ESC}.  
The absence of deadlock and race conditions is a first step in the verification of the functional correctness of multi threaded programs. In order to build a full 
verification scheme for checking functional correctness more has to be done.
The earliest work for  verification of  parallel programs is  the Owicki and Gries approach   
\cite{nipkow99owickigries}  and the rely - guarantee approach. However, 
the first approach is not modular and requires a large amount of verification conditions while for the second, the annotation procedure can not be automatised.

% Such techniques for reasoning over the correctness of parallel programs  exist.
% One of the first logic - based verification techniques for parallel programs is due to Owicki and Gries 
%\cite{nipkow99owickigries}  in which every point of parallel interference is annotated and then the verification consists in establishing that
% all the possible inter leavings of all the threads respect the annotation. This technique is on one hand not modular as the verification process 
%needs the implementation of every program component and on other hand the number of verification conditions may be very big.
% Another approach is the rely guarantee technique which uses a Hoare style verification conditions \cite{nieto03relyguarantee}.
%There, the program points of interference are annotated not only with the predicate that must hold
%at the point but also with rely and guarantee  conditions which express what conditions the program guarantees to the other threads and what 
%the program requires from the other threads. This technique although tempting because of its modularity and the smaller number of verification conditions is difficult to apply
%as for guessing the rely and guarantee conditions requires an in - depth understanding of the program to be verified.  
Extending our verification scheme for bytecode will certainly be based on a more recent work  where one of the basic concerns is to establish method atomicity  \cite{TES03CF}. 
The notion of a statement atomicity states  that however a statement is interleaved with other parallel programs, the result of its execution will not change.
The atomicity can be  detected via static checking \cite{TES03CF} using type systems. Thus, the program verification process is separated in two parts
- first checking for program atomicity  \cite{TES03CF} are done  
and then verifying the functional correctness using  methodologies for sequential programs as Hoare style reasoning. 
In this last approach in the case of Java, the basic concern is to establish the atomicity of method bodies, i.e. method 
execution does not depend on the possible interleaving with threads.
Recently, E.Rodriguez and al. in \cite{RodriguezDFHLR05} proposed an extension for JML for multi threaded
 programs. Their proposal introduces  new specification keywords which allow to express that a variable is locked or
 that a method is atomic.% Giving the semantics of these keywords is still an ongoing work but we consider that the meaning of these specification constructs does not differ on source and bytecode. 
    
 



\subsection{Property coverage for the specification language}
Another direction which may be pursued as a future work of the thesis  is the extension of the expressiveness of the specification language BML. 
So far, BML supports method contracts - method pre and post  conditions, frame conditions, intermediate annotations as for instance
loop invariants, class specifications as well as special specification operators.
These are very useful aspects which allow for dealing with complex properties and 
gives a semantics on bytecode level  to a relatively small subset of the 
high specification language JML which corresponds to JML Level 0 \footnote{ http://www.cs.iastate.edu/~leavens/JML/jmlrefman/jmlrefman\_2.html\#SEC19}. 
 But it is certainly of interest to support more features of JML in BML
as this will turn the latter language richer. However, the meaning  of JML constructs 
(at least from our experience up to  now) is the same as the meaning of their corresponding part in BML.  

 An important example is the  JML construct for pure methods which has been  identified as  a challenge in the position paper \cite{LeavensLeinoMueller06}. 
 These methods does not modify the program state and thus, pure methods can be used in specifications 
 (only side effect free  expressions may occur in expressions).
 This gives more expressive  specifications as with them, for instance, specification can talk about the result of method invocation or use pure methods
 as a predicate relating their  initial and final state. 
 Formalizing and establishing the meaning of pure methods is difficult and a literature exists for this problem \cite{DarvasMueller06}.
 As we said above, the treatment of pure methods is the same on source and bytecode.

Also, support for specification constructions for alias control is certainly useful  especially because it allows for a modular verification 
of class invariants and frame conditions.
The alias control is guaranteed through ownership type systems which check that only an owner of a reference can modify its contents.
 This can considerably improve the current implementation for the verification of object invariants  \cite{DietlMueller05}.
In particular, our way of proving object invariants is non modular - at every method call the invariants of all visible \todo{say what does it mean visibility}
objects must be valid and they are assumed to hold when the call is terminated; similarly, when a method body is verified in its precondition the invariants of all visible
objects are assumed to hold and at the end of the method body all these invariants must be established. 
In practice, it is very difficult to verify that all the invariants for the all visible objects in a method  hold.
In order to keep the number of the verification conditions reasonable, we check the invariants only for the current object this and the 
objects received as parameters which is not sound.

 
\subsection{Preservation of verification conditions}

So far, we have shown that non-optimizing Java compilation
 preserves the  form of the verification conditions on source and
 bytecode.  We identify two basic directions for future work:
\begin{description}
 \item[Source and non optimized bytecode verification conditions equivalent modulo] % implement the compiler from Java source pogs to bytecode pogs
We have experimented with the verification conditions on source and
 bytecode in JACK and saw that in practice they are almost equivalent
 syntactically. From one part, there are the difference in the types 
 supported on bytecode and source level. For instance, the JVM does not
 provide support for boolean type values which are basically encoded as
 integer values. The same is true for byte and short values.  Another
 difference is the identifiers for variables and fields. For instance, in Java
 names for fields, method local variables and parameters are their identifiers which are given by the
 program developer. On bytecode method local variables and parameters are encoded as elements of the
 method register table and field names are encoded as numbers of the constant
 pool table of the class. A  simple but useful extension to the prototype for
 bytecode verification is a compiler from source proof obligations to bytecode proof obligations
 which overcomes those differences. This can be considered also as a step
 towards the  building a PCC architecture where the certificate generation benefits from
 the source level verification and thus allows for treating sophisticated
 security policies.

\item[Relation between verification conditions on Java source and optimized Java bytecode]
 The equivalence  between verification conditions on source and the corresponding non optimized bytecode is important as it
 allows that bytecode programs  benefit from source verification. In particular, it makes feasible Proof Carrying Code
 for sophisticated client requirements.
 However, a step further in this direction is to investigate the 
 relation between source programs and their bytecode counterpart produced by an optimizing compiler.
 This is interesting for the following reasons.
 It is a fact that interpretation of bytecode on the JVM is slower than execution by its corresponding assembly code. 
 In order to speed up the execution time for a Java bytecode program, one might use 
 a just-in-time compilation which  translates on the fly the bytecode into the machine specific language. However, JIT compilation can potentially slow
 the execution exactly because it does compilation on the fly.  Another possibility is to perform 
 optimizations on the bytecode. Currently, most of the  Java compilers do not support much optimizations.
 However, there do already exist Java optimizing compilers, for instance the Soot optimization framework\footnote{http://www.sable.mcgill.ca/soot/} 
 and most probably the number of the Java optimizing compilers will increase with the evolution of the Java language.
 A first step in the latter direction is the work of C. Kunz et al.\cite{BGKRsas06} who give an algorithm for translating 
 certificates and annotations over a non optimized program into a certificate  and annotation for its optimized version.
 Their work addresses  optimizations like constant propagation, loop induction, dead register elimination etc. 
\end{description}
\subsection{Towards a PCC architecture}

The bytecode verification condition generator and the BML compiler is the first step towards a PCC framework. 
The missing  part is  the certificate format which comes along with the bytecode and which  is the evidence for 
that the bytecode respects the client requirements. Defining an encoding of the certificate should take into account several factors:
\begin{itemize} 
  \item certificate size must be reasonably small. This is important, for instance,  if the certified program comes over a network with a limitted bandwith
  \item certificates must be easily checked. This means that the certificate checker is  small and simple.
	       Of course, the code consumer might not want to spend all of its computation 
	      resources for checking that the certificate guarantees the program conformence to its policies.     
\end{itemize}

Note that the certificate size and its checking complexity are dual: the bigger the certificate is more manageable is the checking process and viceversa. 
The problem becomes even more difficult if the certificate must be checked on the device because of the computational and space constraints.
 


% towards.PCC
% For building a PCC framework from the components cited above 
% % there is still missing the proof certificate, the decision procedure
% that will be used by the producer for the certificate generation and the type checker used by the code
% client for checking the certificate. Important problems in this direction are
% \begin{itemize}
%  \item light weight verification condition generators. In particular, we refer 
%        to verification condition generation techniques which are simple and do not need
%	much computational resources. Because a verification condition generator always
%	form part of the trusted computing base on the client side, building such verification 
%	condition generators is important for on - device checking which rely on limitted computational 
%	resources  
  
%   \item generation of certificates. This is important for several reasons.
%         The certificate may certainly  arrive via the network and should not corrupt the performance 
%  
% 
% %  \item efficient type checker on the client site. This is in particular important 
%         if the device is with limitted resources where a complex certificate checking procedure
%         may corrupt the performance of the device
%        
%     
% \end{itemize}


 %To do this,  it is still missing the proof
%certificate, the decision procedure used by the code producer 
%for building the certificate  as well as the type checker used by the code
%client for checking the certificate. 

% to do. type systems
Another perspective in this direction is how   to encode type systems into the bytecode logic. 
Type systems provide a high level of automation. 
Their encoding in the logic can be useful as the certificate can be generated
 automatically and thus, avoids the user interaction. However, type systems are  conservative in the sense 
that they tend to reject a large amount of correct programs. A possible solution to this problem are hybrid certificates which combine both type systems and program 
logic. In this approach, the unknown code comes supplied with a  derivation in the logic generated potentially with the help of user interaction 
for the parts of the  code which can not be inferred by the type system.   The client side then applies a type inference procedure over  the
 unknown code and once it gets to the place in the parts of the code where the 
type inference does not work but for which there is a derivation in the certificate, he will type check that derivation.   
This is actually an approach which will be adopted in the Mobius project. 


The objective of this thesis was to 
give the basis for the a bytecode verification framework and to show that it is feasible. A further objective, pursued in the European project
 Mobius (short for Ubiquity, Mobility and Security) 
is to build basis for guaranteeing security and trust in program application in the presence of mobile and ubicuous computing. We hope that we have convinced
the reader for the importance of such techniques and in particular of the evolution from source verification to
 low level verification  and the necessity of an interactive verification process for building evidence for the security of unknown applications. 

%
% ---- Bibliography ----
%


\bibliographystyle{plain}
\bibliography{bibli,aspects}
%




\end{document}
