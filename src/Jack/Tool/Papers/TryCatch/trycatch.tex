\documentclass[]{llncs}
\title{Run-time Verification can miss Errors\\{\large Why Finally
Clauses can be Dangerous}}
\author{M. Huisman}
\institute{INRIA Sophia Antipolis-Mediterran\'ee, France}

\def \bsl       {\symbol{92}}
\def \unsc      {\symbol{95}}

\pagestyle{plain}
\begin{document}
\maketitle

\begin{abstract}
This paper discusses how a subtle interaction between programming
language semantics and run-time checking can cause a run-time checker
to fail to report errors. We focus on programs written in Java,
annotated with JML. We give some simple examples of annotation
violations that are not reported by the run-time checker, because the
errors are caught within the program text; even without any explicit
reference to them. We explain this behaviour, based on the official
Java Language Specification. We also discuss what can be consequences
of this problem for security, and different ways in which it can be
avoided.

\end{abstract}

\section{Introduction}

It is common folklore that run-time verification has the potential to
find all errors. Provided that one has a correct specification of the
intended program behaviour, if run-time verification does not signal
an error in the execution, the execution did not violate the
specification. Thus, if one would be able to use run-time verification
on all possible executions of a program, this would give a 100 \%
guarantee that the program respects its specification. And only
because in general it is impossible to consider all possible execution
paths of a program, it makes sense to apply static verification and
other static analysis methods.

However, this point of view is actually too optimistic: because of a
subtle interplay between the programming language semantics, and the
implementation of the run-time verifier, it is possible to miss
specification violations - even though they can be detected using
static verification.

This paper illustrates how such an interplay can happen, and cause
run-time checking to miss errors. We concentrate on applications
written in Java, annotated with JML~\cite{LeavensBR99}. JML is an
expressive annotation language for Java, that allows one to encode
many useful security properties. It is supported by a wide variety of
tools, for run-time and static verification, annotation generation
\emph{etc.}, see~\cite{BurdyCCEKLLP03}. The standard JML run-time
checker tool compiles the specification into run-time checks (using
\texttt{jmlc}). Then when the actual run-time checker \texttt{jmlrac}
is executed, appropriate errors are thrown if the annotations are
violated~\cite{Cheon03}. However, since the way to signal
specification violations is encoded in the programming language
itself, the JML errors are treated as any other kind of exception. And
thus, in particular, it is possible to catch them, and return to a
``normal'' program state again. Moreover, because of the subtle
semantics of Java for
\texttt{try-catch-finally} statements, it is even possible to
overwrite a JML error with another exception, and thus to catch it
without explicitly mentioning it in the program text.

This paper emerged as an unexpected difficulty in a larger project on
inlining security monitors (in collaboration with A. Tamalet,
University of Nijmegen, Netherlands). Ultimate goal of the project is
to develop a static verification method that can make monitoring of an
application unnecessary. The translation is defined in two steps:
first the monitor is translated into appropriate annotations of the
methods directly involved in the security property that the monitor
encodes, next the annotations are propagated, so that a verification
condition generator can generate provable proof
obligations. Correctness of the procedure is also proven in two steps:
(1) a monitored program only gets stuck if run-time checking of the
basic annotations would raise an exception; (2) propagation of
annotations would never result in an incorrect program being
accepted. One of the initial assumptions of the project was that any
incorrect application would always be rejected by run-time annotation
checking. However, when formalising the correctness proof, we realised
that this assumption was actually incorrect.

This paper does not intend to say that run-time verification is not
useful. Instead, its intention is to share an observation, and to make
people aware of a possible subtle interaction between an application
and the run-time verification mechanism, caused by the Java semantics
of the \texttt{try-catch-finally} statement~\cite{GoslingJSB05}. It is
well-known that the semantics of this statement can have an unexpected
impact on a program, see~\emph{e.g.},~\cite{Eckel02,Jenkov-online},
and that to avoid problems appropriate programming patterns should be
applied. However, this paper discusses the issue from a different
angle. In particular, it discusses how run-time verification can be
affected by the semantics of the \texttt{try-catch-finally} statement,
and how a too large trust in the run-time verifier could compromise
security. Moreover, the programming patterns that are proposed to
resolve this issue do not necessarily apply also in the case of
run-time verification. In particular, one cannot expect the programmer
to add code in the implementation of a finally block, that checks
whether run-time annotation checking might have produced an error.

Below, we first discuss some examples where the run-time annotation
violations are not signalled to the user. Next, we give an explanation
for this behaviour, based on the Java Language Semantics. Then,
Section~\ref{SecExploiting} discusses how this issue could impact
security, while Section~\ref{SecAvoiding} discusses how the problem
could be avoided. Finally, Section~\ref{SecConcl} concludes and gives
some general observations.

\section{Examples}\label{SecExample}

This section illustrates how a program can be manipulated, so that it
does not signal specification violations. Suppose that we have a
method \texttt{decrypt}, whose first argument is a \texttt{key}, and
its second a special \texttt{access\unsc code} that determines
whether one is allowed to do the decryption. Thus, the access code
should be in a particular range; for simplicity let us assume that it
should be below 10. In JML, this method could be specified as follows.

\begin{verbatim}
//@ requires access_code < 10;
public void decrypt(int [] key, int access_code){
  ...
}
\end{verbatim}


Now suppose that somebody who does not know the proper access code
nevertheless would like to try to decrypt a message. A naive attempt
would be to just guess an access code (say 36), and to call the
decrypt message anyway:

\begin{verbatim}
public void sneakyMethod(){
  int r = 36; 
  decrypt(key, r);                                           
}
\end{verbatim}

However, any attempt with a wrong guess, would be signalled by the JML
run-time checker \texttt{jmlrac}:
\begin{verbatim}
> jmlrac TryCatchExample
Exception in thread "main" 
org.jmlspecs.jmlrac.run-time.JMLInternalPreconditionError: 
by method TryCatch.decrypt
        at TryCatch.sneakyMethod(TryCatchExample.java:432)
        at TryCatch.internal$cheat(TryCatchExample.java:11)
        at TryCatch.cheat(TryCatchExample.java:292)
        at TryCatchExample.internal$main(TryCatchExample.java:1101)
        at TryCatchExample.main(TryCatchExample.java:1373)
\end{verbatim}

A possible way to hide this would be to put the code inside a
try-block, and to catch the JML error.

\begin{verbatim}
public void sneakyMethod(){
  try{ 
    int r = 36; 
    decrypt(key, r);                                           
  }           
  catch (Error e){
  }
}
\end{verbatim}
Run-time checking with \texttt{jmlrac} would not signal any error
anymore, even though one attempts to call \texttt{decrypt} with
illegal arguments. However, an attentive code inspector might get
suspicious by the attempt to catch an \texttt{Error}, since ``the
class \texttt{Error} and its subclasses are exceptions from which
ordinary programs are not ordinarily expected to recover.''~\cite[\S
11.5]{GoslingJSB05}. (JML errors are a subclass of the class
\texttt{Error}, which inherits directly from \texttt{Throwable}, and
is thus incompatible with the class \texttt{Exception}). Therefore,
errors can be distinguished from exceptions, from which recovery might
be possible.

A smarter way to hide this attempt is by using a finally block to
override the possible JML error.
\begin{verbatim}
public void sneakyMethod() throws ArbitraryException{
  try{ 
    int r = 36; 
    decrypt(key, r);                                           
  }           
  finally{
    throw new ArbitraryException();
  }
}
\end{verbatim}
In this case, the program will throw an
\texttt{ArbitraryException}~--~or terminate normally if this
\texttt{ArbitraryException} is caught by the method that triggered
\texttt{sneakyMethod}. For this program, manual code inspection to
reveal whether a JML error occurred might not be straightforward
(remember that the code for the \texttt{decrypt} method and the
invocation of \texttt{sneakyMethod} might be stored in different
classes). Naturally, tools for static validation of JML annotations
(such as ESC/Java~\cite{CokK04} or Jack~\cite{BartheBCGHLPR07:FMCO})
would detect this attempt to call \texttt{decrypt} with illegal
arguments.

\section{A Semantics-based Explanation}\label{SecExplanation}

This failure of run-time checking with \texttt{jmlrac} can be
explained directly on the basis of the Java semantics. The Java
Language Specification gives a detailed description of the behaviour
of the
\texttt{try-catch} and the 
\texttt{try-catch-finally} statements~\cite[\S
14.20]{GoslingJSB05}. For space reasons, we repeat only some fragments of
the specification that are relevant to the examples above.

\begin{quotation}
A \texttt{try-catch-finally} statement with a \texttt{finally} block
is executed by first executing the \texttt{try} block. [..]  If
execution of the \texttt{try} block completes abruptly because of a
throw of a value \(V\), then there is a choice:
\begin{itemize}
\item If the run-time type of \(V\) is assignable to the parameter of
any \texttt{catch} clause of the \texttt{try-catch-finally} statement,
then the first (leftmost) such \texttt{catch} clause is selected. The
value \(V\) is assigned to the parameter of the selected
\texttt{catch} clause, and the block of that \texttt{catch} clause is
executed. [..]

\item If the run-time type of \(V\) is not assignable to the parameter
of any \texttt{catch} clause of the \texttt{try-catch-finally}
statement, then the \texttt{finally} block is executed. Then there is
a choice:

\begin{itemize}
\item If the \texttt{finally} block completes normally, then the \texttt{try}
statement completes abruptly because of a throw of the value \(V\).
\item If the \texttt{finally} block completes abruptly for reason
\(S\), then the \texttt{try-catch-finally} statement completes
abruptly for reason \(S\) (and the throw of value \(V\) is discarded
and forgotten).
\end{itemize} 
[...]
\end{itemize}
\end{quotation}

Notice that if execution of a \texttt{catch} block completes abruptly
because of a throw of an exception, similar rules apply for the
execution of the \texttt{finally} block.


It is not too complicated to give a formal specification (in a formal
language) of the behaviour of the \texttt{try-catch-finally}
statement, based on the description above (from~\cite{GoslingJSB05}),
see for example the formalisation used within the LOOP
project~\cite{Jacobs01}.

Thus, in the second example above, the JML error is caught explicitly
by the catch clause, since its run-time type is assignable to the
parameter \texttt{Error e}.

In the third example, there is no matching \texttt{catch} clause (as
there are none). Thus, the \texttt{finally} block will be executed
directly, and as this will throw an (arbitrary) exception, the JML
error will be discarded. Thus, this behaviour of the \texttt{finally}
block in fact allows implicit catching of all exceptions, including
errors from which recovery should not be possible.

As mentioned above, this behaviour of finally clauses is well-known, and
requires use of special programming patterns to avoid
problems~\cite{Eckel02,Jenkov-online}. Finally clauses are typically
used to perform clean-up on objects that deal with an external
resource, \emph{e.g.},\ to close a file. Correct use of finally
clauses ensures that even though a program might terminate because of
an exception, it will not corrupt external resources. Notice however
that finally clauses are not essential; algorithms exist to eliminate
\texttt{jsr} instructions~--~their bytecode counterpart~--~, see
\emph{e.g.,}~\cite{ArthoB05}.

Notice further that the semantics of the \texttt{try-catch-finally}
statement prescribes that any abrupt completion of the
\texttt{finally} block discards the throw of value \(V\). Thus, any
\texttt{return}, \texttt{break} or \texttt{continue} statement would
have a similar effect. In particular, the following variation of
\texttt{sneakyMethod}: 
\begin{verbatim}
public void sneakyMethod(){
  try{ 
    int r = 36; 
    decrypt(key, r);                                           
  }           
  finally{
    return;
  }
}
\end{verbatim}
would have the same effect as the example above, \emph{i.e.}, it would
discard the JML error, but instead of throwing a new exception,
\texttt{sneakyMethod} would terminate normally.

Finally, we would like to remark that IDEs like Eclipse try to help
the user to avoid writing a \texttt{finally} block that completes
abruptly. From the Eclipse 3.0 release notes:

\begin{quotation}
The Java compiler can now find and flag finally blocks which cannot
complete normally  (as defined in the Java Language
Specification). Finally blocks which cannot complete normally can be
confusing and are considered bad practice.
\end{quotation}

However, this feature has limited value, as it does not give a warning
when in a \texttt{finally} block a method is called that might throw
an exception~--~even when this method has an explicit throws clause as
part of its declaration.


\section{Consequences for Security}\label{SecExploiting}

The main consequence of the behaviour described above, is that it
causes run-time checking (with \texttt{jmlrac}) not to be completely
transparent, where \emph{transparency} means that if no annotation
violations are reported (to the user), running a program with run-time
checking results in the same behaviour as running the program without
run-time checking~\cite{BurdyCCEKLLP03}.

In particular, both the second and the third example program in
Section~\ref{SecExample} will behave differently when executed with
run-time checking enabled or disabled: run-time checking will ensure
that the method \texttt{decrypt} is not executed, while it will be
executed when run-time checking is disabled. In fact, the annotation
violation is detected by the run-time checker, but the underlying
program semantics causes the violation not be reported to the user.

The loss of transparency can in particular be exploited to call a
method in an inconsistent state. Suppose we have a method \texttt{m}
that requires data to be consistent, as specified by a class invariant
(for example: the value of a variable is within a certain range). If
we call method \texttt{m} within a \texttt{try-catch-finally}
statement in a state that does not respect this invariant, the
violation of the invariant will not be reported to the
user~--~provided the state is set back to a consistent one before
completing the caller of method \texttt{m}.  This might make him/her
believe that the invariant is never violated, and thus that it is safe
to run the program without run-time checking enabled (which is more
efficient). But if this is done, the method
\texttt{m} will be called in this inconsistent state, which can cause
highly unexpected behaviour, and can easily violate security.

However, without adding additional \texttt{try-catch-finally}
statements, it is not possible to disguise all the annotation
violations~--~provided the program is sufficiently annotated. In
particular, run-time checking will check whether the postcondition of
the method containing the \texttt{try-catch-finally} statement holds
when leaving the method, and whether all class invariants have been
re-established. Thus, this prevents the inconsistent state to be
propagated. However, this has the consequence that the error is
reported at a different point, than that where it occurred, and thus
it breaks \emph{isolation}, \emph{i.e.,}\ the ability of the run-time
checker to identify the source of a problem~\cite{BurdyCCEKLLP03}.

\section{When does Run-time Checking Work?}\label{SecAvoiding} 

The problem discussed above is caused by the semantics related to the
\texttt{finally} clause, and its ability to discard
exceptions. Therefore, a simple way to avoid the problem is by
forbidding all uses of the finally clauses; but of course this
solution is a bit drastic, as it would mean that the JML run-time checker
cannot be used for every program written in Java. Moreover, as
mentioned above, finally clauses are useful to ensure clean-up of
external resources. 

In fact, to guarantee that run-time checking with \texttt{jmlrac} will
report all annotation violations to the user, it is sufficient to
restrict the behaviour of \texttt{try-catch-finally} statements where
the \texttt{try} or the \texttt{catch} clause could throw a JML
error. To ensure that the JML error is not discarded, either the
finally clause should terminate normally, or it should throw a JML
error itself. This ensures that the whole \texttt{try-catch-finally}
statement will terminate in an exceptional state, because of a JML
error. This is expressed by the following conditions: 

\begin{itemize}
\item If execution of the \texttt{try} block completes abruptly 
because of a throw of a JML error \(J\), then execution of the whole
\texttt{try-catch-finally} statement completes abruptly because of a 
throw of a JML error \(J\).
\item If execution of the \texttt{try} block completes abruptly
because of a throw of a value \(V\), that is not assignable to a JML
error, and if execution of the (leftmost) matching \texttt{catch}
clause completes abruptly because of a throw of a JML error \(J\),
then execution of the whole \texttt{try-catch-finally} statement
completes abruptly because of a throw of a JML error \(J\).
\end{itemize}

Notice that here we leave it unspecified whether the JML error that
the whole statement terminates with is the same as the JML error that
the \texttt{try} or \texttt{catch} block terminates with. Requiring
that these JML errors are identical would ensure even better isolation
of the problem.

However, it is not always easy to determine statically whether a
program satisfies this property. For example, consider the following
code fragment, where we suppose that method \texttt{m} throws a JML
error, and does not change the variable \texttt{b}.
\begin{verbatim}
try {	
  if (b) {m();}
  }	
finally {
  if (!b) {
     throw new ArbitraryException();
  }
}
\end{verbatim}
This satisfies the conditions above, because if \texttt{m} is called,
and the JML error is thrown, the finally clause will not throw a
\texttt{ArbitraryException}, thus the JML error will not be
discarded. However, it requires advanced analysis techniques to
determine this statically.

An alternative solution would be to have \texttt{jmlc} instrument the
\texttt{finally} block (by an additional \texttt{try-catch-finally}
statement, so that it ensures that JML errors are not
discarded~--~even when the \texttt{finally} block throws another
exception.

\begin{figure}[t]
\begin{verbatim}
class A {
    A a = new A();
}

class FinallyCatchesAll {

   public static void main (String [] args) {
     try {
       try { 
         A a = new A();
       }
       finally {
         throw new ArbitraryException();
       }
     }
     catch (Exception e) {
     };
     System.out.println ("still alive");
   }
}
\end{verbatim}
\caption{Recovering from Unrecoverable Errors}\label{FigStackOverflow}
\end{figure}

In fact, we believe that the real source of the problem is that a
finally clause can (implicitly) catch exceptions \emph{and} errors and
discard them, even though recovery from errors should not be possible,
see the Java Language Specification~\cite[\S 11.5]{GoslingJSB05}. This
problem is not only related to errors thrown by the run-time checker,
it applies to \emph{all} errors. Consider for example the program in
Figure~\ref{FigStackOverflow}. The call to the constructor of class
\texttt{A} will recursively invoke itself, to initialise the
field \texttt{a} in class \texttt{A}. This will quickly result in a
\texttt{java.lang.StackOverflowError}. However, the finally clause
will discard this error, and throw some other, arbitrary
exception. This exception is caught in the surrounding
\texttt{try-catch} statement, and thus the program will terminate by
printing \texttt{still alive}~--~and the occurrence of the
\texttt{StackOverflowError} has been forgotten. Notice that in a
similar way other errors, such as \texttt{NoClassDefFoundError}, 
\texttt{OutOfMemoryError} and internal JVM errors, can be
discarded.

This leads us to believe that the semantics of the
\texttt{try-catch-finally} statement should be adapted such that a
finally clause is only executed if the \texttt{try} or \texttt{catch}
clause terminates normally, or with an exception \(V\) that is
assignable to a variable of type \texttt{Exception}. Otherwise, if
\(V\) is not assignable to an \texttt{Exception}, the
whole \texttt{try-catch-finally} statement completes abruptly because
of a throw of value \(V\). This means that the only way to recover
from an error is by an explicit catch. In particular, to
ensure that run-time checking does report all annotation violations,
it is sufficient to check that there are no \texttt{catch} clauses for
class \texttt{Throwable}, \texttt{Error} or any JML-specific error
class. However, this would mean that upon occurrence of an error, the
JVM would have to ensure that external resources are handled
appropriately. 

An alternative solution would be to prioritise throwables, in such a
way that errors have a higher priority than exceptions, and to require
that finally clauses cannot discard errors (unless maybe, it throws an
error itself). Thus, if a \texttt{try} (or \texttt{catch}) block
throws an error \(E\), the finally \texttt{block} is executed. If
execution of the \texttt{finally} block results in exception \(E'\),
then exception \(E'\) is discarded, and the whole
\texttt{try-catch-finally} statement terminates abruptly because of a
throw of error \(E\).



\section{Conclusions}\label{SecConcl}
This paper discusses a case where run-time checking does not report an
annotation violation, even though it properly detects it. This is due
to an interaction between the run-time checker, and the semantics of
the programming language of the application that is being checked. We
argue that in this case, run-time checking is not transparent,
\emph{i.e.},\ applications for which no run-time errors are reported
to the user can have different behaviours depending on whether
run-time checking is enabled or not. In many cases, the annotation
violation will still be signalled later, because other, related
annotations are violated, but this has the consequence that the
run-time checker looses its ability to isolate problems at the point
where they occur.

This paper concentrates on the case of Java applications, annotated
with JML, but the observations presented here are applicable in any
context where the run-time checker is inlined into the code of the
application that is being monitored. If the programming language
contains a mechanism to recover from the errors that are used to
signal run-time annotation violations, then it is possible to recover
from these errors~--~and thus not to report them to the user.

In the case of Java, an additional complexity is that this recovery
also can happen implicitly, by means of the finally clause. We showed
that this finally clause also allows to recover (implicitly) from
other errors, that are in fact supposed to be unrecoverable.

\subsection*{Acknowledgements}
I would like to thank Alejandro Tamalet, who gave me useful feedback
on an earlier version of this paper, and pointed me to several places
where related problems were signalled. Moreover, he also suggested the
idea to prioritise errors over exceptions.

\bibliographystyle{plain}
\bibliography{bibli,../specification,/user/mhuisman/home/Research/Everest/Biblio/everest}
\end{document}
