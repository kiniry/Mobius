\documentclass[a4paper]{llncs}

\usepackage{epsfig}     %% om xfig gegenereerde pstexs te includen

% to be removed in final version
\pagestyle{plain}

\title{Formal specification and static checking of Gemplus' electronic
purse using ESC/Java}

\author{
  N\'estor Cata\~no
\and
  Marieke Huisman
}

\institute{
  INRIA Sophia-Antipolis, France \\ 
  \email{\{Nestor.Catano, Marieke.Huisman\}@sophia.inria.fr}
}

\newcommand{\noth}{\(\backslash\)\texttt{nothing}}
\newcommand{\fieldsof}{\(\backslash\)\texttt{fields\_of}}
\newcommand{\reach}{\(\backslash\)\texttt{reach}}
\newcommand{\comment}[1]{\marginpar{\framebox{\begin{minipage}{\marginparwidth}{#1}\end{minipage}}}}

\begin{document}
%\input{texdefs.tex}

\maketitle


This paper presents a case study in formal specification of smart card
programs, using ESC/Java. It discusses an electronic purse
application, provided by Gemplus, that we have annotated with
functional specifications (\emph{i.e.}~pre- and
postconditions, modifies
clauses and class invariants), that are as detailed as possible. The
specification is based on the informal documentation of the
application. The implementation has been checked \emph{w.r.t.}~the
specification, using ESC/Java.  This revealed several errors or
possibilities for improvement in the source code (\emph{e.g.}~removing
unnecessary tests).

Our paper shows that a relatively
lightweight use of formal specification techniques can already have
a serious impact on the quality of a program and its
documentation. Furthermore, we also present some ideas on how ESC/Java
could be further improved, both
\emph{w.r.t.}~specification and verification.
\\

\end{document}