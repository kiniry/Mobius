
\newcommand{\code}{\textit{code}}
\newcommand{\indexComp}{\textit{index}}





\section{Introduction} \label{bcsl}
This document is an overview of a bytecode level specification language, called for short BCSL and a compiler from a
 subset of the high level Java specification language JML to BCSL. 
 BCSL can express functional properties of Java bytecode programs in the form of method pre and postconditions, class and object 
invariants, assertions for particular program points like loop invariants.  Before going further, we discuss what advocates the need of a low level specification.
Traditionally, specification languages were tailored for high level languages. 
Source  specification allows to express functional or security properties about a program and they are / can successfully be used 
for software audit and validation. Still, source specification in the context of mobile code does not help a lot for several reasons.
First, the executable / interpreted code  may not be accompanied by its specified  source. Second, it is more reasonable for the 
code receiver to check the executable code than its source code, especially if he is not willing to trust the compiler. 
Third, if the client has complex requirements and even if the code respects them, in order to establish them, 
the code should be specified. Of course, for properties like well typedness this specification can be inferred \todo{what kind of 
techniques: abstract interp ? } 
but  for more sophisticated policies, an automatic inference will not work. It is in this perspective, that we propose to make the Java
bytecode benefit from the source specification by defining the BCSL language and a compiler from JML towards BCSL.    

In what follows subsection \ref{prelim} \ introduces the basic features of JML, subsection \ref{grammar} \ gives the formal grammar
 of the specification language and subsection 
 \ref{comJML} \ describes the compilation process from JML to BCSL.
  The full specification of the new user defined Java attributes in which the JML specification is compiled is given in the appendix.



\input prelim.tex
\input classFileExt.tex

