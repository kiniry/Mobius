


\subsection{Application to the Example using the \textsf{JAG} tool}


\begin{figure}

\begin{multicols}{2}
\begin{scriptsize}
\ttfamily
            
//@ invariant true =$>$ true;\\
\\
/*@ invariant ((7 - state) + \\
\quad @ \quad getTryLess()) $>$ 0;\\
\quad @*/\\
\\
/*@ constraint $\backslash$old(true) \\
\quad @ \quad \&\& !(state == 7 || state == 6) =$>$ \\
\quad @ (7 - state + getTryLess()) \\
\quad @ \quad $<$ $\backslash$old(7 - state + getTryLess())\\
\quad @ for initialiseTransaction, sendToBank, \\
\quad @ pinValidation, complete,   printReceipt;\\
\quad @*/ \\ 
\\
/*@ constraint $\backslash$old(true) \\
\quad @ \quad \&\& !(state == 7 || state == 6) =$>$ \\
\quad @ (7 - state + getTryLess()) \\
\quad @ \quad $<$= $\backslash$old(7 - state + getTryLess())\\
\quad @ for getTryLess;\\
\quad @*/ \\ 
\\
/*@ constraint $\backslash$old(true) \\
\quad @ \quad \&\& !(state == 7 || state == 6) =$>$ \\
\quad @ ( state == 1\\
\quad @ || state == 2\\
\quad @ || state == 3\\
\quad @ ||state == 4\\
\quad @ || state == 5)\\
\quad @*/ \\ 

\normalfont
\setlength{\parindent}{1cm}
\end{scriptsize}
\end{multicols}
\caption{Transaction System Example}
\label{FigOutExample}


\end{figure}





The \textsf{JAG} (JML Assertion Generator)\footnote{
\url{http://lifc.univ-fcomte.fr/\~{}groslambert/JAG}}
implement the translation presented in Theorem~\ref{until} 
(see Sect.~\ref{sec-OPLoop}). Given 
a \textsf{Loop} primitive and a Java/JML File, the tool 
automatically generates  an output file containing the
 JML annotations. 
%Following
%Proposition~\ref{PropLeadsTo}, this gives rise to the JML
%assertions
We obtain the class $Transaction$ completed with the
JML annotations  displayed in Fig.~\ref{FigOutExample}
%- generated annotations are displayed in boxes. 
The consistency of the JML specification with the extra annotations
 has automatically been checked 
using the method described in~\cite{bdgZB05}. 
We can also verify the generated annotations  against the Java 
code using any tool for JML; 
in our case, we have discharged
all proof obligations
 automatically using JACK~\cite{BurdyRL03}.

As an input language, the \textsf{JAG} tool can take 
a linear temporal logic language
dedicated to Java~\cite{Huis02,articleJournalTL}, which can deal
with exceptional termination of methods. The formulae of this 
language are reduced by the tool to JML invariants 
and \textsf{Loop} primitives. The table \ref{table-result} summarizes
the results obtained.
\begin{table}

{\scriptsize
\begin{tabularx}{\linewidth}
{|X|X|X|X|}
\hline
Example Name & 
Number of temporal properties to verify & 
Number of line annotation generated & 
Number of PO (automatically proved) \\
\hline
TransactionSystem & 2 & 18 & 92  (91) \\
\hline
AtmTransaction & 2 & - & - \\
Purse & - & - & -\\
\hline
\end{tabularx}
}
\caption{Results}
\label{table-result}
\end{table}


\subsection{Progress Hypothesis $PH$}
\label{sec-env}

The hypothesis ($H_{S_1}$) is sufficient. However, there exists 
executions of the environment which satisfy a liveness property but 
do not satisfy $H_{S_1}$. %Recall that ($H_S$) is expressed 
%by the following.
%$$(\mathsf{F^{\infty}} M ~ \mathbf{called})   (H_S)$$
For our running example, suppose that the environment $Section$ 
verifies  $H_{S_1}$, that means that 
$Section$ continuously performs some $Transaction$s. This is not a realistic
use of the class $Transaction$. One would like to stop the 
progress of section when the current transaction is terminated,
i.e., a progress of $Section$ over $Transaction$ is needed
only when the system is inside the loop. 
%This requires to take account of the loop itself. 

We propose the hypothesis ($H_{S_2}$) to take into account the executions
which satisfy the liveness but do not satisfy ($H_{S_1}$).  For that, the
\textsf{Loop} operator has to be considered.

%More generally, 
%Suppose that we verify 
Let $Q \leadsto_{JML} R (J,V,M)$ be the \textsf{Loop} primitive of a
liveness property. 
%We give a condition ($H_S'$) on the environment
%for which the modality $Q \leadsto_{JML} R (J,V,M)$ is also verified.
The expression of $H_{S_2}$ uses the LTL operator 
$\mathsf{G^{\infty}}$ (``almost everywhere''). 
Intuitively, $\mathsf{G^{\infty}}P$ means 
that after a finite number of states, the property $P$ holds forever.
The semantics of $\mathsf{G^{\infty}}$ is the following~:
$$\sigma_i \models G^{\infty} P \equiv_{def} \exists j \geq i . (\forall k . 
( k \geq j \Rightarrow \sigma_k \models P)).$$ 

\begin{proposition}
\label{prop-hs}
Assuming $\Sigma_C : \mathcal{A}_{1-6}$ (Theorem~\ref{until}),
%the class $C$ satisfies the annotations 
%given in Proposition~\ref{PropLeadsTo}. 
if
$$\hspace*{14em} \Sigma_C \models  \mathsf{G^{\infty}}  
(\neg Q \vee R) \hspace*{8em} (H_{S_2}) $$
then $\Sigma_C \models  Q \leadsto_{JML} R (J,V,M)$. 
\end{proposition}
The sketch of the proof is given
in Appendix~\ref{sec-proof-proposition-MPH}.

%\marginpar{La preuve n'est pas si claire. Peut-être la revoir.}


The disjunction of ($H_{S_1}$) and ($H_{S_2}$) 
constitutes the Progress Hypothesis~($PH$)
on the environment. This hypothesis ensures that if a class conforms 
to JML annotations $\mathcal{A}_{1-6}$ of Theorem~\ref{until} 
then it satisfies the \textsf{Loop} operator.


\begin{proposition}[Progress Hypothesis ($PH$)]
Assuming that $\Sigma_C : \mathcal{A}_{1-6}$ (Theorem~\ref{until}),
if 
$$\hspace*{9em} \Sigma_C \models  \mathsf{G^{\infty}}  
(\neg Q \vee R) \vee
(\mathsf{F^{\infty}} M ~ \mathbf{called}) \hspace{8em} (PH)$$ 
then $\Sigma_C \models  Q \leadsto_{JML} R (J,V,M)$. 
\end{proposition} 
The sketch of the proof is given 
in Appendix~\ref{sec-proof-proposition-MPHminimal}.



The hypothesis ($PH$) being a LTL formula, it can be
model-checked~\cite{spin97,Moped03} on the environment $TS_C$ when $TS_C$
is a finite-state model. For infinite or large size models, $PH$ is checked by
proof. We explain how to generate the appropriate proof obligations 
in Section~\ref{sec-verif}.


%%% Local Variables: 
%%% mode: latex
%%% TeX-master: "main2"
%%% End: 
