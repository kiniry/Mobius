This section presents the Java Applet Correctness Kit (or \JACK).
This tool, already briefly described in \cite{BRL-JACK}, is
a formal tool that allows one to prove properties on Java programs
using the Java Modeling Language \cite{Leavens-Baker-Ruby03} (JML).
It has been developed initially by Gemplus, and since 2003 by INRIA
in the context of a bilateral collaboration with Gemplus, and then
in the context of the INSPIRED project.


It generates proof obligations allowing to prove that the Java code
conforms to its JML specification.  The lemmas are translated into an
internal formula language called JPOL (Java/Jack Proof Obligation
Language). Then JPOL verification conditions are translated into
different prover language, namely Coq, PVS, Simplify and the B
language \cite{bbook}, allowing to use the automatic provers Simplify
and the provers developed within the B method and interactive provers
like the Coq proof assistant, PVS or Click'n'Prove.

 But the tool is not yet another lemma generator for Java, since it
 also provides a lemma viewer integrated in the eclipse
 IDE\footnote{\texttt{http://www.eclipse.org}}, which is one of the
 most commonly used IDE for Java developers.  This allows to hide the
 formalisms used behind a graphical interface.  Lemmas are presented
 to users in a way they can understand them easier, by using the Java
 syntax and highlighting code portions to help the
 understanding. Using \JACK, one does not have to learn a formal
 language to be confident on code correctness.

 The remainder of the section is organized as follows.  Subsection
 \ref{JavaAppletCorrectnessKit} presents the architecture and the main
 principles of the tool we have developed.  Subsection~\ref{JPOL}
 presents the Java/Jack Proof Obligation Language (JPOL).
 Subsection~\ref{Industrialisation} describes more precisely the
 innovative parts of the tool and explains why we consider it as
 accessible to any developers.
