\documentclass[final,nocolorBG,a4,marieke,nototal,ps, accumulate,slideColor]{prosper}

\addtolength{\textheight}{-1cm}

\usepackage{pstricks,pst-node,pst-text,pst-3d}
\usepackage{amsmath}
\usepackage{alltt}
\usepackage{epsfig}

\usepackage{colordvi}
\usepackage{proof}
\usepackage{epsf}
\usepackage{amssymb}

\input{prooftree}
\newcommand{\textttbf}[1]{\texttt{\textbf{#1}}}
 \def \bsl       {\symbol{92}}
 \def \unsc      {\symbol{95}}

\title{High-level languages for security properties}
\subtitle{}
\author{Marieke Huisman}
\institution{INRIA Sophia Antipolis
\medskip\\
\Blue{Contributions by LIFC, Vasco and Everest project}}
\email{Marieke.Huisman@inria.fr}

\slideCaption{High-level languages for security properties}
\Logo{}
\begin{document} 

\maketitle

\begin{slide}{Overview}
\begin{itemize}
\item Motivation
\item Contributions
\item Tools, implementations and on-going work
\end{itemize}
\end{slide}

\begin{slide}{Enforcing security properties}
\begin{itemize}
\item \Blue{Security expert}: from global notions of security to set
of rules
\item \Blue{Developers}: try to obey rules
\item \Blue{Security audit}: manual code inspection whether rules
obeyed
\item Need for tools to help security audit
\item Gap between global notion and rules, and between security rules
and specifications understood by tools 
\end{itemize}
\end{slide}


\begin{slide}{Typical examples of security rules}
\begin{itemize}
  \item Atomicity \item Applet life cycle \item Exception handling
  \item Access control
\end{itemize}
\end{slide}


\begin{slide}{Two different approaches}
\begin{itemize}
\item Express security properties in existing language\\
Advantage: reuse of existing tools
\item Develop specific languages\\
Advantage: tailored techniques
\end{itemize}
\end{slide}

\begin{slide}{Use of existing languages}
\begin{itemize}
\item Different formats available to express such properties
\begin{itemize}
\item Different kinds of automata
\item Temporal logic
\end{itemize}
\item Define mapping of such a format into JML annotations
\item Safety and liveness properties
\item Implementation 
\begin{itemize}
\item JAG: from temporal property to JML
\item JACK: propagation and verification
\end{itemize}
\end{itemize}
\end{slide}

\begin{slide}{Design of specific languages}
\begin{itemize}
\item Concrete applications used to identify relevant properties
\item Results: 
\begin{itemize}
\item Property language for G\'en\'esyst 
\item Access control policy language for MECA
\item JTPL (Java Temporal Pattern Language)
\end{itemize}
\end{itemize}
\end{slide}

\begin{slide}{JAG: JML Annotation Generator}
\begin{itemize}
\item Generates JML annotations for a given temporal property,
including liveness properties
\item Tool accepts different input formats (internal representation
with B\"uchi automata)
\item For liveness properties a progress condition on the environment
has to be established: use of Hoare logic with progress condition
\Red{\[\{|P|\}S\{|Q|\}\]}
\vspace*{-1em}
\begin{itemize}
\item Partial correctness \(\{P\}S\{Q\}\)
\item If \(S\) diverges, it satisfies a progress hypothesis
\end{itemize}
\end{itemize}
\end{slide}

\begin{slide}{Current work: generalisation to other
programming/annotation languages}
\begin{itemize}
\item Define trace-based semantics of programming language
\item Define semantics of the annotations
\item Characterising the annotations needed for the verification of
LTL properties
\item Characterising verification conditions for the appropriate
annotations
\end{itemize}
\end{slide}

\begin{slide}{JACK: Java Applet Correctness Kit}
\begin{itemize}
\item Implements an annotation propagation algorithm
\item Core-annotations for methods directly involved in the method
\item Propagation ensures possibility of static verification
\end{itemize}
\end{slide}

\begin{slide}{Propagation example}
\begin{alltt}
\textbf{public void m() \{
   ...
   \Blue{// requires TRANSACT == 0}
   JCSystem.beginTransaction();
   \Blue{// modifies TRANSACT } 
   \Blue{// ensures TRANSACT == 1}
   ...
   \Blue{// requires TRANSACT == 1}
   JSSystem.commitTransaction();
   \Blue{// modifies TRANSACT } 
   \Blue{// ensures TRANSACT == 0}
   ...
   \}}
\end{alltt}
\end{slide}

\begin{slide}{Propagation example}
\begin{alltt}
\textbf{\Red{//@ requires TRANSACT == 0;}
\Red{//@ modifies TRANSACT;}
\Red{//@ ensures TRANSACT == 0;}
public void m() \{
   ...
   JCSystem.beginTransaction();
   ...
   JSSystem.commitTransaction();
   ...
   \}}
\end{alltt}
\end{slide}

\begin{slide}{Current work: formalisation of the annotation
generation}
\begin{itemize}
\item Properties expressed as MVA automata
\begin{itemize}
\item Transitions labelled with method names
\item Conditional branching
\item Can inspect program state
\item Transitions can update automaton state
\end{itemize}
\item MVA monitors program execution
\item MVA state encoded with ghost variables, transitions defined as
appropriate set-statements, unreachable error state expressed with
invariant
\end{itemize}
\end{slide}


\begin{slide}{Current work: formalisation of the annotation
generation}
\begin{itemize}
\item Two step translation, using special \Blue{preset} and
\Blue{postset} constructs as part of the method specification
\item Monitoring program with MVA does not get stuck iff run-time
checking does not find (new) assertion violation
\item Future work: extend with propagation, and prove static
verification result
\end{itemize}
\end{slide}


\begin{slide}{MECA: Models for Access Control}
\begin{itemize}
\item Tool to describe different types of access control policies 
\begin{itemize}
\item DAC - Discretionary Access Control
\item MAC - Mandatory Access Control  
\item RBAC - Role Based Access Control
\end{itemize}


\item Access control policy and application modeled as B-machines
\item Meca defines necessary information to describe the
access control policy
\item Meca produces a B reference monitor enforcing the security policy 
(precondition for verification, condition for monitoring)
\end{itemize}
\end{slide}

%\begin{slide}{Current work: }
%\end{slide}

\begin{slide}{To summarise}
\begin{itemize}
\item Two different approaches: 
\begin{itemize}
\item Development of special targeted language 
\item Encoding of existing format in annotations
\end{itemize}
\item Work continues in different directions:
\begin{itemize}
\item Generalisation
\item Formalisation
\end{itemize}
\end{itemize}
\end{slide}
\end{document}
