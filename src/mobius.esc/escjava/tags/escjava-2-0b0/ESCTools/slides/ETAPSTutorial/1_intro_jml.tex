\documentclass[
pdf,
%a4,
nocolorBG,
%colorBG,
slideColor,
%accumulate,
%draft,
erik,
%frames,
]{prosper}
%%%%%%%%%%%%%%%%%%%%%%%%%%%%%%%%%%%%%%%%%%%%%%%%%%%%%%%%%%%%%%%%%%%%%%%%%%%
\usepackage{alltt}
\usepackage{pstricks,pst-node,pst-text,pst-3d}
\usepackage{textcomp}
%\usepackage{colordvi}
\newcommand{\Red}[1]{{\red #1}}
\usepackage[all]{xy}

%%%%%%%%%%%%%%%%%%%%%%%%%%%%%%%%%%%%%%%%%%%%%%%%%%%%%%%%%%%%%%%%%%%%%%%%%%%

%\newrgbcolor{Yellowish}{0.90 0.85 0.650}

%\newrgbcolor{red}{1 0 0}
%\newrgbcolor{purple}{0.4 0 0.7}
%\newrgbcolor{lightpurple}{0.63 0.13 0.94}

%\newrgbcolor{lime}{0.73 1 0}
\newrgbcolor{green}{0.133 0.56 0} % lichter
%\newrgbcolor{green}{0.10 0.43 0}

%\newrgbcolor{knalblue}{0 0 1}
\newrgbcolor{blue}{.2 .36 .77}
%\newrgbcolor{darkblue}{0.28 0.24 0.55}

%%%%%%%%%%%%%%%%%%%%%%%%%%%%%%%%%%%%%%%%%%%%%%%%%%%%%%%%%%%%%%%%%%%%%%%%%%%

\newcommand{\embf}[1]{\textit{\textbf{#1}}}
\newcommand{\rmbf}[1]{\textrm{\textbf{#1}}}
\newcommand{\ttbf}[1]{\mbox{\tt \textbf{#1}}}
\newcommand{\code}[1]{{\rm \texttt{\textbf{\small #1}}}}

\myitem{1}{\mbox{{$\bullet$}}}
\newcommand{\old}     {\(\backslash\)old}
\newcommand{\vooralle}{\(\backslash\)forall}
\newcommand{\eris}{\(\backslash\)exists}
\newcommand{\everything}{\(\backslash\)everything}
\newcommand{\nothing}{\(\backslash\)nothing}
\newcommand{\nonnull} {non\verb|_|null}
\newcommand{\result}  {\(\backslash\)result}
\renewcommand{\familydefault}{phv}
\renewcommand{\rmdefault}{phv}
\newif\ifignore

\newrgbcolor{Bluish}{0.9 0.9 1.0}
\newcommand{\doos}[1]{\psshadowbox[fillstyle=solid,
                        fillcolor=Bluish,
                        framearc=0.2,
                        framesep=2mm]
                        {#1}}

%%%%%%%%%%%%%%%%%%%%%%%%%%%%%%%%%%%%%%%%%%%%%%%%%%%%%%%%%%%%%%%%%%%%%%%%%%%%%

\title{\embf{\blue 
       {\huge Introduction to JML
      }}}
\author{\embf{\Large{\red David Cok, Joe Kiniry, and Erik Poll}}
       }
\institution{\embf{Eastman Kodak Company, University College Dublin, \\ and Radboud University Nijmegen}}
\slideCaption{{\blue David Cok, Joe Kiniry \& Erik Poll - ESC/Java2 \& JML Tutorial }}

%%%%%%%%%%%%%%%%%%%%%%%%%%%%%%%%%%%%%%%%%%%%%%%%%%%%%%%%%%%%%%%%%%%%%%%%%%%%%

\begin{document}

\maketitle 

\boldmath

\ifignore
%%%%%%%%%%%%%%%%%%%%%%%%%%%%%%%%%%%%%%%%%%%%%%%%%%%%%%%%%%%%%%%%%%%%%%%%%%%%%
\begin{slide}{Test \hfill}

{\Large Large}
{\large large}
{\normalsize normal}
niks
{\small small}
{\footnotesize footnote}
{\scriptsize script}
{\tiny tiny}

{\bf bf}
{\rm rm}
{\it it}
{\sf sf}
{\sc sc}

\textit{textit}
\textrm{textrm}
\textbf{textbf}
\textsf{textsf}
\textsc{textsc}

\end{slide}
\fi

%%%%%%%%%%%%%%%%%%%%%%%%%%%%%%%%%%%%%%%%%%%%%%%%%%%%%%%%%%%%%%%%%%%%%%%%%%%%%
\begin{slide}{Outline of this tutorial}
\vspace*{-2ex}

First
\begin{itemize}
\item introduction to {\green JML}
\item overview of tool support for JML, esp.\ runtime assertion checking (using {\green jmlrac}) and extended static checking {\green ESC/Java2}
\end{itemize}
Then
\begin{itemize}
\item ESC/Java2: Use and Features
\item ESC/Java2: Warnings
\item Specification tips and pitfalls
\item Advanced JML: more tips and pitfalls
\end{itemize}
interspersed with demos.

\end{slide}

%%%%%%%%%%%%%%%%%%%%%%%%%%%%%%%%%%%%%%%%%%%%%%%%%%%%%%%%%%%%%%%%%%%%%%%%%%%%%

%%%%%%%%%%%%%%%%%%%%%%%%%%%%%%%%%%%%%%%%%%%%%%%%%%%%%%%%%%%%%%%%%%%%%%%%%%%%%
\part{{\Large \red The Java Modeling Language \\
    JML \\
    [2ex] {\large\black \texttt{www.jmlspecs.org}}}}
%%%%%%%%%%%%%%%%%%%%%%%%%%%%%%%%%%%%%%%%%%%%%%%%%%%%%%%%%%%%%%%%%%%%%%%%%%%%%

%%%%%%%%%%%%%%%%%%%%%%%%%%%%%%%%%%%%%%%%%%%%%%%%%%%%%%%%%%%%%%%%%%%%%%%%%%%%
\overlays{2}{
\begin{slide}{JML {\footnotesize {\black by Gary Leavens et al.}}}
\vspace*{-4ex}
{\blue Formal specification language} for Java 
\medskip
\begin{itemize}
\item
to specify behaviour of Java classes
\item
to record design \&implementation decisions
\end{itemize}
\medskip
by adding {\green assertions} to Java source code, eg
\medskip
\begin{itemize}
\item ~{\green preconditions}
\item ~{\green postconditions}
\item ~{\green invariants}
\end{itemize}

\medskip

as in Eiffel (Design by Contract), but more expressive.

\medskip

\fromSlide{2}{\doos{Goal: JML should be easy to use for any Java
    programmer.}}

\end{slide}}

%%%%%%%%%%%%%%%%%%%%%%%%%%%%%%%%%%%%%%%%%%%%%%%%%%%%%%%%%%%%%%%%%%%%%%%%%%%%
\begin{slide}{JML}
\vspace*{-4ex}

To make JML easy to use:

\medskip
\begin{itemize}
\item
JML assertions are added as comments in .java file,

between {\green \code{ /*@}} \ldots {\green \code{ @*/}},
or after {\green \code{ //@}},

\medskip
\item
Properties are specified as Java boolean expressions, extended
  with a few operators
({\green \old, \vooralle, \result, \ldots}).

\medskip
\item
using a few keywords
({\green \code{requires},
\code{ensures},
\code{signals},
\code{assignable},
\code{pure},
\code{invariant},
\code{non\_null},
\ldots})

\end{itemize}

\end{slide}

%%%%%%%%%%%%%%%%%%%%%%%%%%%%%%%%%%%%%%%%%%%%%%%%%%%%%%%%%%%%%%%%%%%%%%%%%%%%
\begin{slide}{requires, ensures}
\vspace*{-4ex}

{\blue Pre-} and {\blue post-conditions} for method can be specified.

\begin{alltt}
\code{
{\green /*@}{\blue requires} amount >= 0;
    {\blue ensures}  balance == \old(balance-amount) &&
              \result == balance;
{ \green  @*/}
  public int{\green debit}(int amount) \{ 
    ...
  \}
}
\end{alltt}

Here \code{\old(balance)} refers to the value of \code{balance}
before execution of the method.

\end{slide}

%%%%%%%%%%%%%%%%%%%%%%%%%%%%%%%%%%%%%%%%%%%%%%%%%%%%%%%%%%%%%%%%%%%%%%%%%%%%
\begin{slide}{requires, ensures}
\vspace*{-4ex}

JML specs can be as strong or as weak as you want.

\begin{alltt}
\code{
{\green /*@}{\blue requires} amount >= 0;
    {\blue ensures} {\blue true};
{ \green  @*/}
  public int{\green debit}(int amount) \{ 
    ...
  \}
}
\end{alltt}

This default postcondition ``\code{ensures true}'' can be omitted.

\end{slide}

%%%%%%%%%%%%%%%%%%%%%%%%%%%%%%%%%%%%%%%%%%%%%%%%%%%%%%%%%%%%%%%%%%%%%%%%%%%%
\begin{slide}{Design-by-Contract}
\vspace*{-4ex}

Pre- and postconditions define a {\blue contract} between a class and
its clients:
\medskip
\begin{itemize}
\item
Client must {\green ensure precondition}
and may {\green assume postcondition}
\item
Method may {\green assume precondition}
and must {\green ensure postcondition}
\end{itemize}

\bigskip

Eg, in the example specs for \code{debit}, it is the obligation of the
client to ensure that \code{amount} is positive.  The \code{requires}
clause makes this {\green explicit}.

\end{slide}

%%%%%%%%%%%%%%%%%%%%%%%%%%%%%%%%%%%%%%%%%%%%%%%%%%%%%%%%%%%%%%%%%%%%%%%%%%%%
\begin{slide}{signals}
\vspace*{-4ex}

{\blue Exceptional postconditions} can also be specified.
\vspace*{-1ex}
\begin{alltt}
\code{
{\green /*@}{\black requires} amount >= 0;
    {\black ensures}  true;
    {\blue signals (BankException e) 
               amount > balance         &&
               balance == \old(balance) &&
               e.getReason().equals("Amount too big");}
{\green   @*/}
  public int{\green debit}(int amount) throws BankException \{ 
   ...
  \}
}
\end{alltt} %}

\end{slide}

%%%%%%%%%%%%%%%%%%%%%%%%%%%%%%%%%%%%%%%%%%%%%%%%%%%%%%%%%%%%%%%%%%%%%%%%%%%%
\begin{slide}{signals}
\vspace*{-4ex}

Exceptions mentioned in throws clause are allowed by default.
To change this, there are three options:

\begin{itemize}
\item
To {\em rule out all} exceptions, use a {\blue normal\_behavior}
\begin{alltt} \code{ {\green /*@}{\blue normal\_behavior}
       {\black requires} ...
       {\black ensures}  ...              
 {\green   @*/}
}
\end{alltt}
\vspace*{0.1ex}
\item
To {\em rule out particular} exception \code{E},
add 
\begin{alltt}\code{ {\blue signals} (E){\green false}; 
}
\end{alltt}
\vspace*{0.1ex}
\item
To {\em allow only some exceptions}, add 
\begin{alltt}\code{ {\blue signals_only} E1, ..., E2; 
}
\end{alltt}
\end{itemize}

\end{slide}

%%%%%%%%%%%%%%%%%%%%%%%%%%%%%%%%%%%%%%%%%%%%%%%%%%%%%%%%%%%%%%%%%%%%%%%%%%%%
\begin{slide}{invariant}
\vspace*{-4ex}
{\blue Invariants} (aka {\em class} invariants) are properties that
must be maintained by all methods, e.g.,
\begin{alltt}
\code{public class {\green Wallet} \{
  public static final short{\green MAX_BAL} = 1000;
  private short{\green balance};
  {\green /*@}{\blue invariant 0 <= balance &&
                      balance <= MAX_BAL;}
  {\green   @*/}
  ...  }
\end{alltt} % }

Invariants are implicitly included in all pre- and postconditions.

Invariants must {\em also} be preserved if exception is thrown!
\end{slide}

%%%%%%%%%%%%%%%%%%%%%%%%%%%%%%%%%%%%%%%%%%%%%%%%%%%%%%%%%%%%%%%%%%%%%%%%%%%%
\begin{slide}{invariant}
\vspace*{-4ex}

Invariants document design decisions, e.g.,
\begin{alltt}
\code{ public class{\green Directory} \{
 private File[]{\green files};
{\green /*@}{\blue invariant} 
     files != null    
     &&
     (\vooralle int i; 0 <= i && i < files.length;
                   ; files[i] != null &&
                     files[i].getParent() == this);
  {\green @*/} }
\end{alltt} % }

Making them {\green explicit} helps in understanding the code.
\end{slide}

%%%%%%%%%%%%%%%%%%%%%%%%%%%%%%%%%%%%%%%%%%%%%%%%%%%%%%%%%%%%%%%%%%%%%%%%%%%%
\begin{slide}{non\_null}
\vspace*{-4ex}

Many invariants, pre- and postconditions are about references not
being \texttt{null}.  {\blue non\_null} is a convenient short-hand for
these.

\begin{alltt}
\code{ public class Directory \{

  private{\green /*@}{\blue non\_null}{\green @*/} File[] files;

  void createSubdir({\green{/*@}}{\blue non\_null}{\green @*/} String name)\{
   ...
  {\green /*@}{\blue non\_null}{\green @*/} Directory getParent()\{
   ...
}
\end{alltt} %} %} %}

\end{slide}

%%%%%%%%%%%%%%%%%%%%%%%%%%%%%%%%%%%%%%%%%%%%%%%%%%%%%%%%%%%%%%%%%%%%%%%%%%%%
\begin{slide}{assert}
\vspace*{-4ex}

An {\blue \code{ assert}} clause specifies
a property that should hold at some point in the code, e.g.,
\begin{alltt}
\code{ if (i <= 0 || j < 0) \{
      ...
  \} else if (j < 5) \{
     {\green //@}{\blue assert i > 0 && 0 < j && j < 5;}
      ...
  \} else \{
     {\green //@}{\blue assert i > 0 && j > 5;}
      ...
  \}  }
\end{alltt}

\end{slide}

%%%%%%%%%%%%%%%%%%%%%%%%%%%%%%%%%%%%%%%%%%%%%%%%%%%%%%%%%%%%%%%%%%%%%%%%%%%%
\begin{slide}{assert}
\vspace*{-4ex}

JML keyword \code{assert} now also in Java (since Java 1.4).

\medskip

Still, assert in JML is more expressive, for example in

\begin{alltt}
\code{  ...
  for (n = 0; n < a.length; n++) 
       if (a[n]==null) break;
{\green /*@}{\blue assert (\vooralle int i; 0 <= i && i < n; 
                            a[i] != null);                }
  {\green @*/} }
\end{alltt} 

\end{slide}

%%%%%%%%%%%%%%%%%%%%%%%%%%%%%%%%%%%%%%%%%%%%%%%%%%%%%%%%%%%%%%%%%%%%%%%%%%%%
\begin{slide}{assignable}
\vspace*{-4ex}

{\blue Frame properties} limit possible side-effects of methods.

\begin{alltt}
\code{ {\green /*@}   requires amount >= 0; 
     {\blue assignable} balance;
         ensures balance == \old(balance)-amount;
{ \green  @*/}
  public int{\green debit}(int amount) \{ \}
    ...
}
\end{alltt} %}
E.g., \code{debit} can {\em only} assign to the field \code{balance}.

NB this does {\em not} follow from the post-condition.

\medskip

Default assignable clause: \code{assignable \everything}.

\end{slide}

%%%%%%%%%%%%%%%%%%%%%%%%%%%%%%%%%%%%%%%%%%%%%%%%%%%%%%%%%%%%%%%%%%%%%%%%%%%%
\begin{slide}{pure}
\vspace*{-4ex}

A {\blue method without side-effects} is called {\blue pure}.

\begin{alltt}
\code{
  public{\green /*@}{\blue pure}{\green @*/} int getBalance()\{...

  Directory{\green /*@}{\blue pure non\_null}{\green @*/} getParent()\{...\}

}
\end{alltt} %}

Pure method are implicitly \code{assignable \nothing}.

\medskip

Pure methods, and only pure methods, can be used {\em in} specifications,
eg.
\begin{alltt}
\code{\scriptsize //@ invariant 0<=getBalance() && getBalance()<=MAX_BALANCE;
}
\end{alltt} 

\end{slide}

%%%%%%%%%%%%%%%%%%%%%%%%%%%%%%%%%%%%%%%%%%%%%%%%%%%%%%%%%%%%%%%
\begin{slide}{JML recap}
\vspace*{-4ex}

The JML keywords discussed so far:
\begin{itemize}
\item {\green \code{requires}}
\item {\green \code{ensures}}
\item {\green \code{signals}}
\item {\green \code{assignable}}
\item {\green \code{normal\_behavior}}
\item {\green \code{invariant}}
\item {\green \code{non\_null}}
\item {\green \code{pure}}
\item {\green \code{\old}, \code{\vooralle}, \code{\eris}, \code{\result}}
\end{itemize}

{\blue This is all you need to know to get started!}

\end{slide}

%%%%%%%%%%%%%%%%%%%%%%%%%%%%%%%%%%%%%%%%%%%%%%%%%%%%%%%%%%%%%%%%%%%%%%%%%%%%%
\part{{\Large \red Tools for JML}}
%%%%%%%%%%%%%%%%%%%%%%%%%%%%%%%%%%%%%%%%%%%%%%%%%%%%%%%%%%%%%%%%%%%%%%%%%%%%%

%%%%%%%%%%%%%%%%%%%%%%%%%%%%%%%%%%%%%%%%%%%%%%%%%%%%%%%%%%%%%%%%%%%%%%%%%%%%
\overlays{3}{\begin{slide}{tools for JML}
\vspace*{-2ex}

\begin{itemize}
\fromSlide{1}{\item$\!$ {\red parsing} and {\red typechecking}} 
\fromSlide{2}{\item$\!$ {\red runtime assertion checking}:\\
      {\blue test} for violations of assertions {\blue during execution}\\
      {\green jmlrac}}
\fromSlide{3}{\item$\!$ {\red extended static checking} ie. automated program verification:\\
      {\blue prove} 
      that contracts are never violated
      {\blue at compile-time}\\
      {\green ESC/Java2}\\

      This is program verification, not just testing.}
\end{itemize}
\end{slide}}

%%%%%%%%%%%%%%%%%%%%%%%%%%%%%%%%%%%%%%%%%%%%%%%%%%%%%%%%%%%%%%%%%%%%%%%%%%%%%
\overlays{5}{
\begin{slide}{runtime assertion checking}
\vspace*{-4ex}
{\blue jmlrac compiler} by Gary Leavens, Yoonsik Cheon, et al. at Iowa State Univ.

\begin{itemize}
\item
translates {\blue JML assertions}
into {\blue runtime checks}:
\begin{quote}
  during execution, {\em all} assertions are tested and any violation
  of an assertion produces an Error.
\end{quote}
\fromSlide{2}{\item$\!$
  {\blue cheap \& easy} to do as part of existing testing practice}
\fromSlide{2}{\item$\!$
  {\blue better testing and better feedback}, because {\green more properties} are tested,
  at {\green more places} in the code}
\fromSlide{3}{\textit{ \ \\ \scriptsize \it Eg,
   ``Invariant violated in line 8000'' \ after 1 minute 
  instead of  \ ``NullPointerException in line 2000'' \  after 4 minutes}}
\end{itemize}

\fromSlide{4}{Of course, an assertion violation can be an {\em error in
    code} {\green or} an {\em error in specification}.}

\fromSlide{5}{
  The {\blue jmlunit} tool combines jmlrac and {\green unit testing}.}
\end{slide}}

%%%%%%%%%%%%%%%%%%%%%%%%%%%%%%%%%%%%%%%%%%%%%%%%%%%%%%%%%%%%%%%%%%%%%%%%%%%%%
\begin{slide}{runtime assertion checking}
\vspace*{-4ex}

jmlrac can generate complicated test-code for free.  E.g., for
\begin{alltt}
\code{{\green /*@} ...
    {\blue signals (Exception) 
                 balance == \old(balance);}
 {\green  @*/}
  public int{\green debit}(int amount) \{ ... \}
}
\end{alltt} 
it will test that {\green if \code{debit} throws an exception,
the balance hasn't changed, and all invariants still hold}.

\bigskip

{\scriptsize jmlrac even checks \code{\vooralle} if the domain of 
quantification is finite.}

\end{slide}

%%%%%%%%%%%%%%%%%%%%%%%%%%%%%%%%%%%%%%%%%%%%%%%%%%%%%%%%%%%%%%%%%%%%%%%%%%%%%
\overlays{6}{
\begin{slide}{extended static checking}
\vspace*{-4ex}

{\blue ESC/Java(2)}
\begin{itemstep}
\item$\!$
extended static checking = fully automated program verification, with some
compromises to achieve full automation
\item$\!$
{\blue {\em tries} to {\em prove} correctness of specifications,}\\
{\green at compile-time, fully automatically}
\item$\!$
\Red{\em not sound}: ESC/Java may miss an error that is actually present
\item$\!$
\Red{\em not complete}: ESC/Java may warn of errors that are impossible
\item$\!$
 but {\blue finds lots of potential bugs quickly}
\item
good at proving absence of runtime exceptions {\scriptsize (eg
Null-, ArrayIndexOutOfBounds-, ClassCast-)} and verifying
relatively simple properties.
\end{itemstep}

\end{slide}}

%%%%%%%%%%%%%%%%%%%%%%%%%%%%%%%%%%%%%%%%%%%%%%%%%%%%%%%%%%%%%%%%%%%%%%%%%%%%%
\begin{slide}{ESC/Java(2) credits }
\vspace*{-4ex}

\begin{itemize}
\item
{\blue ESC/Java} originally developed at DEC SRC -- later Compaq, and now HP Research --
by Rustan Leino, Cormac Flanagan, Mark Lillibridge, Greg Nelson, Raymie Stata, and James Saxe.
\item
{\blue ESC/Java2},  extension that supports more of JML, developed by David Cok and Joe Kiniry.
\end{itemize}

\end{slide}

%%%%%%%%%%%%%%%%%%%%%%%%%%%%%%%%%%%%%%%%%%%%%%%%%%%%%%%%%%%%%%%%%%%%%%%%%%%%
\begin{slide}{static checking vs runtime checking}
\vspace*{-4ex}

One of the assertions below is wrong:
\begin{alltt}
\code{  if (i <= 0 || j < 0) \{
      ...
   \} else if (j < 5) \{
       {\green //@ assert i > 0 && 0 < j && j < 5;}
       ...
   \} else \{
       {\green //@ assert i > 0 && j > 5;}
       ...
   \}  
}\end{alltt}

Runtime assertion checking {\blue {\em may}} detect this with a
comprehensive test suite.

ESC/Java2 {\blue {\em will}} detect this at compile-time.

\end{slide}

%%%%%%%%%%%%%%%%%%%%%%%%%%%%%%%%%%%%%%%%%%%%%%%%%%%%%%%%%%%%%%%%%%%%%%%%%%%%
\begin{slide}{static checking vs runtime checking}
\vspace*{-4ex}

Important differences:

\begin{itemize}
\item ESC/Java2 checks specs at {\blue compile-time}, \\
      jmlrac checks specs at {\green run-time}

\item ESC/Java2 {\blue proves} correctness of specs,\\
      jml only {\green tests} correctness of specs.
\\
Hence
\begin{itemize}
\item ESC/Java2 independent of any test suite, \\
      results of runtime testing only as good as the test suite,
\item ESC/Java2 provides higher degree of confidence.

\medskip
The price for this: you have to specify all pre- and postconditions
of methods (incl. API methods)
and invariants needed for {\blue modular verification}
\end{itemize}

\end{itemize}

\end{slide}

%%%%%%%%%%%%%%%%%%%%%%%%%%%%%%%%%%%%%%%%%%%%%%%%%%%%%%%%%%%%%%%%%%%%%%%%%%%%
\overlays{1}{\begin{slide}{more JML tools}
\vspace*{-4ex}

\begin{itemize}
\item$\!$ {\blue javadoc-style documentation}:  {\green jmldoc}
\item$\!$ {\green Eclipse} plugin
\item
Other full {\blue verification} tools:
\begin{itemize}
\item$\!$ {\green LOOP tool + PVS} (Nijmegen)
\item$\!$ {\green JACK} (Gemplus/INRIA)
\item$\!$ {\green Krakatoa tool + Coq} (INRIA)
\item$\!$ {\green KeY} (Chalmers + Germany)
\end{itemize}
These tools also allow {\blue interactive} verification
(whereas ESC/Java2 only aims at {\blue fully automatic} verification)
and can therefore handle more complex properties.
\item runtime {\blue detection of invariants}: {\green Daikon} (Michael Ernst, MIT)
\item$\!$ {\blue model-checking} multi-threaded programs: {\green Bogor} (Kansas State)
\end{itemize}
See \texttt{www.jmlspecs.org}

\end{slide}}

%%%%%%%%%%%%%%%%%%%%%%%%%%%%%%%%%%%%%%%%%%%%%%%%%%%%%%%%%%%%%%%
\begin{slide}{Related Work}
\vspace*{-4ex}

\begin{itemize}
\item {\blue jContract} tool for Java by {\green Parasoft}
\item {\blue Spec\#} for {\green C\#} by {\green Microsoft}
\item {\blue Spark-Ada} for subset of Ada by {\green Praxis Critical Systems Ltd.}
\item {\blue OCL} specification language for {\green UML}
\end{itemize}
\end{slide}

%%%%%%%%%%%%%%%%%%%%%%%%%%%%%%%%%%%%%%%%%%%%%%%%%%%%%%%%%%%%%%%%%%%%%%%%%%%%
\begin{slide}{Acknowledgements}

\vspace*{-4ex}
Many people and groups have contributed to JML and related tools.

\begin{itemize}
\item {\scriptsize Gary Leavens leads the JML effort at Iowa St. 
    Contributors include Albert Baker, Clyde Ruby, Curtis
    Clifton, Yoonsik Cheon, Anand Ganapathy, Abhay Bhorkar, Arun
    Raghavan, Kristina Boysen, David Behroozi. Katie Becker, Elisabeth
    Seagren, Brandon Shilling, Katie Becker, Ajani Thomas, and Arthur
    Thomas.}
\item {\scriptsize The ESC project at SRC included Rustan 
    Leino, Cormac Flanagan, Mark Lillibridge, Greg Nelson, Raymie
    Stata, and James Saxe.}
\item {\scriptsize More people at many different places are contributing to JML}
\end{itemize}
  
\end{slide}

%%%%%%%%%%%%%%%%%%%%%%%%%%%%%%%%%%%%%%%%%%%%%%%%%%%%%%%%%%%%%%%%%%%%%%%%%%%%
\begin{slide}{More information}

\vspace*{-4ex}
These websites and mailing lists can provide more information (and have links to even more):

\begin{itemize}

\item JML: www.jmlspecs.org
\item mailing lists:  jmlspecs-interest@lists.sourceforge.net\\
jmlspecs-developers@lists.sourceforge.net 

\item[]
\item ESC/Java2:  http://secure.ucd.ie/products/opensource/ESCJava2/
\item ESC/Java:    http://www.research.compaq.com/SRC/esc/
\item mailing list:  jmlspecs-escjava@lists.sourceforge.net

\end{itemize}
  
\end{slide}

\end{document}

